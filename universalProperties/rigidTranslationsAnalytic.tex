\documentclass{ximera}

\title{Rigid Translations: Analytics}
\begin{document}
\begin{abstract}
    This section describes the analytic perspective of what makes a Rigid Translation.
\end{abstract}
\maketitle

\subsection*{The analytic view}
    
    Here is a video!
    
    \youtube{pd_r4HqHb6o}
    
    In analytic terms a rigid translation is when the relation that represents the function is ``changed universally' in the following way: for any two $x$ and $y$ with $f(x) = y$, whatever is being done to that pair of related points, the same thing is being done to \textit{every single pair of related points}. Moreover, this change must always be a matter of adding or subtracting (in a particular way which we will discuss below), \textit{not} multiplying or dividing. This is easier to explain starting with an example. 
    
    \begin{explanation}
        Consider the function $f(x) = x^2 + 3x - 2$. The points $(2,8)$ and $(-1,-4)$ are both on the graph of $f(x)$. Now if we apply the rigid translation of ``move up 2 and left 3'' we can determine where these two points go. In particular the point $(2,8)$ would go to $(-1,10)$ and the point $(-1,-4)$ would go to $(-4,-2)$. There are two important things here to notice. The first is that \textit{both points moved in the same way}. They both went up 2 and left 3. In order to be a rigid translation, \textit{every} point on the graph would move the same way. The second is that the effect of ``moving up 2'' was the same as adding two to the $y$-value and the effect of ``moving left 3'' was the same as subtracting three from the $x$-value. Specifically, the effects were done with adding and subtracting, not with multiplying or dividing.
        
        Knowing this means we can also determine what a rigid translation is from what it does to a single point. Consider the same function as above, $f(x) = x^2 + 3x - 2$ with the same point $(2,8)$ on its graph. If we applied an unknown rigid translation, but we knew that the point $(2,8)$ was translated to the point $(1,3)$ by that rigid translation, we can figure out what the rigid translation was. Since $(2,8)$ went to $(1,3)$ we can see that the point moved left 1 point (the $x$-value went from 2 to 1) and down 5 (the $y$-value went from 8 to 3). Furthermore we can then apply this information to know that the point $(-1,-4)$ on the original graph of $f(x)$ must go left one and down five, so it would end up at $(-2,-9)$.
    \end{explanation}
        
    Notice how, in our example, we talked about the translations that applied to the $x$-value, and translations that applied to the $y$-value separately. This separation is made even more clear when we write the translation in functional notation. Let's consider a new function $f(x)$ whose graph is;

    \begin{minipage}{\textwidth}
        \begin{center}
            \begin{tikzpicture}
                \begin{axis}[
                    axis x line=middle,
                    axis y line=middle,
                    minor tick num=5,
                    x label style={at={(axis description cs:1,0.5)},anchor=south},
                    y label style={at={(axis description cs:0.5,1)},anchor=west},
                    xlabel={$x$},
                    ylabel={$y$},
                    xmin=-8,
                    xmax=8,
                    ymin=-17,
                    ymax=17
                    ]
                \addplot[<->,domain=-3.1:3.1, samples=300]{1/2*(x-3)*(x-2)*(x+1)*(x+3)};
                \end{axis}
            \end{tikzpicture}
        \end{center}
    \end{minipage}
    
    We now want to apply the rigid translation that adds 4 to the $x$ and subtracts 5 from the $y$-value. We can write the result of this translation by first giving it a name, say $g$, and defining it as a modification of the original $f$. This turns out to be slightly more tricky than one may think however. 
    
    Intuitively we might try to write our translation as follows; $g(x) = f(x+4) - 5$. After all, we wanted to add $4$ to the $x$ and subtract $5$ from the $y$, so this seems pretty reasonable. Let's see what the graph $f(x+4) - 5$ looks like.
    
    \begin{minipage}{\textwidth}
        \begin{center}
            \begin{tikzpicture}
                \begin{axis}[
                    axis x line=middle,
                    axis y line=middle,
                    minor tick num=5,
                    x label style={at={(axis description cs:1,0.5)},anchor=south},
                    y label style={at={(axis description cs:0.5,1)},anchor=west},
                    xlabel={$x$},
                    ylabel={$y$},
                    xmin=-8,
                    xmax=8,
                    ymin=-17,
                    ymax=17
                    ]
                \addplot[<->,domain=-7.1:-0.9, samples=300]{1/2*(x-3+4)*(x-2+4)*(x+1+4)*(x+3+4) - 5};
                \end{axis}
            \end{tikzpicture}
        \end{center}
    \end{minipage}
    
    A near miss! It appears that we got the vertical movement correct but it shifted the wrong direction horizontally; it shifted left instead of right! The specifics on why this horizontal shift is backwards are outside the scope of this course, but in this specific context it suffices to remember that anything that effects the $x$ is the opposite of what you'd expect. The mantra to remember is ``Everything about $x$ is backward''. Since you would expect that we want to add 4 to $x$ (to move it to the right) what we \textit{actually} want to do (in order to move it to the right) is \textit{subtract} 4. Thus the $g(x)$ we actually want to use is $g(x) = f(x-4)-5$ whose graph is;
    
    \begin{minipage}{\textwidth}
        \begin{center}
            \begin{tikzpicture}
                \begin{axis}[
                    axis x line=middle,
                    axis y line=middle,
                    minor tick num=5,
                    x label style={at={(axis description cs:1,0.5)},anchor=south},
                    y label style={at={(axis description cs:0.5,1)},anchor=west},
                    xlabel={$x$},
                    ylabel={$y$},
                    xmin=-8,
                    xmax=8,
                    ymin=-17,
                    ymax=17
                    ]
                \addplot[<->,domain=0.9:7.1, samples=300]{1/2*(x-3-4)*(x-2-4)*(x+1-4)*(x+3-4) - 5};
                \end{axis}
            \end{tikzpicture}
        \end{center}
    \end{minipage}
    
    \subsection*{In summary:}
    
    Writing this rigid translation using function notation (ie $f(x-4)-5$) is the `functional representation'. In general, a rigid translation of a function $f$ can be written by adding (or subtracting) values from the functional argument (the part inside the parentheses that follow the letter $f$), or by adding/subtracting values to the overall result after applying $f$. So, if you want to add some number $c$ to the $x$ values and another number $k$ to the $y$ value, you can write the translation as;
    
    \[
        f(x - c) + k
    \]
    
    Notice that the $c$ inside has a \textit{subtraction} sign in front of it. If you remember it in this form then you can write the $c$ as you would expect (since the $c$ is being subtracted not added which will take care of the 'opposite of what you'd except' part). Thus if you want to move the graph to the right 4 units, you can use $c=4$ in the form, giving $f(x - c) = f(x - (4))$. In comparison if you wanted to move the graph to the left by 4 you would do the same process. Since it's \textit{to the left} you would think $c = -4$, so plugging that into the \textit{form above} we have $g(x) = f(x - c) = f(x - (-4) ) = f(x+4)$ which is correct.
    
    Play with the following graph to see what happens as you change the parameters. Be careful to observe the effect of the sign of the parameters in the example. Hopefully using this interactive graph will help make clear how changing the values (rigidly) translates the graph!
    
    \desmos{wn5zcytloz}{}{500}
    
    
    \begin{problem}
        In order to move a graph up or down we need to...
        \begin{multipleChoice}
            \choice{Add something to the $x$-value before the function evaluation in order to go up, and subtract something from the $x$-value before the function evaluation to go down.}
            \choice{Subtract something from the $x$-value before the function evaluation in order to go up, and add something to the $x$-value before the function evaluation to go down.}
            \choice[correct]{Add something to the $y$-value after the function evaluation in order to go up, and subtract something from the $y$-value after the function evaluation to go down.}
            \choice{Subtract something from the $y$-value after the function evaluation in order to go up, and add something to the $y$-value after the function evaluation to go down.}
        \end{multipleChoice}
    \end{problem}
    
    \begin{problem}
        In order to move a graph left and right we need to...
        \begin{multipleChoice}
            \choice{Add something to the $x$-value before the function evaluation in order to go right, and subtract something from the $x$-value before the function evaluation to go left.}
            \choice[correct]{Subtract something from the $x$-value before the function evaluation in order to go right, and add something to the $x$-value before the function evaluation to go left.}
            \choice{Add something to the $y$-value after the function evaluation in order to go right, and subtract something from the $y$-value after the function evaluation to go left.}
            \choice{Subtract something from the $y$-value after the function evaluation in order to go right, and add something to the $y$-value after the function evaluation to go left.}
        \end{multipleChoice}
    \end{problem}
    
        
    

\end{document}