\documentclass{ximera}
\input{../preamble}
\title{Geometric Vs Analytic Viewpoints}
\begin{document}
\begin{abstract}
    We discuss what Geometric and Analytic views of mathematics are and the different roles they play in learning and practicing mathematics.
\end{abstract}
\maketitle

\subsection*{The Hope...}

    When considering mathematics there are two common points of view; the geometric perspective and the analytic perspective. This is true regardless of mathematical level and, in fact, you have already experienced this many times throughout your math education. Unfortunately, the virtues (and flaws) of each perspective aren't typically discussed. In the next few sections we will explicitly discuss the virtues and flaws of these two perspectives in an effort to help students find the best perspective for their individual learning style. Moreover, this will lend transparency to why we learn certain topics in specific ways as well as help students differentiate between understanding a concept versus being able to apply it in practice.

\subsection*{The False Hierarchy: Analytic is (not!) ``better" than Geometric.}

    Mathematicians will often talk about ``seeing" the math. This does not mean they just visualize a bunch of complex equations and bizarre symbols floating around, but rather they are visualizing some geometric representation of what is happening; like a picture of a graph, or stacking of blocks of various sizes. The point is that they \textit{aren't} picturing equations or variables.
    
    The process of visualizing some kind of picture or representation is a geometric process, what we would call geometric reasoning. This is in contrast to thinking about equations, variables, and doing mental computation. These kinds of things are what we call analytic reasoning. Most math classes tend to emphasize analytic skills over geometric skills.\footnote{This is because mathematics ultimately wants precision, and as we've discussed, geometric information tends to be imprecise by its nature. Thus ultimately we want to be able to apply analytic reasoning to our problem; but that doesn't mean that geometric reasoning is useless!} This tends to promote the idea that analytic skills are ``better'' or ``more important'' than geometric skills, but the truth is that each skill has its proper use, and one is not inherently better than the other. In fact, usually the geometric and analytic perspectives complement each other. Geometric perspective is often used to gain an intuitive understanding of a problem or situation, and then the analytic perspective is used to make that intuition into precise information/answers.

You can watch a more detailed discussion on this topic here!

\youtube{oZ0hmA4Y8ck}


\begin{problem}
    One usually uses either analytic or geometric points of view when addressing a problem.
    \begin{multipleChoice}
        \choice{True.}
        \choice[correct]{False.}
    \end{multipleChoice}
\end{problem}

\begin{problem}
    Most commonly we use...
    \begin{multipleChoice}
        \choice{Analytic reasoning first to get a big picture idea, then geomtric reasoning to get specific details for an answer as needed.}
        \choice[correct]{Geometric reasoning first to get a big picture idea, then analytic reasoning to get specific details for an answer as needed.}
        \choice{Geometric reasoning to understand the problem, then analytic reasoning to get the trends or patterns from the problem.}
        \choice{Analytic reasoning to calculate your answer, and geometric reasoning only if you can't solve the equations.}
        \choice{Geometric reasoning to picture the problem, and analytic reasoning only if you can't figure out how to draw a representation of the problem.}
    \end{multipleChoice}
\end{problem}



\end{document}



