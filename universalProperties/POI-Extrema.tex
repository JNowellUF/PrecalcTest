\documentclass{ximera}

\title{Points of Interest on Graphs - Extrema}
\begin{document}
\begin{abstract}
    This section describes extrema of a function as points of interest (PoI) on a graph.
\end{abstract}
\maketitle



\subsection*{Extrema of a function}
    
    Here is a video!
    
    \youtube{lV4guSVBSaU}

    In a similar way to the intercepts, we will often be interested in the extrema of a function, that is the maximum and minimum values. Extrema come in two variations in addition to being either a max or min; specifically there are local (also referred to as relative) extrema, and global (aka absolute) extrema.
    
    \subsubsection*{Absolute Extrema}
        Absolute extrema are typically the easier to state simply; they are the (absolutely) largest and smallest%
        \footnote{%
            The terms 'largest' and 'smallest' are often ambiguous in mathematics. Here, we mean the largest as in the most positive number (or the value that is furthest right on a number line) possible, and the 'smallest' means the number furthest toward the negative side (or the furthest left on the number line) possible.%
            }
        values that a function attains in it's entire domain. Note that the absolute minimum can be positive, just as the absolute maximum can be negative. Moreover, one should note that a function may not attain a maximum (or a minimum) value. For example, $f(x) = x^3$ on the domain of all real numbers has no maximum or minimum value. The function $f(x) = x^2$ has no maximum value but has an absolute minimum. The function $f(x) = -\sqrt{x}$ has a maximum but no minimum value. Even if the function attains its maximum or minimum, even though the value of the max or min is (certainly) unique, \textbf{there can be more than one point that attain those values}. 
        
        For example, a function that constantly oscillates between $-1$ and $1$ (such as $\sin(x)$) attains it's maximum value (specifically the value $1$) and it's minimum value (specifically the value $-1$) \textit{infinitely many times}. It's important to draw the distinction between how many absolute maximum (or minumum) \textit{values} there are (at most one) and \textit{how many points} attain an absolute maximum (or minimum) which could be anything between 0 (no absolute extrema) and infinitely many.
        
    \subsubsection*{Local Extrema}
        The other type of extrema are the local (or relative) extrema. These are points that represent extreme values in some small 'neighborhood' of the function. Essentially these are points that are at the bottom of a valley or at the top of a hill on the graph. The formal definition is a bit intimidating, and locating these values can be exceptionally challenging.%
        \footnote{%
            In fact, this is one of the primary areas of study in calculus 1 and comes down to figuring out the zeros of a related function to the one you are investigating.%
            }
        Nonetheless we will visit how to find local extrema in some specific cases in the coming topics, and further exploration of this aspect will be a major focus in calculus for those that continue.
        
        
        \begin{problem}
            What is a good explanation of what an extrema of a function is?
            \begin{multipleChoice}
                \choice{The extrema are points where the function attains an important value, like a break-even point of a profit function.}
                \choice{The extrema of a function is the lowest or highest point a function might attain over all real numbers.}
                \choice[correct]{Extrema represent the maximum and minimum values of a function within some defined interval of it's domain.}
                \choice{There aren't always extrema of a function, so there is no good explanation.}
            \end{multipleChoice}
        \end{problem}

\end{document}