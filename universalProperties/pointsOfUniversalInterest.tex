\documentclass{ximera}

\title{Points of Interest on Graphs - Zeros}
\begin{document}
\begin{abstract}
    This section describes types of points of interest (PoI) in general and covers zeros of functions as one such type.
\end{abstract}
\maketitle


    There are several types of points that tend to be important for interpreting a graph, regardless of the function involved. In the next few sections we will discuss some of these types and why they may be useful; however finding exact values of these points requires techniques specific to the functions involved and will be covered in later segments of this course. 
    
\subsection*{Zeros of a function}
    
    Here is a video!
    
    \youtube{RYq5r8K5yDk}
    
    The first Point of interest that we discuss are the "zeros". Zeros of a function are the domain ($x$) values that yield zero when you calculate the function at those values (ie $y = 0$). The primary interest of the zeros stems from their importance in applications as well as their usefulness in mathematics.
    
    In applications zero almost always represents an important pivot value in a model. For example; a profit equation equaling zero means you have hit your break-even point (an important point in economics and business). A function that is determining height of an object being launched (like a bullet, rocket, or baseball) equaling zero typically means that the object has returned to `ground level', ie that the object's travel has concluded, which is useful if you are trying to hit a specific point on the ground (such as targeting for a missile or recovery of a space flight after re-entry).
    
    Mathematically zero holds a number of very special roles and properties as well. It's important enough that we will conclude this topic with more discussion about zero (as well as the value 1, and why equal signs are so important and undervalued). Suffice it to say that, something equaling zero enables a whole litany of mathematical options to extract useful information from our functions, so knowing where it equals zero can be incredibly powerful information.
    
    Unfortunately, how to determine when (or even if) a function equals zero can be quite difficult.%
    \footnote{%
        Indeed, one such example is the Riemann-Zeta function, which has a million dollar bounty to anyone that can show where all the zeros of the function are!
        }
    Fortunately we will be discussing particular types of functions and establishing methods to find zeros of those specific functions. This is especially important (and useful) in calculus courses, where determining when a function is zero is a major part of the necessary algebra in almost every topic of the course.
    
    \begin{problem}
        Which of the following might be a reason to care about the zeros of a function? (Select all correct answers)
        \begin{multipleChoice}
            \choice{They are easy to calculate.}
            \choice[correct]{The zero is the output, so it may represent something like the break-even point on a profit curve, or when an object hits the ground on a height function.}
            \choice{The zero is the input, so it tells you something about the initial condition of the problem.}
            \choice{The zero of a function is something I'll need to calcualte for calculus, so I might as well get use to it now.}
        \end{multipleChoice}
    \end{problem}

\subsection*{Intercepts of a function}
    Now, you may be wondering what is so special about the $y$-value, specifically, being zero. The answer is a \textit{lot}, but that doesn't mean we don't care about $x$. We also often want to know what happens when the domain (aka $x$) is zero, but this is also generally easier to find (since the domain values are controlled, so you can simply `plug in' the value 0 for $x$ and compute). The set of \textit{points} where a function equals zero are the `$x$-intercepts', and the \textit{point} where the `$x$-value' is zero is called the $y$-intercept.%
    \footnote{%
        Notice that intercepts are \textit{points}. This means an intercept should \textit{always} be written as some kind of point, like a coordinate pair. Thus an intercept at $x=3$ and $y=0$ would be properly labeled or declared as `the $x$-intercept at $(3,0)$. You should \textbf{not} say `the $x$-intercept is 3.'%
        }
        
    \begin{problem}
        Which of the following might be a good reason to care about the intercepts of a function?
        \begin{multipleChoice}
            \choice[correct]{Using zero as the input to get the y-intercept tells you something about the initial condition of the problem.}
            \choice{The x-intercepts aren't ever useful.}
            \choice{The intercepts can be helpful because they might be on an exam.}
            \choice{The intercepts in general are rarely useful, so I'll never care about them.}
        \end{multipleChoice}
    \end{problem}

\end{document}