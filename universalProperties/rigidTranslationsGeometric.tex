\documentclass{ximera}

\title{Rigid Translations: Geometrics}
\begin{document}
\begin{abstract}
    This section describes the geometric perspective of Rigid Translations.
\end{abstract}
\maketitle

\subsection*{The geometric view}

    Here is a video!
    
    \youtube{EjsFnmKxbPU}
    
    The result of a rigid translation is incredibly apparent when looking at the graph (before and after) of the function. The result is also more easily stated (at least in terms of what is happening to the graph) in geometric terms. In essence, a rigid translation changes the graph of the function in a very specific way; by ``picking up the function and moving it's location on the $x$-$y$ plane". The key observation is that the \textbf{only} thing that changes about the graph is its position on the $x$-$y$ plane. The graph is only moved left/right or up/down, but it is not stretched, squished, rotated, flipped, or anything else. 
    
    Consider the following example:
    
    \begin{minipage}{\textwidth}
        \begin{center}
            \begin{tikzpicture}
                \begin{axis}[
                            axis x line=middle,
                            axis y line=middle,
                            minor tick num=5,
                            x label style={at={(axis description cs:1,0.5)},anchor=south},
                            y label style={at={(axis description cs:0.5,1)},anchor=west},
                            xlabel={$x$},
                            ylabel={$y$},
                            xmin=-8,
                            xmax=8,
                            ymin=-17,
                            ymax=17
                            ]
                \addplot[<->,domain=-3.1:3.1, samples=300]{1/2*(x-3)*(x-2)*(x+1)*(x+3)};
                \end{axis}
            \end{tikzpicture}
        \end{center}
    \end{minipage}
    
    An example of a rigid translation would be to move the graph to the right by $4$ and down by $5$. That would change the above graph to the following:
    
    \begin{minipage}{\textwidth}
        \begin{center}
            \begin{tikzpicture}
                \begin{axis}[
                    axis x line=middle,
                    axis y line=middle,
                    minor tick num=5,
                    x label style={at={(axis description cs:1,0.5)},anchor=south},
                    y label style={at={(axis description cs:0.5,1)},anchor=west},
                    xlabel={$x$},
                    ylabel={$y$},
                    xmin=-8,
                    xmax=8,
                    ymin=-17,
                    ymax=17
                    ]
                \addplot[<->,domain=0.9:7.1, samples=300]{1/2*(x-3-4)*(x-2-4)*(x+1-4)*(x+3-4) - 5};
                \end{axis}
            \end{tikzpicture}
        \end{center}
    \end{minipage}
    
    As we can see, we appear to have the same curve (the same `graph') but it moved to a different location on the plane.\footnote{%
        When we say move, we mean that it moved left/right up/down, it didn't rotate, stretch, or otherwise change it's orientation or shape/size on the $x$-$y$ plane.%
        } 
    This is how we know the transformation we applied is a \textit{rigid transformation}. Another way of putting this is that you can envision the curve on the $x$-$y$ plane is made of cast iron or titanium and you are merely allowed to slide it around; left/right, or up/down. In our example we slid it to the right by $4$ units and down by $5$ units, but the curve itself remained exactly as it was; it has the same shape and didn't rotate or stretch.

    Generally we don't want to have to write out descriptions of rigid translations when discussing them however, so instead we use the analytic form (algebra and function notation) to concisely describe this behavior. We discuss this next.
    
    \begin{problem}
        A rigid translation is...
        \begin{multipleChoice}
            \choice{The process of shifting a graph horizontally or vertically.}
            \choice{The process of stretching a graph horizontally or vertically.}
            \choice[correct]{The process of moving a graph horizontally or vertically \textbf{without changing the size or shape of the graph.}}
            \choice{The process of flipping and/or deforming a graph.}
            \choice{The process of moving a graph around.}
        \end{multipleChoice}
    \end{problem}
    
    
    
\end{document}