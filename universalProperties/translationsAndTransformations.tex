\documentclass{ximera}
\input{../preamble}
\title{Transform And Translates}
\begin{document}
\begin{abstract}
    This covers doing transformations and translations at the same time. In particular we discuss how to determine what order to do the translations/transformations in.
\end{abstract}
\maketitle

\subsection*{Doing Translations and Transformations at the Same Time.}
    The general rule of thumb is that the order of the translations/transformations follows the order of operations in the analytic form; with the important rule that \textit{everything involving $x$ is backward}. 
    
    As an example, lets suppose we have a function $f(x)$ and we define a manipulation of $g(x)$ as $g(x) = 2 \cdot f\left(\frac{x+1}{2}\right) + 1$. Moreover the following is the graph for $f(x)$;
    
    \begin{center}
        \begin{tikzpicture}
            \begin{axis}[
                axis x line=middle, 
                axis y line=middle, 
                minor tick num=5, 
                x label style={at={(axis description cs:1,0.5)},anchor=south},
                y label style={at={(axis description cs:0.5,1)},anchor=west},
                xlabel={$x$}, 
                ylabel={$y$},
                xmin=-12, 
                xmax=12, 
                ymin=-5, 
                ymax=5
                ]
            \addplot[<->,domain=-4:4, samples=300]{sin(deg(x))};
            \end{axis}
        \end{tikzpicture}
    \end{center}
    
    Let's consider the manipulation of $f(x)$ which we called $g(x)$. There is a lot to unpack in the definition of $g(x)$ and we need to make sure that we apply the transformations and translations in the right order, but remember that just means we need to apply them in the order of operations. The easiest way to do this is to pretend to plug in an $x$ value and determine what the order is that each computation is applied.
    
    \textbf{Calculating the $x$ Stuff:}
    
    If we were to plug in a number for $x$, the first thing we would do is add 1. This appears to be a shift to the right one, but remember \textit{everything about $x$ is backwards}, so it's really a shift to the \textit{left} one. Next we would divide the $x$-value by 2, which (again because \textit{everything about $x$ is backwards}) means we stretch the $x$ value to twice it's original width. Thus in total it looks like we would shift left 1 \textit{and then} make the graph twice as wide as it was originally... but \textit{again} \textit{\textbf{everything} about $x$ is backwards}, including the order we apply the changes in, so we \textit{actually} want to make everything twice as wide \textit{and then} shift everything left one.
    
    \textbf{Calculating the $y$ Stuff:}
    
    Now we tackle the $y$ values. To do this, we pretend that we have calculated some value for the $f( )$ portion and replace it with a number. In our case then we would pretend to replace $f\left(\frac{x+1}{2}\right)$ with a number, and then see what computations apply from there. Thus the first change (once we have replaced the `$f$' part with a number) would be to multiply the function result by 2, which means stretching the $y$-value to twice it's height.%
    \footnote{%
        Thankfully $y$ changes work like one expects, so no need to flip our thinking here like we did with the $x$ stuff.%
        }
    Finally, adding one to the overall function means we will shift everything up one. Since the $y$-values work the way we would normally expect (unlike the $x$ stuff), this means that in total we would make everything twice as tall \textit{and then} move everything up one.
    
    It may help to do a concrete computation to see that our translation/transformations occur in this order. Below we compute $g(3)$ and make note of the translations/transformations as we go.
    \begin{explanation}
        Let's compute $g(x) = 2 \cdot f\left(\frac{x+1}{2}\right) + 1$ for $x = 3$.
        \begin{enumerate}
            \item The first thing we do to compute $g(3)$ is compute $3+1$. Adding 1 to the $x$ value is a shift \wordChoice{\choice{to the right}\choice[correct]{to the left}\choice{up}\choice{down}} 1.
            \item The second computation once we have $3+1 = 4$ is to ``divide by 2'', which is a horizontal \textit{stretch} by 2, and yields $4 \div 2 = 2$.
            \item Since $x$ does everything backwards we want to reverse the last two instructions; so we will ``stretch horizontally by 2" \textit{then} ``shift left by 1".
            \item Next we would compute the actual functional value of $f(2)$ which we will denote with $f(2) = k$ (we don't know what $f$ itself does, so we will just call the output $k$). So we want to compute $2k+1$
            \item The first computation to happen to $k$ then is multiply $k$ by 2, which is a vertical \textit{stretch} by a factor of 2.
            \item Next we add 1 to our result, which is a shift \wordChoice{\choice{to the right}\choice{to the left}\choice[correct]{up}\choice{down}} 1.
            \item Finally, since the $y$ stuff is applied in the normal order, that means we would stretch vertically by a factor of 2, \textit{then} translate vertically by 1.
        \end{enumerate}
    \end{explanation}
    
    All this yields the following graph;

    \begin{center}
        \begin{tikzpicture}
            \begin{axis}[
                axis x line=middle, 
                axis y line=middle, 
                minor tick num=5, 
                x label style={at={(axis description cs:1,0.5)},anchor=south},
                y label style={at={(axis description cs:0.5,1)},anchor=west},
                xlabel={$x$}, 
                ylabel={$y$},
                xmin=-12, 
                xmax=12, 
                ymin=-5, 
                ymax=5
                ]
            \addplot[<->,domain=-10:8, samples=300]{2*sin(deg((x+1)/2))+1};
            \end{axis}
        \end{tikzpicture}
    \end{center}
    
    Notice that we can use our knowledge of how the transformations/translations work (and what order to do them) to reverse this process. Let's say we want to take the original $f(x)$ and we want to apply the following changes in the following order: 
    \begin{enumerate}
        \item Move it right 3.
        \item Constrict it by a factor of 4 (aka to $\frac{1}{4}$ of it's normal width).
        \item Move it up 1.
        \item Stretch vertically by a factor of 2. 
    \end{enumerate}
    
    We can write the analytic form of this description by making sure that the order of operations agrees with the order the instructions were given (reversing the effects on $x$ as usual).%
    \footnote{%
        Note that the $x$ and $y$ changes can be separated and done independently. Thus the description `move left 1, then move up 1, then stretch horizontally by a factor of 3' could be re-written as `move left 1, then stretch horizontally by a factor of 3, then move up 1' since we can switch around directions that are applied to different variables.%
        }
    \begin{explanation}
        We wish to apply the following changes (in order) to $f(x)$:
        \begin{enumerate}
            \item Move right 3
            \item Constrict horizontally by a factor of 4 (aka to $\frac{1}{4}$ of it's normal width)
            \item Move up 1
            \item Stretch vertically by a factor of 2.
        \end{enumerate}
        
        \textbf{$x$ Stuff:}
        
        So we want to move (translate) before we stretch (transform) $x$. But because it's $x$ we want to make sure that the order of operations is in the \textit{opposite} order. Thus the order of operations should show compute the stretch \textit{before} the \textit{translation}.
        \[
            f\left(\left(\frac{1}{b}x\right) - c \right) = f\left(\left(\frac{1}{\frac{1}{4}}x\right) - 3 \right) = f(4x - 3)
        \]
        
        \textbf{$y$ Stuff:}
        
        Next we want to put in the vertical changes. Our directions listed the translation before the transformation, and the $y$ order of operations should match the desired order of the transformation/translations. This means we need parentheses again. Thus including the $x$ changes above we will have;
        
        \[
            a\cdot \left( f\left(4x-3\right) + k \right)
            = 2\cdot \left( f\left(4x-3\right) + 1 \right) 
        \]
        
        Thus, our final analytic form to describe the given translations and transformations is;
        \[
            \answer{2}\left(f \left( \answer{4x-3} \right) + \answer{1} \right)
        \]
    \end{explanation}
    
    To see the effects of changing the various parameters try manipulating the values for $a$, $b$, $c$, and $d$ in the following interactive graph. Pay special attention to when these values are positive, negative, and bigger or smaller than 1.
    
    \desmos{ryvd3xeltm}{}{500}
    
    You can also watch a short video giving an in-depth walkthrough example on how to graph the result of translations/transformations of a graph below.
    
    \youtube{FVJewJ93Oc8}


    \begin{problem}
        In order to graph multiple translations and transformations at the same time, you should...
        \begin{multipleChoice}
            \choice{Follow order of operations.}
            \choice{Always graph $x$-stuff first.}
            \choice{Always graph $y$-stuff first.}
            \choice{Follow the order of operations in reverse order.}
            \choice[correct]{Follow the order of operations in reverse order for $x$-stuff (Because everything about $x$ is backwards!) and the order of operations in correct order for $y$-stuff.}
        \end{multipleChoice}
    \end{problem}


\end{document}