\documentclass{ximera}

\title{Invertible Function Practice}
\begin{document}
\begin{abstract}
    Practice for Analytic View of Invertible Functions.
\end{abstract}
\maketitle

\begin{sagesilent}
def RandInt(a,b):
    """ Returns a random integer in [`a`,`b`]. Note that `a` and `b` should be integers themselves to avoid unexpected behavior.
    """
    return QQ(randint(int(a),int(b)))
    # return choice(range(a,b+1))

def NonZeroInt(b,c, avoid = [0]):
    """ Returns a random integer in [`b`,`c`] which is not in `av`. 
        If `av` is not specified, defaults to a non-zero integer.
    """
    while True:
        a = RandInt(b,c)
        if a not in avoid:
            return a


funcvec = [x, x^3, x^5]
ansvec = [x, x^(1/3), x^(1/5)]

p1choice1 = RandInt(0,2)
p1f1t = funcvec[p1choice1]
p1c1 = RandInt(-10,10)
p1c2 = RandInt(-10,10)
p1f1 = p1f1t(x=(x - p1c1)) + p1c2
p1invt = ansvec[p1choice1]
p1ans1 = p1invt(x=(x-p1c2))+p1c1


p2choice1 = RandInt(0,2)
p2f1t = funcvec[p2choice1]
p2c1 = RandInt(-10,10)
p2c2 = RandInt(-10,10)
p2f1 = p2f1t(x=(x - p2c1)) + p2c2
p2invt = ansvec[p2choice1]
p2ans1 = p2invt(x=(x-p2c2))+p2c1


p3choice1 = RandInt(0,2)
p3f1t = funcvec[p3choice1]
p3c1 = RandInt(-10,10)
p3c2 = RandInt(-10,10)
p3f1 = p3f1t(x=(x - p3c1)) + p3c2
p3invt = ansvec[p3choice1]
p3ans1 = p3invt(x=(x-p3c2))+p3c1




\end{sagesilent}

\begin{problem}
    Find the inverse for the function $f(x) = \sage{p1f1}$.\\
    
    $f^{-1}(x) = \answer{\sage{p1ans1}}$
    \begin{feedback}
        Recall; to find an inverse analytically you can switch the input/output variables (replace ``$f(x)$'' with ``$y$'' first) and then re-solve for $y$. Alternatively you can just solve the original expression for $x$. In order to undo a power, use the associated root, and for the love of God don't expand any of these polynomials or you will go insane!
    \end{feedback}
\end{problem}




\begin{problem}
    Find the inverse for the function $f(x) = \sage{p2f1}$.\\
    
    $f^{-1}(x) = \answer{\sage{p2ans1}}$
    \begin{feedback}
        Recall; to find an inverse analytically you can switch the input/output variables (replace ``$f(x)$'' with ``$y$'' first) and then re-solve for $y$. Alternatively you can just solve the original expression for $x$. In order to undo a power, use the associated root, and for the love of God don't expand any of these polynomials or you will go insane!
    \end{feedback}
\end{problem}


\begin{problem}
    Find the inverse for the function $f(x) = \sage{p3f1}$.\\
    
    $f^{-1}(x) = \answer{\sage{p3ans1}}$
    \begin{feedback}
        Recall; to find an inverse analytically you can switch the input/output variables (replace ``$f(x)$'' with ``$y$'' first) and then re-solve for $y$. Alternatively you can just solve the original expression for $x$. In order to undo a power, use the associated root, and for the love of God don't expand any of these polynomials or you will go insane!
    \end{feedback}
\end{problem}


\end{document}