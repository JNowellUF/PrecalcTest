\documentclass{ximera}
\input{../preamble}
\title{Geometric Perspective}
%\outcome{To do better.}
\begin{document}
\begin{abstract}
    We discuss the geometric perspective and what its role is in learning and practicing mathematics.
\end{abstract}
\maketitle

\subsection*{So, What is a ``Geometric Perspective/Reasoning"?}
    
    The most obvious example of geometric reasoning in the math classroom occurs when students learn graphing. We will be discussing graphing techniques extensively in the sections to come, but the goal is to build intuition on how certain actions effect the underlying function. This sounds vague because graphing is used in a lot of contexts; we will be more specific as we introduce specific tools and uses of graphing. In general however, one should think of a geometric perspective as the ``big picture" view of a situation. The primary use is to give an idea of how and why something is working, but it lacks the fine detail to actually perform any precise manipulation or computation.
        
    For example, knowing that your company's profits are ``trending upward" is geometric, but your boss may ask things like ``what precisely is our current profit" or ``at what point will we hit a large enough profit to expand to another location?" These are the kinds of questions that geometric reasoning usually lacks the fine details necessary to answer. 
    
    In a business setting, geometric information usually takes the form of ``presentation information". The information (or style of presentation) that you would present in a meeting to show things like a proof of concept, anticipated growth/profits, or data trends. What geometric perspective/reasoning isn't, is the endless hours you put in ``crunching numbers" to be able to boil everything down to the nice pages in the report with pretty pictures and ``at a glance" information that you hand out in the meetings. That would be the ``analytic perspective" which we will talk about next, and which usually comprises the bulk of your reports.

\begin{problem}
    Which of the following is the \textit{best} description of ``Geometric Perspective"?
    \begin{multipleChoice}
        \choice{Visualizing data points, rather than recording them in a spreadsheet.}
        \choice[correct]{To get a `wide angle lense' view of data; to have a visual summation of large-scale trends and important features.}
        \choice{To find specific data points that are of interest.}
        \choice{To demonstrate ideal data trends and features, as oppose to the actual data trends and features.}
    \end{multipleChoice}
\end{problem}

\end{document}



