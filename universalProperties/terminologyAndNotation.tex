\documentclass{ximera}

\title{Terminology To Know}
\begin{document}
\begin{abstract}
    These are important terms and notations for this section.
\end{abstract}
\maketitle

Below is a quick-reference for definitions in this chapter.

\begin{definition}[(Rigid) Translation]
    A technique to move the function about on a graph without changing it's (relative) size. \\
    \textbf{For example:} Movements of the graph up, down, left, or right would count as `rigid transformations'.
\end{definition}

\begin{definition}[Transformation]
    A technique to change the shape or size of a function in a predictable (and reversible) way. Often used to "rescale" the function's graph.\\
    \textbf{For Example:} Scaling the graph to make it bigger, smaller, or flipping the graph across some line are examples of `transformations'.
\end{definition}

\begin{definition}[(Functional) Argument]
    The content that a function is being applied to.\\
    \textbf{For Example:} The "$x$" in "$f(x)$" or the "$2x+1$" in the "$g(2x+1)$" are both examples of `functional arguments'.
\end{definition}

\begin{definition}[(Functional) Output (or Value)]
    The point in the codomain that a function returns or `output's.\\
    \textbf{For Example:} If $f(x) = 3x+1$ and we compute $f(5) = 3\cdot5 + 1 = 16$, then the `$16$' is an example of the `functional output' (also referred to as function value)
\end{definition}

\begin{definition}[(x or y) intercept(s)]
    The \textbf{points} at which a function intersects either the $x$ or $y$ axis (respectively). These are \textbf{points} and must \textbf{always} be written as points. \\
    \textbf{For example:} One would say ``The $x$-intercept is $(5,0)$". It is \textbf{incorrect} to say ``The $x$-intercept is $5$."
\end{definition}

\begin{definition}[Zeros of a function]
    The zeros of a function are the domain values that yield zero as the output. Put another way, the zeros of a function are the $x$-values only of the $x$-intercepts. These are \textbf{not points}, but they may be written either as points or as values.\\
    \textbf{For example:} One could say ``The zero of the function is $(5,0)$". It is slightly more conventional to say ``The zero of the function is $5$."
\end{definition}

\begin{definition}[Extrema]
    Extrema of a function are the maximum or minimum values that the function attains. These can be broken up into local or relative extrema, and absolute or global extrema.\\
    \textbf{Local/Relative Extrema:} are points that are maximums or minimums within some `small enough' section of $x$ values near the $x$ value of the extrema. By `close enough' we mean that for some specific $x$ value (let's say $x_0$, $f(x_0)$ is bigger (or smaller if it's a local minimum) than $f(x)$ for any $x$ within some distance you can specify (like `within $\frac{1}{2}$')  of $x_0$.\\
    \textbf{Absolute/Global Extrema:} are points that attain the absolute highest (or lowest) $y$ values that a function can attain.
\end{definition}

\begin{definition}[Discontinuities]
    Discontinuities are domain values ($x$-values) where a function fails to be continuous. By convention we only count points where the function is still defined on either side of the discontinuities, thus we wouldn't say $\sqrt{x}$ is 'discontinuous' for $x < 0$ because it's domain simply ends at 0, there is no 'disruption' in the domain because the domain is only on one side of the value 0. Discontinuities can be found in a number of forms; holes, infinite (or asymptotic) discontinuities, and jumps. Classifying these discontinuities analytically is beyond this scope of this course, but we will give geometric examples in this topic.
\end{definition}

\end{document}