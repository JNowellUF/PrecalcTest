\documentclass{ximera}

\title{Intro: Rigid Translations}
\begin{document}
\begin{abstract}
    An introduction to the ideas of rigid translations.
\end{abstract}
\maketitle

In the last section we introduced the tactic of learning things utilizing both analytic and geometric perspectives. Often it is the case that the geometric perspective is more useful for initially learning a new concept, whereas the analytic viewpoint is utilized to get more precise information/results. Rigid translations are no exception to this dynamic. The analytic view of translations is somewhat unenlightening when trying to initially learn the idea. The very phrase `rigid translation' should be a clue that a better way to understand what is happening during a rigid translation, is to \textit{literally see it}. However, once you already understand what is happening, the analytic view tends to be more useful in order to precisely state or describe a rigid translation quickly and concisely.

For these reasons we will start out our work on rigid translations with some geometric descriptions and depictions. Once we have firmly established what is (and is not) a rigid translation intuitively, we will delve into the analytic mechanics of how to write and/or manipulate functions with rigid translations.

\begin{problem}
    We will start with a geometric view of rigid translations because
    \begin{multipleChoice}
        \choice{We had to start with something, might as well be that.}
        \choice{Geometry is a lot more fun and interesting than algebra.}
        \choice[correct]{It is hard to manipulate functions via rigid translations without understanding what they are, and it is hard to understand what they are without literally seeing what is happening.}
        \choice{We aren't starting with geometric view, we're starting with the analytic view.}
    \end{multipleChoice}
\end{problem}
\end{document}