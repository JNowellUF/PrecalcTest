\documentclass{ximera}
\usepackage{longdivision}
\usepackage{polynom}
\usepackage{float}% Use `H' as the figure optional argument to force it's vertical placement to conform to source.
%\usepackage{caption}% Allows us to describe the figures without having "figure 1:" in it. :: Apparently Caption isn't supported.
%    \captionsetup{labelformat=empty}% Actually does the figure configuration stated above.
\usetikzlibrary{arrows.meta,arrows}% Allow nicer arrow heads for tikz.
\usepackage{gensymb, pgfplots}
\usepackage{tabularx}
\usepackage{arydshln}
\usepackage[margin=1.5cm]{geometry}
\usepackage{indentfirst}

\setlength\parindent{16pt}

\graphicspath{
  {./}
  {./explorePolynomials/}
  {./exploreRadicals/}
  {./graphing/}
}

%% Default style for tikZ
\pgfplotsset{my style/.append style={axis x line=middle, axis y line=
middle, xlabel={$x$}, ylabel={$y$}, axis equal }}


%% Because log being natural log is too hard for people.
\let\logOld\log% Keep the old \log definition, just in case we need it.
\renewcommand{\log}{\ln}


%%% Changes in polynom to show the zero coefficient terms
\makeatletter
\def\pld@CF@loop#1+{%
    \ifx\relax#1\else
        \begingroup
          \pld@AccuSetX11%
          \def\pld@frac{{}{}}\let\pld@symbols\@empty\let\pld@vars\@empty
          \pld@false
          #1%
          \let\pld@temp\@empty
          \pld@AccuIfOne{}{\pld@AccuGet\pld@temp
                            \edef\pld@temp{\noexpand\pld@R\pld@temp}}%
           \pld@if \pld@Extend\pld@temp{\expandafter\pld@F\pld@frac}\fi
           \expandafter\pld@CF@loop@\pld@symbols\relax\@empty
           \expandafter\pld@CF@loop@\pld@vars\relax\@empty
           \ifx\@empty\pld@temp
               \def\pld@temp{\pld@R11}%
           \fi
          \global\let\@gtempa\pld@temp
        \endgroup
        \ifx\@empty\@gtempa\else
            \pld@ExtendPoly\pld@tempoly\@gtempa
        \fi
        \expandafter\pld@CF@loop
    \fi}
\def\pld@CMAddToTempoly{%
    \pld@AccuGet\pld@temp\edef\pld@temp{\noexpand\pld@R\pld@temp}%
    \pld@CondenseMonomials\pld@false\pld@symbols
    \ifx\pld@symbols\@empty \else
        \pld@ExtendPoly\pld@temp\pld@symbols
    \fi
    \ifx\pld@temp\@empty \else
        \pld@if
            \expandafter\pld@IfSum\expandafter{\pld@temp}%
                {\expandafter\def\expandafter\pld@temp\expandafter
                    {\expandafter\pld@F\expandafter{\pld@temp}{}}}%
                {}%
        \fi
        \pld@ExtendPoly\pld@tempoly\pld@temp
        \pld@Extend\pld@tempoly{\pld@monom}%
    \fi}
\makeatother




%%%%% Code for making prime factor trees for numbers, taken from user Qrrbrbirlbel at: https://tex.stackexchange.com/questions/131689/how-to-automatically-draw-tree-diagram-of-prime-factorization-with-latex

\usepackage{forest,mathtools,siunitx}
\makeatletter
\def\ifNum#1{\ifnum#1\relax
  \expandafter\pgfutil@firstoftwo\else
  \expandafter\pgfutil@secondoftwo\fi}
\forestset{
  num content/.style={
    delay={
      content/.expanded={\noexpand\num{\forestoption{content}}}}},
  pt@prime/.style={draw, circle},
  pt@start/.style={},
  pt@normal/.style={},
  start primeTree/.style={%
    /utils/exec=%
      % \pt@start holds the current minimum factor, we'll start with 2
      \def\pt@start{2}%
      % \pt@result will hold the to-be-typeset factorization, we'll start with
      % \pgfutil@gobble since we don't want a initial \times
      \let\pt@result\pgfutil@gobble
      % \pt@start@cnt holds the number of ^factors for the current factor
      \def\pt@start@cnt{0}%
      % \pt@lStart will later hold "l"ast factor used
      \let\pt@lStart\pgfutil@empty,
    alias=pt-start,
    pt@start/.try,
    delay={content/.expanded={$\noexpand\num{\forestove{content}}
                            \noexpand\mathrlap{{}= \noexpand\pt@result}$}},
    primeTree},
  primeTree/.code=%
    % take the content of the node and save it in the count
    \c@pgf@counta\forestove{content}\relax
    % if it's 2 we're already finished with the factorization
    \ifNum{\c@pgf@counta=2}{%
      % add the factor
      \pt@addfactor{2}%
      % finalize the factorization of the result
      \pt@addfactor{}%
      % and set the style to the prime style
      \forestset{pt@prime/.try}%
    }{%
      % this simply calculates content/2 and saves it in \pt@end
      % this is later used for an early break of the recursion since no factor
      % can be greater then content/2 (for integers of course)
      \edef\pt@content{\the\c@pgf@counta}%
      \divide\c@pgf@counta2\relax
      \advance\c@pgf@counta1\relax % to be on the safe side
      \edef\pt@end{\the\c@pgf@counta}%
      \pt@do}}

%%% our main "function"
\def\pt@do{%
  % let's test if the current factor is already greather then the max factor
  \ifNum{\pt@end<\pt@start}{%
    % great, we're finished, the same as above
    \expandafter\pt@addfactor\expandafter{\pt@content}%
    \pt@addfactor{}%
    \def\pt@next{\forestset{pt@prime/.try}}%
  }{%
    % this calculates int(content/factor)*factor
    % if factor is a factor of content (without remainder), the result will
    % equal content. The int(content/factor) is saved in \pgf@temp.
    \c@pgf@counta\pt@content\relax
    \divide\c@pgf@counta\pt@start\relax
    \edef\pgf@temp{\the\c@pgf@counta}%
    \multiply\c@pgf@counta\pt@start\relax
    \ifNum{\the\c@pgf@counta=\pt@content}{%
      % yeah, we found a factor, add it to the result and ...
      \expandafter\pt@addfactor\expandafter{\pt@start}%
      % ... add the factor as the first child with style pt@prime
      % and the result of int(content/factor) as another child.
      \edef\pt@next{\noexpand\forestset{%
        append={[\pt@start, pt@prime/.try]},
        append={[\pgf@temp, pt@normal/.try]},
        % forest is complex, this makes sure that for the second child, the
        % primeTree style is not executed too early (there must be a better way).
        delay={
          for descendants={
            delay={if n'=1{primeTree, num content}{}}}}}}%
    }{%
      % Alright this is not a factor, let's get the next factor
      \ifNum{\pt@start=2}{%
        % if the previous factor was 2, the next one will be 3
        \def\pt@start{3}%
      }{%
        % hmm, the previos factor was not 2,
        % let's add 2, maybe we'll hit the next prime number
        % and maybe a factor
        \c@pgf@counta\pt@start
        \advance\c@pgf@counta2\relax
        \edef\pt@start{\the\c@pgf@counta}%
      }%
      % let's do that again
      \let\pt@next\pt@do
    }%
  }%
  \pt@next
}

%%% this builds the \pt@result macro with the factors
\def\pt@addfactor#1{%
  \def\pgf@tempa{#1}%
  % is it the same factor as the previous one
  \ifx\pgf@tempa\pt@lStart
    % add 1 to the counter
    \c@pgf@counta\pt@start@cnt\relax
    \advance\c@pgf@counta1\relax
    \edef\pt@start@cnt{\the\c@pgf@counta}%
  \else
    % a new factor! Add the previous one to the product of factors
    \ifx\pt@lStart\pgfutil@empty\else
      % as long as there actually is one, the \ifnum makes sure we do not add ^1
      \edef\pgf@tempa{\noexpand\num{\pt@lStart}\ifnum\pt@start@cnt>1 
                                           ^{\noexpand\num{\pt@start@cnt}}\fi}%
      \expandafter\pt@addfactor@\expandafter{\pgf@tempa}%
    \fi
    % setup the macros for the next round
    \def\pt@lStart{#1}% <- current (new) factor
    \def\pt@start@cnt{1}% <- first time
  \fi
}
%%% This simply appends "\times #1" to \pt@result, with etoolbox this would be
%%% \appto\pt@result{\times#1}
\def\pt@addfactor@#1{%
  \expandafter\def\expandafter\pt@result\expandafter{\pt@result \times #1}}

%%% Our main macro:
%%% #1 = possible optional argument for forest (can be tikz too)
%%% #2 = the number to factorize
\newcommand*{\PrimeTree}[2][]{%
  \begin{forest}%
    % as the result is set via \mathrlap it doesn't update the bounding box
    % let's fix this:
    tikz={execute at end scope={\pgfmathparse{width("${}=\pt@result$")}%
                         \path ([xshift=\pgfmathresult pt]pt-start.east);}},
    % other optional arguments
    #1
    % And go!
    [#2, start primeTree]
  \end{forest}}
\makeatother


\providecommand\tabitem{\makebox[1em][r]{\textbullet~}}
\providecommand{\letterPlus}{\makebox[0pt][l]{$+$}}
\providecommand{\letterMinus}{\makebox[0pt][l]{$-$}}

\renewcommand{\texttt}[1]{#1}% Renew the command to prevent it from showing up in the sage strings for some weird reason.
%\renewcommand{\text}[1]{#1}% Renew the command to prevent it from showing up in the sage strings for some weird reason.



\title{Piecewise: Analytic View Practice 1}
\begin{document}
\begin{abstract}
    This is a practice understanding of piecewise functions from an analytic viewpoint.
\end{abstract}
\maketitle

\begin{sagesilent}
##### Useful Macros
def RandInt(a,b):
    """ Returns a random integer in [`a`,`b`]. Note that `a` and `b` should be integers themselves to avoid unexpected behavior.
    """
    return QQ(randint(int(a),int(b)))
    # return choice(range(a,b+1))

def NonZeroInt(b,c, avoid = [0]):
    """ Returns a random integer in [`b`,`c`] which is not in `av`. 
        If `av` is not specified, defaults to a non-zero integer.
    """
    while True:
        a = RandInt(b,c)
        if a not in avoid:
            return a

funcvec = [x,x^2,x^3,sqrt(abs(x)),ln(abs(x)+1),e^x]

###############
#### Problem p1
p1f1pick = RandInt(0,5)
p1f2pick = RandInt(0,5)
p1f3pick = RandInt(0,5)

### Build each of the functions.
p1f1 = NonZeroInt(-5,5)*funcvec[p1f1pick](x=(x-RandInt(-5,5))) + RandInt(-5,5)
p1f2 = NonZeroInt(-5,5)*funcvec[p1f2pick](x=(x-RandInt(-5,5))) + RandInt(-5,5)
p1f3 = NonZeroInt(-5,5)*funcvec[p1f3pick](x=(x-RandInt(-5,5))) + RandInt(-5,5)

### Build the left and right endpoints for the functions.
p1l1 = RandInt(-15,15)
p1r1 = p1l1 + RandInt(1,7)
p1l2 = RandInt(p1l1+1,p1r1+5)
p1r2 = p1l2 + RandInt(1,7)
p1l3 = RandInt(p1l2+1,p1r2+5)
p1r3 = p1l3 + RandInt(1,7)

### Now check to see if what we got is actually a function.
# Default answer:
p1ans = 1

if p1l2 < p1r1:
    p1ans = 0
if p1l3 < p1r2:
    p1ans = 0
if p1r1 == p1l2 and p1f1(x=p1r1) != p1f2(x=p1l2):
    p1ans = 0
if p1r2 == p1l3 and p1f1(x=p1r2) != p1f2(x=p1l3):
    p1ans = 0

##### End of problem p1



###############
#### Problem p2
p2f1pick = RandInt(0,5)
p2f2pick = RandInt(0,5)
p2f3pick = RandInt(0,5)

### Build each of the functions.
p2f1 = NonZeroInt(-5,5)*funcvec[p2f1pick](x=(x-RandInt(-5,5))) + RandInt(-5,5)
p2f2 = NonZeroInt(-5,5)*funcvec[p2f2pick](x=(x-RandInt(-5,5))) + RandInt(-5,5)
p2f3 = NonZeroInt(-5,5)*funcvec[p2f3pick](x=(x-RandInt(-5,5))) + RandInt(-5,5)

### Build the left and right endpoints for the functions.
p2l1 = RandInt(-15,15)
p2r1 = p2l1 + RandInt(1,7)
p2l2 = RandInt(p2l1+1,p2r1+5)
p2r2 = p2l2 + RandInt(1,7)
p2l3 = RandInt(p2l2+1,p2r2+5)
p2r3 = p2l3 + RandInt(1,7)

### Now check to see if what we got is actually a function.
# Default answer:
p2ans = 1

if p2l2 < p2r1:
    p2ans = 0
if p2l3 < p2r2:
    p2ans = 0
if p2r1 == p2l2 and p2f1(x=p2r1) != p2f2(x=p2l2):
    p2ans = 0
if p2r2 == p2l3 and p2f1(x=p2r2) != p2f2(x=p2l3):
    p2ans = 0

##### End of problem p2




###############
#### Problem p3
p3f1pick = RandInt(0,5)
p3f2pick = RandInt(0,5)
p3f3pick = RandInt(0,5)

### Build each of the functions.
p3f1 = NonZeroInt(-5,5)*funcvec[p3f1pick](x=(x-RandInt(-5,5))) + RandInt(-5,5)
p3f2 = NonZeroInt(-5,5)*funcvec[p3f2pick](x=(x-RandInt(-5,5))) + RandInt(-5,5)
p3f3 = NonZeroInt(-5,5)*funcvec[p3f3pick](x=(x-RandInt(-5,5))) + RandInt(-5,5)

### Build the left and right endpoints for the functions.
p3l1 = RandInt(-15,15)
p3r1 = p3l1 + RandInt(1,7)
p3l2 = RandInt(p3l1+1,p3r1+5)
p3r2 = p3l2 + RandInt(1,7)
p3l3 = RandInt(p3l2+1,p3r2+5)
p3r3 = p3l3 + RandInt(1,7)

### Now check to see if what we got is actually a function.
# Default answer:
p3ans = 1

if p3l2 < p3r1:
    p3ans = 0
if p3l3 < p3r2:
    p3ans = 0
if p3r1 == p3l2 and p3f1(x=p3r1) != p3f2(x=p3l2):
    p3ans = 0
if p3r2 == p3l3 and p3f1(x=p3r2) != p3f2(x=p3l3):
    p3ans = 0

##### End of problem p3




###############
#### Problem p4
p4f1pick = RandInt(0,5)
p4f2pick = RandInt(0,5)
p4f3pick = RandInt(0,5)

### Build each of the functions.
p4f1 = NonZeroInt(-5,5)*funcvec[p4f1pick](x=(x-RandInt(-5,5))) + RandInt(-5,5)
p4f2 = NonZeroInt(-5,5)*funcvec[p4f2pick](x=(x-RandInt(-5,5))) + RandInt(-5,5)
p4f3 = NonZeroInt(-5,5)*funcvec[p4f3pick](x=(x-RandInt(-5,5))) + RandInt(-5,5)

### Build the left and right endpoints for the functions.
p4l1 = RandInt(-15,15)
p4r1 = p4l1 + RandInt(1,7)
p4l2 = RandInt(p4l1+1,p4r1+5)
p4r2 = p4l2 + RandInt(1,7)
p4l3 = RandInt(p4l2+1,p4r2+5)
p4r3 = p4l3 + RandInt(1,7)

### Now check to see if what we got is actually a function.
# Default answer:
p4ans = 1

if p4l2 < p4r1:
    p4ans = 0
if p4l3 < p4r2:
    p4ans = 0
if p4r1 == p4l2 and p4f1(x=p4r1) != p4f2(x=p4l2):
    p4ans = 0
if p4r2 == p4l3 and p4f1(x=p4r2) != p4f2(x=p4l3):
    p4ans = 0

##### End of problem p4




###############
#### Problem p5
p5f1pick = RandInt(0,5)
p5f2pick = RandInt(0,5)
p5f3pick = RandInt(0,5)

### Build each of the functions.
p5f1 = NonZeroInt(-5,5)*funcvec[p5f1pick](x=(x-RandInt(-5,5))) + RandInt(-5,5)
p5f2 = NonZeroInt(-5,5)*funcvec[p5f2pick](x=(x-RandInt(-5,5))) + RandInt(-5,5)
p5f3 = NonZeroInt(-5,5)*funcvec[p5f3pick](x=(x-RandInt(-5,5))) + RandInt(-5,5)

### Build the left and right endpoints for the functions.
p5l1 = RandInt(-15,15)
p5r1 = p5l1 + RandInt(1,7)
p5l2 = RandInt(p5l1+1,p5r1+5)
p5r2 = p5l2 + RandInt(1,7)
p5l3 = RandInt(p5l2+1,p5r2+5)
p5r3 = p5l3 + RandInt(1,7)

### Now check to see if what we got is actually a function.
# Default answer:
p5ans = 1

if p5l2 < p5r1:
    p5ans = 0
if p5l3 < p5r2:
    p5ans = 0
if p5r1 == p5l2 and p5f1(x=p5r1) != p5f2(x=p5l2):
    p5ans = 0
if p5r2 == p5l3 and p5f1(x=p5r2) != p5f2(x=p5l3):
    p5ans = 0

##### End of problem p5






\end{sagesilent}

\begin{problem}
    Determine if the following piecewise definition is a function or not. 
    \[
        f(x) =
            \begin{cases}
                \sage{p1f1}     & \sage{p1l1}\leq x \leq \sage{p1r1} \\
                \sage{p1f2}     & \sage{p1l2} \leq x \leq \sage{p1r2} \\
                \sage{p1f3}     & \sage{p1l3} \leq x \leq \sage{p1r3}
            \end{cases}
    \]
    
    If the above piecewise definition is a function, enter $1$. If it is not a function, then enter $0$.
    $\answer{\sage{p1ans}}$
    \begin{feedback}
        To know if the definition is a function you need to check the domains (listed in the right most column) and see if there are any overlapping values between any of the rows. If there are any overlapping values, then you need to check the overlapping values in both functions where those domains overlap, to see if you get the same values or not. If there are no overlapping x-values, or you get the same values in both functions for overlapping x-values, then the definition is a function. If you don't, then it isn't.
    \end{feedback}
    
\end{problem}





\begin{problem}
    Determine if the following piecewise definition is a function or not. 
    \[
        f(x) =
            \begin{cases}
                \sage{p2f1}     & \sage{p2l1}\leq x \leq \sage{p2r1} \\
                \sage{p2f2}     & \sage{p2l2} \leq x \leq \sage{p2r2} \\
                \sage{p2f3}     & \sage{p2l3} \leq x \leq \sage{p2r3}
            \end{cases}
    \]
    
    If the above piecewise definition is a function, enter $1$. If it is not a function, then enter $0$.
    $\answer{\sage{p2ans}}$
    \begin{feedback}
        To know if the definition is a function you need to check the domains (listed in the right most column) and see if there are any overlapping values between any of the rows. If there are any overlapping values, then you need to check the overlapping values in both functions where those domains overlap, to see if you get the same values or not. If there are no overlapping x-values, or you get the same values in both functions for overlapping x-values, then the definition is a function. If you don't, then it isn't.
    \end{feedback}
    
\end{problem}




\begin{problem}
    Determine if the following piecewise definition is a function or not. 
    \[
        f(x) =
            \begin{cases}
                \sage{p3f1}     & \sage{p3l1}\leq x \leq \sage{p3r1} \\
                \sage{p3f2}     & \sage{p3l2} \leq x \leq \sage{p3r2} \\
                \sage{p3f3}     & \sage{p3l3} \leq x \leq \sage{p3r3}
            \end{cases}
    \]
    
    If the above piecewise definition is a function, enter $1$. If it is not a function, then enter $0$.
    $\answer{\sage{p3ans}}$
    \begin{feedback}
        To know if the definition is a function you need to check the domains (listed in the right most column) and see if there are any overlapping values between any of the rows. If there are any overlapping values, then you need to check the overlapping values in both functions where those domains overlap, to see if you get the same values or not. If there are no overlapping x-values, or you get the same values in both functions for overlapping x-values, then the definition is a function. If you don't, then it isn't.
    \end{feedback}
    
\end{problem}




\begin{problem}
    Determine if the following piecewise definition is a function or not. 
    \[
        f(x) =
            \begin{cases}
                \sage{p4f1}     & \sage{p4l1}\leq x \leq \sage{p4r1} \\
                \sage{p4f2}     & \sage{p4l2} \leq x \leq \sage{p4r2} \\
                \sage{p4f3}     & \sage{p4l3} \leq x \leq \sage{p4r3}
            \end{cases}
    \]
    
    If the above piecewise definition is a function, enter $1$. If it is not a function, then enter $0$.
    $\answer{\sage{p4ans}}$
    \begin{feedback}
        To know if the definition is a function you need to check the domains (listed in the right most column) and see if there are any overlapping values between any of the rows. If there are any overlapping values, then you need to check the overlapping values in both functions where those domains overlap, to see if you get the same values or not. If there are no overlapping x-values, or you get the same values in both functions for overlapping x-values, then the definition is a function. If you don't, then it isn't.
    \end{feedback}
    
\end{problem}




\begin{problem}
    Determine if the following piecewise definition is a function or not. 
    \[
        f(x) =
            \begin{cases}
                \sage{p5f1}     & \sage{p5l1}\leq x \leq \sage{p5r1} \\
                \sage{p5f2}     & \sage{p5l2} \leq x \leq \sage{p5r2} \\
                \sage{p5f3}     & \sage{p5l3} \leq x \leq \sage{p5r3}
            \end{cases}
    \]
    
    If the above piecewise definition is a function, enter $1$. If it is not a function, then enter $0$.
    $\answer{\sage{p5ans}}$
    \begin{feedback}
        To know if the definition is a function you need to check the domains (listed in the right most column) and see if there are any overlapping values between any of the rows. If there are any overlapping values, then you need to check the overlapping values in both functions where those domains overlap, to see if you get the same values or not. If there are no overlapping x-values, or you get the same values in both functions for overlapping x-values, then the definition is a function. If you don't, then it isn't.
    \end{feedback}
    
\end{problem}




\end{document}







