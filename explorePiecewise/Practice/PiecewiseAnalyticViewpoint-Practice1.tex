\documentclass{ximera}
\input{../../preamble}
\title{Piecewise: Analytic View Practice 1}
\begin{document}
\begin{abstract}
    This is a practice understanding of piecewise functions from an analytic viewpoint.
\end{abstract}
\maketitle

\begin{sagesilent}
##### Useful Macros
def RandInt(a,b):
    """ Returns a random integer in [`a`,`b`]. Note that `a` and `b` should be integers themselves to avoid unexpected behavior.
    """
    return QQ(randint(int(a),int(b)))
    # return choice(range(a,b+1))

def NonZeroInt(b,c, avoid = [0]):
    """ Returns a random integer in [`b`,`c`] which is not in `av`. 
        If `av` is not specified, defaults to a non-zero integer.
    """
    while True:
        a = RandInt(b,c)
        if a not in avoid:
            return a

funcvec = [x,x^2,x^3,sqrt(abs(x)),ln(abs(x)+1),e^x]

###############
#### Problem p1
p1f1pick = RandInt(0,5)
p1f2pick = RandInt(0,5)
p1f3pick = RandInt(0,5)

### Build each of the functions.
p1f1 = NonZeroInt(-5,5)*funcvec[p1f1pick](x=(x-RandInt(-5,5))) + RandInt(-5,5)
p1f2 = NonZeroInt(-5,5)*funcvec[p1f2pick](x=(x-RandInt(-5,5))) + RandInt(-5,5)
p1f3 = NonZeroInt(-5,5)*funcvec[p1f3pick](x=(x-RandInt(-5,5))) + RandInt(-5,5)

### Build the left and right endpoints for the functions.
p1l1 = RandInt(-15,15)
p1r1 = p1l1 + RandInt(1,7)
p1l2 = RandInt(p1l1+1,p1r1+5)
p1r2 = p1l2 + RandInt(1,7)
p1l3 = RandInt(p1l2+1,p1r2+5)
p1r3 = p1l3 + RandInt(1,7)

### Now check to see if what we got is actually a function.
# Default answer:
p1ans = 1

if p1l2 < p1r1:
    p1ans = 0
if p1l3 < p1r2:
    p1ans = 0
if p1r1 == p1l2 and p1f1(x=p1r1) != p1f2(x=p1l2):
    p1ans = 0
if p1r2 == p1l3 and p1f1(x=p1r2) != p1f2(x=p1l3):
    p1ans = 0

##### End of problem p1



###############
#### Problem p2
p2f1pick = RandInt(0,5)
p2f2pick = RandInt(0,5)
p2f3pick = RandInt(0,5)

### Build each of the functions.
p2f1 = NonZeroInt(-5,5)*funcvec[p2f1pick](x=(x-RandInt(-5,5))) + RandInt(-5,5)
p2f2 = NonZeroInt(-5,5)*funcvec[p2f2pick](x=(x-RandInt(-5,5))) + RandInt(-5,5)
p2f3 = NonZeroInt(-5,5)*funcvec[p2f3pick](x=(x-RandInt(-5,5))) + RandInt(-5,5)

### Build the left and right endpoints for the functions.
p2l1 = RandInt(-15,15)
p2r1 = p2l1 + RandInt(1,7)
p2l2 = RandInt(p2l1+1,p2r1+5)
p2r2 = p2l2 + RandInt(1,7)
p2l3 = RandInt(p2l2+1,p2r2+5)
p2r3 = p2l3 + RandInt(1,7)

### Now check to see if what we got is actually a function.
# Default answer:
p2ans = 1

if p2l2 < p2r1:
    p2ans = 0
if p2l3 < p2r2:
    p2ans = 0
if p2r1 == p2l2 and p2f1(x=p2r1) != p2f2(x=p2l2):
    p2ans = 0
if p2r2 == p2l3 and p2f1(x=p2r2) != p2f2(x=p2l3):
    p2ans = 0

##### End of problem p2




###############
#### Problem p3
p3f1pick = RandInt(0,5)
p3f2pick = RandInt(0,5)
p3f3pick = RandInt(0,5)

### Build each of the functions.
p3f1 = NonZeroInt(-5,5)*funcvec[p3f1pick](x=(x-RandInt(-5,5))) + RandInt(-5,5)
p3f2 = NonZeroInt(-5,5)*funcvec[p3f2pick](x=(x-RandInt(-5,5))) + RandInt(-5,5)
p3f3 = NonZeroInt(-5,5)*funcvec[p3f3pick](x=(x-RandInt(-5,5))) + RandInt(-5,5)

### Build the left and right endpoints for the functions.
p3l1 = RandInt(-15,15)
p3r1 = p3l1 + RandInt(1,7)
p3l2 = RandInt(p3l1+1,p3r1+5)
p3r2 = p3l2 + RandInt(1,7)
p3l3 = RandInt(p3l2+1,p3r2+5)
p3r3 = p3l3 + RandInt(1,7)

### Now check to see if what we got is actually a function.
# Default answer:
p3ans = 1

if p3l2 < p3r1:
    p3ans = 0
if p3l3 < p3r2:
    p3ans = 0
if p3r1 == p3l2 and p3f1(x=p3r1) != p3f2(x=p3l2):
    p3ans = 0
if p3r2 == p3l3 and p3f1(x=p3r2) != p3f2(x=p3l3):
    p3ans = 0

##### End of problem p3




###############
#### Problem p4
p4f1pick = RandInt(0,5)
p4f2pick = RandInt(0,5)
p4f3pick = RandInt(0,5)

### Build each of the functions.
p4f1 = NonZeroInt(-5,5)*funcvec[p4f1pick](x=(x-RandInt(-5,5))) + RandInt(-5,5)
p4f2 = NonZeroInt(-5,5)*funcvec[p4f2pick](x=(x-RandInt(-5,5))) + RandInt(-5,5)
p4f3 = NonZeroInt(-5,5)*funcvec[p4f3pick](x=(x-RandInt(-5,5))) + RandInt(-5,5)

### Build the left and right endpoints for the functions.
p4l1 = RandInt(-15,15)
p4r1 = p4l1 + RandInt(1,7)
p4l2 = RandInt(p4l1+1,p4r1+5)
p4r2 = p4l2 + RandInt(1,7)
p4l3 = RandInt(p4l2+1,p4r2+5)
p4r3 = p4l3 + RandInt(1,7)

### Now check to see if what we got is actually a function.
# Default answer:
p4ans = 1

if p4l2 < p4r1:
    p4ans = 0
if p4l3 < p4r2:
    p4ans = 0
if p4r1 == p4l2 and p4f1(x=p4r1) != p4f2(x=p4l2):
    p4ans = 0
if p4r2 == p4l3 and p4f1(x=p4r2) != p4f2(x=p4l3):
    p4ans = 0

##### End of problem p4




###############
#### Problem p5
p5f1pick = RandInt(0,5)
p5f2pick = RandInt(0,5)
p5f3pick = RandInt(0,5)

### Build each of the functions.
p5f1 = NonZeroInt(-5,5)*funcvec[p5f1pick](x=(x-RandInt(-5,5))) + RandInt(-5,5)
p5f2 = NonZeroInt(-5,5)*funcvec[p5f2pick](x=(x-RandInt(-5,5))) + RandInt(-5,5)
p5f3 = NonZeroInt(-5,5)*funcvec[p5f3pick](x=(x-RandInt(-5,5))) + RandInt(-5,5)

### Build the left and right endpoints for the functions.
p5l1 = RandInt(-15,15)
p5r1 = p5l1 + RandInt(1,7)
p5l2 = RandInt(p5l1+1,p5r1+5)
p5r2 = p5l2 + RandInt(1,7)
p5l3 = RandInt(p5l2+1,p5r2+5)
p5r3 = p5l3 + RandInt(1,7)

### Now check to see if what we got is actually a function.
# Default answer:
p5ans = 1

if p5l2 < p5r1:
    p5ans = 0
if p5l3 < p5r2:
    p5ans = 0
if p5r1 == p5l2 and p5f1(x=p5r1) != p5f2(x=p5l2):
    p5ans = 0
if p5r2 == p5l3 and p5f1(x=p5r2) != p5f2(x=p5l3):
    p5ans = 0

##### End of problem p5






\end{sagesilent}

\begin{problem}
    Determine if the following piecewise definition is a function or not. 
    \[
        f(x) =
            \begin{cases}
                \sage{p1f1}     & \sage{p1l1}\leq x \leq \sage{p1r1} \\
                \sage{p1f2}     & \sage{p1l2} \leq x \leq \sage{p1r2} \\
                \sage{p1f3}     & \sage{p1l3} \leq x \leq \sage{p1r3}
            \end{cases}
    \]
    
    If the above piecewise definition is a function, enter $1$. If it is not a function, then enter $0$.
    $\answer{\sage{p1ans}}$
    \begin{feedback}
        To know if the definition is a function you need to check the domains (listed in the right most column) and see if there are any overlapping values between any of the rows. If there are any overlapping values, then you need to check the overlapping values in both functions where those domains overlap, to see if you get the same values or not. If there are no overlapping x-values, or you get the same values in both functions for overlapping x-values, then the definition is a function. If you don't, then it isn't.
    \end{feedback}
    
\end{problem}





\begin{problem}
    Determine if the following piecewise definition is a function or not. 
    \[
        f(x) =
            \begin{cases}
                \sage{p2f1}     & \sage{p2l1}\leq x \leq \sage{p2r1} \\
                \sage{p2f2}     & \sage{p2l2} \leq x \leq \sage{p2r2} \\
                \sage{p2f3}     & \sage{p2l3} \leq x \leq \sage{p2r3}
            \end{cases}
    \]
    
    If the above piecewise definition is a function, enter $1$. If it is not a function, then enter $0$.
    $\answer{\sage{p2ans}}$
    \begin{feedback}
        To know if the definition is a function you need to check the domains (listed in the right most column) and see if there are any overlapping values between any of the rows. If there are any overlapping values, then you need to check the overlapping values in both functions where those domains overlap, to see if you get the same values or not. If there are no overlapping x-values, or you get the same values in both functions for overlapping x-values, then the definition is a function. If you don't, then it isn't.
    \end{feedback}
    
\end{problem}




\begin{problem}
    Determine if the following piecewise definition is a function or not. 
    \[
        f(x) =
            \begin{cases}
                \sage{p3f1}     & \sage{p3l1}\leq x \leq \sage{p3r1} \\
                \sage{p3f2}     & \sage{p3l2} \leq x \leq \sage{p3r2} \\
                \sage{p3f3}     & \sage{p3l3} \leq x \leq \sage{p3r3}
            \end{cases}
    \]
    
    If the above piecewise definition is a function, enter $1$. If it is not a function, then enter $0$.
    $\answer{\sage{p3ans}}$
    \begin{feedback}
        To know if the definition is a function you need to check the domains (listed in the right most column) and see if there are any overlapping values between any of the rows. If there are any overlapping values, then you need to check the overlapping values in both functions where those domains overlap, to see if you get the same values or not. If there are no overlapping x-values, or you get the same values in both functions for overlapping x-values, then the definition is a function. If you don't, then it isn't.
    \end{feedback}
    
\end{problem}




\begin{problem}
    Determine if the following piecewise definition is a function or not. 
    \[
        f(x) =
            \begin{cases}
                \sage{p4f1}     & \sage{p4l1}\leq x \leq \sage{p4r1} \\
                \sage{p4f2}     & \sage{p4l2} \leq x \leq \sage{p4r2} \\
                \sage{p4f3}     & \sage{p4l3} \leq x \leq \sage{p4r3}
            \end{cases}
    \]
    
    If the above piecewise definition is a function, enter $1$. If it is not a function, then enter $0$.
    $\answer{\sage{p4ans}}$
    \begin{feedback}
        To know if the definition is a function you need to check the domains (listed in the right most column) and see if there are any overlapping values between any of the rows. If there are any overlapping values, then you need to check the overlapping values in both functions where those domains overlap, to see if you get the same values or not. If there are no overlapping x-values, or you get the same values in both functions for overlapping x-values, then the definition is a function. If you don't, then it isn't.
    \end{feedback}
    
\end{problem}




\begin{problem}
    Determine if the following piecewise definition is a function or not. 
    \[
        f(x) =
            \begin{cases}
                \sage{p5f1}     & \sage{p5l1}\leq x \leq \sage{p5r1} \\
                \sage{p5f2}     & \sage{p5l2} \leq x \leq \sage{p5r2} \\
                \sage{p5f3}     & \sage{p5l3} \leq x \leq \sage{p5r3}
            \end{cases}
    \]
    
    If the above piecewise definition is a function, enter $1$. If it is not a function, then enter $0$.
    $\answer{\sage{p5ans}}$
    \begin{feedback}
        To know if the definition is a function you need to check the domains (listed in the right most column) and see if there are any overlapping values between any of the rows. If there are any overlapping values, then you need to check the overlapping values in both functions where those domains overlap, to see if you get the same values or not. If there are no overlapping x-values, or you get the same values in both functions for overlapping x-values, then the definition is a function. If you don't, then it isn't.
    \end{feedback}
    
\end{problem}




\end{document}







