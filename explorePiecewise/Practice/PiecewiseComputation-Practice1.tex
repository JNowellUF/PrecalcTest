\documentclass{ximera}
\input{../../preamble}
\title{Piecewise: Computation Practice 1}
\begin{document}
\begin{abstract}
    This is a practice understanding of piecewise functions from an analytic viewpoint.
\end{abstract}
\maketitle

\begin{sagesilent}
##### Useful Macros
def RandInt(a,b):
    """ Returns a random integer in [`a`,`b`]. Note that `a` and `b` should be integers themselves to avoid unexpected behavior.
    """
    return QQ(randint(int(a),int(b)))
    # return choice(range(a,b+1))

def NonZeroInt(b,c, avoid = [0]):
    """ Returns a random integer in [`b`,`c`] which is not in `av`. 
        If `av` is not specified, defaults to a non-zero integer.
    """
    while True:
        a = RandInt(b,c)
        if a not in avoid:
            return a

funcvec = [x,x^2,x^3,sqrt(abs(x)),ln(abs(x)+1),e^x]

###############
#### Problem p1
p1f1pick = RandInt(0,5)
p1f2pick = RandInt(0,5)
p1f3pick = RandInt(0,5)

### Build each of the functions.
p1f1 = NonZeroInt(-5,5)*funcvec[p1f1pick](x=(x-RandInt(-5,5))) + RandInt(-5,5)
p1f2 = NonZeroInt(-5,5)*funcvec[p1f2pick](x=(x-RandInt(-5,5))) + RandInt(-5,5)
p1f3 = NonZeroInt(-5,5)*funcvec[p1f3pick](x=(x-RandInt(-5,5))) + RandInt(-5,5)

### Build the left and right endpoints for the functions.
p1l1 = RandInt(-15,15)
p1r1 = p1l1 + RandInt(1,7)
p1r2 = p1r1 + RandInt(1,7)
p1r3 = p1r2 + RandInt(1,7)

## Now to pick a point to evaluate; and then evaluate it:
p1eval = RandInt(p1l1,p1r3)

if p1eval < p1r1+1:
    p1ans = p1f1(x=p1eval)
elif p1eval < p1r2+1:
    p1ans = p1f2(x=p1eval)
else:
    p1ans = p1f3(x=p1eval)


##### End of problem p1




###############
#### Problem p2
p2f1pick = RandInt(0,5)
p2f2pick = RandInt(0,5)
p2f3pick = RandInt(0,5)

### Build each of the functions.
p2f1 = NonZeroInt(-5,5)*funcvec[p2f1pick](x=(x-RandInt(-5,5))) + RandInt(-5,5)
p2f2 = NonZeroInt(-5,5)*funcvec[p2f2pick](x=(x-RandInt(-5,5))) + RandInt(-5,5)
p2f3 = NonZeroInt(-5,5)*funcvec[p2f3pick](x=(x-RandInt(-5,5))) + RandInt(-5,5)

### Build the left and right endpoints for the functions.
p2l1 = RandInt(-15,15)
p2r1 = p2l1 + RandInt(1,7)
p2r2 = p2r1 + RandInt(1,7)
p2r3 = p2r2 + RandInt(1,7)

## Now to pick a point to evaluate; and then evaluate it:
p2eval = RandInt(p2l1,p2r3)

if p2eval < p2r1+1:
    p2ans = p2f1(x=p2eval)
elif p2eval < p2r2+1:
    p2ans = p2f2(x=p2eval)
else:
    p2ans = p2f3(x=p2eval)


##### End of problem p2




###############
#### Problem p3
p3f1pick = RandInt(0,5)
p3f2pick = RandInt(0,5)
p3f3pick = RandInt(0,5)

### Build each of the functions.
p3f1 = NonZeroInt(-5,5)*funcvec[p3f1pick](x=(x-RandInt(-5,5))) + RandInt(-5,5)
p3f2 = NonZeroInt(-5,5)*funcvec[p3f2pick](x=(x-RandInt(-5,5))) + RandInt(-5,5)
p3f3 = NonZeroInt(-5,5)*funcvec[p3f3pick](x=(x-RandInt(-5,5))) + RandInt(-5,5)

### Build the left and right endpoints for the functions.
p3l1 = RandInt(-15,15)
p3r1 = p3l1 + RandInt(1,7)
p3r2 = p3r1 + RandInt(1,7)
p3r3 = p3r2 + RandInt(1,7)

## Now to pick a point to evaluate; and then evaluate it:
p3eval = RandInt(p3l1,p3r3)

if p3eval < p3r1+1:
    p3ans = p3f1(x=p3eval)
elif p3eval < p3r2+1:
    p3ans = p3f2(x=p3eval)
else:
    p3ans = p3f3(x=p3eval)


##### End of problem p3




###############
#### Problem p4
p4f1pick = RandInt(0,5)
p4f2pick = RandInt(0,5)
p4f3pick = RandInt(0,5)

### Build each of the functions.
p4f1 = NonZeroInt(-5,5)*funcvec[p4f1pick](x=(x-RandInt(-5,5))) + RandInt(-5,5)
p4f2 = NonZeroInt(-5,5)*funcvec[p4f2pick](x=(x-RandInt(-5,5))) + RandInt(-5,5)
p4f3 = NonZeroInt(-5,5)*funcvec[p4f3pick](x=(x-RandInt(-5,5))) + RandInt(-5,5)

### Build the left and right endpoints for the functions.
p4l1 = RandInt(-15,15)
p4r1 = p4l1 + RandInt(1,7)
p4r2 = p4r1 + RandInt(1,7)
p4r3 = p4r2 + RandInt(1,7)

## Now to pick a point to evaluate; and then evaluate it:
p4eval = RandInt(p4l1,p4r3)

if p4eval < p4r1+1:
    p4ans = p4f1(x=p4eval)
elif p4eval < p4r2+1:
    p4ans = p4f2(x=p4eval)
else:
    p4ans = p4f3(x=p4eval)


##### End of problem p4




###############
#### Problem p5
p5f1pick = RandInt(0,5)
p5f2pick = RandInt(0,5)
p5f3pick = RandInt(0,5)

### Build each of the functions.
p5f1 = NonZeroInt(-5,5)*funcvec[p5f1pick](x=(x-RandInt(-5,5))) + RandInt(-5,5)
p5f2 = NonZeroInt(-5,5)*funcvec[p5f2pick](x=(x-RandInt(-5,5))) + RandInt(-5,5)
p5f3 = NonZeroInt(-5,5)*funcvec[p5f3pick](x=(x-RandInt(-5,5))) + RandInt(-5,5)

### Build the left and right endpoints for the functions.
p5l1 = RandInt(-15,15)
p5r1 = p5l1 + RandInt(1,7)
p5r2 = p5r1 + RandInt(1,7)
p5r3 = p5r2 + RandInt(1,7)

## Now to pick a point to evaluate; and then evaluate it:
p5eval = RandInt(p5l1,p5r3)

if p5eval < p5r1+1:
    p5ans = p5f1(x=p5eval)
elif p5eval < p5r2+1:
    p5ans = p5f2(x=p5eval)
else:
    p5ans = p5f3(x=p5eval)


##### End of problem p5




\end{sagesilent}

\begin{problem}
    Consider the following piecewise function:
    \[
        f(x) =
            \begin{cases}
                \sage{p1f1}     & \sage{p1l1}\leq x \leq \sage{p1r1} \\
                \sage{p1f2}     & \sage{p1r1} < x \leq \sage{p1r2} \\
                \sage{p1f3}     & \sage{p1r2} < x \leq \sage{p1r3}
            \end{cases}
    \]
    
    Evaluate $f(\sage{p1eval}) = \answer{\sage{p1ans}}$
    \begin{feedback}
        To evaluate, find the row that has the input in the listed domain span. Once you find the correct row, plug the input in as the x-value into the function in that row to find the value of the piecewise function at that input.
    \end{feedback}
    
\end{problem}




\begin{problem}
    Consider the following piecewise function:
    \[
        f(x) =
            \begin{cases}
                \sage{p2f1}     & \sage{p2l1}\leq x \leq \sage{p2r1} \\
                \sage{p2f2}     & \sage{p2r1} < x \leq \sage{p2r2} \\
                \sage{p2f3}     & \sage{p2r2} < x \leq \sage{p2r3}
            \end{cases}
    \]
    
    Evaluate $f(\sage{p2eval}) = \answer{\sage{p2ans}}$
    \begin{feedback}
        To evaluate, find the row that has the input in the listed domain span. Once you find the correct row, plug the input in as the x-value into the function in that row to find the value of the piecewise function at that input.
    \end{feedback}
    
\end{problem}




\begin{problem}
    Consider the following piecewise function:
    \[
        f(x) =
            \begin{cases}
                \sage{p3f1}     & \sage{p3l1}\leq x \leq \sage{p3r1} \\
                \sage{p3f2}     & \sage{p3r1} < x \leq \sage{p3r2} \\
                \sage{p3f3}     & \sage{p3r2} < x \leq \sage{p3r3}
            \end{cases}
    \]
    
    Evaluate $f(\sage{p3eval}) = \answer{\sage{p3ans}}$
    \begin{feedback}
        To evaluate, find the row that has the input in the listed domain span. Once you find the correct row, plug the input in as the x-value into the function in that row to find the value of the piecewise function at that input.
    \end{feedback}
    
\end{problem}




\begin{problem}
    Consider the following piecewise function:
    \[
        f(x) =
            \begin{cases}
                \sage{p4f1}     & \sage{p4l1}\leq x \leq \sage{p4r1} \\
                \sage{p4f2}     & \sage{p4r1} < x \leq \sage{p4r2} \\
                \sage{p4f3}     & \sage{p4r2} < x \leq \sage{p4r3}
            \end{cases}
    \]
    
    Evaluate $f(\sage{p4eval}) = \answer{\sage{p4ans}}$
    \begin{feedback}
        To evaluate, find the row that has the input in the listed domain span. Once you find the correct row, plug the input in as the x-value into the function in that row to find the value of the piecewise function at that input.
    \end{feedback}
    
\end{problem}




\begin{problem}
    Consider the following piecewise function:
    \[
        f(x) =
            \begin{cases}
                \sage{p5f1}     & \sage{p5l1}\leq x \leq \sage{p5r1} \\
                \sage{p5f2}     & \sage{p5r1} < x \leq \sage{p5r2} \\
                \sage{p5f3}     & \sage{p5r2} < x \leq \sage{p5r3}
            \end{cases}
    \]
    
    Evaluate $f(\sage{p5eval}) = \answer{\sage{p5ans}}$
    \begin{feedback}
        To evaluate, find the row that has the input in the listed domain span. Once you find the correct row, plug the input in as the x-value into the function in that row to find the value of the piecewise function at that input.
    \end{feedback}
    
\end{problem}






\end{document}