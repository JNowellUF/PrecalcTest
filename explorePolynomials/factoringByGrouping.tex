\documentclass{ximera}

\title{Factoring; Grouping Method}
\begin{document}
\begin{abstract}
    Factor higher polynomials by grouping terms
\end{abstract}
\maketitle

\youtube{WQzJA7BhFzM}

Generally the first step of factoring anything is to factor out any common factors. Even if this doesn't result in a complete or significant ``piece'' of the factoring process, the more that you can factor out at the beginning the smaller the numbers and terms will be, making all future factoring techniques easier to do.

The next most common way to factor after the method of coefficients is a polynomial version of this idea of ``factoring out common terms'': factoring by grouping. This is when we factor by grouping terms we think are ``similar" together and then factoring out any expressions common to all the terms of the individual group. The hope is that what remains becomes a common term between all the grouped terms. This is difficult to describe, but easier to see with an example.

\begin{example}
    Consider the polynomial $p(x) = 5x^3 + 15x^2 - 3x - 9$. We see that there are some similar terms within this polynomial; specifically the first two terms $5x^3$ and $15x^2$, both share a common coefficient ($5$, as well as $x^2$). Likewise the last two terms; $-3x$ and $-9$ both share a common coefficient ($-3$). So we can try grouping these terms together and factor out the greatest common factor (also known as GCF) from each group which gives us the following;
    
    \[
        p(x) = 5x^3 + 15x^2 - 3x - 9 = (5x^3 + 15x^2) + (-3x -9) = (5x^2)(x + 3) + (-3)(x + 3) 
    \]
    
    To know whether or not the factor by grouping worked, you have to check each of the groups for a common factor. In this case there is in fact a `common factor' of $(x + 3)$ in each group. This is \textbf{vital}, that \textit{both of the leftover factors are \textbf{exactly} the same}. If they weren't exactly the same (for instance, we had gotten $x+3$ for one and $x-3$ for the other) then that means \textbf{factoring by grouping has failed}. We can now factor out the common factor just as we did a moment ago, only this time the common factor is $(x + 3)$. So we have;
    
    \[
        p(x) = (5x^2)(x + 3) + (-3)(x + 3) = (x + 3)(5x^2 - 3)
    \]
    
    \noindent The $5x^2$ and the $-3$ are the `left over' parts when we pull out the common $(x + 3)$ factor.
\end{example}
The key idea here is that the grouping can be done with any number of terms in any combination, so long as what we have leftover is \textit{\textbf{exactly} the same} in each group after factoring out the GCF from each group. But this means that the grouping needs to have groups of the \textit{same size}. That is to say, you could group a function that has 9 terms as 3 groups of 3 terms, but not 1 group of 4 terms and 1 group of 5 terms.




\end{document}