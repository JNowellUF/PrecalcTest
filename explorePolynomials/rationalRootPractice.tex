\documentclass{ximera}
\usepackage{longdivision}
\usepackage{polynom}
\usepackage{float}% Use `H' as the figure optional argument to force it's vertical placement to conform to source.
%\usepackage{caption}% Allows us to describe the figures without having "figure 1:" in it. :: Apparently Caption isn't supported.
%    \captionsetup{labelformat=empty}% Actually does the figure configuration stated above.
\usetikzlibrary{arrows.meta,arrows}% Allow nicer arrow heads for tikz.
\usepackage{gensymb, pgfplots}
\usepackage{tabularx}
\usepackage{arydshln}
\usepackage[margin=1.5cm]{geometry}
\usepackage{indentfirst}

\setlength\parindent{16pt}

\graphicspath{
  {./}
  {./explorePolynomials/}
  {./exploreRadicals/}
  {./graphing/}
}

%% Default style for tikZ
\pgfplotsset{my style/.append style={axis x line=middle, axis y line=
middle, xlabel={$x$}, ylabel={$y$}, axis equal }}


%% Because log being natural log is too hard for people.
\let\logOld\log% Keep the old \log definition, just in case we need it.
\renewcommand{\log}{\ln}


%%% Changes in polynom to show the zero coefficient terms
\makeatletter
\def\pld@CF@loop#1+{%
    \ifx\relax#1\else
        \begingroup
          \pld@AccuSetX11%
          \def\pld@frac{{}{}}\let\pld@symbols\@empty\let\pld@vars\@empty
          \pld@false
          #1%
          \let\pld@temp\@empty
          \pld@AccuIfOne{}{\pld@AccuGet\pld@temp
                            \edef\pld@temp{\noexpand\pld@R\pld@temp}}%
           \pld@if \pld@Extend\pld@temp{\expandafter\pld@F\pld@frac}\fi
           \expandafter\pld@CF@loop@\pld@symbols\relax\@empty
           \expandafter\pld@CF@loop@\pld@vars\relax\@empty
           \ifx\@empty\pld@temp
               \def\pld@temp{\pld@R11}%
           \fi
          \global\let\@gtempa\pld@temp
        \endgroup
        \ifx\@empty\@gtempa\else
            \pld@ExtendPoly\pld@tempoly\@gtempa
        \fi
        \expandafter\pld@CF@loop
    \fi}
\def\pld@CMAddToTempoly{%
    \pld@AccuGet\pld@temp\edef\pld@temp{\noexpand\pld@R\pld@temp}%
    \pld@CondenseMonomials\pld@false\pld@symbols
    \ifx\pld@symbols\@empty \else
        \pld@ExtendPoly\pld@temp\pld@symbols
    \fi
    \ifx\pld@temp\@empty \else
        \pld@if
            \expandafter\pld@IfSum\expandafter{\pld@temp}%
                {\expandafter\def\expandafter\pld@temp\expandafter
                    {\expandafter\pld@F\expandafter{\pld@temp}{}}}%
                {}%
        \fi
        \pld@ExtendPoly\pld@tempoly\pld@temp
        \pld@Extend\pld@tempoly{\pld@monom}%
    \fi}
\makeatother




%%%%% Code for making prime factor trees for numbers, taken from user Qrrbrbirlbel at: https://tex.stackexchange.com/questions/131689/how-to-automatically-draw-tree-diagram-of-prime-factorization-with-latex

\usepackage{forest,mathtools,siunitx}
\makeatletter
\def\ifNum#1{\ifnum#1\relax
  \expandafter\pgfutil@firstoftwo\else
  \expandafter\pgfutil@secondoftwo\fi}
\forestset{
  num content/.style={
    delay={
      content/.expanded={\noexpand\num{\forestoption{content}}}}},
  pt@prime/.style={draw, circle},
  pt@start/.style={},
  pt@normal/.style={},
  start primeTree/.style={%
    /utils/exec=%
      % \pt@start holds the current minimum factor, we'll start with 2
      \def\pt@start{2}%
      % \pt@result will hold the to-be-typeset factorization, we'll start with
      % \pgfutil@gobble since we don't want a initial \times
      \let\pt@result\pgfutil@gobble
      % \pt@start@cnt holds the number of ^factors for the current factor
      \def\pt@start@cnt{0}%
      % \pt@lStart will later hold "l"ast factor used
      \let\pt@lStart\pgfutil@empty,
    alias=pt-start,
    pt@start/.try,
    delay={content/.expanded={$\noexpand\num{\forestove{content}}
                            \noexpand\mathrlap{{}= \noexpand\pt@result}$}},
    primeTree},
  primeTree/.code=%
    % take the content of the node and save it in the count
    \c@pgf@counta\forestove{content}\relax
    % if it's 2 we're already finished with the factorization
    \ifNum{\c@pgf@counta=2}{%
      % add the factor
      \pt@addfactor{2}%
      % finalize the factorization of the result
      \pt@addfactor{}%
      % and set the style to the prime style
      \forestset{pt@prime/.try}%
    }{%
      % this simply calculates content/2 and saves it in \pt@end
      % this is later used for an early break of the recursion since no factor
      % can be greater then content/2 (for integers of course)
      \edef\pt@content{\the\c@pgf@counta}%
      \divide\c@pgf@counta2\relax
      \advance\c@pgf@counta1\relax % to be on the safe side
      \edef\pt@end{\the\c@pgf@counta}%
      \pt@do}}

%%% our main "function"
\def\pt@do{%
  % let's test if the current factor is already greather then the max factor
  \ifNum{\pt@end<\pt@start}{%
    % great, we're finished, the same as above
    \expandafter\pt@addfactor\expandafter{\pt@content}%
    \pt@addfactor{}%
    \def\pt@next{\forestset{pt@prime/.try}}%
  }{%
    % this calculates int(content/factor)*factor
    % if factor is a factor of content (without remainder), the result will
    % equal content. The int(content/factor) is saved in \pgf@temp.
    \c@pgf@counta\pt@content\relax
    \divide\c@pgf@counta\pt@start\relax
    \edef\pgf@temp{\the\c@pgf@counta}%
    \multiply\c@pgf@counta\pt@start\relax
    \ifNum{\the\c@pgf@counta=\pt@content}{%
      % yeah, we found a factor, add it to the result and ...
      \expandafter\pt@addfactor\expandafter{\pt@start}%
      % ... add the factor as the first child with style pt@prime
      % and the result of int(content/factor) as another child.
      \edef\pt@next{\noexpand\forestset{%
        append={[\pt@start, pt@prime/.try]},
        append={[\pgf@temp, pt@normal/.try]},
        % forest is complex, this makes sure that for the second child, the
        % primeTree style is not executed too early (there must be a better way).
        delay={
          for descendants={
            delay={if n'=1{primeTree, num content}{}}}}}}%
    }{%
      % Alright this is not a factor, let's get the next factor
      \ifNum{\pt@start=2}{%
        % if the previous factor was 2, the next one will be 3
        \def\pt@start{3}%
      }{%
        % hmm, the previos factor was not 2,
        % let's add 2, maybe we'll hit the next prime number
        % and maybe a factor
        \c@pgf@counta\pt@start
        \advance\c@pgf@counta2\relax
        \edef\pt@start{\the\c@pgf@counta}%
      }%
      % let's do that again
      \let\pt@next\pt@do
    }%
  }%
  \pt@next
}

%%% this builds the \pt@result macro with the factors
\def\pt@addfactor#1{%
  \def\pgf@tempa{#1}%
  % is it the same factor as the previous one
  \ifx\pgf@tempa\pt@lStart
    % add 1 to the counter
    \c@pgf@counta\pt@start@cnt\relax
    \advance\c@pgf@counta1\relax
    \edef\pt@start@cnt{\the\c@pgf@counta}%
  \else
    % a new factor! Add the previous one to the product of factors
    \ifx\pt@lStart\pgfutil@empty\else
      % as long as there actually is one, the \ifnum makes sure we do not add ^1
      \edef\pgf@tempa{\noexpand\num{\pt@lStart}\ifnum\pt@start@cnt>1 
                                           ^{\noexpand\num{\pt@start@cnt}}\fi}%
      \expandafter\pt@addfactor@\expandafter{\pgf@tempa}%
    \fi
    % setup the macros for the next round
    \def\pt@lStart{#1}% <- current (new) factor
    \def\pt@start@cnt{1}% <- first time
  \fi
}
%%% This simply appends "\times #1" to \pt@result, with etoolbox this would be
%%% \appto\pt@result{\times#1}
\def\pt@addfactor@#1{%
  \expandafter\def\expandafter\pt@result\expandafter{\pt@result \times #1}}

%%% Our main macro:
%%% #1 = possible optional argument for forest (can be tikz too)
%%% #2 = the number to factorize
\newcommand*{\PrimeTree}[2][]{%
  \begin{forest}%
    % as the result is set via \mathrlap it doesn't update the bounding box
    % let's fix this:
    tikz={execute at end scope={\pgfmathparse{width("${}=\pt@result$")}%
                         \path ([xshift=\pgfmathresult pt]pt-start.east);}},
    % other optional arguments
    #1
    % And go!
    [#2, start primeTree]
  \end{forest}}
\makeatother


\providecommand\tabitem{\makebox[1em][r]{\textbullet~}}
\providecommand{\letterPlus}{\makebox[0pt][l]{$+$}}
\providecommand{\letterMinus}{\makebox[0pt][l]{$-$}}

\renewcommand{\texttt}[1]{#1}% Renew the command to prevent it from showing up in the sage strings for some weird reason.
%\renewcommand{\text}[1]{#1}% Renew the command to prevent it from showing up in the sage strings for some weird reason.



\title{Factoring Practice}

\begin{document}
\begin{abstract}
    Unlimited Practice for Polynomial Factoring.
\end{abstract}
\maketitle

\textbf{NOTE:} These are all randomized problems. As a result, it is entirely possible to get pretty awful numbers if you are suitably unlucky. Some of these may look bad until you start doing them, but if you see problems that look excessively awful, remember that you can always hit the `Another' button in the top (green refresh arrow) to get new numbers. If you find yourself doing this frequently, you may want to discuss it with your TA to see if you have a gap in your understanding, or to see if the problems are just really that bad (in which case the TA will forward the info to the content authors).

%\begin{sagesilent}

######  Define a function to convert a sage number into a saved counter number.

#####Define default Sage variables.
#Default function variables
var('x,y,z,X,Y,Z')
#Default function names
var('f,g,h,dx,dy,dz,dh,df')
#Default Wild cards
w0 = SR.wild(0)

def DispSign(b):
    """ Returns the string of the 'signed' version of `b`, e.g. 3 -> "+3", -3 -> "-3", 0 -> "".
    """
    if b == 0:
        return ""
    elif b > 0:
        return "+" + str(b)
    elif b < 0:
        return str(b)
    else:
        # If we're here, then something has gone wrong.
        raise ValueError

def ISP(b):
    return DispSign(b)

def NoEval(f, c):
    # TODO
    """ Returns a non-evaluted version of the result f(c).
    """
    cStr = str(c)
    # fLatex = latex(f)
    fString = latex(f)
    fStrList = list(fString)
    length = len(fStrList)
    fStrList2 = range(length)
    for i in range(0, length):
        if fStrList[i] == "x":
            fStrList2[i] = "("+cstr+")"
        else:
            fStrList2[i] = fStrList[i]
    f2 = join(fStrList2,"")
    return LatexExpr(f2)

def HyperSimp(f):
    """ Returns the expression `f` without hyperbolic expressions.
    """
    subsDict = {
        sinh(w0) : (exp(w0) - exp(-w0))/2,
        cosh(w0) : (exp(w0) + exp(-w0))/2,
        tanh(w0) : (exp(w0) - exp(-w0))/(exp(w0) + exp(-w0)),
        sech(w0) : 2/(exp(w0) + exp(-w0)),                      # This seems to work, but Nowell said it didn't at one point.
        csch(w0) : 2/(exp(w0) - exp(-w0)),                      # This seems to work, but Nowell said it didn't at one point.
        coth(w0) : (exp(w0) + exp(-w0))/(exp(w0) - exp(-w0)),   # This seems to work, but Nowell said it didn't at one point.
        arcsinh(w0) :       ln( w0 + sqrt((w0)^2 + 1) ),
        arccosh(w0) :       ln( w0 + sqrt((w0)^2 - 1) ),
        arctanh(w0) : 1/2 * ln( (1 + w0) / (1 - w0) ),
        arccsch(w0) :       ln( (1 + sqrt((w0)^2 + 1))/w0 ),
        arcsech(w0) :       ln( (1 + sqrt(1 - (w0)^2))/w0 ),
        arccoth(w0) : 1/2 * ln( (1 + w0) / (w0 - 1) )
    }
    g = f.substitute(subsDict)
    return simplify(g)

def RandInt(a,b):
    """ Returns a random integer in [`a`,`b`]. Note that `a` and `b` should be integers themselves to avoid unexpected behavior.
    """
    return QQ(randint(int(a),int(b)))
    # return choice(range(a,b+1))

def NonZeroInt(b,c, avoid = [0]):
    """ Returns a random integer in [`b`,`c`] which is not in `av`. 
        If `av` is not specified, defaults to a non-zero integer.
    """
    while True:
        a = RandInt(b,c)
        if a not in avoid:
            return a

def RandVector(b, c, avoid=[], rep=1):
    """ Returns essentially a multiset permutation of ([b,c]-av) * rep.
        That is, a vector which contains each integer in [`b`,`c`] which is not in `av` a total of `rep` number of times.
        Example:
        sage: RandVector(1,3, [2], 2)
        [3, 1, 1, 3]
    """
    oneVec = [val for val in range(b,c+1) if val not in avoid]
    vec = oneVec * rep
    shuffle(vec)
    return vec

def fudge(b):
    up = b+RandInt(2,5)/10
    down = b-RandInt(2,5)/10
    fudgebup = round(up,1)
    fudgebdown = round(down,1)
    fudgedb = [fudgebdown,fudgebup]
    return fudgedb

def disjointCheck(checkvec):
    if length(uniq(checkvec)) < length(checkvec):
        return 1
    else:
        return 0

def disjointIntervals(IntStart,IntEnd,CheckVal):
    if IntStart < CheckVal and CheckVal < IntEnd:
        return 1
    else:
        return 0

def IntervalVecCheck(checkVec):
    veclen = len(checkVec)
    returnval = 0
    for i in range(veclen):
        for j in range(veclen):
            if (disjointIntervals(checkVec[j][0],checkVec[j][1],checkVec[i][0]) + disjointIntervals(checkVec[j][0],checkVec[j][1],checkVec[i][1])) > 0:
                returnval = returnval + 1
    if returnval > 0:
        return 1
    else:
        return 0



\end{sagesilent}

\begin{sagesilent}
#####Define default Sage variables.
#Default function variables
var('x,y,z,X,Y,Z')
#Default function names
var('f,g,h,dx,dy,dz,dh,df')
#Default Wild cards
w0 = SR.wild(0)

def RandInt(a,b):
    """ Returns a random integer in [`a`,`b`]. Note that `a` and `b` should be integers themselves to avoid unexpected behavior.
    """
    return QQ(randint(int(a),int(b)))
    # return choice(range(a,b+1))

def NonZeroInt(b,c, avoid = [0]):
    """ Returns a random integer in [`b`,`c`] which is not in `av`. 
        If `av` is not specified, defaults to a non-zero integer.
    """
    while True:
        a = RandInt(b,c)
        if a not in avoid:
            return a

\end{sagesilent}
\begin{sagesilent}
### Problem p1
p1c1 = RandInt(-5,5)
p1c2 = NonZeroInt(-5,5,[p1c1,-p1c1])
p1c3 = NonZeroInt(-5,5,[p1c1,-p1c1, p1c2, -p1c2])
p1c4 = NonZeroInt(-5,5,[p1c1,-p1c1, p1c2, -p1c2, p1c3, -p1c3])

p1f1 = expand( (x-p1c1)*(x-p1c2)*(x-p1c3)*(x-p1c4) )
p1ans1 = min(p1c1, p1c2, p1c3, p1c4)
p1ans2 = max(p1c1, p1c2, p1c3, p1c4)
p1ans3 = p1c1+p1c2+p1c3+p1c4


### Problem p2
p2c1 = RandInt(-5,5)
p2c2 = NonZeroInt(-5,5,[p2c1,-p2c1])
p2c3 = NonZeroInt(-5,5,[p2c1,-p2c1, p2c2, -p2c2])
p2c4 = NonZeroInt(-5,5,[p2c1,-p2c1, p2c2, -p2c2, p2c3, -p2c3])

p2f1 = expand( (x-p2c1)*(x-p2c2)*(x-p2c3)*(x-p2c4) )
p2ans1 = min(p2c1, p2c2, p2c3, p2c4)
p2ans2 = max(p2c1, p2c2, p2c3, p2c4)
p2ans3 = p2c1+p2c2+p2c3+p2c4


### Problem p3
p3c1 = RandInt(-5,5)
p3c2 = NonZeroInt(-5,5,[p3c1,-p3c1])
p3c3 = NonZeroInt(-5,5,[p3c1,-p3c1, p3c2, -p3c2])
p3c4 = NonZeroInt(-5,5,[p3c1,-p3c1, p3c2, -p3c2, p3c3, -p3c3])

p3f1 = expand( (x-p3c1)*(x-p3c2)*(x-p3c3)*(x-p3c4) )
p3ans1 = min(p3c1, p3c2, p3c3, p3c4)
p3ans2 = max(p3c1, p3c2, p3c3, p3c4)
p3ans3 = p3c1+p3c2+p3c3+p3c4


### Problem p4
p4c1 = RandInt(1,5)
p4c2 = RandInt(1,5)
p4c3 = NonZeroInt(-5,5)
p4c4 = NonZeroInt(-5,5)
p4c5 = NonZeroInt(-5,5)
p4c6 = NonZeroInt(-5,5)

p4f1 = expand( (x^2 + p4c1)*(x^2 + p4c2)*(p4c3*x-p4c4)*(p4c5*x - p4c6) )

p4ans1 = p4c4/p4c3 + p4c6/p4c5


### Problem p5
p5c1 = RandInt(1,5)
p5c2 = RandInt(1,5)
p5c3 = NonZeroInt(-5,5)
p5c4 = NonZeroInt(-5,5)
p5c5 = NonZeroInt(-5,5)
p5c6 = NonZeroInt(-5,5)

p5f1 = expand( (x^2 + p5c1)*(x^2 + p5c2)*(p5c3*x-p5c4)*(p5c5*x - p5c6) )

p5ans1 = p5c4/p5c3 + p5c6/p5c5


### Problem p6
p6c1 = RandInt(1,5)
p6c2 = RandInt(1,5)
p6c3 = NonZeroInt(-5,5)
p6c4 = NonZeroInt(-5,5)
p6c5 = NonZeroInt(-5,5)
p6c6 = NonZeroInt(-5,5)

p6f1 = expand( (x^2 + p6c1)*(x^2 + p6c2)*(p6c3*x-p6c4)*(p6c5*x - p6c6) )

p6ans1 = p6c4/p6c3 + p6c6/p6c5

\end{sagesilent}

\begin{problem}% Problem p1
    Fully factor the following polynomial (Hint: You likely need to use Rational Root Theorem to find at least one factor)
    \[
        p(x) = \sage{p1f1}
    \]
    The smallest (most negative) zero is: $\answer{\sage{p1ans1}}$\\
    The largest (most positive) zero is: $\answer{\sage{p1ans2}}$\\
    The sum of the zeros of $p(x)$ is: $\answer{\sage{p1ans3}}$
\end{problem}


\begin{problem}% Problem p2
    Fully factor the following polynomial (Hint: You likely need to use Rational Root Theorem to find at least one factor)
    \[
        p(x) = \sage{p2f1}
    \]
    The smallest (most negative) zero is: $\answer{\sage{p2ans1}}$\\
    The largest (most positive) zero is: $\answer{\sage{p2ans2}}$\\
    The sum of the zeros of $p(x)$ is: $\answer{\sage{p2ans3}}$
\end{problem}


\begin{problem}% Problem p3
    Fully factor the following polynomial (Hint: You likely need to use Rational Root Theorem to find at least one factor)
    \[
        p(x) = \sage{p3f1}
    \]
    The smallest (most negative) zero is: $\answer{\sage{p3ans1}}$\\
    The largest (most positive) zero is: $\answer{\sage{p3ans2}}$\\
    The sum of the zeros of $p(x)$ is: $\answer{\sage{p3ans3}}$
\end{problem}


\begin{problem}% Problem p4
    Fully factor the following polynomial using \textit{real} coefficients.
    \[
        p(x) = \sage{p4f1}
    \]
    
    How many real-valued zeros does $p(x)$ have? $\answer{2}$
    
    What is the sum of the real-valued zeros? $\answer{\sage{p4ans1}}$
    
    How many non-real-valued zeros does $p(x)$ have? $\answer{4}$
\end{problem}

\begin{problem}% Problem p5
    Fully factor the following polynomial using \textit{real} coefficients.
    \[
        p(x) = \sage{p5f1}
    \]
    
    How many real-valued zeros does $p(x)$ have? $\answer{2}$
    
    What is the sum of the real-valued zeros? $\answer{\sage{p5ans1}}$
    
    How many non-real-valued zeros does $p(x)$ have? $\answer{4}$
\end{problem}

\begin{problem}% Problem p6
    Fully factor the following polynomial using \textit{real} coefficients.
    \[
        p(x) = \sage{p6f1}
    \]
    
    How many real-valued zeros does $p(x)$ have? $\answer{2}$
    
    What is the sum of the real-valued zeros? $\answer{\sage{p6ans1}}$
    
    How many non-real-valued zeros does $p(x)$ have? $\answer{4}$
\end{problem}


\end{document}