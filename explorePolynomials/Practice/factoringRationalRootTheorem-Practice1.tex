\documentclass{ximera}
\title{Factor Coefficients Method Practice 1}



\begin{document}
Please be aware! Xronos is very good at making sure the factored polynomial you have entered is the same polynomial. It is less reliable to know if you have entered a \textit{fully factored version} of that polynomial. To be sure that your factored version is \textbf{fully} factored, you should check two things:
\begin{itemize}
    \item Every factor is degree 1 or degree 2 polynomial (a linear expression or a quadratic).
    \item If the factor is a quadratic, then the discriminate ($b^2-4ac$) \textbf{must be negative}. If it is Zero or positive, then that term can be factored further and you should try to do so.
\end{itemize}

\begin{sagesilent}
def RandInt(a,b):
    """ Returns a random integer in [`a`,`b`]. Note that `a` and `b` should be integers themselves to avoid unexpected behavior.
    """
    return QQ(randint(int(a),int(b)))
    # return choice(range(a,b+1))

def NonZeroInt(b,c, avoid = [0]):
    """ Returns a random integer in [`b`,`c`] which is not in `av`. 
        If `av` is not specified, defaults to a non-zero integer.
    """
    while True:
        a = RandInt(b,c)
        if a not in avoid:
            return a

primevec = [1, 2, 3, 5, 7, 11]

###### Problem p1
p1ch1 = RandInt(0,3)
p1ch2 = RandInt(0,3)
p1ch3 = NonZeroInt(0,5,[p1ch1])
p1ch4 = NonZeroInt(0,5,[p1ch2])
p1ch5 = RandInt(0,5)
p1ch6 = RandInt(0,5)

p1c1 = primevec[p1ch1]
p1c2 = primevec[p1ch2]

p1c3 = (-1)^(RandInt(0,1))*primevec[p1ch3]
p1c4 = (-1)^(RandInt(0,1))*primevec[p1ch4]
p1c5 = (-1)^(RandInt(0,1))*primevec[p1ch5]
p1c6 = (-1)^(RandInt(0,1))*primevec[p1ch6]

p1c7 = p1c3*p1c4*p1c5*p1c6
p1c8 = p1c1*p1c2

while abs(p1c7) > 600 or abs(p1c8) > 15:
    p1ch1 = RandInt(0,3)
    p1ch2 = RandInt(0,3)
    p1ch3 = NonZeroInt(0,5,[p1ch1])
    p1ch4 = NonZeroInt(0,5,[p1ch2])
    p1ch5 = RandInt(0,5)
    
    p1c1 = primevec[p1ch1]
    p1c2 = primevec[p1ch2]
    
    p1c3 = (-1)^(RandInt(0,1))*primevec[p1ch3]
    p1c4 = (-1)^(RandInt(0,1))*primevec[p1ch4]
    p1c5 = (-1)^(RandInt(0,1))*primevec[p1ch5]
    p1c6 = (-1)^(RandInt(0,1))*primevec[p1ch6]
    
    p1c7 = p1c3*p1c4*p1c5*p1c6
    p1c8 = p1c1*p1c2


p1f1 = (p1c1*x-p1c3)
p1f2 = (p1c2*x-p1c4)
p1f3 = (x-p1c5)
p1f4 = (x-p1c6)
p1f5 = expand(p1f1*p1f2*p1f3*p1f4)
p1ans = (p1f1).mul(p1f2,p1f3,p1f4,hold=true)


###### Problem p2
p2ch1 = RandInt(0,3)
p2ch2 = RandInt(0,3)
p2ch3 = NonZeroInt(0,5,[p2ch1])
p2ch4 = NonZeroInt(0,5,[p2ch2])
p2ch5 = RandInt(0,5)
p2ch6 = RandInt(0,5)

p2c1 = primevec[p2ch1]
p2c2 = primevec[p2ch2]

p2c3 = (-1)^(RandInt(0,1))*primevec[p2ch3]
p2c4 = (-1)^(RandInt(0,1))*primevec[p2ch4]
p2c5 = (-1)^(RandInt(0,1))*primevec[p2ch5]
p2c6 = (-1)^(RandInt(0,1))*primevec[p2ch6]

p2c7 = p2c3*p2c4*p2c5*p2c6
p2c8 = p2c1*p2c2

while abs(p2c7) > 600 or abs(p2c8) > 15:
    p2ch1 = RandInt(0,3)
    p2ch2 = RandInt(0,3)
    p2ch3 = NonZeroInt(0,5,[p2ch1])
    p2ch4 = NonZeroInt(0,5,[p2ch2])
    p2ch5 = RandInt(0,5)
    
    p2c1 = primevec[p2ch1]
    p2c2 = primevec[p2ch2]
    
    p2c3 = (-1)^(RandInt(0,1))*primevec[p2ch3]
    p2c4 = (-1)^(RandInt(0,1))*primevec[p2ch4]
    p2c5 = (-1)^(RandInt(0,1))*primevec[p2ch5]
    p2c6 = (-1)^(RandInt(0,1))*primevec[p2ch6]
    
    p2c7 = p2c3*p2c4*p2c5*p2c6
    p2c8 = p2c1*p2c2


p2f1 = (p2c1*x-p2c3)
p2f2 = (p2c2*x-p2c4)
p2f3 = (x-p2c5)
p2f4 = (x-p2c6)
p2f5 = expand(p2f1*p2f2*p2f3*p2f4)
p2ans = (p2f1).mul(p2f2,p2f3,p2f4,hold=true)

###### Problem p3
p3ch1 = RandInt(0,3)
p3ch2 = RandInt(0,3)
p3ch3 = NonZeroInt(0,5,[p3ch1])
p3ch4 = NonZeroInt(0,5,[p3ch2])
p3ch5 = RandInt(0,5)
p3ch6 = RandInt(0,5)

p3c1 = primevec[p3ch1]
p3c2 = primevec[p3ch2]

p3c3 = (-1)^(RandInt(0,1))*primevec[p3ch3]
p3c4 = (-1)^(RandInt(0,1))*primevec[p3ch4]
p3c5 = (-1)^(RandInt(0,1))*primevec[p3ch5]
p3c6 = (-1)^(RandInt(0,1))*primevec[p3ch6]

p3c7 = p3c3*p3c4*p3c5*p3c6
p3c8 = p3c1*p3c2

while abs(p3c7) > 600 or abs(p3c8) > 15:
    p3ch1 = RandInt(0,3)
    p3ch2 = RandInt(0,3)
    p3ch3 = NonZeroInt(0,5,[p3ch1])
    p3ch4 = NonZeroInt(0,5,[p3ch2])
    p3ch5 = RandInt(0,5)
    
    p3c1 = primevec[p3ch1]
    p3c2 = primevec[p3ch2]
    
    p3c3 = (-1)^(RandInt(0,1))*primevec[p3ch3]
    p3c4 = (-1)^(RandInt(0,1))*primevec[p3ch4]
    p3c5 = (-1)^(RandInt(0,1))*primevec[p3ch5]
    p3c6 = (-1)^(RandInt(0,1))*primevec[p3ch6]
    
    p3c7 = p3c3*p3c4*p3c5*p3c6
    p3c8 = p3c1*p3c2


p3f1 = (p3c1*x-p3c3)
p3f2 = (p3c2*x-p3c4)
p3f3 = (x-p3c5)
p3f4 = (x-p3c6)
p3f5 = expand(p3f1*p3f2*p3f3*p3f4)
p3ans = (p3f1).mul(p3f2,p3f3,p3f4,hold=true)

###### Problem p4
p4ch1 = RandInt(0,3)
p4ch2 = RandInt(0,3)
p4ch3 = NonZeroInt(0,5,[p4ch1])
p4ch4 = NonZeroInt(0,5,[p4ch2])
p4ch5 = RandInt(0,5)
p4ch6 = RandInt(0,5)

p4c1 = primevec[p4ch1]
p4c2 = primevec[p4ch2]

p4c3 = (-1)^(RandInt(0,1))*primevec[p4ch3]
p4c4 = (-1)^(RandInt(0,1))*primevec[p4ch4]
p4c5 = (-1)^(RandInt(0,1))*primevec[p4ch5]
p4c6 = (-1)^(RandInt(0,1))*primevec[p4ch6]

p4c7 = p4c3*p4c4*p4c5*p4c6
p4c8 = p4c1*p4c2

while abs(p4c7) > 600 or abs(p4c8) > 15:
    p4ch1 = RandInt(0,3)
    p4ch2 = RandInt(0,3)
    p4ch3 = NonZeroInt(0,5,[p4ch1])
    p4ch4 = NonZeroInt(0,5,[p4ch2])
    p4ch5 = RandInt(0,5)
    
    p4c1 = primevec[p4ch1]
    p4c2 = primevec[p4ch2]
    
    p4c3 = (-1)^(RandInt(0,1))*primevec[p4ch3]
    p4c4 = (-1)^(RandInt(0,1))*primevec[p4ch4]
    p4c5 = (-1)^(RandInt(0,1))*primevec[p4ch5]
    p4c6 = (-1)^(RandInt(0,1))*primevec[p4ch6]
    
    p4c7 = p4c3*p4c4*p4c5*p4c6
    p4c8 = p4c1*p4c2


p4f1 = (p4c1*x-p4c3)
p4f2 = (p4c2*x-p4c4)
p4f3 = (x-p4c5)
p4f4 = (x-p4c6)
p4f5 = expand(p4f1*p4f2*p4f3*p4f4)
p4ans = (p4f1).mul(p4f2,p4f3,p4f4,hold=true)



\end{sagesilent}

\begin{javascript}
// A validator to check and verify something has a factored form...
function factorCheck(f,g) {
    // This validator is designed to check that a student is submitting a factored polynomial. It works by:
    //  Checking that there are the correct number of non-numeric and non-inverse factors as expected,
    //  Checking that the submitted answer and the expected answer are the same via real Xronos evaluation,
    //  Checking that the outer most (last to be computed when following order of operations) operation is multiplication.
    
    var operCheck = f.tree[0];// Check to see if the root operation is multiplication at end.
    var studentFactors = f.tree.length;// Temporary number of student-provided factors (+1 because of root operation)
    
    // Now we adjust the length to remove any numeric factors, or division factors, etc to avoid ``padding'' by students.
    for (var i = 0; i < f.tree.length; i++) {
        if ((typeof f.tree[i] === 'number')||(f.tree[i][0] == '-')||(f.tree[i][0] == '/')) {
            studentFactors = studentFactors - 1;
        }
    }
    
    // Now we do the same with the provided answer, in case sage or something provides a weird format.
    var answerFactors = g.tree.length;
    
    // Adjust length in the same way, so that it will match the students if it should.
    for (var i = 0; i < g.tree.length; i++) {
        if (typeof g.tree[i] === 'number') {
            answerFactors = answerFactors - 1;
        }
    }
    
    // Note: An especially dedicated student could pad with weird factors that are happen to cancel down to 1.
    // For example, a student could enter sin^2(x)+cos^2(x) as a multiplicative factor to pad the number of factors.
    // This would be somewhat difficult to think of, even on purpose.
    // Until I can reliably evaluate the factors themselves as functions though, there isn't a lot to be done here.
    
    return ((f.equals(g))&&(studentFactors==answerFactors)&&(operCheck=='*'))
    }
\end{javascript}

\textbf{Note:} This is using an experimental factoring validator. If you verified that your answer should be correct and Xronos won't take it, please email your instructor to see if there is a problem.

\begin{problem}
    
    Factor the following polynomial (hint: Using the Rational Root Theorem to get a zero, then divide out the factor to reduce the polynomial)
    
    \[
        \sage{p1f5} = \answer[validator=factorCheck]{\sage{p1ans}}
    \]
    \begin{feedback}
        Remember that the Rational Root Theorem says you want to find all the factors of the constant (i.e. $\sage{p1c7}$) and all the factors of the leading coefficient (i.e. $\sage{p1c8}$) and create a list of ($\pm$) every combination of a factor of $\sage{p1c7}$ divided by a factor of $\sage{p1c8}$. 
        
        Note that this is probably going to give you a long list! Unfortunately there really isn't a great way to know which of the factors to try first; so I would typically suggest starting with the ones that are easiest to calculate... This is why the Rational Root Theorem should \textit{always be a tool of last resort!} It takes a long time, and it is really inefficient, but sometimes it's all you have. In this spirit, once you have found a zero and factored it out using something like polynomial long division, make sure to start over on the resulting polynomial in terms of factoring; with any luck you won't need to use RRT and can use some other technique (like grouping or quadratic form). The goal is to use RRT as infrequently as possible!
    \end{feedback}
    
\end{problem}


\begin{problem}
    
    Factor the following polynomial (hint: Using the Rational Root Theorem to get a zero, then divide out the factor to reduce the polynomial)
    
    \[
        \sage{p2f5} = \answer[validator=factorCheck]{\sage{p2ans}}
    \]
    \begin{feedback}
        Remember that the Rational Root Theorem says you want to find all the factors of the constant (i.e. $\sage{p2c7}$) and all the factors of the leading coefficient (i.e. $\sage{p2c8}$) and create a list of ($\pm$) every combination of a factor of $\sage{p2c7}$ divided by a factor of $\sage{p2c8}$. 
        
        Note that this is probably going to give you a long list! Unfortunately there really isn't a great way to know which of the factors to try first; so I would typically suggest starting with the ones that are easiest to calculate... This is why the Rational Root Theorem should \textit{always be a tool of last resort!} It takes a long time, and it is really inefficient, but sometimes it's all you have. In this spirit, once you have found a zero and factored it out using something like polynomial long division, make sure to start over on the resulting polynomial in terms of factoring; with any luck you won't need to use RRT and can use some other technique (like grouping or quadratic form). The goal is to use RRT as infrequently as possible!
    \end{feedback}
    
\end{problem}


\begin{problem}
    
    Factor the following polynomial (hint: Using the Rational Root Theorem to get a zero, then divide out the factor to reduce the polynomial)
    
    \[
        \sage{p3f5} = \answer[validator=factorCheck]{\sage{p3ans}}
    \]
    \begin{feedback}
        Remember that the Rational Root Theorem says you want to find all the factors of the constant (i.e. $\sage{p3c7}$) and all the factors of the leading coefficient (i.e. $\sage{p3c8}$) and create a list of ($\pm$) every combination of a factor of $\sage{p3c7}$ divided by a factor of $\sage{p3c8}$. 
        
        Note that this is probably going to give you a long list! Unfortunately there really isn't a great way to know which of the factors to try first; so I would typically suggest starting with the ones that are easiest to calculate... This is why the Rational Root Theorem should \textit{always be a tool of last resort!} It takes a long time, and it is really inefficient, but sometimes it's all you have. In this spirit, once you have found a zero and factored it out using something like polynomial long division, make sure to start over on the resulting polynomial in terms of factoring; with any luck you won't need to use RRT and can use some other technique (like grouping or quadratic form). The goal is to use RRT as infrequently as possible!
    \end{feedback}
    
\end{problem}


\begin{problem}
    
    Factor the following polynomial (hint: Using the Rational Root Theorem to get a zero, then divide out the factor to reduce the polynomial)
    
    \[
        \sage{p4f5} = \answer[validator=factorCheck]{\sage{p4ans}}
    \]
    \begin{feedback}
        Remember that the Rational Root Theorem says you want to find all the factors of the constant (i.e. $\sage{p4c7}$) and all the factors of the leading coefficient (i.e. $\sage{p4c8}$) and create a list of ($\pm$) every combination of a factor of $\sage{p4c7}$ divided by a factor of $\sage{p4c8}$. 
        
        Note that this is probably going to give you a long list! Unfortunately there really isn't a great way to know which of the factors to try first; so I would typically suggest starting with the ones that are easiest to calculate... This is why the Rational Root Theorem should \textit{always be a tool of last resort!} It takes a long time, and it is really inefficient, but sometimes it's all you have. In this spirit, once you have found a zero and factored it out using something like polynomial long division, make sure to start over on the resulting polynomial in terms of factoring; with any luck you won't need to use RRT and can use some other technique (like grouping or quadratic form). The goal is to use RRT as infrequently as possible!
    \end{feedback}
    
\end{problem}



\end{document}