\documentclass{ximera}
\title{Factor Coefficients Method Practice 1}



\begin{document}
%Please be aware! Xronos is very good at making sure the factored polynomial you have entered is the same polynomial. It is less reliable to know if you have entered a \textit{fully factored version} of that polynomial. To be sure that your factored version is \textbf{fully} factored, you should check two things:
%\begin{itemize}
%    \item Every factor is degree 1 or degree 2 polynomial (a linear expression or a quadratic).
%    \item If the factor is a quadratic, then the discriminate ($b^2-4ac$) \textbf{must be negative}. If it is Zero or positive, then that term can be factored further and you should try to do so.
%\end{itemize}

\begin{sagesilent}
def RandInt(a,b):
    """ Returns a random integer in [`a`,`b`]. Note that `a` and `b` should be integers themselves to avoid unexpected behavior.
    """
    return QQ(randint(int(a),int(b)))
    # return choice(range(a,b+1))

def NonZeroInt(b,c, avoid = [0]):
    """ Returns a random integer in [`b`,`c`] which is not in `av`. 
        If `av` is not specified, defaults to a non-zero integer.
    """
    while True:
        a = RandInt(b,c)
        if a not in avoid:
            return a


###### Problem p1
p1c1 = NonZeroInt(-10,10)
p1c2 = NonZeroInt(-5,5)

p1f1 = (p1c2*x - p1c1)
p1tog = RandInt(0,1)
if p1tog == 0:
    p1f2 = (p1c2*x + p1c1)
else:
    p1f2 = ((p1c2*x)^2 + p1c2*p1c1*x + p1c1^2)

p1f3 = p1f1.mul(p1f2,hold=true)
p1f4 = expand(p1f3)



###### Problem p2
p2c1 = NonZeroInt(-10,10)
p2c2 = NonZeroInt(-5,5)

p2f1 = (p2c2*x - p2c1)
p2tog = RandInt(0,1)
if p2tog == 0:
    p2f2 = (p2c2*x + p2c1)
else:
    p2f2 = ((p2c2*x)^2 + p2c2*p2c1*x + p2c1^2)

p2f3 = p2f1.mul(p2f2,hold=true)
p2f4 = expand(p2f3)


###### Problem p3
p3c1 = NonZeroInt(-10,10)
p3c2 = NonZeroInt(-5,5)

p3f1 = (p3c2*x - p3c1)
p3tog = RandInt(0,1)
if p3tog == 0:
    p3f2 = (p3c2*x + p3c1)
else:
    p3f2 = ((p3c2*x)^2 + p3c2*p3c1*x + p3c1^2)

p3f3 = p3f1.mul(p3f2,hold=true)
p3f4 = expand(p3f3)



\end{sagesilent}

\begin{javascript}
// A validator to check and verify something has a factored form...
function factorCheck(f,g) {
    // This validator is designed to check that a student is submitting a factored polynomial. It works by:
    //  Checking that there are the correct number of non-numeric and non-inverse factors as expected,
    //  Checking that the submitted answer and the expected answer are the same via real Xronos evaluation,
    //  Checking that the outer most (last to be computed when following order of operations) operation is multiplication.
    
    var operCheck = f.tree[0];// Check to see if the root operation is multiplication at end.
    var studentFactors = f.tree.length;// Temporary number of student-provided factors (+1 because of root operation)
    
    // Now we adjust the length to remove any numeric factors, or division factors, etc to avoid ``padding'' by students.
    for (var i = 0; i < f.tree.length; i++) {
        if ((typeof f.tree[i] === 'number')||(f.tree[i][0] == '-')||(f.tree[i][0] == '/')) {
            studentFactors = studentFactors - 1;
        }
    }
    
    // Now we do the same with the provided answer, in case sage or something provides a weird format.
    var answerFactors = g.tree.length;
    
    // Adjust length in the same way, so that it will match the students if it should.
    for (var i = 0; i < g.tree.length; i++) {
        if (typeof g.tree[i] === 'number') {
            answerFactors = answerFactors - 1;
        }
    }
    
    // Note: An especially dedicated student could pad with weird factors that are happen to cancel down to 1.
    // For example, a student could enter sin^2(x)+cos^2(x) as a multiplicative factor to pad the number of factors.
    // This would be somewhat difficult to think of, even on purpose.
    // Until I can reliably evaluate the factors themselves as functions though, there isn't a lot to be done here.
    
    return ((f.equals(g))&&(studentFactors==answerFactors)&&(operCheck=='*'))
    }
\end{javascript}

\textbf{Note:} This is using an experimental factoring validator. If you verified that your answer should be correct and Xronos won't take it, please email your instructor to see if there is a problem.

\begin{problem}
Factor the following polynomial:
    \[
        \sage{p1f4} = \answer[validator=factorCheck]{\sage{p1f3}}
    \]
    \begin{feedback}
        If the degree of the polynomial is even, then it's a good idea to start with difference of squares (if possible). If it isn't even, but is divisible by 3, then you can try a difference/sum of cubes.
    \end{feedback}
\end{problem}

\begin{problem}
Factor the following polynomial:
    \[
        \sage{p2f4} = \answer[validator=factorCheck]{\sage{p2f3}}
    \]
    \begin{feedback}
        If the degree of the polynomial is even, then it's a good idea to start with difference of squares (if possible). If it isn't even, but is divisible by 3, then you can try a difference/sum of cubes.
    \end{feedback}
\end{problem}



\begin{problem}
Factor the following polynomial:
    \[
        \sage{p3f4} = \answer[validator=factorCheck]{\sage{p3f3}}
    \]
    \begin{feedback}
        If the degree of the polynomial is even, then it's a good idea to start with difference of squares (if possible). If it isn't even, but is divisible by 3, then you can try a difference/sum of cubes.
    \end{feedback}
\end{problem}


\end{document}