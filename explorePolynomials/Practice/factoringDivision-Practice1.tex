\documentclass{ximera}
\title{Factor Coefficients Method Practice 1}

%\usepackage[margin=1cm]{geometry}

\begin{document}
\begin{sagesilent}

######  Define a function to convert a sage number into a saved counter number.

#####Define default Sage variables.
#Default function variables
var('x,y,z,X,Y,Z')
#Default function names
var('f,g,h,dx,dy,dz,dh,df')
#Default Wild cards
w0 = SR.wild(0)

def DispSign(b):
    """ Returns the string of the 'signed' version of `b`, e.g. 3 -> "+3", -3 -> "-3", 0 -> "".
    """
    if b == 0:
        return ""
    elif b > 0:
        return "+" + str(b)
    elif b < 0:
        return str(b)
    else:
        # If we're here, then something has gone wrong.
        raise ValueError

def ISP(b):
    return DispSign(b)

def NoEval(f, c):
    # TODO
    """ Returns a non-evaluted version of the result f(c).
    """
    cStr = str(c)
    # fLatex = latex(f)
    fString = latex(f)
    fStrList = list(fString)
    length = len(fStrList)
    fStrList2 = range(length)
    for i in range(0, length):
        if fStrList[i] == "x":
            fStrList2[i] = "("+cstr+")"
        else:
            fStrList2[i] = fStrList[i]
    f2 = join(fStrList2,"")
    return LatexExpr(f2)

def HyperSimp(f):
    """ Returns the expression `f` without hyperbolic expressions.
    """
    subsDict = {
        sinh(w0) : (exp(w0) - exp(-w0))/2,
        cosh(w0) : (exp(w0) + exp(-w0))/2,
        tanh(w0) : (exp(w0) - exp(-w0))/(exp(w0) + exp(-w0)),
        sech(w0) : 2/(exp(w0) + exp(-w0)),                      # This seems to work, but Nowell said it didn't at one point.
        csch(w0) : 2/(exp(w0) - exp(-w0)),                      # This seems to work, but Nowell said it didn't at one point.
        coth(w0) : (exp(w0) + exp(-w0))/(exp(w0) - exp(-w0)),   # This seems to work, but Nowell said it didn't at one point.
        arcsinh(w0) :       ln( w0 + sqrt((w0)^2 + 1) ),
        arccosh(w0) :       ln( w0 + sqrt((w0)^2 - 1) ),
        arctanh(w0) : 1/2 * ln( (1 + w0) / (1 - w0) ),
        arccsch(w0) :       ln( (1 + sqrt((w0)^2 + 1))/w0 ),
        arcsech(w0) :       ln( (1 + sqrt(1 - (w0)^2))/w0 ),
        arccoth(w0) : 1/2 * ln( (1 + w0) / (w0 - 1) )
    }
    g = f.substitute(subsDict)
    return simplify(g)

def RandInt(a,b):
    """ Returns a random integer in [`a`,`b`]. Note that `a` and `b` should be integers themselves to avoid unexpected behavior.
    """
    return QQ(randint(int(a),int(b)))
    # return choice(range(a,b+1))

def NonZeroInt(b,c, avoid = [0]):
    """ Returns a random integer in [`b`,`c`] which is not in `av`. 
        If `av` is not specified, defaults to a non-zero integer.
    """
    while True:
        a = RandInt(b,c)
        if a not in avoid:
            return a

def RandVector(b, c, avoid=[], rep=1):
    """ Returns essentially a multiset permutation of ([b,c]-av) * rep.
        That is, a vector which contains each integer in [`b`,`c`] which is not in `av` a total of `rep` number of times.
        Example:
        sage: RandVector(1,3, [2], 2)
        [3, 1, 1, 3]
    """
    oneVec = [val for val in range(b,c+1) if val not in avoid]
    vec = oneVec * rep
    shuffle(vec)
    return vec

def fudge(b):
    up = b+RandInt(2,5)/10
    down = b-RandInt(2,5)/10
    fudgebup = round(up,1)
    fudgebdown = round(down,1)
    fudgedb = [fudgebdown,fudgebup]
    return fudgedb

def disjointCheck(checkvec):
    if length(uniq(checkvec)) < length(checkvec):
        return 1
    else:
        return 0

def disjointIntervals(IntStart,IntEnd,CheckVal):
    if IntStart < CheckVal and CheckVal < IntEnd:
        return 1
    else:
        return 0

def IntervalVecCheck(checkVec):
    veclen = len(checkVec)
    returnval = 0
    for i in range(veclen):
        for j in range(veclen):
            if (disjointIntervals(checkVec[j][0],checkVec[j][1],checkVec[i][0]) + disjointIntervals(checkVec[j][0],checkVec[j][1],checkVec[i][1])) > 0:
                returnval = returnval + 1
    if returnval > 0:
        return 1
    else:
        return 0



\end{sagesilent}

\begin{sagesilent}
###### Problem p1
p1c1 = NonZeroInt(-5,5)
p1c2 = RandInt(-10,10)
p1pwr1 = RandInt(1,2)
p1pwr2 = RandInt(2,5)
p1f0 = sum([RandInt(-5,5)*x^t for t in [0..p1pwr1-1]])


p1f1 = p1c1*x^p1pwr1 + sum([RandInt(-5,5)*x^t for t in [0..p1pwr1-1]])
p1f2 = NonZeroInt(-5,5)*x^p1pwr2 + sum([RandInt(-5,5)*x^t for t in [0..p1pwr2-1]])
p1f3 = expand(p1f1*p1f2) + p1f0


###### Problem p2
p2c1 = NonZeroInt(-5,5)
p2c2 = RandInt(-10,10)
p2pwr1 = RandInt(1,2)
p2pwr2 = RandInt(2,5)
p2f0 = sum([RandInt(-5,5)*x^t for t in [0..p2pwr1-1]])


p2f1 = p2c1*x^p2pwr1 + sum([RandInt(-5,5)*x^t for t in [0..p2pwr1-1]])
p2f2 = NonZeroInt(-5,5)*x^p2pwr2 + sum([RandInt(-5,5)*x^t for t in [0..p2pwr2-1]])
p2f3 = expand(p2f1*p2f2) + p2f0


###### Problem p3
p3c1 = NonZeroInt(-5,5)
p3c2 = RandInt(-10,10)
p3pwr1 = RandInt(1,2)
p3pwr2 = RandInt(2,5)
p3f0 = sum([RandInt(-5,5)*x^t for t in [0..p3pwr1-1]])


p3f1 = p3c1*x^p3pwr1 + sum([RandInt(-5,5)*x^t for t in [0..p3pwr1-1]])
p3f2 = NonZeroInt(-5,5)*x^p3pwr2 + sum([RandInt(-5,5)*x^t for t in [0..p3pwr2-1]])
p3f3 = expand(p3f1*p3f2) + p3f0


###### Problem p4
p4c1 = NonZeroInt(-5,5)
p4c2 = RandInt(-10,10)
p4pwr1 = RandInt(1,2)
p4pwr2 = RandInt(2,5)
p4f0 = sum([RandInt(-5,5)*x^t for t in [0..p4pwr1-1]])


p4f1 = p4c1*x^p4pwr1 + sum([RandInt(-5,5)*x^t for t in [0..p4pwr1-1]])
p4f2 = NonZeroInt(-5,5)*x^p4pwr2 + sum([RandInt(-5,5)*x^t for t in [0..p4pwr2-1]])
p4f3 = expand(p4f1*p4f2) + p4f0



\end{sagesilent}

%\begin{javascript}
%linearFactoring = function(correctAns,f1,f2,f3) {
%    var i;
%    var Truth;
%    var Prod;
%    var Temp;
%    Truth=1;
%    Prod=1;
%    if (isNaN(f1.derivative('x'))){Truth = Truth * 0;}else{Truth = Truth * 1;}
%    if (isNaN(f2.derivative('x'))){Truth = Truth * 0;}else{Truth = Truth * 1;}
%    if (isNaN(f3.derivative('x'))){Truth = Truth * 0;}else{Truth = Truth * 1;}
%    parser.evaluate('f(x) = f1(x)*f2(x)*f3(x));
%    if (f('x').equals(correctAns('x'))){Truth = Truth*1;}else{Truth=Truth*0;}
%    if (Truth === 0){return !1}else{return !0}
%    };
%\end{javascript}

\begin{problem}

    Compute the following division:
    \begin{align*}
        \left(\sage{p1f3}\right) \div \left(\sage{p1f1}\right) \\
        = \answer{\sage{p1f2}} + \frac{\answer{\sage{p1f0}}}{\answer{\sage{p1f1}}}
    \end{align*}
    \begin{feedback}
        When dividing, make sure to account for \textbf{all} powers of $x$, especially those missing in the polynomial. For example, if you are dividing $x^3 + 3x - 2$, then first rewrite the polynomial as $x^3 + 0x^2 + 3x - 2$ to ensure you are accounting for the missing $x^2$ term.
        
        When you are dividing by a polynomial that is higher than degree 1 (for example dividing by a quadratic like $x^2 -2$) or if the leading term's coefficient is not 1 (for example, dividing by something like $3x+1$ or $-x + 7$), it is \textit{much} better to use polynomial long division, and not synthetic division. Synthetic division will almost certainly give you the wrong polynomial result in both these cases, without doing some clever extra steps.
    \end{feedback}
    
\end{problem}

\begin{problem}
    
    Compute the following division:
    \begin{align*}
        \left(\sage{p2f3}\right) \div \left(\sage{p2f1}\right) \\
        = \answer{\sage{p2f2}} + \frac{\answer{\sage{p2f0}}}{\answer{\sage{p2f1}}}
    \end{align*}
    \begin{feedback}
        When dividing, make sure to account for \textbf{all} powers of $x$, especially those missing in the polynomial. For example, if you are dividing $x^3 + 3x - 2$, then first rewrite the polynomial as $x^3 + 0x^2 + 3x - 2$ to ensure you are accounting for the missing $x^2$ term.
        
        When you are dividing by a polynomial that is higher than degree 1 (for example dividing by a quadratic like $x^2 -2$) or if the leading term's coefficient is not 1 (for example, dividing by something like $3x+1$ or $-x + 7$), it is \textit{much} better to use polynomial long division, and not synthetic division. Synthetic division will almost certainly give you the wrong polynomial result in both these cases, without doing some clever extra steps.
    \end{feedback}
    
\end{problem}

\begin{problem}
    
    Compute the following division:
    \begin{align*}
        \left(\sage{p3f3}\right) \div \left(\sage{p3f1}\right) \\
        = \answer{\sage{p3f2}} + \frac{\answer{\sage{p3f0}}}{\answer{\sage{p3f1}}}
    \end{align*}
    \begin{feedback}
        When dividing, make sure to account for \textbf{all} powers of $x$, especially those missing in the polynomial. For example, if you are dividing $x^3 + 3x - 2$, then first rewrite the polynomial as $x^3 + 0x^2 + 3x - 2$ to ensure you are accounting for the missing $x^2$ term.
        
        When you are dividing by a polynomial that is higher than degree 1 (for example dividing by a quadratic like $x^2 -2$) or if the leading term's coefficient is not 1 (for example, dividing by something like $3x+1$ or $-x + 7$), it is \textit{much} better to use polynomial long division, and not synthetic division. Synthetic division will almost certainly give you the wrong polynomial result in both these cases, without doing some clever extra steps.
    \end{feedback}
    
\end{problem}

\begin{problem}
    
    Compute the following division:
    \begin{align*}
        \left(\sage{p4f3}\right) \div \left(\sage{p4f1}\right) \\
        = \answer{\sage{p4f2}} + \frac{\answer{\sage{p4f0}}}{\answer{\sage{p4f1}}}
    \end{align*}
    \begin{feedback}
        When dividing, make sure to account for \textbf{all} powers of $x$, especially those missing in the polynomial. For example, if you are dividing $x^3 + 3x - 2$, then first rewrite the polynomial as $x^3 + 0x^2 + 3x - 2$ to ensure you are accounting for the missing $x^2$ term.
        
        When you are dividing by a polynomial that is higher than degree 1 (for example dividing by a quadratic like $x^2 -2$) or if the leading term's coefficient is not 1 (for example, dividing by something like $3x+1$ or $-x + 7$), it is \textit{much} better to use polynomial long division, and not synthetic division. Synthetic division will almost certainly give you the wrong polynomial result in both these cases, without doing some clever extra steps.
    \end{feedback}
    
\end{problem}



\end{document}