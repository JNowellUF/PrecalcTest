\documentclass{ximera}
\title{Comprehensive Factoring Quiz}
\begin{document}
\begin{abstract}
    A Comprehensive Factoring Practice Quiz.
\end{abstract}
\maketitle

\begin{sagesilent}

### Initial function definitions:
def RandInt(a,b):
    """ Returns a random integer in [`a`,`b`]. Note that `a` and `b` should be integers themselves to avoid unexpected behavior.
    """
    return QQ(randint(int(a),int(b)))
    # return choice(range(a,b+1))

def NonZeroInt(b,c, avoid = [0]):
    """ Returns a random integer in [`b`,`c`] which is not in `av`. 
        If `av` is not specified, defaults to a non-zero integer.
    """
    while True:
        a = RandInt(b,c)
        if a not in avoid:
            return a

######  Problem p1

primevec = [1, 2, 3, 5, 7, 11]

p1variant = RandInt(0,4)

##  Coefficient Method
if p1variant == 0:
    p1c1 = RandInt(-10,10)
    p1c2 = RandInt(-10,10)
    p1f1 = x+p1c1
    p1f2 = x+p1c2
    p1disp = expand(p1f1*p1f2)
    p1ans = (p1f1).mul(p1f2,hold=true)


##  AC Method 
if p1variant == 1:
    p1c1 = RandInt(-10,10)
    p1c2 = RandInt(-10,10)
    p1c3 = NonZeroInt(-5,5)
    p1c4 = RandInt(1,10)
    p1f1 = p1c3*x + p1c1
    p1f2 = p1c4*x + p1c2
    
    p1disp = expand(p1f1*p1f2)
    p1ans = (p1f1).mul(p1f2, hold=true)



##  Factoring with Rational Root Theorem 1
if p1variant == 2:
    p1ch1 = RandInt(0,3)
    p1ch2 = RandInt(0,3)
    p1ch3 = NonZeroInt(0,5,[p1ch1])
    p1ch4 = NonZeroInt(0,5,[p1ch2])
    p1ch5 = RandInt(0,5)
    p1ch6 = RandInt(0,5)
    
    p1c1 = primevec[p1ch1]
    p1c2 = primevec[p1ch2]
    
    p1c3 = (-1)^(RandInt(0,1))*primevec[p1ch3]
    p1c4 = (-1)^(RandInt(0,1))*primevec[p1ch4]
    p1c5 = (-1)^(RandInt(0,1))*primevec[p1ch5]
    p1c6 = (-1)^(RandInt(0,1))*primevec[p1ch6]
    
    p1c7 = p1c3*p1c4*p1c5*p1c6
    p1c8 = p1c1*p1c2
    
    while abs(p1c7) > 600 or abs(p1c8) > 15:
        p1ch1 = RandInt(0,3)
        p1ch2 = RandInt(0,3)
        p1ch3 = NonZeroInt(0,5,[p1ch1])
        p1ch4 = NonZeroInt(0,5,[p1ch2])
        p1ch5 = RandInt(0,5)
        
        p1c1 = primevec[p1ch1]
        p1c2 = primevec[p1ch2]
        
        p1c3 = (-1)^(RandInt(0,1))*primevec[p1ch3]
        p1c4 = (-1)^(RandInt(0,1))*primevec[p1ch4]
        p1c5 = (-1)^(RandInt(0,1))*primevec[p1ch5]
        p1c6 = (-1)^(RandInt(0,1))*primevec[p1ch6]
        
        p1c7 = p1c3*p1c4*p1c5*p1c6
        p1c8 = p1c1*p1c2
    
    
    p1f1 = (p1c1*x-p1c3)
    p1f2 = (p1c2*x-p1c4)
    p1f3 = (x-p1c5)
    p1f4 = (x-p1c6)
    p1disp = expand(p1f1*p1f2*p1f3*p1f4)
    p1ans = (p1f1).mul(p1f2,p1f3,p1f4,hold=true)


##  Grouping Method
if p1variant == 3:
    p1c1 = RandInt(-10,10)
    p1c2 = RandInt(-10,10)
    p1c3 = NonZeroInt(-5,5)
    p1c4 = RandInt(1,10)
    p1f1 = p1c3*x + p1c1
    p1f2 = p1c3*x - p1c1
    p1f3 = p1c4*x + p1c2
    
    p1disp = expand(p1f1*p1f2*p1f3)
    p1ans = (p1f1).mul(p1f2,p1f3,hold=true)



##  Special Forms
if p1variant == 4:
    p1c2 = NonZeroInt(-5,5)
    p1c1 = NonZeroInt(-10,10,[0,p1c2,-p1c2,p1c2^2,-p1c2^2,-p1c2^3,p1c2^3])
    
    p1f1 = (p1c2*x - p1c1)
    p1tog = RandInt(0,1)
    if p1tog == 0:
        p1f2 = (p1c2*x + p1c1)
    else:
        p1f2 = ((p1c2*x)^2 + p1c2*p1c1*x + p1c1^2)
    
    p1ans = p1f1.mul(p1f2,hold=true)
    p1disp = expand(p1f1*p1f2)





######  Problem p2

p2variant = RandInt(0,4)

##  Coefficient Method
if p2variant == 0:
    p2c1 = RandInt(-10,10)
    p2c2 = RandInt(-10,10)
    p2f1 = x+p2c1
    p2f2 = x+p2c2
    p2disp = expand(p2f1*p2f2)
    p2ans = (p2f1).mul(p2f2,hold=true)


##  AC Method 
if p2variant == 1:
    p2c1 = RandInt(-10,10)
    p2c2 = RandInt(-10,10)
    p2c3 = NonZeroInt(-5,5)
    p2c4 = RandInt(1,10)
    p2f1 = p2c3*x + p2c1
    p2f2 = p2c4*x + p2c2
    
    p2disp = expand(p2f1*p2f2)
    p2ans = (p2f1).mul(p2f2, hold=true)



##  Factoring with Rational Root Theorem 1
if p2variant == 2:
    p2ch1 = RandInt(0,3)
    p2ch2 = RandInt(0,3)
    p2ch3 = NonZeroInt(0,5,[p2ch1])
    p2ch4 = NonZeroInt(0,5,[p2ch2])
    p2ch5 = RandInt(0,5)
    p2ch6 = RandInt(0,5)
    
    p2c1 = primevec[p2ch1]
    p2c2 = primevec[p2ch2]
    
    p2c3 = (-1)^(RandInt(0,1))*primevec[p2ch3]
    p2c4 = (-1)^(RandInt(0,1))*primevec[p2ch4]
    p2c5 = (-1)^(RandInt(0,1))*primevec[p2ch5]
    p2c6 = (-1)^(RandInt(0,1))*primevec[p2ch6]
    
    p2c7 = p2c3*p2c4*p2c5*p2c6
    p2c8 = p2c1*p2c2
    
    while abs(p2c7) > 600 or abs(p2c8) > 15:
        p2ch1 = RandInt(0,3)
        p2ch2 = RandInt(0,3)
        p2ch3 = NonZeroInt(0,5,[p2ch1])
        p2ch4 = NonZeroInt(0,5,[p2ch2])
        p2ch5 = RandInt(0,5)
        
        p2c1 = primevec[p2ch1]
        p2c2 = primevec[p2ch2]
        
        p2c3 = (-1)^(RandInt(0,1))*primevec[p2ch3]
        p2c4 = (-1)^(RandInt(0,1))*primevec[p2ch4]
        p2c5 = (-1)^(RandInt(0,1))*primevec[p2ch5]
        p2c6 = (-1)^(RandInt(0,1))*primevec[p2ch6]
        
        p2c7 = p2c3*p2c4*p2c5*p2c6
        p2c8 = p2c1*p2c2
    
    
    p2f1 = (p2c1*x-p2c3)
    p2f2 = (p2c2*x-p2c4)
    p2f3 = (x-p2c5)
    p2f4 = (x-p2c6)
    p2disp = expand(p2f1*p2f2*p2f3*p2f4)
    p2ans = (p2f1).mul(p2f2,p2f3,p2f4,hold=true)



##  Grouping Method
if p2variant == 3:
    p2c1 = RandInt(-10,10)
    p2c2 = RandInt(-10,10)
    p2c3 = NonZeroInt(-5,5)
    p2c4 = RandInt(1,10)
    p2f1 = p2c3*x + p2c1
    p2f2 = p2c3*x - p2c1
    p2f3 = p2c4*x + p2c2
    
    p2disp = expand(p2f1*p2f2*p2f3)
    p2ans = (p2f1).mul(p2f2,p2f3,hold=true)



##  Special Forms
if p2variant == 4:
    p2c2 = NonZeroInt(-5,5)
    p2c1 = NonZeroInt(-10,10,[0,p2c2,-p2c2,p2c2^2,-p2c2^2,-p2c2^3,p2c2^3])
    
    p2f1 = (p2c2*x - p2c1)
    p2tog = RandInt(0,1)
    if p2tog == 0:
        p2f2 = (p2c2*x + p2c1)
    else:
        p2f2 = ((p2c2*x)^2 + p2c2*p2c1*x + p2c1^2)
    
    p2ans = p2f1.mul(p2f2,hold=true)
    p2disp = expand(p2f1*p2f2)





######  Problem p3

p3variant = RandInt(0,4)

##  Coefficient Method
if p3variant == 0:
    p3c1 = RandInt(-10,10)
    p3c2 = RandInt(-10,10)
    p3f1 = x+p3c1
    p3f2 = x+p3c2
    p3disp = expand(p3f1*p3f2)
    p3ans = (p3f1).mul(p3f2,hold=true)


##  AC Method 
if p3variant == 1:
    p3c1 = RandInt(-10,10)
    p3c2 = RandInt(-10,10)
    p3c3 = NonZeroInt(-5,5)
    p3c4 = RandInt(1,10)
    p3f1 = p3c3*x + p3c1
    p3f2 = p3c4*x + p3c2
    
    p3disp = expand(p3f1*p3f2)
    p3ans = (p3f1).mul(p3f2, hold=true)



##  Factoring with Rational Root Theorem 1
if p3variant == 2:
    p3ch1 = RandInt(0,3)
    p3ch2 = RandInt(0,3)
    p3ch3 = NonZeroInt(0,5,[p3ch1])
    p3ch4 = NonZeroInt(0,5,[p3ch2])
    p3ch5 = RandInt(0,5)
    p3ch6 = RandInt(0,5)
    
    p3c1 = primevec[p3ch1]
    p3c2 = primevec[p3ch2]
    
    p3c3 = (-1)^(RandInt(0,1))*primevec[p3ch3]
    p3c4 = (-1)^(RandInt(0,1))*primevec[p3ch4]
    p3c5 = (-1)^(RandInt(0,1))*primevec[p3ch5]
    p3c6 = (-1)^(RandInt(0,1))*primevec[p3ch6]
    
    p3c7 = p3c3*p3c4*p3c5*p3c6
    p3c8 = p3c1*p3c2
    
    while abs(p3c7) > 600 or abs(p3c8) > 15:
        p3ch1 = RandInt(0,3)
        p3ch2 = RandInt(0,3)
        p3ch3 = NonZeroInt(0,5,[p3ch1])
        p3ch4 = NonZeroInt(0,5,[p3ch2])
        p3ch5 = RandInt(0,5)
        
        p3c1 = primevec[p3ch1]
        p3c2 = primevec[p3ch2]
        
        p3c3 = (-1)^(RandInt(0,1))*primevec[p3ch3]
        p3c4 = (-1)^(RandInt(0,1))*primevec[p3ch4]
        p3c5 = (-1)^(RandInt(0,1))*primevec[p3ch5]
        p3c6 = (-1)^(RandInt(0,1))*primevec[p3ch6]
        
        p3c7 = p3c3*p3c4*p3c5*p3c6
        p3c8 = p3c1*p3c2
    
    
    p3f1 = (p3c1*x-p3c3)
    p3f2 = (p3c2*x-p3c4)
    p3f3 = (x-p3c5)
    p3f4 = (x-p3c6)
    p3disp = expand(p3f1*p3f2*p3f3*p3f4)
    p3ans = (p3f1).mul(p3f2,p3f3,p3f4,hold=true)



##  Grouping Method
if p3variant == 3:
    p3c1 = RandInt(-10,10)
    p3c2 = RandInt(-10,10)
    p3c3 = NonZeroInt(-5,5)
    p3c4 = RandInt(1,10)
    p3f1 = p3c3*x + p3c1
    p3f2 = p3c3*x - p3c1
    p3f3 = p3c4*x + p3c2
    
    p3disp = expand(p3f1*p3f2*p3f3)
    p3ans = (p3f1).mul(p3f2,p3f3,hold=true)



##  Special Forms
if p3variant == 4:
    p3c2 = NonZeroInt(-5,5)
    p3c1 = NonZeroInt(-10,10,[0,p3c2,-p3c2,p3c2^2,-p3c2^2,-p3c2^3,p3c2^3])
    
    p3f1 = (p3c2*x - p3c1)
    p3tog = RandInt(0,1)
    if p3tog == 0:
        p3f2 = (p3c2*x + p3c1)
    else:
        p3f2 = ((p3c2*x)^2 + p3c2*p3c1*x + p3c1^2)
    
    p3ans = p3f1.mul(p3f2,hold=true)
    p3disp = expand(p3f1*p3f2)





######  Problem p4

p4variant = RandInt(0,4)

##  Coefficient Method
if p4variant == 0:
    p4c1 = RandInt(-10,10)
    p4c2 = RandInt(-10,10)
    p4f1 = x+p4c1
    p4f2 = x+p4c2
    p4disp = expand(p4f1*p4f2)
    p4ans = (p4f1).mul(p4f2,hold=true)


##  AC Method 
if p4variant == 1:
    p4c1 = RandInt(-10,10)
    p4c2 = RandInt(-10,10)
    p4c3 = NonZeroInt(-5,5)
    p4c4 = RandInt(1,10)
    p4f1 = p4c3*x + p4c1
    p4f2 = p4c4*x + p4c2
    
    p4disp = expand(p4f1*p4f2)
    p4ans = (p4f1).mul(p4f2, hold=true)



##  Factoring with Rational Root Theorem 1
if p4variant == 2:
    p4ch1 = RandInt(0,3)
    p4ch2 = RandInt(0,3)
    p4ch3 = NonZeroInt(0,5,[p4ch1])
    p4ch4 = NonZeroInt(0,5,[p4ch2])
    p4ch5 = RandInt(0,5)
    p4ch6 = RandInt(0,5)
    
    p4c1 = primevec[p4ch1]
    p4c2 = primevec[p4ch2]
    
    p4c3 = (-1)^(RandInt(0,1))*primevec[p4ch3]
    p4c4 = (-1)^(RandInt(0,1))*primevec[p4ch4]
    p4c5 = (-1)^(RandInt(0,1))*primevec[p4ch5]
    p4c6 = (-1)^(RandInt(0,1))*primevec[p4ch6]
    
    p4c7 = p4c3*p4c4*p4c5*p4c6
    p4c8 = p4c1*p4c2
    
    while abs(p4c7) > 600 or abs(p4c8) > 15:
        p4ch1 = RandInt(0,3)
        p4ch2 = RandInt(0,3)
        p4ch3 = NonZeroInt(0,5,[p4ch1])
        p4ch4 = NonZeroInt(0,5,[p4ch2])
        p4ch5 = RandInt(0,5)
        
        p4c1 = primevec[p4ch1]
        p4c2 = primevec[p4ch2]
        
        p4c3 = (-1)^(RandInt(0,1))*primevec[p4ch3]
        p4c4 = (-1)^(RandInt(0,1))*primevec[p4ch4]
        p4c5 = (-1)^(RandInt(0,1))*primevec[p4ch5]
        p4c6 = (-1)^(RandInt(0,1))*primevec[p4ch6]
        
        p4c7 = p4c3*p4c4*p4c5*p4c6
        p4c8 = p4c1*p4c2
    
    
    p4f1 = (p4c1*x-p4c3)
    p4f2 = (p4c2*x-p4c4)
    p4f3 = (x-p4c5)
    p4f4 = (x-p4c6)
    p4disp = expand(p4f1*p4f2*p4f3*p4f4)
    p4ans = (p4f1).mul(p4f2,p4f3,p4f4,hold=true)



##  Grouping Method
if p4variant == 3:
    p4c1 = RandInt(-10,10)
    p4c2 = RandInt(-10,10)
    p4c3 = NonZeroInt(-5,5)
    p4c4 = RandInt(1,10)
    p4f1 = p4c3*x + p4c1
    p4f2 = p4c3*x - p4c1
    p4f3 = p4c4*x + p4c2
    
    p4disp = expand(p4f1*p4f2*p4f3)
    p4ans = (p4f1).mul(p4f2,p4f3,hold=true)



##  Special Forms
if p4variant == 4:
    p4c2 = NonZeroInt(-5,5)
    p4c1 = NonZeroInt(-10,10,[0,p4c2,-p4c2,p4c2^2,-p4c2^2,-p4c2^3,p4c2^3])
    
    p4f1 = (p4c2*x - p4c1)
    p4tog = RandInt(0,1)
    if p4tog == 0:
        p4f2 = (p4c2*x + p4c1)
    else:
        p4f2 = ((p4c2*x)^2 + p4c2*p4c1*x + p4c1^2)
    
    p4ans = p4f1.mul(p4f2,hold=true)
    p4disp = expand(p4f1*p4f2)





######  Problem p5

p5variant = RandInt(0,4)

##  Coefficient Method
if p5variant == 0:
    p5c1 = RandInt(-10,10)
    p5c2 = RandInt(-10,10)
    p5f1 = x+p5c1
    p5f2 = x+p5c2
    p5disp = expand(p5f1*p5f2)
    p5ans = (p5f1).mul(p5f2,hold=true)



##  AC Method 
if p5variant == 1:
    p5c1 = RandInt(-10,10)
    p5c2 = RandInt(-10,10)
    p5c3 = NonZeroInt(-5,5)
    p5c4 = RandInt(1,10)
    p5f1 = p5c3*x + p5c1
    p5f2 = p5c4*x + p5c2
    
    p5disp = expand(p5f1*p5f2)
    p5ans = (p5f1).mul(p5f2, hold=true)

##  Factoring with Rational Root Theorem 1
if p5variant == 2:
    p5ch1 = RandInt(0,3)
    p5ch2 = RandInt(0,3)
    p5ch3 = NonZeroInt(0,5,[p5ch1])
    p5ch4 = NonZeroInt(0,5,[p5ch2])
    p5ch5 = RandInt(0,5)
    p5ch6 = RandInt(0,5)
    
    p5c1 = primevec[p5ch1]
    p5c2 = primevec[p5ch2]
    
    p5c3 = (-1)^(RandInt(0,1))*primevec[p5ch3]
    p5c4 = (-1)^(RandInt(0,1))*primevec[p5ch4]
    p5c5 = (-1)^(RandInt(0,1))*primevec[p5ch5]
    p5c6 = (-1)^(RandInt(0,1))*primevec[p5ch6]
    
    p5c7 = p5c3*p5c4*p5c5*p5c6
    p5c8 = p5c1*p5c2
    
    while abs(p5c7) > 600 or abs(p5c8) > 15:
        p5ch1 = RandInt(0,3)
        p5ch2 = RandInt(0,3)
        p5ch3 = NonZeroInt(0,5,[p5ch1])
        p5ch4 = NonZeroInt(0,5,[p5ch2])
        p5ch5 = RandInt(0,5)
        
        p5c1 = primevec[p5ch1]
        p5c2 = primevec[p5ch2]
        
        p5c3 = (-1)^(RandInt(0,1))*primevec[p5ch3]
        p5c4 = (-1)^(RandInt(0,1))*primevec[p5ch4]
        p5c5 = (-1)^(RandInt(0,1))*primevec[p5ch5]
        p5c6 = (-1)^(RandInt(0,1))*primevec[p5ch6]
        
        p5c7 = p5c3*p5c4*p5c5*p5c6
        p5c8 = p5c1*p5c2
    
    
    p5f1 = (p5c1*x-p5c3)
    p5f2 = (p5c2*x-p5c4)
    p5f3 = (x-p5c5)
    p5f4 = (x-p5c6)
    p5disp = expand(p5f1*p5f2*p5f3*p5f4)
    p5ans = (p5f1).mul(p5f2,p5f3,p5f4,hold=true)



##  Grouping Method
if p5variant == 3:
    p5c1 = RandInt(-10,10)
    p5c2 = RandInt(-10,10)
    p5c3 = NonZeroInt(-5,5)
    p5c4 = RandInt(1,10)
    p5f1 = p5c3*x + p5c1
    p5f2 = p5c3*x - p5c1
    p5f3 = p5c4*x + p5c2
    
    p5disp = expand(p5f1*p5f2*p5f3)
    p5ans = (p5f1).mul(p5f2,p5f3,hold=true)



##  Special Forms
if p5variant == 4:
    p5c2 = NonZeroInt(-5,5)
    p5c1 = NonZeroInt(-10,10,[0,p5c2,-p5c2,p5c2^2,-p5c2^2,-p5c2^3,p5c2^3])
    
    p5f1 = (p5c2*x - p5c1)
    p5tog = RandInt(0,1)
    if p5tog == 0:
        p5f2 = (p5c2*x + p5c1)
    else:
        p5f2 = ((p5c2*x)^2 + p5c2*p5c1*x + p5c1^2)
    
    p5ans = p5f1.mul(p5f2,hold=true)
    p5disp = expand(p5f1*p5f2)





######  Problem p6

p6variant = RandInt(0,4)

##  Coefficient Method
if p6variant == 0:
    p6c1 = RandInt(-10,10)
    p6c2 = RandInt(-10,10)
    p6f1 = x+p6c1
    p6f2 = x+p6c2
    p6disp = expand(p6f1*p6f2)
    p6ans = (p6f1).mul(p6f2,hold=true)


##  AC Method 
if p6variant == 1:
    p6c1 = RandInt(-10,10)
    p6c2 = RandInt(-10,10)
    p6c3 = NonZeroInt(-5,5)
    p6c4 = RandInt(1,10)
    p6f1 = p6c3*x + p6c1
    p6f2 = p6c4*x + p6c2
    
    p6disp = expand(p6f1*p6f2)
    p6ans = (p6f1).mul(p6f2, hold=true)



##  Factoring with Rational Root Theorem 1
if p6variant == 2:
    p6ch1 = RandInt(0,3)
    p6ch2 = RandInt(0,3)
    p6ch3 = NonZeroInt(0,5,[p6ch1])
    p6ch4 = NonZeroInt(0,5,[p6ch2])
    p6ch5 = RandInt(0,5)
    p6ch6 = RandInt(0,5)
    
    p6c1 = primevec[p6ch1]
    p6c2 = primevec[p6ch2]
    
    p6c3 = (-1)^(RandInt(0,1))*primevec[p6ch3]
    p6c4 = (-1)^(RandInt(0,1))*primevec[p6ch4]
    p6c5 = (-1)^(RandInt(0,1))*primevec[p6ch5]
    p6c6 = (-1)^(RandInt(0,1))*primevec[p6ch6]
    
    p6c7 = p6c3*p6c4*p6c5*p6c6
    p6c8 = p6c1*p6c2
    
    while abs(p6c7) > 600 or abs(p6c8) > 15:
        p6ch1 = RandInt(0,3)
        p6ch2 = RandInt(0,3)
        p6ch3 = NonZeroInt(0,5,[p6ch1])
        p6ch4 = NonZeroInt(0,5,[p6ch2])
        p6ch5 = RandInt(0,5)
        
        p6c1 = primevec[p6ch1]
        p6c2 = primevec[p6ch2]
        
        p6c3 = (-1)^(RandInt(0,1))*primevec[p6ch3]
        p6c4 = (-1)^(RandInt(0,1))*primevec[p6ch4]
        p6c5 = (-1)^(RandInt(0,1))*primevec[p6ch5]
        p6c6 = (-1)^(RandInt(0,1))*primevec[p6ch6]
        
        p6c7 = p6c3*p6c4*p6c5*p6c6
        p6c8 = p6c1*p6c2
    
    
    p6f1 = (p6c1*x-p6c3)
    p6f2 = (p6c2*x-p6c4)
    p6f3 = (x-p6c5)
    p6f4 = (x-p6c6)
    p6disp = expand(p6f1*p6f2*p6f3*p6f4)
    p6ans = (p6f1).mul(p6f2,p6f3,p6f4,hold=true)



##  Grouping Method
if p6variant == 3:
    p6c1 = RandInt(-10,10)
    p6c2 = RandInt(-10,10)
    p6c3 = NonZeroInt(-5,5)
    p6c4 = RandInt(1,10)
    p6f1 = p6c3*x + p6c1
    p6f2 = p6c3*x - p6c1
    p6f3 = p6c4*x + p6c2
    
    p6disp = expand(p6f1*p6f2*p6f3)
    p6ans = (p6f1).mul(p6f2,p6f3,hold=true)



##  Special Forms
if p6variant == 4:
    p6c2 = NonZeroInt(-5,5)
    p6c1 = NonZeroInt(-10,10,[0,p6c2,-p6c2,p6c2^2,-p6c2^2,-p6c2^3,p6c2^3])
    
    p6f1 = (p6c2*x - p6c1)
    p6tog = RandInt(0,1)
    if p6tog == 0:
        p6f2 = (p6c2*x + p6c1)
    else:
        p6f2 = ((p6c2*x)^2 + p6c2*p6c1*x + p6c1^2)
    
    p6ans = p6f1.mul(p6f2,hold=true)
    p6disp = expand(p6f1*p6f2)





######  Problem p7

p7variant = RandInt(0,4)

##  Coefficient Method
if p7variant == 0:
    p7c1 = RandInt(-10,10)
    p7c2 = RandInt(-10,10)
    p7f1 = x+p7c1
    p7f2 = x+p7c2
    p7disp = expand(p7f1*p7f2)
    p7ans = (p7f1).mul(p7f2,hold=true)


##  AC Method 
if p7variant == 1:
    p7c1 = RandInt(-10,10)
    p7c2 = RandInt(-10,10)
    p7c3 = NonZeroInt(-5,5)
    p7c4 = RandInt(1,10)
    p7f1 = p7c3*x + p7c1
    p7f2 = p7c4*x + p7c2
    
    p7disp = expand(p7f1*p7f2)
    p7ans = (p7f1).mul(p7f2, hold=true)



##  Factoring with Rational Root Theorem 1
if p7variant == 2:
    p7ch1 = RandInt(0,3)
    p7ch2 = RandInt(0,3)
    p7ch3 = NonZeroInt(0,5,[p7ch1])
    p7ch4 = NonZeroInt(0,5,[p7ch2])
    p7ch5 = RandInt(0,5)
    p7ch6 = RandInt(0,5)
    
    p7c1 = primevec[p7ch1]
    p7c2 = primevec[p7ch2]
    
    p7c3 = (-1)^(RandInt(0,1))*primevec[p7ch3]
    p7c4 = (-1)^(RandInt(0,1))*primevec[p7ch4]
    p7c5 = (-1)^(RandInt(0,1))*primevec[p7ch5]
    p7c6 = (-1)^(RandInt(0,1))*primevec[p7ch6]
    
    p7c7 = p7c3*p7c4*p7c5*p7c6
    p7c8 = p7c1*p7c2
    
    while abs(p7c7) > 600 or abs(p7c8) > 15:
        p7ch1 = RandInt(0,3)
        p7ch2 = RandInt(0,3)
        p7ch3 = NonZeroInt(0,5,[p7ch1])
        p7ch4 = NonZeroInt(0,5,[p7ch2])
        p7ch5 = RandInt(0,5)
        
        p7c1 = primevec[p7ch1]
        p7c2 = primevec[p7ch2]
        
        p7c3 = (-1)^(RandInt(0,1))*primevec[p7ch3]
        p7c4 = (-1)^(RandInt(0,1))*primevec[p7ch4]
        p7c5 = (-1)^(RandInt(0,1))*primevec[p7ch5]
        p7c6 = (-1)^(RandInt(0,1))*primevec[p7ch6]
        
        p7c7 = p7c3*p7c4*p7c5*p7c6
        p7c8 = p7c1*p7c2
    
    
    p7f1 = (p7c1*x-p7c3)
    p7f2 = (p7c2*x-p7c4)
    p7f3 = (x-p7c5)
    p7f4 = (x-p7c6)
    p7disp = expand(p7f1*p7f2*p7f3*p7f4)
    p7ans = (p7f1).mul(p7f2,p7f3,p7f4,hold=true)



##  Grouping Method
if p7variant == 3:
    p7c1 = RandInt(-10,10)
    p7c2 = RandInt(-10,10)
    p7c3 = NonZeroInt(-5,5)
    p7c4 = RandInt(1,10)
    p7f1 = p7c3*x + p7c1
    p7f2 = p7c3*x - p7c1
    p7f3 = p7c4*x + p7c2
    
    p7disp = expand(p7f1*p7f2*p7f3)
    p7ans = (p7f1).mul(p7f2,p7f3,hold=true)



##  Special Forms
if p7variant == 4:
    p7c2 = NonZeroInt(-5,5)
    p7c1 = NonZeroInt(-10,10,[0,p7c2,-p7c2,p7c2^2,-p7c2^2,-p7c2^3,p7c2^3])
    
    p7f1 = (p7c2*x - p7c1)
    p7tog = RandInt(0,1)
    if p7tog == 0:
        p7f2 = (p7c2*x + p7c1)
    else:
        p7f2 = ((p7c2*x)^2 + p7c2*p7c1*x + p7c1^2)
    
    p7ans = p7f1.mul(p7f2,hold=true)
    p7disp = expand(p7f1*p7f2)





######  Problem p8

p8variant = RandInt(0,4)

##  Coefficient Method
if p8variant == 0:
    p8c1 = RandInt(-10,10)
    p8c2 = RandInt(-10,10)
    p8f1 = x+p8c1
    p8f2 = x+p8c2
    p8disp = expand(p8f1*p8f2)
    p8ans = (p8f1).mul(p8f2,hold=true)


##  AC Method 
if p8variant == 1:
    p8c1 = RandInt(-10,10)
    p8c2 = RandInt(-10,10)
    p8c3 = NonZeroInt(-5,5)
    p8c4 = RandInt(1,10)
    p8f1 = p8c3*x + p8c1
    p8f2 = p8c4*x + p8c2
    
    p8disp = expand(p8f1*p8f2)
    p8ans = (p8f1).mul(p8f2, hold=true)



##  Factoring with Rational Root Theorem 1
if p8variant == 2:
    p8ch1 = RandInt(0,3)
    p8ch2 = RandInt(0,3)
    p8ch3 = NonZeroInt(0,5,[p8ch1])
    p8ch4 = NonZeroInt(0,5,[p8ch2])
    p8ch5 = RandInt(0,5)
    p8ch6 = RandInt(0,5)
    
    p8c1 = primevec[p8ch1]
    p8c2 = primevec[p8ch2]
    
    p8c3 = (-1)^(RandInt(0,1))*primevec[p8ch3]
    p8c4 = (-1)^(RandInt(0,1))*primevec[p8ch4]
    p8c5 = (-1)^(RandInt(0,1))*primevec[p8ch5]
    p8c6 = (-1)^(RandInt(0,1))*primevec[p8ch6]
    
    p8c7 = p8c3*p8c4*p8c5*p8c6
    p8c8 = p8c1*p8c2
    
    while abs(p8c7) > 600 or abs(p8c8) > 15:
        p8ch1 = RandInt(0,3)
        p8ch2 = RandInt(0,3)
        p8ch3 = NonZeroInt(0,5,[p8ch1])
        p8ch4 = NonZeroInt(0,5,[p8ch2])
        p8ch5 = RandInt(0,5)
        
        p8c1 = primevec[p8ch1]
        p8c2 = primevec[p8ch2]
        
        p8c3 = (-1)^(RandInt(0,1))*primevec[p8ch3]
        p8c4 = (-1)^(RandInt(0,1))*primevec[p8ch4]
        p8c5 = (-1)^(RandInt(0,1))*primevec[p8ch5]
        p8c6 = (-1)^(RandInt(0,1))*primevec[p8ch6]
        
        p8c7 = p8c3*p8c4*p8c5*p8c6
        p8c8 = p8c1*p8c2
    
    
    p8f1 = (p8c1*x-p8c3)
    p8f2 = (p8c2*x-p8c4)
    p8f3 = (x-p8c5)
    p8f4 = (x-p8c6)
    p8disp = expand(p8f1*p8f2*p8f3*p8f4)
    p8ans = (p8f1).mul(p8f2,p8f3,p8f4,hold=true)



##  Grouping Method
if p8variant == 3:
    p8c1 = RandInt(-10,10)
    p8c2 = RandInt(-10,10)
    p8c3 = NonZeroInt(-5,5)
    p8c4 = RandInt(1,10)
    p8f1 = p8c3*x + p8c1
    p8f2 = p8c3*x - p8c1
    p8f3 = p8c4*x + p8c2
    
    p8disp = expand(p8f1*p8f2*p8f3)
    p8ans = (p8f1).mul(p8f2,p8f3,hold=true)



##  Special Forms
if p8variant == 4:
    p8c2 = NonZeroInt(-5,5)
    p8c1 = NonZeroInt(-10,10,[0,p8c2,-p8c2,p8c2^2,-p8c2^2,-p8c2^3,p8c2^3])
    
    p8f1 = (p8c2*x - p8c1)
    p8tog = RandInt(0,1)
    if p8tog == 0:
        p8f2 = (p8c2*x + p8c1)
    else:
        p8f2 = ((p8c2*x)^2 + p8c2*p8c1*x + p8c1^2)
    
    p8ans = p8f1.mul(p8f2,hold=true)
    p8disp = expand(p8f1*p8f2)





######  Problem p9

p9variant = RandInt(0,4)

##  Coefficient Method
if p9variant == 0:
    p9c1 = RandInt(-10,10)
    p9c2 = RandInt(-10,10)
    p9f1 = x+p9c1
    p9f2 = x+p9c2
    p9disp = expand(p9f1*p9f2)
    p9ans = (p9f1).mul(p9f2,hold=true)


##  AC Method 
if p9variant == 1:
    p9c1 = RandInt(-10,10)
    p9c2 = RandInt(-10,10)
    p9c3 = NonZeroInt(-5,5)
    p9c4 = RandInt(1,10)
    p9f1 = p9c3*x + p9c1
    p9f2 = p9c4*x + p9c2
    
    p9disp = expand(p9f1*p9f2)
    p9ans = (p9f1).mul(p9f2, hold=true)



##  Factoring with Rational Root Theorem 1
if p9variant == 2:
    p9ch1 = RandInt(0,3)
    p9ch2 = RandInt(0,3)
    p9ch3 = NonZeroInt(0,5,[p9ch1])
    p9ch4 = NonZeroInt(0,5,[p9ch2])
    p9ch5 = RandInt(0,5)
    p9ch6 = RandInt(0,5)
    
    p9c1 = primevec[p9ch1]
    p9c2 = primevec[p9ch2]
    
    p9c3 = (-1)^(RandInt(0,1))*primevec[p9ch3]
    p9c4 = (-1)^(RandInt(0,1))*primevec[p9ch4]
    p9c5 = (-1)^(RandInt(0,1))*primevec[p9ch5]
    p9c6 = (-1)^(RandInt(0,1))*primevec[p9ch6]
    
    p9c7 = p9c3*p9c4*p9c5*p9c6
    p9c8 = p9c1*p9c2
    
    while abs(p9c7) > 600 or abs(p9c8) > 15:
        p9ch1 = RandInt(0,3)
        p9ch2 = RandInt(0,3)
        p9ch3 = NonZeroInt(0,5,[p9ch1])
        p9ch4 = NonZeroInt(0,5,[p9ch2])
        p9ch5 = RandInt(0,5)
        
        p9c1 = primevec[p9ch1]
        p9c2 = primevec[p9ch2]
        
        p9c3 = (-1)^(RandInt(0,1))*primevec[p9ch3]
        p9c4 = (-1)^(RandInt(0,1))*primevec[p9ch4]
        p9c5 = (-1)^(RandInt(0,1))*primevec[p9ch5]
        p9c6 = (-1)^(RandInt(0,1))*primevec[p9ch6]
        
        p9c7 = p9c3*p9c4*p9c5*p9c6
        p9c8 = p9c1*p9c2
    
    
    p9f1 = (p9c1*x-p9c3)
    p9f2 = (p9c2*x-p9c4)
    p9f3 = (x-p9c5)
    p9f4 = (x-p9c6)
    p9disp = expand(p9f1*p9f2*p9f3*p9f4)
    p9ans = (p9f1).mul(p9f2,p9f3,p9f4,hold=true)



##  Grouping Method
if p9variant == 3:
    p9c1 = RandInt(-10,10)
    p9c2 = RandInt(-10,10)
    p9c3 = NonZeroInt(-5,5)
    p9c4 = RandInt(1,10)
    p9f1 = p9c3*x + p9c1
    p9f2 = p9c3*x - p9c1
    p9f3 = p9c4*x + p9c2
    
    p9disp = expand(p9f1*p9f2*p9f3)
    p9ans = (p9f1).mul(p9f2,p9f3,hold=true)



##  Special Forms
if p9variant == 4:
    p9c2 = NonZeroInt(-5,5)
    p9c1 = NonZeroInt(-10,10,[0,p9c2,-p9c2,p9c2^2,-p9c2^2,-p9c2^3,p9c2^3])
    
    p9f1 = (p9c2*x - p9c1)
    p9tog = RandInt(0,1)
    if p9tog == 0:
        p9f2 = (p9c2*x + p9c1)
    else:
        p9f2 = ((p9c2*x)^2 + p9c2*p9c1*x + p9c1^2)
    
    p9ans = p9f1.mul(p9f2,hold=true)
    p9disp = expand(p9f1*p9f2)





######  Problem p10

p10variant = RandInt(0,4)

##  Coefficient Method
if p10variant == 0:
    p10c1 = RandInt(-10,10)
    p10c2 = RandInt(-10,10)
    p10f1 = x+p10c1
    p10f2 = x+p10c2
    p10disp = expand(p10f1*p10f2)
    p10ans = (p10f1).mul(p10f2,hold=true)


##  AC Method 
if p10variant == 1:
    p10c1 = RandInt(-10,10)
    p10c2 = RandInt(-10,10)
    p10c3 = NonZeroInt(-5,5)
    p10c4 = RandInt(1,10)
    p10f1 = p10c3*x + p10c1
    p10f2 = p10c4*x + p10c2
    
    p10disp = expand(p10f1*p10f2)
    p10ans = (p10f1).mul(p10f2, hold=true)



##  Factoring with Rational Root Theorem 1
if p10variant == 2:
    p10ch1 = RandInt(0,3)
    p10ch2 = RandInt(0,3)
    p10ch3 = NonZeroInt(0,5,[p10ch1])
    p10ch4 = NonZeroInt(0,5,[p10ch2])
    p10ch5 = RandInt(0,5)
    p10ch6 = RandInt(0,5)
    
    p10c1 = primevec[p10ch1]
    p10c2 = primevec[p10ch2]
    
    p10c3 = (-1)^(RandInt(0,1))*primevec[p10ch3]
    p10c4 = (-1)^(RandInt(0,1))*primevec[p10ch4]
    p10c5 = (-1)^(RandInt(0,1))*primevec[p10ch5]
    p10c6 = (-1)^(RandInt(0,1))*primevec[p10ch6]
    
    p10c7 = p10c3*p10c4*p10c5*p10c6
    p10c8 = p10c1*p10c2
    
    while abs(p10c7) > 600 or abs(p10c8) > 15:
        p10ch1 = RandInt(0,3)
        p10ch2 = RandInt(0,3)
        p10ch3 = NonZeroInt(0,5,[p10ch1])
        p10ch4 = NonZeroInt(0,5,[p10ch2])
        p10ch5 = RandInt(0,5)
        
        p10c1 = primevec[p10ch1]
        p10c2 = primevec[p10ch2]
        
        p10c3 = (-1)^(RandInt(0,1))*primevec[p10ch3]
        p10c4 = (-1)^(RandInt(0,1))*primevec[p10ch4]
        p10c5 = (-1)^(RandInt(0,1))*primevec[p10ch5]
        p10c6 = (-1)^(RandInt(0,1))*primevec[p10ch6]
        
        p10c7 = p10c3*p10c4*p10c5*p10c6
        p10c8 = p10c1*p10c2
    
    
    p10f1 = (p10c1*x-p10c3)
    p10f2 = (p10c2*x-p10c4)
    p10f3 = (x-p10c5)
    p10f4 = (x-p10c6)
    p10disp = expand(p10f1*p10f2*p10f3*p10f4)
    p10ans = (p10f1).mul(p10f2,p10f3,p10f4,hold=true)


##  Grouping Method
if p10variant == 3:
    p10c1 = RandInt(-10,10)
    p10c2 = RandInt(-10,10)
    p10c3 = NonZeroInt(-5,5)
    p10c4 = RandInt(1,10)
    p10f1 = p10c3*x + p10c1
    p10f2 = p10c3*x - p10c1
    p10f3 = p10c4*x + p10c2
    
    p10disp = expand(p10f1*p10f2*p10f3)
    p10ans = (p10f1).mul(p10f2,p10f3,hold=true)



##  Special Forms
if p10variant == 4:
    p10c2 = NonZeroInt(-5,5)
    p10c1 = NonZeroInt(-10,10,[0,p10c2,-p10c2,p10c2^2,-p10c2^2,-p10c2^3,p10c2^3])
    
    p10f1 = (p10c2*x - p10c1)
    p10tog = RandInt(0,1)
    if p10tog == 0:
        p10f2 = (p10c2*x + p10c1)
    else:
        p10f2 = ((p10c2*x)^2 + p10c2*p10c1*x + p10c1^2)
    
    p10ans = p10f1.mul(p10f2,hold=true)
    p10disp = expand(p10f1*p10f2)





######  Problem p11

p11variant = RandInt(0,4)

##  Coefficient Method
if p11variant == 0:
    p11c1 = RandInt(-10,10)
    p11c2 = RandInt(-10,10)
    p11f1 = x+p11c1
    p11f2 = x+p11c2
    p11disp = expand(p11f1*p11f2)
    p11ans = (p11f1).mul(p11f2,hold=true)


##  AC Method 
if p11variant == 1:
    p11c1 = RandInt(-10,10)
    p11c2 = RandInt(-10,10)
    p11c3 = NonZeroInt(-5,5)
    p11c4 = RandInt(1,10)
    p11f1 = p11c3*x + p11c1
    p11f2 = p11c4*x + p11c2
    
    p11disp = expand(p11f1*p11f2)
    p11ans = (p11f1).mul(p11f2, hold=true)



##  Factoring with Rational Root Theorem 1
if p11variant == 2:
    p11ch1 = RandInt(0,3)
    p11ch2 = RandInt(0,3)
    p11ch3 = NonZeroInt(0,5,[p11ch1])
    p11ch4 = NonZeroInt(0,5,[p11ch2])
    p11ch5 = RandInt(0,5)
    p11ch6 = RandInt(0,5)
    
    p11c1 = primevec[p11ch1]
    p11c2 = primevec[p11ch2]
    
    p11c3 = (-1)^(RandInt(0,1))*primevec[p11ch3]
    p11c4 = (-1)^(RandInt(0,1))*primevec[p11ch4]
    p11c5 = (-1)^(RandInt(0,1))*primevec[p11ch5]
    p11c6 = (-1)^(RandInt(0,1))*primevec[p11ch6]
    
    p11c7 = p11c3*p11c4*p11c5*p11c6
    p11c8 = p11c1*p11c2
    
    while abs(p11c7) > 600 or abs(p11c8) > 15:
        p11ch1 = RandInt(0,3)
        p11ch2 = RandInt(0,3)
        p11ch3 = NonZeroInt(0,5,[p11ch1])
        p11ch4 = NonZeroInt(0,5,[p11ch2])
        p11ch5 = RandInt(0,5)
        
        p11c1 = primevec[p11ch1]
        p11c2 = primevec[p11ch2]
        
        p11c3 = (-1)^(RandInt(0,1))*primevec[p11ch3]
        p11c4 = (-1)^(RandInt(0,1))*primevec[p11ch4]
        p11c5 = (-1)^(RandInt(0,1))*primevec[p11ch5]
        p11c6 = (-1)^(RandInt(0,1))*primevec[p11ch6]
        
        p11c7 = p11c3*p11c4*p11c5*p11c6
        p11c8 = p11c1*p11c2
    
    
    p11f1 = (p11c1*x-p11c3)
    p11f2 = (p11c2*x-p11c4)
    p11f3 = (x-p11c5)
    p11f4 = (x-p11c6)
    p11disp = expand(p11f1*p11f2*p11f3*p11f4)
    p11ans = (p11f1).mul(p11f2,p11f3,p11f4,hold=true)


##  Grouping Method
if p11variant == 3:
    p11c1 = RandInt(-10,10)
    p11c2 = RandInt(-10,10)
    p11c3 = NonZeroInt(-5,5)
    p11c4 = RandInt(1,10)
    p11f1 = p11c3*x + p11c1
    p11f2 = p11c3*x - p11c1
    p11f3 = p11c4*x + p11c2
    
    p11disp = expand(p11f1*p11f2*p11f3)
    p11ans = (p11f1).mul(p11f2,p11f3,hold=true)



##  Special Forms
if p11variant == 4:
    p11c2 = NonZeroInt(-5,5)
    p11c1 = NonZeroInt(-10,10,[0,p11c2,-p11c2,p11c2^2,-p11c2^2,-p11c2^3,p11c2^3])
    
    p11f1 = (p11c2*x - p11c1)
    p11tog = RandInt(0,1)
    if p11tog == 0:
        p11f2 = (p11c2*x + p11c1)
    else:
        p11f2 = ((p11c2*x)^2 + p11c2*p11c1*x + p11c1^2)
    
    p11ans = p11f1.mul(p11f2,hold=true)
    p11disp = expand(p11f1*p11f2)





######  Problem p12

p12variant = RandInt(0,4)

##  Coefficient Method
if p12variant == 0:
    p12c1 = RandInt(-10,10)
    p12c2 = RandInt(-10,10)
    p12f1 = x+p12c1
    p12f2 = x+p12c2
    p12disp = expand(p12f1*p12f2)
    p12ans = (p12f1).mul(p12f2,hold=true)


##  AC Method 
if p12variant == 1:
    p12c1 = RandInt(-10,10)
    p12c2 = RandInt(-10,10)
    p12c3 = NonZeroInt(-5,5)
    p12c4 = RandInt(1,10)
    p12f1 = p12c3*x + p12c1
    p12f2 = p12c4*x + p12c2
    
    p12disp = expand(p12f1*p12f2)
    p12ans = (p12f1).mul(p12f2, hold=true)



##  Factoring with Rational Root Theorem 1
if p12variant == 2:
    p12ch1 = RandInt(0,3)
    p12ch2 = RandInt(0,3)
    p12ch3 = NonZeroInt(0,5,[p12ch1])
    p12ch4 = NonZeroInt(0,5,[p12ch2])
    p12ch5 = RandInt(0,5)
    p12ch6 = RandInt(0,5)
    
    p12c1 = primevec[p12ch1]
    p12c2 = primevec[p12ch2]
    
    p12c3 = (-1)^(RandInt(0,1))*primevec[p12ch3]
    p12c4 = (-1)^(RandInt(0,1))*primevec[p12ch4]
    p12c5 = (-1)^(RandInt(0,1))*primevec[p12ch5]
    p12c6 = (-1)^(RandInt(0,1))*primevec[p12ch6]
    
    p12c7 = p12c3*p12c4*p12c5*p12c6
    p12c8 = p12c1*p12c2
    
    while abs(p12c7) > 600 or abs(p12c8) > 15:
        p12ch1 = RandInt(0,3)
        p12ch2 = RandInt(0,3)
        p12ch3 = NonZeroInt(0,5,[p12ch1])
        p12ch4 = NonZeroInt(0,5,[p12ch2])
        p12ch5 = RandInt(0,5)
        
        p12c1 = primevec[p12ch1]
        p12c2 = primevec[p12ch2]
        
        p12c3 = (-1)^(RandInt(0,1))*primevec[p12ch3]
        p12c4 = (-1)^(RandInt(0,1))*primevec[p12ch4]
        p12c5 = (-1)^(RandInt(0,1))*primevec[p12ch5]
        p12c6 = (-1)^(RandInt(0,1))*primevec[p12ch6]
        
        p12c7 = p12c3*p12c4*p12c5*p12c6
        p12c8 = p12c1*p12c2
    
    
    p12f1 = (p12c1*x-p12c3)
    p12f2 = (p12c2*x-p12c4)
    p12f3 = (x-p12c5)
    p12f4 = (x-p12c6)
    p12disp = expand(p12f1*p12f2*p12f3*p12f4)
    p12ans = (p12f1).mul(p12f2,p12f3,p12f4,hold=true)



##  Grouping Method
if p12variant == 3:
    p12c1 = RandInt(-10,10)
    p12c2 = RandInt(-10,10)
    p12c3 = NonZeroInt(-5,5)
    p12c4 = RandInt(1,10)
    p12f1 = p12c3*x + p12c1
    p12f2 = p12c3*x - p12c1
    p12f3 = p12c4*x + p12c2
    
    p12disp = expand(p12f1*p12f2*p12f3)
    p12ans = (p12f1).mul(p12f2,p12f3,hold=true)



##  Special Forms
if p12variant == 4:
    p12c2 = NonZeroInt(-5,5)
    p12c1 = NonZeroInt(-10,10,[0,p12c2,-p12c2,p12c2^2,-p12c2^2,-p12c2^3,p12c2^3])
    
    p12f1 = (p12c2*x - p12c1)
    p12tog = RandInt(0,1)
    if p12tog == 0:
        p12f2 = (p12c2*x + p12c1)
    else:
        p12f2 = ((p12c2*x)^2 + p12c2*p12c1*x + p12c1^2)
    
    p12ans = p12f1.mul(p12f2,hold=true)
    p12disp = expand(p12f1*p12f2)





######  Problem p13

p13variant = RandInt(0,4)

##  Coefficient Method
if p13variant == 0:
    p13c1 = RandInt(-10,10)
    p13c2 = RandInt(-10,10)
    p13f1 = x+p13c1
    p13f2 = x+p13c2
    p13disp = expand(p13f1*p13f2)
    p13ans = (p13f1).mul(p13f2,hold=true)


##  AC Method 
if p13variant == 1:
    p13c1 = RandInt(-10,10)
    p13c2 = RandInt(-10,10)
    p13c3 = NonZeroInt(-5,5)
    p13c4 = RandInt(1,10)
    p13f1 = p13c3*x + p13c1
    p13f2 = p13c4*x + p13c2
    
    p13disp = expand(p13f1*p13f2)
    p13ans = (p13f1).mul(p13f2, hold=true)



##  Factoring with Rational Root Theorem 1
if p13variant == 2:
    p13ch1 = RandInt(0,3)
    p13ch2 = RandInt(0,3)
    p13ch3 = NonZeroInt(0,5,[p13ch1])
    p13ch4 = NonZeroInt(0,5,[p13ch2])
    p13ch5 = RandInt(0,5)
    p13ch6 = RandInt(0,5)
    
    p13c1 = primevec[p13ch1]
    p13c2 = primevec[p13ch2]
    
    p13c3 = (-1)^(RandInt(0,1))*primevec[p13ch3]
    p13c4 = (-1)^(RandInt(0,1))*primevec[p13ch4]
    p13c5 = (-1)^(RandInt(0,1))*primevec[p13ch5]
    p13c6 = (-1)^(RandInt(0,1))*primevec[p13ch6]
    
    p13c7 = p13c3*p13c4*p13c5*p13c6
    p13c8 = p13c1*p13c2
    
    while abs(p13c7) > 600 or abs(p13c8) > 15:
        p13ch1 = RandInt(0,3)
        p13ch2 = RandInt(0,3)
        p13ch3 = NonZeroInt(0,5,[p13ch1])
        p13ch4 = NonZeroInt(0,5,[p13ch2])
        p13ch5 = RandInt(0,5)
        
        p13c1 = primevec[p13ch1]
        p13c2 = primevec[p13ch2]
        
        p13c3 = (-1)^(RandInt(0,1))*primevec[p13ch3]
        p13c4 = (-1)^(RandInt(0,1))*primevec[p13ch4]
        p13c5 = (-1)^(RandInt(0,1))*primevec[p13ch5]
        p13c6 = (-1)^(RandInt(0,1))*primevec[p13ch6]
        
        p13c7 = p13c3*p13c4*p13c5*p13c6
        p13c8 = p13c1*p13c2
    
    
    p13f1 = (p13c1*x-p13c3)
    p13f2 = (p13c2*x-p13c4)
    p13f3 = (x-p13c5)
    p13f4 = (x-p13c6)
    p13disp = expand(p13f1*p13f2*p13f3*p13f4)
    p13ans = (p13f1).mul(p13f2,p13f3,p13f4,hold=true)



##  Grouping Method
if p13variant == 3:
    p13c1 = RandInt(-10,10)
    p13c2 = RandInt(-10,10)
    p13c3 = NonZeroInt(-5,5)
    p13c4 = RandInt(1,10)
    p13f1 = p13c3*x + p13c1
    p13f2 = p13c3*x - p13c1
    p13f3 = p13c4*x + p13c2
    
    p13disp = expand(p13f1*p13f2*p13f3)
    p13ans = (p13f1).mul(p13f2,p13f3,hold=true)



##  Special Forms
if p13variant == 4:
    p13c2 = NonZeroInt(-5,5)
    p13c1 = NonZeroInt(-10,10,[0,p13c2,-p13c2,p13c2^2,-p13c2^2,-p13c2^3,p13c2^3])
    
    p13f1 = (p13c2*x - p13c1)
    p13tog = RandInt(0,1)
    if p13tog == 0:
        p13f2 = (p13c2*x + p13c1)
    else:
        p13f2 = ((p13c2*x)^2 + p13c2*p13c1*x + p13c1^2)
    
    p13ans = p13f1.mul(p13f2,hold=true)
    p13disp = expand(p13f1*p13f2)





######  Problem p14

p14variant = RandInt(0,4)

##  Coefficient Method
if p14variant == 0:
    p14c1 = RandInt(-10,10)
    p14c2 = RandInt(-10,10)
    p14f1 = x+p14c1
    p14f2 = x+p14c2
    p14disp = expand(p14f1*p14f2)
    p14ans = (p14f1).mul(p14f2,hold=true)


##  AC Method 
if p14variant == 1:
    p14c1 = RandInt(-10,10)
    p14c2 = RandInt(-10,10)
    p14c3 = NonZeroInt(-5,5)
    p14c4 = RandInt(1,10)
    p14f1 = p14c3*x + p14c1
    p14f2 = p14c4*x + p14c2
    
    p14disp = expand(p14f1*p14f2)
    p14ans = (p14f1).mul(p14f2, hold=true)



##  Factoring with Rational Root Theorem 1
if p14variant == 2:
    p14ch1 = RandInt(0,3)
    p14ch2 = RandInt(0,3)
    p14ch3 = NonZeroInt(0,5,[p14ch1])
    p14ch4 = NonZeroInt(0,5,[p14ch2])
    p14ch5 = RandInt(0,5)
    p14ch6 = RandInt(0,5)
    
    p14c1 = primevec[p14ch1]
    p14c2 = primevec[p14ch2]
    
    p14c3 = (-1)^(RandInt(0,1))*primevec[p14ch3]
    p14c4 = (-1)^(RandInt(0,1))*primevec[p14ch4]
    p14c5 = (-1)^(RandInt(0,1))*primevec[p14ch5]
    p14c6 = (-1)^(RandInt(0,1))*primevec[p14ch6]
    
    p14c7 = p14c3*p14c4*p14c5*p14c6
    p14c8 = p14c1*p14c2
    
    while abs(p14c7) > 600 or abs(p14c8) > 15:
        p14ch1 = RandInt(0,3)
        p14ch2 = RandInt(0,3)
        p14ch3 = NonZeroInt(0,5,[p14ch1])
        p14ch4 = NonZeroInt(0,5,[p14ch2])
        p14ch5 = RandInt(0,5)
        
        p14c1 = primevec[p14ch1]
        p14c2 = primevec[p14ch2]
        
        p14c3 = (-1)^(RandInt(0,1))*primevec[p14ch3]
        p14c4 = (-1)^(RandInt(0,1))*primevec[p14ch4]
        p14c5 = (-1)^(RandInt(0,1))*primevec[p14ch5]
        p14c6 = (-1)^(RandInt(0,1))*primevec[p14ch6]
        
        p14c7 = p14c3*p14c4*p14c5*p14c6
        p14c8 = p14c1*p14c2
    
    
    p14f1 = (p14c1*x-p14c3)
    p14f2 = (p14c2*x-p14c4)
    p14f3 = (x-p14c5)
    p14f4 = (x-p14c6)
    p14disp = expand(p14f1*p14f2*p14f3*p14f4)
    p14ans = (p14f1).mul(p14f2,p14f3,p14f4,hold=true)



##  Grouping Method
if p14variant == 3:
    p14c1 = RandInt(-10,10)
    p14c2 = RandInt(-10,10)
    p14c3 = NonZeroInt(-5,5)
    p14c4 = RandInt(1,10)
    p14f1 = p14c3*x + p14c1
    p14f2 = p14c3*x - p14c1
    p14f3 = p14c4*x + p14c2
    
    p14disp = expand(p14f1*p14f2*p14f3)
    p14ans = (p14f1).mul(p14f2,p14f3,hold=true)



##  Special Forms
if p14variant == 4:
    p14c2 = NonZeroInt(-5,5)
    p14c1 = NonZeroInt(-10,10,[0,p14c2,-p14c2,p14c2^2,-p14c2^2,-p14c2^3,p14c2^3])
    
    p14f1 = (p14c2*x - p14c1)
    p14tog = RandInt(0,1)
    if p14tog == 0:
        p14f2 = (p14c2*x + p14c1)
    else:
        p14f2 = ((p14c2*x)^2 + p14c2*p14c1*x + p14c1^2)
    
    p14ans = p14f1.mul(p14f2,hold=true)
    p14disp = expand(p14f1*p14f2)





######  Problem p15

p15variant = RandInt(0,4)

##  Coefficient Method
if p15variant == 0:
    p15c1 = RandInt(-10,10)
    p15c2 = RandInt(-10,10)
    p15f1 = x+p15c1
    p15f2 = x+p15c2
    p15disp = expand(p15f1*p15f2)
    p15ans = (p15f1).mul(p15f2,hold=true)


##  AC Method 
if p15variant == 1:
    p15c1 = RandInt(-10,10)
    p15c2 = RandInt(-10,10)
    p15c3 = NonZeroInt(-5,5)
    p15c4 = RandInt(1,10)
    p15f1 = p15c3*x + p15c1
    p15f2 = p15c4*x + p15c2
    
    p15disp = expand(p15f1*p15f2)
    p15ans = (p15f1).mul(p15f2, hold=true)



##  Factoring with Rational Root Theorem 1
if p15variant == 2:
    p15ch1 = RandInt(0,3)
    p15ch2 = RandInt(0,3)
    p15ch3 = NonZeroInt(0,5,[p15ch1])
    p15ch4 = NonZeroInt(0,5,[p15ch2])
    p15ch5 = RandInt(0,5)
    p15ch6 = RandInt(0,5)
    
    p15c1 = primevec[p15ch1]
    p15c2 = primevec[p15ch2]
    
    p15c3 = (-1)^(RandInt(0,1))*primevec[p15ch3]
    p15c4 = (-1)^(RandInt(0,1))*primevec[p15ch4]
    p15c5 = (-1)^(RandInt(0,1))*primevec[p15ch5]
    p15c6 = (-1)^(RandInt(0,1))*primevec[p15ch6]
    
    p15c7 = p15c3*p15c4*p15c5*p15c6
    p15c8 = p15c1*p15c2
    
    while abs(p15c7) > 600 or abs(p15c8) > 15:
        p15ch1 = RandInt(0,3)
        p15ch2 = RandInt(0,3)
        p15ch3 = NonZeroInt(0,5,[p15ch1])
        p15ch4 = NonZeroInt(0,5,[p15ch2])
        p15ch5 = RandInt(0,5)
        
        p15c1 = primevec[p15ch1]
        p15c2 = primevec[p15ch2]
        
        p15c3 = (-1)^(RandInt(0,1))*primevec[p15ch3]
        p15c4 = (-1)^(RandInt(0,1))*primevec[p15ch4]
        p15c5 = (-1)^(RandInt(0,1))*primevec[p15ch5]
        p15c6 = (-1)^(RandInt(0,1))*primevec[p15ch6]
        
        p15c7 = p15c3*p15c4*p15c5*p15c6
        p15c8 = p15c1*p15c2
    
    
    p15f1 = (p15c1*x-p15c3)
    p15f2 = (p15c2*x-p15c4)
    p15f3 = (x-p15c5)
    p15f4 = (x-p15c6)
    p15disp = expand(p15f1*p15f2*p15f3*p15f4)
    p15ans = (p15f1).mul(p15f2,p15f3,p15f4,hold=true)



##  Grouping Method
if p15variant == 3:
    p15c1 = RandInt(-10,10)
    p15c2 = RandInt(-10,10)
    p15c3 = NonZeroInt(-5,5)
    p15c4 = RandInt(1,10)
    p15f1 = p15c3*x + p15c1
    p15f2 = p15c3*x - p15c1
    p15f3 = p15c4*x + p15c2
    
    p15disp = expand(p15f1*p15f2*p15f3)
    p15ans = (p15f1).mul(p15f2,p15f3,hold=true)



##  Special Forms
if p15variant == 4:
    p15c2 = NonZeroInt(-5,5)
    p15c1 = NonZeroInt(-10,10,[0,p15c2,-p15c2,p15c2^2,-p15c2^2,-p15c2^3,p15c2^3])
    
    p15f1 = (p15c2*x - p15c1)
    p15tog = RandInt(0,1)
    if p15tog == 0:
        p15f2 = (p15c2*x + p15c1)
    else:
        p15f2 = ((p15c2*x)^2 + p15c2*p15c1*x + p15c1^2)
    
    p15ans = p15f1.mul(p15f2,hold=true)
    p15disp = expand(p15f1*p15f2)







\end{sagesilent}


\begin{javascript}
    var x;

    // sameDerivative checks to see if the derivative with respect to x and C are equal.
    sameDerivative = function(a,b) {
        return (a.derivative('x').equals( b.derivative('x') ) && a.derivative('C').equals( b.derivative('C') )) ;
    };


    function factorCheck(f,g) {
        // This validator is designed to check that a student is submitting a factored polynomial. It works by:
        //  Checking that there are the correct number of non-numeric and non-inverse factors as expected,
        //  Checking that the submitted answer and the expected answer are the same via real Xronos evaluation,
        //  Checking that the outer most (last to be computed when following order of operations) operation is multiplication.
        
        var operCheck = f.tree[0];// Check to see if the root operation is multiplication at end.
        var studentFactors = f.tree.length;// Temporary number of student-provided factors (+1 because of root operation)
        
        // Now we adjust the length to remove any numeric factors, or division factors, etc to avoid ``padding'' by students.
        for (var i = 0; i < f.tree.length; i++) {
            if ((typeof f.tree[i] === 'number')||(f.tree[i][0] == '-')||(f.tree[i][0] == '/')) {
                studentFactors = studentFactors - 1;
            }
        }
        
        // Now we do the same with the provided answer, in case sage or something provides a weird format.
        var answerFactors = g.tree.length;
        
        // Adjust length in the same way, so that it will match the students if it should.
        for (var i = 0; i < g.tree.length; i++) {
            if (typeof g.tree[i] === 'number') {
                answerFactors = answerFactors - 1;
            }
        }
        
        // This is where we should do a derivative check of each factor (from student and from supplied answer) 
        // to determine if it is non-constant linear, non-constant zero, zero, or other. 
        // If any of the derivatives are ``other'' it's wrong. 
        // Otherwise, the number of each of these should match against the provided answer.
        // Current version doesn't let me manipulate the factors as functions so I can't implement this nicely.
        
        
        // Note: An especially dedicated student could pad with weird factors that all happen to cancel down to 1.
        // For example, a student could enter sin^2(x)+cos^2(x) as a multiplicative factor to pad the number of factors.
        // This would be somewhat difficult to think of, even on purpose.
        // Until I can reliably evaluate the factors themselves as functions though, there isn't a lot to be done here.
        
        return ((f.equals(g))&&(studentFactors==answerFactors)&&(operCheck=='*'))
    }
    
\end{javascript}

This is a comprehensive factoring practice page. Each of the problems below are randomly generated and randomly drawn from all the different types of factoring we have learned in the polynomial chapter. Each time you hit the ``Another'' button in the top right corner, the type of technique necessary and the values of each of the problems will change randomly.

I highly encourage you to keep redoing this page by hitting that ``Another'' button in the top right until you are able to answer each of the problems relatively easily and can quickly recognize which technique to use on any given problem.

\textbf{Note:} There is currently a known bug that Xronos may not correctly mark a factorization if it has a negative sign on the outside of the parentheses. To avoid this, simply multiply the negative value into one of your parentheses.


\textbf{Hint:} Rather than give individual hints (which can be challenging given the dynamically generated nature of the content below) I will give an overall hint here; note that this applies to all of the below. Remember that your aim is to fully factor the polynomial, but the techniques you may need to can be different from problem to problem. For that reason it is helpful to think of a general flow of techniques, starting from easiest (to do, or to rule out as possible) to hardest/longest to complete. I would recommend the following order, but this is entirely personal preference, so use the order that works best for you:
\begin{enumerate}
    \item The first step of any factoring process should always be to \textbf{factor out any common terms!} Even if it is just a constant, this will make the constants/coefficients you have to deal with later much easier to deal with as they will be smaller. Trust me, it makes a big difference!
    \item Quadratic Forms: It is usually easy to tell if a polynomial is a quadratic form as it must be 2 or 3 terms, and have that very specific format of $ax^{2n} + bx^n + c$ (i.e. that the leading term's power is exactly twice the only other power of $x$, and the last term is a constant, possibly $0$). For this reason, you can usually see if any of the following techniques are even possible with the polynomial you are trying to factor:
    \begin{itemize}
        \item Factoring Coefficients
        \item AC-Method
        \item Difference of Squares (or cubes, although that isn't a quadratic form technique, technically).
        \item Completing the Square
        \item Quadratic Formula
    \end{itemize}
    \item Next, if you can't use any of the quadratic form techniques, a good option is to see if you can factor by grouping. Remember this requires a non-prime number of terms generally (i.e. you need something like 4, 6, or 8 terms) to have any hope of doing a grouping method.
    \item If you can't use any of the above methods, Rational Root Theorem is your tool of last resort. Remember \textbf{Rational Root Theorem is terrible!} It takes a while to do, and it's basically a better version of guess and check; which is always a method of last resort in math. Nonetheless, it is sometimes all you can do.
\end{enumerate}
Finally, once you have found \textit{a} factor that you can pull out of the polynomial, treat the resulting ``chunk'' of a polynomial that you have left (after pulling out the factor you found) as a new problem; start your process over again by trying to factor out any common terms, looking at quadratic form techniques, factor by grouping, etc. The goal is to use the easier techniques if at all possible, at any stage; using rational root theorem a bunch of times in a row is going to take way too long on an exam, and realistically you rarely need to use it more than a couple times (at most) for a given problem.

\begin{problem}
    Factor the following polynomial completely.

\[
    \sage{p1disp} = \answer[validator=factorCheck]{\sage{p1ans}}
\]
\end{problem}




\begin{problem}
    Factor the following polynomial completely.

\[
    \sage{p2disp} = \answer[validator=factorCheck]{\sage{p2ans}}
\]
\end{problem}




\begin{problem}
    Factor the following polynomial completely.

\[
    \sage{p3disp} = \answer[validator=factorCheck]{\sage{p3ans}}
\]
\end{problem}




\begin{problem}
    Factor the following polynomial completely.

\[
    \sage{p4disp} = \answer[validator=factorCheck]{\sage{p4ans}}
\]
\end{problem}




\begin{problem}
    Factor the following polynomial completely.

\[
    \sage{p5disp} = \answer[validator=factorCheck]{\sage{p5ans}}
\]
\end{problem}




\begin{problem}
    Factor the following polynomial completely.

\[
    \sage{p6disp} = \answer[validator=factorCheck]{\sage{p6ans}}
\]
\end{problem}




\begin{problem}
    Factor the following polynomial completely.

\[
    \sage{p7disp} = \answer[validator=factorCheck]{\sage{p7ans}}
\]
\end{problem}




\begin{problem}
    Factor the following polynomial completely.

\[
    \sage{p8disp} = \answer[validator=factorCheck]{\sage{p8ans}}
\]
\end{problem}




\begin{problem}
    Factor the following polynomial completely.

\[
    \sage{p9disp} = \answer[validator=factorCheck]{\sage{p9ans}}
\]
\end{problem}




\begin{problem}
    Factor the following polynomial completely.

\[
    \sage{p10disp} = \answer[validator=factorCheck]{\sage{p10ans}}
\]
\end{problem}




\begin{problem}
    Factor the following polynomial completely.

\[
    \sage{p11disp} = \answer[validator=factorCheck]{\sage{p11ans}}
\]
\end{problem}




\begin{problem}
    Factor the following polynomial completely.

\[
    \sage{p12disp} = \answer[validator=factorCheck]{\sage{p12ans}}
\]
\end{problem}




\begin{problem}
    Factor the following polynomial completely.

\[
    \sage{p13disp} = \answer[validator=factorCheck]{\sage{p13ans}}
\]
\end{problem}




\begin{problem}
    Factor the following polynomial completely.

\[
    \sage{p14disp} = \answer[validator=factorCheck]{\sage{p14ans}}
\]
\end{problem}




\begin{problem}
    Factor the following polynomial completely.

\[
    \sage{p15disp} = \answer[validator=factorCheck]{\sage{p15ans}}
\]
\end{problem}







\end{document}