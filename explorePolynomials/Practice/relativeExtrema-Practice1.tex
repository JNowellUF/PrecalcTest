\documentclass{ximera}
\title{Factor Coefficients Method Practice 1}



\begin{document}
\begin{sagesilent}

######  Define a function to convert a sage number into a saved counter number.

#####Define default Sage variables.
#Default function variables
var('x,y,z,X,Y,Z')
#Default function names
var('f,g,h,dx,dy,dz,dh,df')
#Default Wild cards
w0 = SR.wild(0)

def DispSign(b):
    """ Returns the string of the 'signed' version of `b`, e.g. 3 -> "+3", -3 -> "-3", 0 -> "".
    """
    if b == 0:
        return ""
    elif b > 0:
        return "+" + str(b)
    elif b < 0:
        return str(b)
    else:
        # If we're here, then something has gone wrong.
        raise ValueError

def ISP(b):
    return DispSign(b)

def NoEval(f, c):
    # TODO
    """ Returns a non-evaluted version of the result f(c).
    """
    cStr = str(c)
    # fLatex = latex(f)
    fString = latex(f)
    fStrList = list(fString)
    length = len(fStrList)
    fStrList2 = range(length)
    for i in range(0, length):
        if fStrList[i] == "x":
            fStrList2[i] = "("+cstr+")"
        else:
            fStrList2[i] = fStrList[i]
    f2 = join(fStrList2,"")
    return LatexExpr(f2)

def HyperSimp(f):
    """ Returns the expression `f` without hyperbolic expressions.
    """
    subsDict = {
        sinh(w0) : (exp(w0) - exp(-w0))/2,
        cosh(w0) : (exp(w0) + exp(-w0))/2,
        tanh(w0) : (exp(w0) - exp(-w0))/(exp(w0) + exp(-w0)),
        sech(w0) : 2/(exp(w0) + exp(-w0)),                      # This seems to work, but Nowell said it didn't at one point.
        csch(w0) : 2/(exp(w0) - exp(-w0)),                      # This seems to work, but Nowell said it didn't at one point.
        coth(w0) : (exp(w0) + exp(-w0))/(exp(w0) - exp(-w0)),   # This seems to work, but Nowell said it didn't at one point.
        arcsinh(w0) :       ln( w0 + sqrt((w0)^2 + 1) ),
        arccosh(w0) :       ln( w0 + sqrt((w0)^2 - 1) ),
        arctanh(w0) : 1/2 * ln( (1 + w0) / (1 - w0) ),
        arccsch(w0) :       ln( (1 + sqrt((w0)^2 + 1))/w0 ),
        arcsech(w0) :       ln( (1 + sqrt(1 - (w0)^2))/w0 ),
        arccoth(w0) : 1/2 * ln( (1 + w0) / (w0 - 1) )
    }
    g = f.substitute(subsDict)
    return simplify(g)

def RandInt(a,b):
    """ Returns a random integer in [`a`,`b`]. Note that `a` and `b` should be integers themselves to avoid unexpected behavior.
    """
    return QQ(randint(int(a),int(b)))
    # return choice(range(a,b+1))

def NonZeroInt(b,c, avoid = [0]):
    """ Returns a random integer in [`b`,`c`] which is not in `av`. 
        If `av` is not specified, defaults to a non-zero integer.
    """
    while True:
        a = RandInt(b,c)
        if a not in avoid:
            return a

def RandVector(b, c, avoid=[], rep=1):
    """ Returns essentially a multiset permutation of ([b,c]-av) * rep.
        That is, a vector which contains each integer in [`b`,`c`] which is not in `av` a total of `rep` number of times.
        Example:
        sage: RandVector(1,3, [2], 2)
        [3, 1, 1, 3]
    """
    oneVec = [val for val in range(b,c+1) if val not in avoid]
    vec = oneVec * rep
    shuffle(vec)
    return vec

def fudge(b):
    up = b+RandInt(2,5)/10
    down = b-RandInt(2,5)/10
    fudgebup = round(up,1)
    fudgebdown = round(down,1)
    fudgedb = [fudgebdown,fudgebup]
    return fudgedb

def disjointCheck(checkvec):
    if length(uniq(checkvec)) < length(checkvec):
        return 1
    else:
        return 0

def disjointIntervals(IntStart,IntEnd,CheckVal):
    if IntStart < CheckVal and CheckVal < IntEnd:
        return 1
    else:
        return 0

def IntervalVecCheck(checkVec):
    veclen = len(checkVec)
    returnval = 0
    for i in range(veclen):
        for j in range(veclen):
            if (disjointIntervals(checkVec[j][0],checkVec[j][1],checkVec[i][0]) + disjointIntervals(checkVec[j][0],checkVec[j][1],checkVec[i][1])) > 0:
                returnval = returnval + 1
    if returnval > 0:
        return 1
    else:
        return 0



\end{sagesilent}

\begin{javascript}
function sameParity(a,b) {
    return (a-b)%2 == 0;
    };
function boundedSameParity(a,b,c) {
    var ansOne;
    var ansTwo;
    var ansThree;
    console.log('We are checking if '+a.toString()+' is the same parity as '+b.toString());    
    if ((a-b)%2==0) {
        ansOne=1;
        console.log('we passed parity!');
    } else {
        ansOne=0;
        console.log('we failed parity!');
    }
    var upB = parseInt(c)+1;
    console.log('We are checking if '+a.toString()+' is less than '+c.toString()+' plus 1 ie '+upB.toString());
    if (a<upB) {
        ansTwo=1;
        console.log('we passed upper bound!');
    } else {
        ansTwo=0;
        console.log('we failed upper bound!');
    }
    var lowB = parseInt(b)-1;
    console.log('We are checking if '+a.toString()+' is greater than '+b.toString()+' minus 1, ie '+lowB.toString());
    if (a>lowB) {
        ansThree=1;
        console.log('we passed lower bound!');
    } else {
        ansThree=0;
    }
    return ( ansOne*ansTwo*ansThree );
    };
\end{javascript}

\begin{sagesilent}

#### Problem p1
p1c1 = NonZeroInt(-10,10)
p1c2 = NonZeroInt(-10,10,[0,p1c1])
p1c3 = NonZeroInt(-10,10,[0,p1c1,p1c2])
p1c4 = NonZeroInt(-10,10,[0,p1c1,p1c2,p1c3])

p1pwr1 = RandInt(1,20)
p1pwr2 = RandInt(1,20)
p1pwr3 = RandInt(1,20)
p1pwr4 = RandInt(1,20)

p1f1 = p1c1*x^p1pwr1
p1f2 = p1c2*x^p1pwr2
p1f3 = p1c3*x^p1pwr3
p1f4 = p1c4*x^p1pwr4

p1temp1 = max(p1pwr1,p1pwr2,p1pwr3,p1pwr4)
p1ans1 = p1temp1 - 1

p1ans2 = p1ans1%2


#### Problem p2
p2c1 = NonZeroInt(-10,10)
p2c2 = NonZeroInt(-10,10,[0,p2c1])
p2c3 = NonZeroInt(-10,10,[0,p2c1,p2c2])
p2c4 = NonZeroInt(-10,10,[0,p2c1,p2c2,p2c3])

p2pwr1 = RandInt(1,20)
p2pwr2 = RandInt(1,20)
p2pwr3 = RandInt(1,20)
p2pwr4 = RandInt(1,20)

p2f1 = p2c1*x^p2pwr1
p2f2 = p2c2*x^p2pwr2
p2f3 = p2c3*x^p2pwr3
p2f4 = p2c4*x^p2pwr4

p2temp2 = max(p2pwr1,p2pwr2,p2pwr3,p2pwr4)
p2ans1 = p2temp2 - 1

p2ans2 = p2ans1%2


#### Problem p3
p3c1 = NonZeroInt(-10,10)
p3c2 = NonZeroInt(-10,10,[0,p3c1])
p3c3 = NonZeroInt(-10,10,[0,p3c1,p3c2])
p3c4 = NonZeroInt(-10,10,[0,p3c1,p3c2,p3c3])

p3pwr1 = RandInt(1,20)
p3pwr2 = RandInt(1,20)
p3pwr3 = RandInt(1,20)
p3pwr4 = RandInt(1,20)

p3f1 = p3c1*x^p3pwr1
p3f2 = p3c2*x^p3pwr2
p3f3 = p3c3*x^p3pwr3
p3f4 = p3c4*x^p3pwr4

p3temp3 = max(p3pwr1,p3pwr2,p3pwr3,p3pwr4)
p3ans1 = p3temp3 - 1

p3ans2 = p3ans1%2





\end{sagesilent}

\begin{problem}
    Consider the function $f(x) = (\sage{p1f1}) + (\sage{p1f2}) + (\sage{p1f3}) + (\sage{p1f4})$. What are the maximum number of relative extrema that $f(x)$ could have? $\answer[id=c]{\sage{p1ans1}}$.
    \begin{feedback}
        Remember, you don't need to factor or graph the function, but you may need to simplify (combine like terms) the function to determine your answer. The answer is only asking for the maximum \textbf{possible} relative extrema, not how many local extrema the function actually has.
    \end{feedback}
    \begin{problem}
        What is the minimum number relative extrema that $f(x)$ could possibly have? $\answer[id=b]{\sage{p1ans2}}$
        \begin{feedback}
            Remember that local extrema must come in pairs. To list all the possible number of local extrema, start with the maximum number, and then subtract two at a time to make the list. For example, if a polynomial has a maximum of 7 local extrema, then we can create the list of possible local extrema by subtracting 2 at a time, so there can only be 7, 5, 3, or 1 local extrema. Thus the minimum local extrema in this case would be 1. 
        \end{feedback}
        \begin{problem}
            Enter any number that could be a valid number of \textbf{possible} local extrema for $f(x)$.
            \begin{validator}[boundedSameParity(a,b,c)]
                $\answer[id=a]{\sage{p1ans1}}$
                \begin{feedback}
                    This should allow you an opportunity to make sure you understand what possible numbers of relative extrema $f(x)$ can have. This should dynamically accept any valid answer, but it is still very experimental.
                \end{feedback}
            \end{validator}
        \end{problem}
    \end{problem}
\end{problem}




\begin{problem}
    Consider the function $f(x) = (\sage{p2f1}) + (\sage{p2f2}) + (\sage{p2f3}) + (\sage{p2f4})$. What are the maximum number of relative extrema that $f(x)$ could have? $\answer[id=cc]{\sage{p2ans1}}$.
    \begin{feedback}
        Remember, you don't need to factor or graph the function, but you may need to simplify (combine like terms) the function to determine your answer. The answer is only asking for the maximum \textbf{possible} relative extrema, not how many local extrema the function actually has.
    \end{feedback}
    \begin{problem}
        What is the minimum number relative extrema that $f(x)$ could possibly have? $\answer[id=bb]{\sage{p2ans2}}$
        \begin{feedback}
            Remember that local extrema must come in pairs. To list all the possible number of local extrema, start with the maximum number, and then subtract two at a time to make the list. For example, if a polynomial has a maximum of 7 local extrema, then we can create the list of possible local extrema by subtracting 2 at a time, so there can only be 7, 5, 3, or 1 local extrema. Thus the minimum local extrema in this case would be 1. 
        \end{feedback}
        \begin{problem}
            Enter any number that could be a valid number of \textbf{possible} local extrema for $f(x)$.
            \begin{validator}[boundedSameParity(aa,bb,cc)]
                $\answer[id=aa]{\sage{p2ans1}}$
                \begin{feedback}
                    This should allow you an opportunity to make sure you understand what possible numbers of relative extrema $f(x)$ can have. This should dynamically accept any valid answer, but it is still very experimental.
                \end{feedback}
            \end{validator}
        \end{problem}
    \end{problem}
\end{problem}




\begin{problem}
    Consider the function $f(x) = (\sage{p3f1}) + (\sage{p3f2}) + (\sage{p3f3}) + (\sage{p3f4})$. What are the maximum number of relative extrema that $f(x)$ could have? $\answer[id=ccc]{\sage{p3ans1}}$.
    \begin{feedback}
        Remember, you don't need to factor or graph the function, but you may need to simplify (combine like terms) the function to determine your answer. The answer is only asking for the maximum \textbf{possible} relative extrema, not how many local extrema the function actually has.
    \end{feedback}
    \begin{problem}
        What is the minimum number relative extrema that $f(x)$ could possibly have? $\answer[id=bbb]{\sage{p3ans2}}$
        \begin{feedback}
            Remember that local extrema must come in pairs. To list all the possible number of local extrema, start with the maximum number, and then subtract two at a time to make the list. For example, if a polynomial has a maximum of 7 local extrema, then we can create the list of possible local extrema by subtracting 2 at a time, so there can only be 7, 5, 3, or 1 local extrema. Thus the minimum local extrema in this case would be 1. 
        \end{feedback}
        \begin{problem}
            Enter any number that could be a valid number of \textbf{possible} local extrema for $f(x)$.
            \begin{validator}[boundedSameParity(aaa,bbb,ccc)]
                $\answer[id=aaa]{\sage{p3ans1}}$
                \begin{feedback}
                    This should allow you an opportunity to make sure you understand what possible numbers of relative extrema $f(x)$ can have. This should dynamically accept any valid answer, but it is still very experimental.
                \end{feedback}
            \end{validator}
        \end{problem}
    \end{problem}
\end{problem}




\end{document}