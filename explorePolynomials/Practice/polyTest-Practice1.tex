\documentclass{ximera}
\title{Factor Coefficients Method Practice 1}



\begin{document}
\input{Useful-Sage-Macros}

\begin{sagesilent}
funcvec = [x, x^2, x^3, x^4, x^5, sqrt(x),e^x, 1/x, 1/x^2, log(x)]

###### Problem p1
p1pick1 = RandInt(0,9)
p1pick2 = NonZeroInt(0,9,[p1pick1])
p1pick3 = NonZeroInt(0,9,[p1pick1,p1pick2])

p1c1 = NonZeroInt(-5,5)
p1c2 = NonZeroInt(-5,5)
p1c3 = NonZeroInt(-5,5)

p1c4 = RandInt(-10,10)
p1c5 = RandInt(-10,10)
p1c6 = RandInt(-10,10)

p1ans1 = 0
p1ans1t = 0
p1ans2t = 0
p1ans3t = 0

p1f1t = funcvec[p1pick1]
p1f2t = funcvec[p1pick2]
p1f3t = funcvec[p1pick3]

p1f1 = p1c1*p1f1t + p1c4
p1f2 = p1c2*p1f2t + p1c5
p1f3 = p1c3*p1f3t + p1c6

p1f4 = p1f1 + p1f2 + p1f3

if p1pick1 > 4:
    p1ans1t = 1
if p1pick2 > 4:
    p1ans2t = 1
if p1pick3 > 4:
    p1ans3t = 1

p1ans1 = p1ans1t + p1ans2t + p1ans3t



###### Problem p2
p2pick1 = RandInt(0,9)
p2pick2 = NonZeroInt(0,9,[p2pick1])
p2pick3 = NonZeroInt(0,9,[p2pick1,p2pick2])

p2c1 = NonZeroInt(-5,5)
p2c2 = NonZeroInt(-5,5)
p2c3 = NonZeroInt(-5,5)

p2c4 = RandInt(-10,10)
p2c5 = RandInt(-10,10)
p2c6 = RandInt(-10,10)

p2ans1 = 0
p2ans1t = 0
p2ans2t = 0
p2ans3t = 0

p2f1t = funcvec[p2pick1]
p2f2t = funcvec[p2pick2]
p2f3t = funcvec[p2pick3]

p2f1 = p2c1*p2f1t + p2c4
p2f2 = p2c2*p2f2t + p2c5
p2f3 = p2c3*p2f3t + p2c6

p2f4 = p2f1 + p2f2 + p2f3

if p2pick1 > 4:
    p2ans1t = 1
if p2pick2 > 4:
    p2ans2t = 1
if p2pick3 > 4:
    p2ans3t = 1

p2ans1 = p2ans1t + p2ans2t + p2ans3t




###### Problem p3
p3pick1 = RandInt(0,9)
p3pick2 = NonZeroInt(0,9,[p3pick1])
p3pick3 = NonZeroInt(0,9,[p3pick1,p3pick2])

p3c1 = NonZeroInt(-5,5)
p3c2 = NonZeroInt(-5,5)
p3c3 = NonZeroInt(-5,5)

p3c4 = RandInt(-10,10)
p3c5 = RandInt(-10,10)
p3c6 = RandInt(-10,10)

p3ans1 = 0
p3ans1t = 0
p3ans2t = 0
p3ans3t = 0

p3f1t = funcvec[p3pick1]
p3f2t = funcvec[p3pick2]
p3f3t = funcvec[p3pick3]

p3f1 = p3c1*p3f1t + p3c4
p3f2 = p3c2*p3f2t + p3c5
p3f3 = p3c3*p3f3t + p3c6

p3f4 = p3f1 + p3f2 + p3f3

if p3pick1 > 4:
    p3ans1t = 1
if p3pick2 > 4:
    p3ans2t = 1
if p3pick3 > 4:
    p3ans3t = 1

p3ans1 = p3ans1t + p3ans2t + p3ans3t



\end{sagesilent}

\begin{problem}
    Consider the function $f(x) = \sage{p1f4}$. How many terms in $f(x)$ are \textbf{not} monomials? $\answer{\sage{p1ans1}}$.
    \begin{feedback}
        Monomials are the terms that look like $ax^n$ for some real number $a$ and $n$ needs to be a whole number. In other words, terms that look like (number)$\times x^n$ (where $n$ can be 0, 1, 2, etc) are monomials. Notice that, if $n=0$ then this will actually just look like a constant (since $x^0=1$), so even constants count as monomials.
    \end{feedback}
\end{problem}

\begin{problem}
Consider the function $f(x) = \sage{p2f4}$. How many terms in $f(x)$ are \textbf{not} monomials? $\answer{\sage{p2ans1}}$.
    \begin{feedback}
        Monomials are the terms that look like $ax^n$ for some real number $a$ and $n$ needs to be a whole number. In other words, terms that look like (number)$\times x^n$ (where $n$ can be 0, 1, 2, etc) are monomials. Notice that, if $n=0$ then this will actually just look like a constant (since $x^0=1$), so even constants count as monomials.
    \end{feedback}
\end{problem}

\begin{problem}
Consider the function $f(x) = \sage{p3f4}$. How many terms in $f(x)$ are \textbf{not} monomials? $\answer{\sage{p3ans1}}$.
    \begin{feedback}
        Monomials are the terms that look like $ax^n$ for some real number $a$ and $n$ needs to be a whole number. In other words, terms that look like (number)$\times x^n$ (where $n$ can be 0, 1, 2, etc) are monomials. Notice that, if $n=0$ then this will actually just look like a constant (since $x^0=1$), so even constants count as monomials.
    \end{feedback}
\end{problem}



\end{document}