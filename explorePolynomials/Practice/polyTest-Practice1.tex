\documentclass{ximera}
\title{Factor Coefficients Method Practice 1}



\begin{document}
\begin{sagesilent}

######  Define a function to convert a sage number into a saved counter number.

#####Define default Sage variables.
#Default function variables
var('x,y,z,X,Y,Z')
#Default function names
var('f,g,h,dx,dy,dz,dh,df')
#Default Wild cards
w0 = SR.wild(0)

def DispSign(b):
    """ Returns the string of the 'signed' version of `b`, e.g. 3 -> "+3", -3 -> "-3", 0 -> "".
    """
    if b == 0:
        return ""
    elif b > 0:
        return "+" + str(b)
    elif b < 0:
        return str(b)
    else:
        # If we're here, then something has gone wrong.
        raise ValueError

def ISP(b):
    return DispSign(b)

def NoEval(f, c):
    # TODO
    """ Returns a non-evaluted version of the result f(c).
    """
    cStr = str(c)
    # fLatex = latex(f)
    fString = latex(f)
    fStrList = list(fString)
    length = len(fStrList)
    fStrList2 = range(length)
    for i in range(0, length):
        if fStrList[i] == "x":
            fStrList2[i] = "("+cstr+")"
        else:
            fStrList2[i] = fStrList[i]
    f2 = join(fStrList2,"")
    return LatexExpr(f2)

def HyperSimp(f):
    """ Returns the expression `f` without hyperbolic expressions.
    """
    subsDict = {
        sinh(w0) : (exp(w0) - exp(-w0))/2,
        cosh(w0) : (exp(w0) + exp(-w0))/2,
        tanh(w0) : (exp(w0) - exp(-w0))/(exp(w0) + exp(-w0)),
        sech(w0) : 2/(exp(w0) + exp(-w0)),                      # This seems to work, but Nowell said it didn't at one point.
        csch(w0) : 2/(exp(w0) - exp(-w0)),                      # This seems to work, but Nowell said it didn't at one point.
        coth(w0) : (exp(w0) + exp(-w0))/(exp(w0) - exp(-w0)),   # This seems to work, but Nowell said it didn't at one point.
        arcsinh(w0) :       ln( w0 + sqrt((w0)^2 + 1) ),
        arccosh(w0) :       ln( w0 + sqrt((w0)^2 - 1) ),
        arctanh(w0) : 1/2 * ln( (1 + w0) / (1 - w0) ),
        arccsch(w0) :       ln( (1 + sqrt((w0)^2 + 1))/w0 ),
        arcsech(w0) :       ln( (1 + sqrt(1 - (w0)^2))/w0 ),
        arccoth(w0) : 1/2 * ln( (1 + w0) / (w0 - 1) )
    }
    g = f.substitute(subsDict)
    return simplify(g)

def RandInt(a,b):
    """ Returns a random integer in [`a`,`b`]. Note that `a` and `b` should be integers themselves to avoid unexpected behavior.
    """
    return QQ(randint(int(a),int(b)))
    # return choice(range(a,b+1))

def NonZeroInt(b,c, avoid = [0]):
    """ Returns a random integer in [`b`,`c`] which is not in `av`. 
        If `av` is not specified, defaults to a non-zero integer.
    """
    while True:
        a = RandInt(b,c)
        if a not in avoid:
            return a

def RandVector(b, c, avoid=[], rep=1):
    """ Returns essentially a multiset permutation of ([b,c]-av) * rep.
        That is, a vector which contains each integer in [`b`,`c`] which is not in `av` a total of `rep` number of times.
        Example:
        sage: RandVector(1,3, [2], 2)
        [3, 1, 1, 3]
    """
    oneVec = [val for val in range(b,c+1) if val not in avoid]
    vec = oneVec * rep
    shuffle(vec)
    return vec

def fudge(b):
    up = b+RandInt(2,5)/10
    down = b-RandInt(2,5)/10
    fudgebup = round(up,1)
    fudgebdown = round(down,1)
    fudgedb = [fudgebdown,fudgebup]
    return fudgedb

def disjointCheck(checkvec):
    if length(uniq(checkvec)) < length(checkvec):
        return 1
    else:
        return 0

def disjointIntervals(IntStart,IntEnd,CheckVal):
    if IntStart < CheckVal and CheckVal < IntEnd:
        return 1
    else:
        return 0

def IntervalVecCheck(checkVec):
    veclen = len(checkVec)
    returnval = 0
    for i in range(veclen):
        for j in range(veclen):
            if (disjointIntervals(checkVec[j][0],checkVec[j][1],checkVec[i][0]) + disjointIntervals(checkVec[j][0],checkVec[j][1],checkVec[i][1])) > 0:
                returnval = returnval + 1
    if returnval > 0:
        return 1
    else:
        return 0



\end{sagesilent}

\begin{sagesilent}
funcvec = [x, x^2, x^3, x^4, x^5, sqrt(x),e^x, 1/x, 1/x^2, log(x)]

###### Problem p1
p1pick1 = RandInt(0,9)
p1pick2 = NonZeroInt(0,9,[p1pick1])
p1pick3 = NonZeroInt(0,9,[p1pick1,p1pick2])

p1c1 = NonZeroInt(-5,5)
p1c2 = NonZeroInt(-5,5)
p1c3 = NonZeroInt(-5,5)

p1c4 = RandInt(-10,10)
p1c5 = RandInt(-10,10)
p1c6 = RandInt(-10,10)

p1ans1 = 0
p1ans1t = 0
p1ans2t = 0
p1ans3t = 0

p1f1t = funcvec[p1pick1]
p1f2t = funcvec[p1pick2]
p1f3t = funcvec[p1pick3]

p1f1 = p1c1*p1f1t + p1c4
p1f2 = p1c2*p1f2t + p1c5
p1f3 = p1c3*p1f3t + p1c6

p1f4 = p1f1 + p1f2 + p1f3

if p1pick1 > 4:
    p1ans1t = 1
if p1pick2 > 4:
    p1ans2t = 1
if p1pick3 > 4:
    p1ans3t = 1

p1ans1 = p1ans1t + p1ans2t + p1ans3t



###### Problem p2
p2pick1 = RandInt(0,9)
p2pick2 = NonZeroInt(0,9,[p2pick1])
p2pick3 = NonZeroInt(0,9,[p2pick1,p2pick2])

p2c1 = NonZeroInt(-5,5)
p2c2 = NonZeroInt(-5,5)
p2c3 = NonZeroInt(-5,5)

p2c4 = RandInt(-10,10)
p2c5 = RandInt(-10,10)
p2c6 = RandInt(-10,10)

p2ans1 = 0
p2ans1t = 0
p2ans2t = 0
p2ans3t = 0

p2f1t = funcvec[p2pick1]
p2f2t = funcvec[p2pick2]
p2f3t = funcvec[p2pick3]

p2f1 = p2c1*p2f1t + p2c4
p2f2 = p2c2*p2f2t + p2c5
p2f3 = p2c3*p2f3t + p2c6

p2f4 = p2f1 + p2f2 + p2f3

if p2pick1 > 4:
    p2ans1t = 1
if p2pick2 > 4:
    p2ans2t = 1
if p2pick3 > 4:
    p2ans3t = 1

p2ans1 = p2ans1t + p2ans2t + p2ans3t




###### Problem p3
p3pick1 = RandInt(0,9)
p3pick2 = NonZeroInt(0,9,[p3pick1])
p3pick3 = NonZeroInt(0,9,[p3pick1,p3pick2])

p3c1 = NonZeroInt(-5,5)
p3c2 = NonZeroInt(-5,5)
p3c3 = NonZeroInt(-5,5)

p3c4 = RandInt(-10,10)
p3c5 = RandInt(-10,10)
p3c6 = RandInt(-10,10)

p3ans1 = 0
p3ans1t = 0
p3ans2t = 0
p3ans3t = 0

p3f1t = funcvec[p3pick1]
p3f2t = funcvec[p3pick2]
p3f3t = funcvec[p3pick3]

p3f1 = p3c1*p3f1t + p3c4
p3f2 = p3c2*p3f2t + p3c5
p3f3 = p3c3*p3f3t + p3c6

p3f4 = p3f1 + p3f2 + p3f3

if p3pick1 > 4:
    p3ans1t = 1
if p3pick2 > 4:
    p3ans2t = 1
if p3pick3 > 4:
    p3ans3t = 1

p3ans1 = p3ans1t + p3ans2t + p3ans3t



\end{sagesilent}

\begin{problem}
    Consider the function $f(x) = \sage{p1f4}$. How many terms in $f(x)$ are \textbf{not} monomials? $\answer{\sage{p1ans1}}$.
    \begin{feedback}
        Monomials are the terms that look like $ax^n$ for some real number $a$ and $n$ needs to be a whole number. In other words, terms that look like (number)$\times x^n$ (where $n$ can be 0, 1, 2, etc) are monomials. Notice that, if $n=0$ then this will actually just look like a constant (since $x^0=1$), so even constants count as monomials.
    \end{feedback}
\end{problem}

\begin{problem}
Consider the function $f(x) = \sage{p2f4}$. How many terms in $f(x)$ are \textbf{not} monomials? $\answer{\sage{p2ans1}}$.
    \begin{feedback}
        Monomials are the terms that look like $ax^n$ for some real number $a$ and $n$ needs to be a whole number. In other words, terms that look like (number)$\times x^n$ (where $n$ can be 0, 1, 2, etc) are monomials. Notice that, if $n=0$ then this will actually just look like a constant (since $x^0=1$), so even constants count as monomials.
    \end{feedback}
\end{problem}

\begin{problem}
Consider the function $f(x) = \sage{p3f4}$. How many terms in $f(x)$ are \textbf{not} monomials? $\answer{\sage{p3ans1}}$.
    \begin{feedback}
        Monomials are the terms that look like $ax^n$ for some real number $a$ and $n$ needs to be a whole number. In other words, terms that look like (number)$\times x^n$ (where $n$ can be 0, 1, 2, etc) are monomials. Notice that, if $n=0$ then this will actually just look like a constant (since $x^0=1$), so even constants count as monomials.
    \end{feedback}
\end{problem}



\end{document}