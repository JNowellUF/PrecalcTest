\documentclass{ximera}
\title{Factor Coefficients Method Practice 1}
\input{../../preamble}
%\usepackage{sagetex}

\begin{document}
%\input{Useful-Sage-Macros}

\begin{sagesilent}
def RandInt(a,b):
    """ Returns a random integer in [`a`,`b`]. Note that `a` and `b` should be integers themselves to avoid unexpected behavior.
    """
    return QQ(randint(int(a),int(b)))
    # return choice(range(a,b+1))

def NonZeroInt(b,c, avoid = [0]):
    """ Returns a random integer in [`b`,`c`] which is not in `av`.
        If `av` is not specified, defaults to a non-zero integer.
    """
    while True:
        a = RandInt(b,c)
        if a not in avoid:
            return a


def cpx(z):
    if z == 0:
        return LatexExpr('0')
    a = z.real()
    b = z.imag()
    if (a == 0) or (b==0) :
        return LatexExpr(latex(z))
    elif b > 0:
      s = '+'
    else:
      s = '-'
    return latex(a) + LatexExpr(s) + latex(abs(b) * i)



###### Problem p1
p1c1 = NonZeroInt(-10,10)
p1c2 = NonZeroInt(-10,10)

p1f1 = p1c1 - p1c2*i
p1f2 = p1c1 + p1c2*i

p1f1d = cpx(p1f1)

###### Problem p2
p2c1 = NonZeroInt(-10,10)
p2c2 = NonZeroInt(-10,10)

p2f1 = p2c1 - p2c2*i
p2f2 = p2c1 + p2c2*i

p2f1d = cpx(p2f1)

###### Problem p3
p3c1 = NonZeroInt(-10,10)
p3c2 = NonZeroInt(-10,10)

p3f1 = p3c1 - p3c2*i
p3f2 = p3c1 + p3c2*i

p3f1d = cpx(p3f1)

###### Problem p4
p4c1 = NonZeroInt(-10,10)
p4c2 = NonZeroInt(-10,10)

p4f1 = p4c1 - p4c2*i
p4f2 = p4c1 + p4c2*i

p4f1d = cpx(p4f1)


\end{sagesilent}

\begin{problem}
    If the complex number $\sage{p1f1d}$ is a zero of the polynomial $p(x)$, what other number do you know \emph{must} be a zero of $p(x)$? $\answer{\sage{p1f2}}$
    \begin{feedback}
        Remember that if $a + bi$ is a zero, then it's complex conjugate, i.e. $a - bi$ is also a zero. So, to find the other zero, you need to swap the sign in front of the imaginary term.
    \end{feedback}
    
\end{problem}


\begin{problem}
    If the complex number $\sage{p2f1d}$ is a zero of the polynomial $p(x)$, what other number do you know \emph{must} be a zero of $p(x)$? $\answer{\sage{p2f2}}$
    \begin{feedback}
        Remember that if $a + bi$ is a zero, then it's complex conjugate, i.e. $a - bi$ is also a zero. So, to find the other zero, you need to swap the sign in front of the imaginary term.
    \end{feedback}
    
\end{problem}


\begin{problem}
    If the complex number $\sage{p3f1d}$ is a zero of the polynomial $p(x)$, what other number do you know \emph{must} be a zero of $p(x)$? $\answer{\sage{p3f2}}$
    \begin{feedback}
        Remember that if $a + bi$ is a zero, then it's complex conjugate, i.e. $a - bi$ is also a zero. So, to find the other zero, you need to swap the sign in front of the imaginary term.
    \end{feedback}
    
\end{problem}


\begin{problem}
    If the complex number $\sage{p4f1d}$ is a zero of the polynomial $p(x)$, what other number do you know \emph{must} be a zero of $p(x)$? $\answer{\sage{p4f2}}$
    \begin{feedback}
        Remember that if $a + bi$ is a zero, then it's complex conjugate, i.e. $a - bi$ is also a zero. So, to find the other zero, you need to swap the sign in front of the imaginary term.
    \end{feedback}
    
\end{problem}



\end{document}