\documentclass{ximera}
\title{Factor Coefficients Method Practice 1}
\begin{document}


\begin{sagesilent}
def RandInt(a,b):
    """ Returns a random integer in [`a`,`b`]. Note that `a` and `b` should be integers themselves to avoid unexpected behavior.
    """
    return QQ(randint(int(a),int(b)))
    # return choice(range(a,b+1))

def NonZeroInt(b,c, avoid = [0]):
    """ Returns a random integer in [`b`,`c`] which is not in `av`. 
        If `av` is not specified, defaults to a non-zero integer.
    """
    while True:
        a = RandInt(b,c)
        if a not in avoid:
            return a

###### Problem p1
p1c1 = RandInt(-10,10)
p1c2 = RandInt(-10,10)
p1f1 = x+p1c1
p1f2 = x+p1c2
p1f3 = (p1f1).mul(p1f2,hold=true)
p1f4 = expand(p1f3)

###### Problem p2
p2c1 = RandInt(-10,10)
p2c2 = RandInt(-10,10)
p2f1 = x+p2c1
p2f2 = x+p2c2
p2f3 = (p2f1).mul(p2f2,hold=true)
p2f4 = expand(p2f3)

###### Problem p3
p3c1 = RandInt(-10,10)
p3c2 = RandInt(-10,10)
p3f1 = x+p3c1
p3f2 = x+p3c2
p3f3 = (p3f1).mul(p3f2,hold=true)
p3f4 = expand(p3f3)

###### Problem p4
p4c1 = RandInt(-10,10)
p4c2 = RandInt(-10,10)
p4f1 = x+p4c1
p4f2 = x+p4c2
p4f3 = (p4f1).mul(p4f2,hold=true)
p4f4 = expand(p4f3)

\end{sagesilent}


\begin{javascript}
// A validator to check and verify something has a factored form...
function factorCheck(f,g) {
    // This validator is designed to check that a student is submitting a factored polynomial. It works by:
    //  Checking that there are the correct number of non-numeric and non-inverse factors as expected,
    //  Checking that the submitted answer and the expected answer are the same via real Xronos evaluation,
    //  Checking that the outer most (last to be computed when following order of operations) operation is multiplication.
    
    var operCheck = f.tree[0];// Check to see if the root operation is multiplication at end.
    var studentFactors = f.tree.length;// Temporary number of student-provided factors (+1 because of root operation)
    
    // Now we adjust the length to remove any numeric factors, or division factors, etc to avoid ``padding'' by students.
    for (var i = 0; i < f.tree.length; i++) {
        if ((typeof f.tree[i] === 'number')||(f.tree[i][0] == '-')||(f.tree[i][0] == '/')) {
            studentFactors = studentFactors - 1;
        }
    }
    
    // Now we do the same with the provided answer, in case sage or something provides a weird format.
    var answerFactors = g.tree.length;
    
    // Adjust length in the same way, so that it will match the students if it should.
    for (var i = 0; i < g.tree.length; i++) {
        if (typeof g.tree[i] === 'number') {
            answerFactors = answerFactors - 1;
        }
    }
    
    // Note: An especially dedicated student could pad with weird factors that are happen to cancel down to 1.
    // For example, a student could enter sin^2(x)+cos^2(x) as a multiplicative factor to pad the number of factors.
    // This would be somewhat difficult to think of, even on purpose.
    // Until I can reliably evaluate the factors themselves as functions though, there isn't a lot to be done here.
    
    return ((f.equals(g))&&(studentFactors==answerFactors)&&(operCheck=='*'))
    }
\end{javascript}

\textbf{Note:} This is using an experimental factoring validator. If you verified that your answer should be correct and Xronos won't take it, please email your instructor to see if there is a problem

\begin{problem}
    Factor the following quadratic by factoring it's coefficients. 
    \[
        p(x) = \sage{p1f4} = \answer[validator=factorCheck]{\sage{p1f3}}
    \]
    \begin{feedback}
        Remember you want to find two numbers that multiply to $\sage{p1c1*p1c2}$ and add to $\sage{p1c1+p1c2}$.
    \end{feedback}
\end{problem}



\begin{problem}
    Factor the following quadratic by factoring it's coefficients.
    \[
        p(x) = \sage{p2f4} = \answer[validator=factorCheck]{\sage{p2f3}}
    \]
    \begin{feedback}
        Remember you want to find two numbers that multiply to $\sage{p2c1*p2c2}$ and add to $\sage{p2c1+p2c2}$.
    \end{feedback}
\end{problem}



\begin{problem}
    Factor the following quadratic by factoring it's coefficients.
    \[
        p(x) = \sage{p3f4} = \answer[validator=factorCheck]{\sage{p3f3}}
    \]
    \begin{feedback}
        Remember you want to find two numbers that multiply to $\sage{p3c1*p3c2}$ and add to $\sage{p3c1+p3c2}$.
    \end{feedback}
\end{problem}



\begin{problem}
    Factor the following quadratic by factoring it's coefficients.
    \[
        p(x) = \sage{p4f4} = \answer[validator=factorCheck]{\sage{p4f3}}
    \]
    \begin{feedback}
        Remember you want to find two numbers that multiply to $\sage{p4c1*p4c2}$ and add to $\sage{p4c1+p4c2}$.
    \end{feedback}
\end{problem}




\end{document}