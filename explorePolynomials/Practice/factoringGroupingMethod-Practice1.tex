\documentclass{ximera}
\title{Factor Coefficients Method Practice 1}



\begin{document}

\begin{sagesilent}
def RandInt(a,b):
    """ Returns a random integer in [`a`,`b`]. Note that `a` and `b` should be integers themselves to avoid unexpected behavior.
    """
    return QQ(randint(int(a),int(b)))
    # return choice(range(a,b+1))

def NonZeroInt(b,c, avoid = [0]):
    """ Returns a random integer in [`b`,`c`] which is not in `av`. 
        If `av` is not specified, defaults to a non-zero integer.
    """
    while True:
        a = RandInt(b,c)
        if a not in avoid:
            return a

###### Problem p1
p1c1 = NonZeroInt(-10,10)
p1c2 = RandInt(-10,10)
p1c3 = RandInt(1,10)
while p1c3*p1c1^2>1000 or p1c1^2+p1c2>1000:
    p1c1 = RandInt(-10,10)
    p1c2 = RandInt(-10,10)
    p1c3 = RandInt(1,10)

p1f1 = x + p1c1
p1f2 = x - p1c1
p1f3 = p1c3*x + p1c2

p1f4 = expand(p1f1*p1f2*p1f3)
p1ans = (p1f1).mul(p1f2,p1f3,hold=true)




###### Problem p2
p2c1 = NonZeroInt(-10,10)
p2c2 = RandInt(-10,10)
p2c3 = RandInt(1,10)
while p2c3*p2c1^2>1000 or p2c1^2+p2c2>1000:
    p2c1 = RandInt(-10,10)
    p2c2 = RandInt(-10,10)
    p2c3 = RandInt(1,10)

p2f1 = x + p2c1
p2f2 = x - p2c1
p2f3 = p2c3*x + p2c2

p2f4 = expand(p2f1*p2f2*p2f3)
p2ans = (p2f1).mul(p2f2,p2f3,hold=true)



###### Problem p3
p3c1 = NonZeroInt(-10,10)
p3c2 = RandInt(-10,10)
p3c3 = RandInt(1,10)
while p3c3*p3c1^2>1000 or p3c1^2+p3c2>1000:
    p3c1 = RandInt(-10,10)
    p3c2 = RandInt(-10,10)
    p3c3 = RandInt(1,10)

p3f1 = x + p3c1
p3f2 = x - p3c1
p3f3 = p3c3*x + p3c2

p3f4 = expand(p3f1*p3f2*p3f3)
p3ans = (p3f1).mul(p3f2,p3f3,hold=true)



###### Problem p4
p4c1 = NonZeroInt(-10,10)
p4c2 = RandInt(-10,10)
p4c3 = RandInt(1,10)
while p4c3*p4c1^2>1000 or p4c1^2+p4c2>1000:
    p4c1 = RandInt(-10,10)
    p4c2 = RandInt(-10,10)
    p4c3 = RandInt(1,10)

p4f1 = x + p4c1
p4f2 = x - p4c1
p4f3 = p4c3*x + p4c2

p4f4 = expand(p4f1*p4f2*p4f3)
p4ans = (p4f1).mul(p4f2,p4f3,hold=true)



###### Problem p5
p5c1 = NonZeroInt(-5,5)
p5c2 = NonZeroInt(-5,5)
p5c3 = NonZeroInt(-6,6)
p5c4 = RandInt(1,7)
while abs(p5c3)*p5c1^2*p5c2^2>2000 or gcd(p5c3,p5c4)>2:
    p5c1 = RandInt(-10,10)
    p5c2 = RandInt(-10,10)
    p5c3 = RandInt(1,10)


p5f1 = x + p5c1
p5f2 = x - p5c1
p5f3 = x + p5c2
p5f4 = x - p5c2
p5f5 = p5c4*x + p5c3

p5f6 = expand(p5f1*p5f2*p5f3*p5f4*p5f5)
p5ans = (p5f1).mul(p5f2,p5f3,p5f4,p5f5,hold=true)




\end{sagesilent}


\begin{javascript}
// A validator to check and verify something has a factored form...
function factorCheck(f,g) {
    // This validator is designed to check that a student is submitting a factored polynomial. It works by:
    //  Checking that there are the correct number of non-numeric and non-inverse factors as expected,
    //  Checking that the submitted answer and the expected answer are the same via real Xronos evaluation,
    //  Checking that the outer most (last to be computed when following order of operations) operation is multiplication.
    
    var operCheck = f.tree[0];// Check to see if the root operation is multiplication at end.
    var studentFactors = f.tree.length;// Temporary number of student-provided factors (+1 because of root operation)
    
    // Now we adjust the length to remove any numeric factors, or division factors, etc to avoid ``padding'' by students.
    for (var i = 0; i < f.tree.length; i++) {
        if ((typeof f.tree[i] === 'number')||(f.tree[i][0] == '-')||(f.tree[i][0] == '/')) {
            studentFactors = studentFactors - 1;
        }
    }
    
    // Now we do the same with the provided answer, in case sage or something provides a weird format.
    var answerFactors = g.tree.length;
    
    // Adjust length in the same way, so that it will match the students if it should.
    for (var i = 0; i < g.tree.length; i++) {
        if (typeof g.tree[i] === 'number') {
            answerFactors = answerFactors - 1;
        }
    }
    
    // Note: An especially dedicated student could pad with weird factors that are happen to cancel down to 1.
    // For example, a student could enter sin^2(x)+cos^2(x) as a multiplicative factor to pad the number of factors.
    // This would be somewhat difficult to think of, even on purpose.
    // Until I can reliably evaluate the factors themselves as functions though, there isn't a lot to be done here.
    
    return ((f.equals(g))&&(studentFactors==answerFactors)&&(operCheck=='*'))
    }
\end{javascript}

\textbf{Note:} Make sure to \textbf{fully} factor each of the below. Remember that, factor by grouping is just \textbf{one} step in the factoring process, you should check any resulting factors you get after factor by grouping (or any other factoring method) to see if they are still factorable.%This is using an experimental factoring validator. If you verified that your answer should be correct and Xronos won't take it, please email your instructor to see if there is a problem.


\begin{problem}
    Factor the following polynomial:
    \[
        \sage{p1f4} = \answer[validator=factorCheck]{\sage{p1ans}}
    \]
    \begin{feedback}
        You want to factor by grouping, so you want to group up terms into pairs and then factor out any common terms from each pair. For example, if you had $9x^3 + 8x^2 - 144x - 128$ then you could group it as $(9x^3+8x^2) + (-144x -128)$, then factor out any common factors in each group; $x^2(9x+8) + -16(9x+8)$, then pull out the common term in the parentheses (only works if they are the same!) to get $(9x+8)(x^2-16)$. This isn't the end though; you need to fully factor, so you need to check both terms to see if either is factorable, in this case $(x^2-16)$ is factorable to $(x-4)(x+4)$ which gets you $(9x+8)(x-4)(x+4)$ as your final factoring.
    \end{feedback}
\end{problem}

\begin{problem}
    Factor the following polynomial:
    \[
        \sage{p2f4} = \answer[validator=factorCheck]{\sage{p2ans}}
    \]
    \begin{feedback}
        You want to factor by grouping, so you want to group up terms into pairs and then factor out any common terms from each pair. For example, if you had $9x^3 + 8x^2 - 144x - 128$ then you could group it as $(9x^3+8x^2) + (-144x -128)$, then factor out any common factors in each group; $x^2(9x+8) + -16(9x+8)$, then pull out the common term in the parentheses (only works if they are the same!) to get $(9x+8)(x^2-16)$. This isn't the end though; you need to fully factor, so you need to check both terms to see if either is factorable, in this case $(x^2-16)$ is factorable to $(x-4)(x+4)$ which gets you $(9x+8)(x-4)(x+4)$ as your final factoring.
    \end{feedback}
\end{problem}

\begin{problem}
    Factor the following polynomial:
    \[
        \sage{p3f4} = \answer[validator=factorCheck]{\sage{p3ans}}
    \]
    \begin{feedback}
        You want to factor by grouping, so you want to group up terms into pairs and then factor out any common terms from each pair. For example, if you had $9x^3 + 8x^2 - 144x - 128$ then you could group it as $(9x^3+8x^2) + (-144x -128)$, then factor out any common factors in each group; $x^2(9x+8) + -16(9x+8)$, then pull out the common term in the parentheses (only works if they are the same!) to get $(9x+8)(x^2-16)$. This isn't the end though; you need to fully factor, so you need to check both terms to see if either is factorable, in this case $(x^2-16)$ is factorable to $(x-4)(x+4)$ which gets you $(9x+8)(x-4)(x+4)$ as your final factoring.
    \end{feedback}
\end{problem}

\begin{problem}
    Factor the following polynomial:
    \[
        \sage{p4f4} = \answer[validator=factorCheck]{\sage{p4ans}}
    \]
    \begin{feedback}
        You want to factor by grouping, so you want to group up terms into pairs and then factor out any common terms from each pair. For example, if you had $9x^3 + 8x^2 - 144x - 128$ then you could group it as $(9x^3+8x^2) + (-144x -128)$, then factor out any common factors in each group; $x^2(9x+8) + -16(9x+8)$, then pull out the common term in the parentheses (only works if they are the same!) to get $(9x+8)(x^2-16)$. This isn't the end though; you need to fully factor, so you need to check both terms to see if either is factorable, in this case $(x^2-16)$ is factorable to $(x-4)(x+4)$ which gets you $(9x+8)(x-4)(x+4)$ as your final factoring.
    \end{feedback}
\end{problem}

\begin{problem}
    Factor the following polynomial:
    \[
        \sage{p5f6} = \answer[validator=factorCheck]{\sage{p5ans}}
    \]
    \begin{feedback}
        You want to factor by grouping, so you want to group up terms into pairs and then factor out any common terms from each pair. For example, if you had $9x^3 + 8x^2 - 144x - 128$ then you could group it as $(9x^3+8x^2) + (-144x -128)$, then factor out any common factors in each group; $x^2(9x+8) + -16(9x+8)$, then pull out the common term in the parentheses (only works if they are the same!) to get $(9x+8)(x^2-16)$. This isn't the end though; you need to fully factor, so you need to check both terms to see if either is factorable, in this case $(x^2-16)$ is factorable to $(x-4)(x+4)$ which gets you $(9x+8)(x-4)(x+4)$ as your final factoring.
    \end{feedback}
\end{problem}


\end{document}