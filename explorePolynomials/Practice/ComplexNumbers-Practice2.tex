\documentclass{ximera}
\title{Factor Coefficients Method Practice 1}



\begin{document}
\begin{sagesilent}

######  Define a function to convert a sage number into a saved counter number.

#####Define default Sage variables.
#Default function variables
var('x,y,z,X,Y,Z')
#Default function names
var('f,g,h,dx,dy,dz,dh,df')
#Default Wild cards
w0 = SR.wild(0)

def DispSign(b):
    """ Returns the string of the 'signed' version of `b`, e.g. 3 -> "+3", -3 -> "-3", 0 -> "".
    """
    if b == 0:
        return ""
    elif b > 0:
        return "+" + str(b)
    elif b < 0:
        return str(b)
    else:
        # If we're here, then something has gone wrong.
        raise ValueError

def ISP(b):
    return DispSign(b)

def NoEval(f, c):
    # TODO
    """ Returns a non-evaluted version of the result f(c).
    """
    cStr = str(c)
    # fLatex = latex(f)
    fString = latex(f)
    fStrList = list(fString)
    length = len(fStrList)
    fStrList2 = range(length)
    for i in range(0, length):
        if fStrList[i] == "x":
            fStrList2[i] = "("+cstr+")"
        else:
            fStrList2[i] = fStrList[i]
    f2 = join(fStrList2,"")
    return LatexExpr(f2)

def HyperSimp(f):
    """ Returns the expression `f` without hyperbolic expressions.
    """
    subsDict = {
        sinh(w0) : (exp(w0) - exp(-w0))/2,
        cosh(w0) : (exp(w0) + exp(-w0))/2,
        tanh(w0) : (exp(w0) - exp(-w0))/(exp(w0) + exp(-w0)),
        sech(w0) : 2/(exp(w0) + exp(-w0)),                      # This seems to work, but Nowell said it didn't at one point.
        csch(w0) : 2/(exp(w0) - exp(-w0)),                      # This seems to work, but Nowell said it didn't at one point.
        coth(w0) : (exp(w0) + exp(-w0))/(exp(w0) - exp(-w0)),   # This seems to work, but Nowell said it didn't at one point.
        arcsinh(w0) :       ln( w0 + sqrt((w0)^2 + 1) ),
        arccosh(w0) :       ln( w0 + sqrt((w0)^2 - 1) ),
        arctanh(w0) : 1/2 * ln( (1 + w0) / (1 - w0) ),
        arccsch(w0) :       ln( (1 + sqrt((w0)^2 + 1))/w0 ),
        arcsech(w0) :       ln( (1 + sqrt(1 - (w0)^2))/w0 ),
        arccoth(w0) : 1/2 * ln( (1 + w0) / (w0 - 1) )
    }
    g = f.substitute(subsDict)
    return simplify(g)

def RandInt(a,b):
    """ Returns a random integer in [`a`,`b`]. Note that `a` and `b` should be integers themselves to avoid unexpected behavior.
    """
    return QQ(randint(int(a),int(b)))
    # return choice(range(a,b+1))

def NonZeroInt(b,c, avoid = [0]):
    """ Returns a random integer in [`b`,`c`] which is not in `av`. 
        If `av` is not specified, defaults to a non-zero integer.
    """
    while True:
        a = RandInt(b,c)
        if a not in avoid:
            return a

def RandVector(b, c, avoid=[], rep=1):
    """ Returns essentially a multiset permutation of ([b,c]-av) * rep.
        That is, a vector which contains each integer in [`b`,`c`] which is not in `av` a total of `rep` number of times.
        Example:
        sage: RandVector(1,3, [2], 2)
        [3, 1, 1, 3]
    """
    oneVec = [val for val in range(b,c+1) if val not in avoid]
    vec = oneVec * rep
    shuffle(vec)
    return vec

def fudge(b):
    up = b+RandInt(2,5)/10
    down = b-RandInt(2,5)/10
    fudgebup = round(up,1)
    fudgebdown = round(down,1)
    fudgedb = [fudgebdown,fudgebup]
    return fudgedb

def disjointCheck(checkvec):
    if length(uniq(checkvec)) < length(checkvec):
        return 1
    else:
        return 0

def disjointIntervals(IntStart,IntEnd,CheckVal):
    if IntStart < CheckVal and CheckVal < IntEnd:
        return 1
    else:
        return 0

def IntervalVecCheck(checkVec):
    veclen = len(checkVec)
    returnval = 0
    for i in range(veclen):
        for j in range(veclen):
            if (disjointIntervals(checkVec[j][0],checkVec[j][1],checkVec[i][0]) + disjointIntervals(checkVec[j][0],checkVec[j][1],checkVec[i][1])) > 0:
                returnval = returnval + 1
    if returnval > 0:
        return 1
    else:
        return 0



\end{sagesilent}

\begin{sagesilent}
def RandInt(a,b):
    """ Returns a random integer in [`a`,`b`]. Note that `a` and `b` should be integers themselves to avoid unexpected behavior.
    """
    return QQ(randint(int(a),int(b)))
    # return choice(range(a,b+1))

def NonZeroInt(b,c, avoid = [0]):
    """ Returns a random integer in [`b`,`c`] which is not in `av`. 
        If `av` is not specified, defaults to a non-zero integer.
    """
    while True:
        a = RandInt(b,c)
        if a not in avoid:
            return a

def cpx(z):
    if z == 0:
        return LatexExpr('0')
    a = z.real()
    b = z.imag()
    if (a == 0) or (b==0) :
        return LatexExpr(latex(z))
    elif b > 0:
      s = '+'
    else:
      s = '-'
    return latex(a) + LatexExpr(s) + latex(abs(b) * i)



### Problem p1
p1c1 = RandInt(-5,5)
p1c2 = RandInt(-5,5)
p1c3 = RandInt(-5,5)
p1c4 = NonZeroInt(-5,5)

p1complex1 = p1c1 + p1c2*i
p1complex2 = p1c3 + p1c4*i

p1complex1d = cpx(p1c1 + p1c2*i)
p1complex2d = cpx(p1c3 + p1c4*i)

p1ans1 = (p1complex1/p1complex2).real()
p1ans2 = (p1complex1/p1complex2).imag()

p1h1 = cpx(p1complex2.conjugate())



### Problem p2
p2c1 = RandInt(-5,5)
p2c2 = RandInt(-5,5)
p2c3 = RandInt(-5,5)
p2c4 = NonZeroInt(-5,5)

p2complex1 = p2c1 + p2c2*i
p2complex2 = p2c3 + p2c4*i

p2complex1d = cpx(p2c1 + p2c2*i)
p2complex2d = cpx(p2c3 + p2c4*i)

p2ans1 = (p2complex1/p2complex2).real()
p2ans2 = (p2complex1/p2complex2).imag()

p2h1 = cpx(p2complex2.conjugate())


### Problem p3
p3c1 = RandInt(-5,5)
p3c2 = RandInt(-5,5)
p3c3 = RandInt(-5,5)
p3c4 = NonZeroInt(-5,5)

p3complex1 = p3c1 + p3c2*i
p3complex2 = p3c3 + p3c4*i

p3complex1d = cpx(p3c1 + p3c2*i)
p3complex2d = cpx(p3c3 + p3c4*i)

p3ans1 = (p3complex1/p3complex2).real()
p3ans2 = (p3complex1/p3complex2).imag()

p3h1 = cpx(p3complex2.conjugate())


### Problem p4
p4c1 = RandInt(-5,5)
p4c2 = RandInt(-5,5)
p4c3 = RandInt(-5,5)
p4c4 = NonZeroInt(-5,5)

p4complex1 = p4c1 + p4c2*i
p4complex2 = p4c3 + p4c4*i

p4complex1d = cpx(p4c1 + p4c2*i)
p4complex2d = cpx(p4c3 + p4c4*i)

p4ans1 = (p4complex1/p4complex2).real()
p4ans2 = (p4complex1/p4complex2).imag()

p4h1 = cpx(p4complex2.conjugate())




\end{sagesilent}


\begin{problem}
    Simplify the following complex expression into standard form.
    \[
        \frac{\sage{p1complex1d}}{\sage{p1complex2d}} = \answer{\sage{p1ans1}} + \answer{\sage{p1ans2}}\cdot i
    \]
    \begin{feedback}
        Remember to multiply the top and bottom of the fraction by the conjugate of the bottom; i.e. $\sage{p1h1}$ and foil out. This should get you a real denominator.
        
        Also notice that the ``$i$'' is already provided, so you don't need to type into your answer.
    \end{feedback}
    
\end{problem}


\begin{problem}
    Simplify the following complex expression into standard form.
    \[
        \frac{\sage{p2complex1d}}{\sage{p2complex2d}} = \answer{\sage{p2ans1}} + \answer{\sage{p2ans2}}\cdot i
    \]
    \begin{feedback}
        Remember to multiply the top and bottom of the fraction by the conjugate of the bottom; i.e. $\sage{p2h1}$ and foil out. This should get you a real denominator.
        
        Also notice that the ``$i$'' is already provided, so you don't need to type into your answer.
    \end{feedback}
    
\end{problem}


\begin{problem}
    Simplify the following complex expression into standard form.
    \[
        \frac{\sage{p3complex1d}}{\sage{p3complex2d}} = \answer{\sage{p3ans1}} + \answer{\sage{p3ans2}}\cdot i
    \]
    \begin{feedback}
        Remember to multiply the top and bottom of the fraction by the conjugate of the bottom; i.e. $\sage{p3h1}$ and foil out. This should get you a real denominator.
        
        Also notice that the ``$i$'' is already provided, so you don't need to type into your answer.
    \end{feedback}
    
\end{problem}


\begin{problem}
    Simplify the following complex expression into standard form.
    \[
        \frac{\sage{p4complex1d}}{\sage{p4complex2d}} = \answer{\sage{p4ans1}} + \answer{\sage{p4ans2}}\cdot i
    \]
    \begin{feedback}
        Remember to multiply the top and bottom of the fraction by the conjugate of the bottom; i.e. $\sage{p4h1}$ and foil out. This should get you a real denominator.
        
        Also notice that the ``$i$'' is already provided, so you don't need to type into your answer.
    \end{feedback}
    
\end{problem}



\end{document}