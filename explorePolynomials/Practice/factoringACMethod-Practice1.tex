\documentclass{ximera}
\title{Factor Coefficients Method Practice 1}
\begin{document}
\begin{sagesilent}

######  Define a function to convert a sage number into a saved counter number.

#####Define default Sage variables.
#Default function variables
var('x,y,z,X,Y,Z')
#Default function names
var('f,g,h,dx,dy,dz,dh,df')
#Default Wild cards
w0 = SR.wild(0)

def DispSign(b):
    """ Returns the string of the 'signed' version of `b`, e.g. 3 -> "+3", -3 -> "-3", 0 -> "".
    """
    if b == 0:
        return ""
    elif b > 0:
        return "+" + str(b)
    elif b < 0:
        return str(b)
    else:
        # If we're here, then something has gone wrong.
        raise ValueError

def ISP(b):
    return DispSign(b)

def NoEval(f, c):
    # TODO
    """ Returns a non-evaluted version of the result f(c).
    """
    cStr = str(c)
    # fLatex = latex(f)
    fString = latex(f)
    fStrList = list(fString)
    length = len(fStrList)
    fStrList2 = range(length)
    for i in range(0, length):
        if fStrList[i] == "x":
            fStrList2[i] = "("+cstr+")"
        else:
            fStrList2[i] = fStrList[i]
    f2 = join(fStrList2,"")
    return LatexExpr(f2)

def HyperSimp(f):
    """ Returns the expression `f` without hyperbolic expressions.
    """
    subsDict = {
        sinh(w0) : (exp(w0) - exp(-w0))/2,
        cosh(w0) : (exp(w0) + exp(-w0))/2,
        tanh(w0) : (exp(w0) - exp(-w0))/(exp(w0) + exp(-w0)),
        sech(w0) : 2/(exp(w0) + exp(-w0)),                      # This seems to work, but Nowell said it didn't at one point.
        csch(w0) : 2/(exp(w0) - exp(-w0)),                      # This seems to work, but Nowell said it didn't at one point.
        coth(w0) : (exp(w0) + exp(-w0))/(exp(w0) - exp(-w0)),   # This seems to work, but Nowell said it didn't at one point.
        arcsinh(w0) :       ln( w0 + sqrt((w0)^2 + 1) ),
        arccosh(w0) :       ln( w0 + sqrt((w0)^2 - 1) ),
        arctanh(w0) : 1/2 * ln( (1 + w0) / (1 - w0) ),
        arccsch(w0) :       ln( (1 + sqrt((w0)^2 + 1))/w0 ),
        arcsech(w0) :       ln( (1 + sqrt(1 - (w0)^2))/w0 ),
        arccoth(w0) : 1/2 * ln( (1 + w0) / (w0 - 1) )
    }
    g = f.substitute(subsDict)
    return simplify(g)

def RandInt(a,b):
    """ Returns a random integer in [`a`,`b`]. Note that `a` and `b` should be integers themselves to avoid unexpected behavior.
    """
    return QQ(randint(int(a),int(b)))
    # return choice(range(a,b+1))

def NonZeroInt(b,c, avoid = [0]):
    """ Returns a random integer in [`b`,`c`] which is not in `av`. 
        If `av` is not specified, defaults to a non-zero integer.
    """
    while True:
        a = RandInt(b,c)
        if a not in avoid:
            return a

def RandVector(b, c, avoid=[], rep=1):
    """ Returns essentially a multiset permutation of ([b,c]-av) * rep.
        That is, a vector which contains each integer in [`b`,`c`] which is not in `av` a total of `rep` number of times.
        Example:
        sage: RandVector(1,3, [2], 2)
        [3, 1, 1, 3]
    """
    oneVec = [val for val in range(b,c+1) if val not in avoid]
    vec = oneVec * rep
    shuffle(vec)
    return vec

def fudge(b):
    up = b+RandInt(2,5)/10
    down = b-RandInt(2,5)/10
    fudgebup = round(up,1)
    fudgebdown = round(down,1)
    fudgedb = [fudgebdown,fudgebup]
    return fudgedb

def disjointCheck(checkvec):
    if length(uniq(checkvec)) < length(checkvec):
        return 1
    else:
        return 0

def disjointIntervals(IntStart,IntEnd,CheckVal):
    if IntStart < CheckVal and CheckVal < IntEnd:
        return 1
    else:
        return 0

def IntervalVecCheck(checkVec):
    veclen = len(checkVec)
    returnval = 0
    for i in range(veclen):
        for j in range(veclen):
            if (disjointIntervals(checkVec[j][0],checkVec[j][1],checkVec[i][0]) + disjointIntervals(checkVec[j][0],checkVec[j][1],checkVec[i][1])) > 0:
                returnval = returnval + 1
    if returnval > 0:
        return 1
    else:
        return 0



\end{sagesilent}

\begin{sagesilent}
###### Problem p1
p1c1 = RandInt(-10,10)
p1c2 = RandInt(-10,10)
p1c3 = NonZeroInt(-5,5)
p1c4 = RandInt(1,10)
while abs(p1c1*p1c2*p1c3*p1c4) > 200:
    p1c1 = RandInt(-10,10)
    p1c2 = RandInt(-10,10)
    p1c3 = NonZeroInt(-5,5)
    p1c4 = RandInt(1,10)

p1f1 = p1c3*x + p1c1
p1f2 = p1c4*x + p1c2

p1f3 = expand(p1f1*p1f2)
p1ans = (p1f1).mul(p1f2, hold=true)


###### Problem p2
p2c1 = RandInt(-10,10)
p2c2 = RandInt(-10,10)
p2c3 = NonZeroInt(-5,5)
p2c4 = RandInt(1,10)
while abs(p2c1*p2c2*p2c3*p2c4) > 200:
    p2c1 = RandInt(-10,10)
    p2c2 = RandInt(-10,10)
    p2c3 = NonZeroInt(-5,5)
    p2c4 = RandInt(1,10)

p2f1 = p2c3*x + p2c1
p2f2 = p2c4*x + p2c2

p2f3 = expand(p2f1*p2f2)
p2ans = (p2f1).mul(p2f2, hold=true)


###### Problem p3
p3c1 = RandInt(-10,10)
p3c2 = RandInt(-10,10)
p3c3 = NonZeroInt(-5,5)
p3c4 = RandInt(1,10)
while abs(p3c1*p3c2*p3c3*p3c4) > 200:
    p3c1 = RandInt(-10,10)
    p3c2 = RandInt(-10,10)
    p3c3 = NonZeroInt(-5,5)
    p3c4 = RandInt(1,10)

p3f1 = p3c3*x + p3c1
p3f2 = p3c4*x + p3c2

p3f3 = expand(p3f1*p3f2)
p3ans = (p3f1).mul(p3f2, hold=true)


###### Problem p4
p4c1 = RandInt(-10,10)
p4c2 = RandInt(-10,10)
p4c3 = NonZeroInt(-5,5)
p4c4 = RandInt(1,10)
while abs(p4c1*p4c2*p4c3*p4c4) > 200:
    p4c1 = RandInt(-10,10)
    p4c2 = RandInt(-10,10)
    p4c3 = NonZeroInt(-5,5)
    p4c4 = RandInt(1,10)

p4f1 = p4c3*x + p4c1
p4f2 = p4c4*x + p4c2

p4f3 = expand(p4f1*p4f2)
p4ans = (p4f1).mul(p4f2, hold=true)


\end{sagesilent}

\begin{javascript}
// A validator to check and verify something has a factored form...
function factorCheck(f,g) {
    // This validator is designed to check that a student is submitting a factored polynomial. It works by:
    //  Checking that there are the correct number of non-numeric and non-inverse factors as expected,
    //  Checking that the submitted answer and the expected answer are the same via real Xronos evaluation,
    //  Checking that the outer most (last to be computed when following order of operations) operation is multiplication.
    
    var operCheck = f.tree[0];// Check to see if the root operation is multiplication at end.
    var studentFactors = f.tree.length;// Temporary number of student-provided factors (+1 because of root operation)
    
    // Now we adjust the length to remove any numeric factors, or division factors, etc to avoid ``padding'' by students.
    for (var i = 0; i < f.tree.length; i++) {
        if ((typeof f.tree[i] === 'number')||(f.tree[i][0] == '-')||(f.tree[i][0] == '/')) {
            studentFactors = studentFactors - 1;
        }
    }
    
    // Now we do the same with the provided answer, in case sage or something provides a weird format.
    var answerFactors = g.tree.length;
    
    // Adjust length in the same way, so that it will match the students if it should.
    for (var i = 0; i < g.tree.length; i++) {
        if (typeof g.tree[i] === 'number') {
            answerFactors = answerFactors - 1;
        }
    }
    console.log('student input tree...');
    console.log(f.tree);
    
    console.log('expected tree.');
    console.log(g.tree);
    // Note: An especially dedicated student could pad with weird factors that are happen to cancel down to 1.
    // For example, a student could enter sin^2(x)+cos^2(x) as a multiplicative factor to pad the number of factors.
    // This would be somewhat difficult to think of, even on purpose.
    // Until I can reliably evaluate the factors themselves as functions though, there isn't a lot to be done here.
    
    return ((f.equals(g))&&(studentFactors==answerFactors)&&(operCheck=='*'))
    }
\end{javascript}

\textbf{Note:} This is using an experimental factoring validator. If you verified that your answer should be correct and Xronos won't take it, please email your instructor to see if there is a problem.

\begin{problem}
    Factor the following quadratic using the AC-Method.
    \[
        p(x) = \sage{p1f3} = \answer[validator=factorCheck]{\sage{p1ans}}
    \]
    \begin{feedback}
        Remember to multiply the A term and the B term and try to find values that multiple to that new value ($\sage{p1c1*p1c2*p1c3*p1c4}$), but add to $\sage{p1c3*p1c2+p1c1*p1c4}$. Use those to numbers to split the middle term and then factor by grouping (as shown in the video).
    \end{feedback}
\end{problem}



\begin{problem}
    Factor the following quadratic using the AC-Method
    \[
        p(x) = \sage{p2f3} = \answer[validator=factorCheck]{\sage{p2ans}}
    \]
    \begin{feedback}
        Remember to multiply the A term and the B term and try to find values that multiple to that new value ($\sage{p2c1*p2c2*p2c3*p2c4}$), but add to $\sage{p2c3*p2c2+p2c1*p2c4}$. Use those to numbers to split the middle term and then factor by grouping (as shown in the video).
    \end{feedback}
\end{problem}


\begin{problem}
    Factor the following quadratic using the AC-Method
    \[
        p(x) = \sage{p3f3} = \answer[validator=factorCheck]{\sage{p3ans}}
    \]
    \begin{feedback}
        Remember to multiply the A term and the B term and try to find values that multiple to that new value ($\sage{p3c1*p3c2*p3c3*p3c4}$), but add to $\sage{p3c3*p3c2+p3c1*p3c4}$. Use those to numbers to split the middle term and then factor by grouping (as shown in the video).
    \end{feedback}
\end{problem}



\begin{problem}
    Factor the following quadratic using the AC-Method
    \[
        p(x) = \sage{p4f3} = \answer[validator=factorCheck]{\sage{p4ans}}
    \]
    \begin{feedback}
        Remember to multiply the A term and the B term and try to find values that multiple to that new value ($\sage{p4c1*p4c2*p4c3*p4c4}$), but add to $\sage{p4c3*p4c2+p4c1*p4c4}$. Use those to numbers to split the middle term and then factor by grouping (as shown in the video).
    \end{feedback}
\end{problem}




\end{document}