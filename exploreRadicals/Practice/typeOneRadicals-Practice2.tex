\documentclass{ximera}
\title{Simplifying Algebraic Radicals: Practice 1}


\begin{document}


\begin{sagesilent}
var('x')

def RandInt(a,b):
    """ Returns a random integer in [`a`,`b`]. Note that `a` and `b` should be integers themselves to avoid unexpected behavior.
    """
    return QQ(randint(int(a),int(b)))
    # return choice(range(a,b+1))

def NonZeroInt(b,c, avoid = [0]):
    """ Returns a random integer in [`b`,`c`] which is not in `av`. 
        If `av` is not specified, defaults to a non-zero integer.
    """
    while True:
        a = RandInt(b,c)
        if a not in avoid:
            return a


#### Note: Because Sage does everything in the complex plane, 
#           it cannot be trusted to simplify radicals in any sane way for a precalc student.
#           Thus we will need to manually build out the correct answer.

###### Problem p1
### Decide on the root value.
p1rootVal = RandInt(2,9)

### Build coefficients. In order to avoid crazy leading coefficients it is useful to ensure that there are no common factors.
p1c1 = 2
p1c2 = 2
p1c3 = 2
p1c4 = 2
p1c5 = 2
p1c6 = 2
p1c7 = 2
p1c8 = 2
while abs(p1c2/p1c1)==abs(p1c4/p1c3) or abs(p1c2/p1c1)==abs(p1c6/p1c5) or abs(p1c2/p1c1)==abs(p1c8/p1c7) or abs(p1c4/p1c3)==abs(p1c6/p1c5) or abs(p1c4/p1c3)==abs(p1c8/p1c7) or abs(p1c6/p1c5)==abs(p1c8/p1c7):
    p1c1 = 2
    p1c2 = 2
    p1c3 = 2
    p1c4 = 2
    p1c5 = 2
    p1c6 = 2
    p1c7 = 2
    p1c8 = 2
    while gcd(p1c1,p1c2)>1:
        p1c1 = RandInt(1,5)
        p1c2 = RandInt(-10,10)
    while gcd(p1c3,p1c4)>1:
        p1c3 = RandInt(1,5)
        p1c4 = RandInt(-10,10)
    while gcd(p1c5,p1c6)>1:
        p1c5 = RandInt(1,5)
        p1c6 = RandInt(-10,10)
    while gcd(p1c7,p1c8)>1:
        p1c7 = RandInt(1,5)
        p1c8 = RandInt(-10,10)

### Choose the remainder power for each factor; the part that remains in radicand after simplifying.
p1f1remainPwr = RandInt(0,p1rootVal-1)
p1f2remainPwr = RandInt(0,p1rootVal-1)
p1f3remainPwr = RandInt(0,p1rootVal-1)
p1f4remainPwr = RandInt(0,p1rootVal-1)

### Choose the factorable power; the part that will be removed from the radicand during simplifying.
p1f1factorPwr = RandInt(0,5)
p1f2factorPwr = RandInt(0,5)
p1f3factorPwr = RandInt(0,5)
p1f4factorPwr = RandInt(0,5)

### Build individual function elements.
p1f1remain = (p1c1*x-p1c2)^p1f1remainPwr
p1f2remain = (p1c3*x-p1c4)^p1f2remainPwr
p1f3remain = (p1c5*x-p1c6)^p1f3remainPwr
p1f4remain = (p1c7*x-p1c8)^p1f4remainPwr

p1f1total = (p1c1*x-p1c2)^(p1f1remainPwr+p1f1factorPwr*p1rootVal)
p1f2total = (p1c3*x-p1c4)^(p1f2remainPwr+p1f2factorPwr*p1rootVal)
p1f3total = (p1c5*x-p1c6)^(p1f3remainPwr+p1f3factorPwr*p1rootVal)
p1f4total = (p1c7*x-p1c8)^(p1f4remainPwr+p1f4factorPwr*p1rootVal)

p1fDisp = p1f1total*p1f2total*p1f3total*p1f4total

p1fRemainder = p1f1remain*p1f2remain*p1f3remain*p1f4remain

### The fun part; writing out the factored piece, which depends on things like if they need absolute value or not.
if p1rootVal%2==0:
    if p1f1factorPwr%2==0:
        p1f1factored = (p1c1*x-p1c2)^p1f1factorPwr
    else:
        p1f1factored = abs(p1c1*x-p1c2)^p1f1factorPwr
    if p1f2factorPwr%2==0:
        p1f2factored = (p1c3*x-p1c4)^p1f2factorPwr
    else:
        p1f2factored = abs(p1c3*x-p1c4)^p1f2factorPwr
    if p1f3factorPwr%2==0:
        p1f3factored = (p1c5*x-p1c6)^p1f3factorPwr
    else:
        p1f3factored = abs(p1c5*x-p1c6)^p1f3factorPwr
    if p1f4factorPwr%2==0:
        p1f4factored = (p1c7*x-p1c8)^p1f4factorPwr
    else:
        p1f4factored = abs(p1c7*x-p1c8)^p1f4factorPwr
    p1fFactored = p1f1factored*p1f2factored*p1f3factored*p1f4factored
else:
    p1fFactored = (p1c1*x-p1c2)^p1f1factorPwr*(p1c3*x-p1c4)^p1f2factorPwr*(p1c5*x-p1c6)^p1f3factorPwr*(p1c7*x-p1c8)^p1f4factorPwr






###### Problem p2
### Decide on the root value.
p2rootVal = RandInt(2,9)

### Build coefficients. In order to avoid crazy leading coefficients it is useful to ensure that there are no common factors.
p2c1 = 2
p2c2 = 2
p2c3 = 2
p2c4 = 2
p2c5 = 2
p2c6 = 2
p2c7 = 2
p2c8 = 2
while abs(p2c2/p2c1)==abs(p2c4/p2c3) or abs(p2c2/p2c1)==abs(p2c6/p2c5) or abs(p2c2/p2c1)==abs(p2c8/p2c7) or abs(p2c4/p2c3)==abs(p2c6/p2c5) or abs(p2c4/p2c3)==abs(p2c8/p2c7) or abs(p2c6/p2c5)==abs(p2c8/p2c7):
    p2c1 = 2
    p2c2 = 2
    p2c3 = 2
    p2c4 = 2
    p2c5 = 2
    p2c6 = 2
    p2c7 = 2
    p2c8 = 2
    while gcd(p2c1,p2c2)>1:
        p2c1 = RandInt(1,5)
        p2c2 = RandInt(-10,10)
    while gcd(p2c3,p2c4)>1:
        p2c3 = RandInt(1,5)
        p2c4 = RandInt(-10,10)
    while gcd(p2c5,p2c6)>1:
        p2c5 = RandInt(1,5)
        p2c6 = RandInt(-10,10)
    while gcd(p2c7,p2c8)>1:
        p2c7 = RandInt(1,5)
        p2c8 = RandInt(-10,10)

### Choose the remainder power for each factor; the part that remains in radicand after simplifying.
p2f1remainPwr = RandInt(0,p2rootVal-1)
p2f2remainPwr = RandInt(0,p2rootVal-1)
p2f3remainPwr = RandInt(0,p2rootVal-1)
p2f4remainPwr = RandInt(0,p2rootVal-1)

### Choose the factorable power; the part that will be removed from the radicand during simplifying.
p2f1factorPwr = RandInt(0,5)
p2f2factorPwr = RandInt(0,5)
p2f3factorPwr = RandInt(0,5)
p2f4factorPwr = RandInt(0,5)

### Build individual function elements.
p2f1remain = (p2c1*x-p2c2)^p2f1remainPwr
p2f2remain = (p2c3*x-p2c4)^p2f2remainPwr
p2f3remain = (p2c5*x-p2c6)^p2f3remainPwr
p2f4remain = (p2c7*x-p2c8)^p2f4remainPwr

p2f1total = (p2c1*x-p2c2)^(p2f1remainPwr+p2f1factorPwr*p2rootVal)
p2f2total = (p2c3*x-p2c4)^(p2f2remainPwr+p2f2factorPwr*p2rootVal)
p2f3total = (p2c5*x-p2c6)^(p2f3remainPwr+p2f3factorPwr*p2rootVal)
p2f4total = (p2c7*x-p2c8)^(p2f4remainPwr+p2f4factorPwr*p2rootVal)

p2fDisp = p2f1total*p2f2total*p2f3total*p2f4total

p2fRemainder = p2f1remain*p2f2remain*p2f3remain*p2f4remain

### The fun part; writing out the factored piece, which depends on things like if they need absolute value or not.
if p2rootVal%2==0:
    if p2f1factorPwr%2==0:
        p2f1factored = (p2c1*x-p2c2)^p2f1factorPwr
    else:
        p2f1factored = abs(p2c1*x-p2c2)^p2f1factorPwr
    if p2f2factorPwr%2==0:
        p2f2factored = (p2c3*x-p2c4)^p2f2factorPwr
    else:
        p2f2factored = abs(p2c3*x-p2c4)^p2f2factorPwr
    if p2f3factorPwr%2==0:
        p2f3factored = (p2c5*x-p2c6)^p2f3factorPwr
    else:
        p2f3factored = abs(p2c5*x-p2c6)^p2f3factorPwr
    if p2f4factorPwr%2==0:
        p2f4factored = (p2c7*x-p2c8)^p2f4factorPwr
    else:
        p2f4factored = abs(p2c7*x-p2c8)^p2f4factorPwr
    p2fFactored = p2f1factored*p2f2factored*p2f3factored*p2f4factored
else:
    p2fFactored = (p2c1*x-p2c2)^p2f1factorPwr*(p2c3*x-p2c4)^p2f2factorPwr*(p2c5*x-p2c6)^p2f3factorPwr*(p2c7*x-p2c8)^p2f4factorPwr





###### Problem p3
### Decide on the root value.
p3rootVal = RandInt(2,9)

### Build coefficients. In order to avoid crazy leading coefficients it is useful to ensure that there are no common factors.
p3c1 = 2
p3c2 = 2
p3c3 = 2
p3c4 = 2
p3c5 = 2
p3c6 = 2
p3c7 = 2
p3c8 = 2
while abs(p3c2/p3c1)==abs(p3c4/p3c3) or abs(p3c2/p3c1)==abs(p3c6/p3c5) or abs(p3c2/p3c1)==abs(p3c8/p3c7) or abs(p3c4/p3c3)==abs(p3c6/p3c5) or abs(p3c4/p3c3)==abs(p3c8/p3c7) or abs(p3c6/p3c5)==abs(p3c8/p3c7):
    p3c1 = 2
    p3c2 = 2
    p3c3 = 2
    p3c4 = 2
    p3c5 = 2
    p3c6 = 2
    p3c7 = 2
    p3c8 = 2
    while gcd(p3c1,p3c2)>1:
        p3c1 = RandInt(1,5)
        p3c2 = RandInt(-10,10)
    while gcd(p3c3,p3c4)>1:
        p3c3 = RandInt(1,5)
        p3c4 = RandInt(-10,10)
    while gcd(p3c5,p3c6)>1:
        p3c5 = RandInt(1,5)
        p3c6 = RandInt(-10,10)
    while gcd(p3c7,p3c8)>1:
        p3c7 = RandInt(1,5)
        p3c8 = RandInt(-10,10)

### Choose the remainder power for each factor; the part that remains in radicand after simplifying.
p3f1remainPwr = RandInt(0,p3rootVal-1)
p3f2remainPwr = RandInt(0,p3rootVal-1)
p3f3remainPwr = RandInt(0,p3rootVal-1)
p3f4remainPwr = RandInt(0,p3rootVal-1)

### Choose the factorable power; the part that will be removed from the radicand during simplifying.
p3f1factorPwr = RandInt(0,5)
p3f2factorPwr = RandInt(0,5)
p3f3factorPwr = RandInt(0,5)
p3f4factorPwr = RandInt(0,5)

### Build individual function elements.
p3f1remain = (p3c1*x-p3c2)^p3f1remainPwr
p3f2remain = (p3c3*x-p3c4)^p3f2remainPwr
p3f3remain = (p3c5*x-p3c6)^p3f3remainPwr
p3f4remain = (p3c7*x-p3c8)^p3f4remainPwr

p3f1total = (p3c1*x-p3c2)^(p3f1remainPwr+p3f1factorPwr*p3rootVal)
p3f2total = (p3c3*x-p3c4)^(p3f2remainPwr+p3f2factorPwr*p3rootVal)
p3f3total = (p3c5*x-p3c6)^(p3f3remainPwr+p3f3factorPwr*p3rootVal)
p3f4total = (p3c7*x-p3c8)^(p3f4remainPwr+p3f4factorPwr*p3rootVal)

p3fDisp = p3f1total*p3f2total*p3f3total*p3f4total

p3fRemainder = p3f1remain*p3f2remain*p3f3remain*p3f4remain

### The fun part; writing out the factored piece, which depends on things like if they need absolute value or not.
if p3rootVal%2==0:
    if p3f1factorPwr%2==0:
        p3f1factored = (p3c1*x-p3c2)^p3f1factorPwr
    else:
        p3f1factored = abs(p3c1*x-p3c2)^p3f1factorPwr
    if p3f2factorPwr%2==0:
        p3f2factored = (p3c3*x-p3c4)^p3f2factorPwr
    else:
        p3f2factored = abs(p3c3*x-p3c4)^p3f2factorPwr
    if p3f3factorPwr%2==0:
        p3f3factored = (p3c5*x-p3c6)^p3f3factorPwr
    else:
        p3f3factored = abs(p3c5*x-p3c6)^p3f3factorPwr
    if p3f4factorPwr%2==0:
        p3f4factored = (p3c7*x-p3c8)^p3f4factorPwr
    else:
        p3f4factored = abs(p3c7*x-p3c8)^p3f4factorPwr
    p3fFactored = p3f1factored*p3f2factored*p3f3factored*p3f4factored
else:
    p3fFactored = (p3c1*x-p3c2)^p3f1factorPwr*(p3c3*x-p3c4)^p3f2factorPwr*(p3c5*x-p3c6)^p3f3factorPwr*(p3c7*x-p3c8)^p3f4factorPwr





###### Problem p4
### Decide on the root value.
p4rootVal = RandInt(2,9)

### Build coefficients. In order to avoid crazy leading coefficients it is useful to ensure that there are no common factors.
p4c1 = 2
p4c2 = 2
p4c3 = 2
p4c4 = 2
p4c5 = 2
p4c6 = 2
p4c7 = 2
p4c8 = 2
while abs(p4c2/p4c1)==abs(p4c4/p4c3) or abs(p4c2/p4c1)==abs(p4c6/p4c5) or abs(p4c2/p4c1)==abs(p4c8/p4c7) or abs(p4c4/p4c3)==abs(p4c6/p4c5) or abs(p4c4/p4c3)==abs(p4c8/p4c7) or abs(p4c6/p4c5)==abs(p4c8/p4c7):
    p4c1 = 2
    p4c2 = 2
    p4c3 = 2
    p4c4 = 2
    p4c5 = 2
    p4c6 = 2
    p4c7 = 2
    p4c8 = 2
    while gcd(p4c1,p4c2)>1:
        p4c1 = RandInt(1,5)
        p4c2 = RandInt(-10,10)
    while gcd(p4c3,p4c4)>1:
        p4c3 = RandInt(1,5)
        p4c4 = RandInt(-10,10)
    while gcd(p4c5,p4c6)>1:
        p4c5 = RandInt(1,5)
        p4c6 = RandInt(-10,10)
    while gcd(p4c7,p4c8)>1:
        p4c7 = RandInt(1,5)
        p4c8 = RandInt(-10,10)

### Choose the remainder power for each factor; the part that remains in radicand after simplifying.
p4f1remainPwr = RandInt(0,p4rootVal-1)
p4f2remainPwr = RandInt(0,p4rootVal-1)
p4f3remainPwr = RandInt(0,p4rootVal-1)
p4f4remainPwr = RandInt(0,p4rootVal-1)

### Choose the factorable power; the part that will be removed from the radicand during simplifying.
p4f1factorPwr = RandInt(0,5)
p4f2factorPwr = RandInt(0,5)
p4f3factorPwr = RandInt(0,5)
p4f4factorPwr = RandInt(0,5)

### Build individual function elements.
p4f1remain = (p4c1*x-p4c2)^p4f1remainPwr
p4f2remain = (p4c3*x-p4c4)^p4f2remainPwr
p4f3remain = (p4c5*x-p4c6)^p4f3remainPwr
p4f4remain = (p4c7*x-p4c8)^p4f4remainPwr

p4f1total = (p4c1*x-p4c2)^(p4f1remainPwr+p4f1factorPwr*p4rootVal)
p4f2total = (p4c3*x-p4c4)^(p4f2remainPwr+p4f2factorPwr*p4rootVal)
p4f3total = (p4c5*x-p4c6)^(p4f3remainPwr+p4f3factorPwr*p4rootVal)
p4f4total = (p4c7*x-p4c8)^(p4f4remainPwr+p4f4factorPwr*p4rootVal)

p4fDisp = p4f1total*p4f2total*p4f3total*p4f4total

p4fRemainder = p4f1remain*p4f2remain*p4f3remain*p4f4remain

### The fun part; writing out the factored piece, which depends on things like if they need absolute value or not.
if p4rootVal%2==0:
    if p4f1factorPwr%2==0:
        p4f1factored = (p4c1*x-p4c2)^p4f1factorPwr
    else:
        p4f1factored = abs(p4c1*x-p4c2)^p4f1factorPwr
    if p4f2factorPwr%2==0:
        p4f2factored = (p4c3*x-p4c4)^p4f2factorPwr
    else:
        p4f2factored = abs(p4c3*x-p4c4)^p4f2factorPwr
    if p4f3factorPwr%2==0:
        p4f3factored = (p4c5*x-p4c6)^p4f3factorPwr
    else:
        p4f3factored = abs(p4c5*x-p4c6)^p4f3factorPwr
    if p4f4factorPwr%2==0:
        p4f4factored = (p4c7*x-p4c8)^p4f4factorPwr
    else:
        p4f4factored = abs(p4c7*x-p4c8)^p4f4factorPwr
    p4fFactored = p4f1factored*p4f2factored*p4f3factored*p4f4factored
else:
    p4fFactored = (p4c1*x-p4c2)^p4f1factorPwr*(p4c3*x-p4c4)^p4f2factorPwr*(p4c5*x-p4c6)^p4f3factorPwr*(p4c7*x-p4c8)^p4f4factorPwr





###### Problem p5
### Decide on the root value.
p5rootVal = RandInt(2,9)

### Build coefficients. In order to avoid crazy leading coefficients it is useful to ensure that there are no common factors.
p5c1 = 2
p5c2 = 2
p5c3 = 2
p5c4 = 2
p5c5 = 2
p5c6 = 2
p5c7 = 2
p5c8 = 2
while abs(p5c2/p5c1)==abs(p5c4/p5c3) or abs(p5c2/p5c1)==abs(p5c6/p5c5) or abs(p5c2/p5c1)==abs(p5c8/p5c7) or abs(p5c4/p5c3)==abs(p5c6/p5c5) or abs(p5c4/p5c3)==abs(p5c8/p5c7) or abs(p5c6/p5c5)==abs(p5c8/p5c7):
    p5c1 = 2
    p5c2 = 2
    p5c3 = 2
    p5c4 = 2
    p5c5 = 2
    p5c6 = 2
    p5c7 = 2
    p5c8 = 2
    while gcd(p5c1,p5c2)>1:
        p5c1 = RandInt(1,5)
        p5c2 = RandInt(-10,10)
    while gcd(p5c3,p5c4)>1:
        p5c3 = RandInt(1,5)
        p5c4 = RandInt(-10,10)
    while gcd(p5c5,p5c6)>1:
        p5c5 = RandInt(1,5)
        p5c6 = RandInt(-10,10)
    while gcd(p5c7,p5c8)>1:
        p5c7 = RandInt(1,5)
        p5c8 = RandInt(-10,10)

### Choose the remainder power for each factor; the part that remains in radicand after simplifying.
p5f1remainPwr = RandInt(0,p5rootVal-1)
p5f2remainPwr = RandInt(0,p5rootVal-1)
p5f3remainPwr = RandInt(0,p5rootVal-1)
p5f4remainPwr = RandInt(0,p5rootVal-1)

### Choose the factorable power; the part that will be removed from the radicand during simplifying.
p5f1factorPwr = RandInt(0,5)
p5f2factorPwr = RandInt(0,5)
p5f3factorPwr = RandInt(0,5)
p5f4factorPwr = RandInt(0,5)

### Build individual function elements.
p5f1remain = (p5c1*x-p5c2)^p5f1remainPwr
p5f2remain = (p5c3*x-p5c4)^p5f2remainPwr
p5f3remain = (p5c5*x-p5c6)^p5f3remainPwr
p5f4remain = (p5c7*x-p5c8)^p5f4remainPwr

p5f1total = (p5c1*x-p5c2)^(p5f1remainPwr+p5f1factorPwr*p5rootVal)
p5f2total = (p5c3*x-p5c4)^(p5f2remainPwr+p5f2factorPwr*p5rootVal)
p5f3total = (p5c5*x-p5c6)^(p5f3remainPwr+p5f3factorPwr*p5rootVal)
p5f4total = (p5c7*x-p5c8)^(p5f4remainPwr+p5f4factorPwr*p5rootVal)

p5fDisp = p5f1total*p5f2total*p5f3total*p5f4total

p5fRemainder = p5f1remain*p5f2remain*p5f3remain*p5f4remain

### The fun part; writing out the factored piece, which depends on things like if they need absolute value or not.
if p5rootVal%2==0:
    if p5f1factorPwr%2==0:
        p5f1factored = (p5c1*x-p5c2)^p5f1factorPwr
    else:
        p5f1factored = abs(p5c1*x-p5c2)^p5f1factorPwr
    if p5f2factorPwr%2==0:
        p5f2factored = (p5c3*x-p5c4)^p5f2factorPwr
    else:
        p5f2factored = abs(p5c3*x-p5c4)^p5f2factorPwr
    if p5f3factorPwr%2==0:
        p5f3factored = (p5c5*x-p5c6)^p5f3factorPwr
    else:
        p5f3factored = abs(p5c5*x-p5c6)^p5f3factorPwr
    if p5f4factorPwr%2==0:
        p5f4factored = (p5c7*x-p5c8)^p5f4factorPwr
    else:
        p5f4factored = abs(p5c7*x-p5c8)^p5f4factorPwr
    p5fFactored = p5f1factored*p5f2factored*p5f3factored*p5f4factored
else:
    p5fFactored = (p5c1*x-p5c2)^p5f1factorPwr*(p5c3*x-p5c4)^p5f2factorPwr*(p5c5*x-p5c6)^p5f3factorPwr*(p5c7*x-p5c8)^p5f4factorPwr






\end{sagesilent}


\begin{problem}
    Simplify the following type one radical. Notice that the root symbol is already supplied for you so you only need to supply the inside and outside functions (no need to expand them!)
    \[
        \sqrt[\sage{p1rootVal}]{\sage{p1fDisp}} = \left(\answer{\sage{p1fFactored}}\right)\sqrt[\sage{p1rootVal}]{\answer{\sage{p1fRemainder}}}
    \]
    \begin{feedback}
        The process for this problem is much like the previous practice tile, but there is one twist here. Remember that, for odd valued roots (like cube or fifth roots) the process is the same, but for even roots you have to worry about absolute values when you pull out a factor. The rule of thumb is to apply absolute values to anything you pull out of an even radical when you are simplifying, and \textbf{then} justify whether or not you can remove those absolute values on a term by term basis (for example, numbers don't need absolute values because you can calculate the absolute value of a constant. Even powered terms outside also don't need absolute values because they are already being raised to an even power and thus must become positive anyway).
    \end{feedback}
\end{problem}

\begin{problem}
    Simplify the following type one radical. Notice that the root symbol is already supplied for you so you only need to supply the inside and outside functions (no need to expand them!)
    \[
        \sqrt[\sage{p2rootVal}]{\sage{p2fDisp}} = \left(\answer{\sage{p2fFactored}}\right)\sqrt[\sage{p2rootVal}]{\answer{\sage{p2fRemainder}}}
    \]
    \begin{feedback}
        The process for this problem is much like the previous practice tile, but there is one twist here. Remember that, for odd valued roots (like cube or fifth roots) the process is the same, but for even roots you have to worry about absolute values when you pull out a factor. The rule of thumb is to apply absolute values to anything you pull out of an even radical when you are simplifying, and \textbf{then} justify whether or not you can remove those absolute values on a term by term basis (for example, numbers don't need absolute values because you can calculate the absolute value of a constant. Even powered terms outside also don't need absolute values because they are already being raised to an even power and thus must become positive anyway).
    \end{feedback}
\end{problem}


\begin{problem}
    Simplify the following type one radical. Notice that the root symbol is already supplied for you so you only need to supply the inside and outside functions (no need to expand them!)
    \[
        \sqrt[\sage{p3rootVal}]{\sage{p3fDisp}} = \left(\answer{\sage{p3fFactored}}\right)\sqrt[\sage{p3rootVal}]{\answer{\sage{p3fRemainder}}}
    \]
    \begin{feedback}
        The process for this problem is much like the previous practice tile, but there is one twist here. Remember that, for odd valued roots (like cube or fifth roots) the process is the same, but for even roots you have to worry about absolute values when you pull out a factor. The rule of thumb is to apply absolute values to anything you pull out of an even radical when you are simplifying, and \textbf{then} justify whether or not you can remove those absolute values on a term by term basis (for example, numbers don't need absolute values because you can calculate the absolute value of a constant. Even powered terms outside also don't need absolute values because they are already being raised to an even power and thus must become positive anyway).
    \end{feedback}
\end{problem}


\begin{problem}
    Simplify the following type one radical. Notice that the root symbol is already supplied for you so you only need to supply the inside and outside functions (no need to expand them!)
    \[
        \sqrt[\sage{p4rootVal}]{\sage{p4fDisp}} = \left(\answer{\sage{p4fFactored}}\right)\sqrt[\sage{p4rootVal}]{\answer{\sage{p4fRemainder}}}
    \]
    \begin{feedback}
        The process for this problem is much like the previous practice tile, but there is one twist here. Remember that, for odd valued roots (like cube or fifth roots) the process is the same, but for even roots you have to worry about absolute values when you pull out a factor. The rule of thumb is to apply absolute values to anything you pull out of an even radical when you are simplifying, and \textbf{then} justify whether or not you can remove those absolute values on a term by term basis (for example, numbers don't need absolute values because you can calculate the absolute value of a constant. Even powered terms outside also don't need absolute values because they are already being raised to an even power and thus must become positive anyway).
    \end{feedback}
\end{problem}


\begin{problem}
    Simplify the following type one radical. Notice that the root symbol is already supplied for you so you only need to supply the inside and outside functions (no need to expand them!)
    \[
        \sqrt[\sage{p5rootVal}]{\sage{p5fDisp}} = \left(\answer{\sage{p5fFactored}}\right)\sqrt[\sage{p5rootVal}]{\answer{\sage{p5fRemainder}}}
    \]
    \begin{feedback}
        The process for this problem is much like the previous practice tile, but there is one twist here. Remember that, for odd valued roots (like cube or fifth roots) the process is the same, but for even roots you have to worry about absolute values when you pull out a factor. The rule of thumb is to apply absolute values to anything you pull out of an even radical when you are simplifying, and \textbf{then} justify whether or not you can remove those absolute values on a term by term basis (for example, numbers don't need absolute values because you can calculate the absolute value of a constant. Even powered terms outside also don't need absolute values because they are already being raised to an even power and thus must become positive anyway).
    \end{feedback}
\end{problem}


\end{document}