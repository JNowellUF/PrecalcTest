\documentclass{ximera}
\title{Factor Coefficients Method Practice 1}



\begin{document}
%\input{Useful-Sage-Macros}

\begin{sagesilent}

def RandInt(a,b):
    """ Returns a random integer in [`a`,`b`]. Note that `a` and `b` should be integers themselves to avoid unexpected behavior.
    """
    return QQ(randint(int(a),int(b)))
    # return choice(range(a,b+1))

def NonZeroInt(b,c, avoid = [0]):
    """ Returns a random integer in [`b`,`c`] which is not in `av`. 
        If `av` is not specified, defaults to a non-zero integer.
    """
    while True:
        a = RandInt(b,c)
        if a not in avoid:
            return a





p1ans1 = 100
p1ans2 = 200

while p1ans1>50 or p1ans2>50:
    # Make p(x)
    p1c1 = NonZeroInt(1,5)# a
    p1c2 = RandInt(1,5)# b
    p1c3 = RandInt(1,5)# c
    
    p1px = p1c1*(x-p1c2)^2 + p1c3
    
    p1w = RandInt(1,3)
    
    p1h = p1px(x=(p1c2+p1w))
    
    p1wprime = RandInt(2,5)
    
    p1bprime = RandInt(p1c2+1, 10)
    p1aprime = NonZeroInt(1,5,[p1c1])
    p1cprime = p1h - p1aprime*p1wprime^2
    
    p1gap = p1bprime + p1wprime -p1c2 - p1w#RandInt(p1bprime - p1c2 - p1w + 1, 2*(p1bprime - p1c2 - p1w + 1))
    
    
    p1left = sqrt((x - p1c3)/p1c1) + p1c2 + p1gap
    p1right = sqrt((x - p1cprime)/p1aprime) + p1bprime
    
    
    p1d = p1c2+p1gap-p1bprime
    p1e = p1c3*p1aprime-p1cprime*p1c1-p1d^2*p1c1*p1aprime
    
    p1A = (p1c1-p1aprime)^2
    p1B = 2*(p1c1-p1aprime)*p1e - 4*p1d^2*p1aprime^2*p1c1
    p1C = p1e^2 + 4*p1d^2*p1aprime^2*p1c1*p1c3
    
    p1ans1 = (-p1B + sqrt(p1B^2 - 4*p1A*p1C))/(2*p1A)
    p1ans2 = (-p1B - sqrt(p1B^2 - 4*p1A*p1C))/(2*p1A)
    
    if p1left(x=p1ans1) == p1right(x=p1ans1):
        if p1left(x=p1ans2)==p1right(x=p1ans2):
            p1ans = p1ans1+p1ans2
        else:
            p1ans=p1ans1
    else:
        p1ans=p1ans2
    
    if p1ans1==p1ans2:
        if p1left(x=p1ans1)==p1right(x=p1ans1):
            p1ans=p1ans1
        else:
            p1ans='DNE'


\end{sagesilent}

\begin{problem}
    What is the \textbf{sum} of the answers to the following equality?
    \[
        \sage{p1left} = \sage{p1right}
    \]
    The sum of solutions is: $\answer{\sage{p1ans}}$.
    \begin{feedback}
        Recall that you need to isolate one of the radicals, then square both sides. Once you expand everything out, you will need to re-isolate the remaining radical and square both sides again, which should leave you with a quadratic. Once you solve the quadratic, you then need to check your solutions to see if they are extraneous or valid. 
    \end{feedback}
\end{problem}







\end{document}