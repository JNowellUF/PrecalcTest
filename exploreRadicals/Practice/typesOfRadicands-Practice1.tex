\documentclass{ximera}
\title{Factor Coefficients Method Practice 1}



\begin{document}
\begin{sagesilent}

######  Define a function to convert a sage number into a saved counter number.

#####Define default Sage variables.
#Default function variables
var('x,y,z,X,Y,Z')
#Default function names
var('f,g,h,dx,dy,dz,dh,df')
#Default Wild cards
w0 = SR.wild(0)

def DispSign(b):
    """ Returns the string of the 'signed' version of `b`, e.g. 3 -> "+3", -3 -> "-3", 0 -> "".
    """
    if b == 0:
        return ""
    elif b > 0:
        return "+" + str(b)
    elif b < 0:
        return str(b)
    else:
        # If we're here, then something has gone wrong.
        raise ValueError

def ISP(b):
    return DispSign(b)

def NoEval(f, c):
    # TODO
    """ Returns a non-evaluted version of the result f(c).
    """
    cStr = str(c)
    # fLatex = latex(f)
    fString = latex(f)
    fStrList = list(fString)
    length = len(fStrList)
    fStrList2 = range(length)
    for i in range(0, length):
        if fStrList[i] == "x":
            fStrList2[i] = "("+cstr+")"
        else:
            fStrList2[i] = fStrList[i]
    f2 = join(fStrList2,"")
    return LatexExpr(f2)

def HyperSimp(f):
    """ Returns the expression `f` without hyperbolic expressions.
    """
    subsDict = {
        sinh(w0) : (exp(w0) - exp(-w0))/2,
        cosh(w0) : (exp(w0) + exp(-w0))/2,
        tanh(w0) : (exp(w0) - exp(-w0))/(exp(w0) + exp(-w0)),
        sech(w0) : 2/(exp(w0) + exp(-w0)),                      # This seems to work, but Nowell said it didn't at one point.
        csch(w0) : 2/(exp(w0) - exp(-w0)),                      # This seems to work, but Nowell said it didn't at one point.
        coth(w0) : (exp(w0) + exp(-w0))/(exp(w0) - exp(-w0)),   # This seems to work, but Nowell said it didn't at one point.
        arcsinh(w0) :       ln( w0 + sqrt((w0)^2 + 1) ),
        arccosh(w0) :       ln( w0 + sqrt((w0)^2 - 1) ),
        arctanh(w0) : 1/2 * ln( (1 + w0) / (1 - w0) ),
        arccsch(w0) :       ln( (1 + sqrt((w0)^2 + 1))/w0 ),
        arcsech(w0) :       ln( (1 + sqrt(1 - (w0)^2))/w0 ),
        arccoth(w0) : 1/2 * ln( (1 + w0) / (w0 - 1) )
    }
    g = f.substitute(subsDict)
    return simplify(g)

def RandInt(a,b):
    """ Returns a random integer in [`a`,`b`]. Note that `a` and `b` should be integers themselves to avoid unexpected behavior.
    """
    return QQ(randint(int(a),int(b)))
    # return choice(range(a,b+1))

def NonZeroInt(b,c, avoid = [0]):
    """ Returns a random integer in [`b`,`c`] which is not in `av`. 
        If `av` is not specified, defaults to a non-zero integer.
    """
    while True:
        a = RandInt(b,c)
        if a not in avoid:
            return a

def RandVector(b, c, avoid=[], rep=1):
    """ Returns essentially a multiset permutation of ([b,c]-av) * rep.
        That is, a vector which contains each integer in [`b`,`c`] which is not in `av` a total of `rep` number of times.
        Example:
        sage: RandVector(1,3, [2], 2)
        [3, 1, 1, 3]
    """
    oneVec = [val for val in range(b,c+1) if val not in avoid]
    vec = oneVec * rep
    shuffle(vec)
    return vec

def fudge(b):
    up = b+RandInt(2,5)/10
    down = b-RandInt(2,5)/10
    fudgebup = round(up,1)
    fudgebdown = round(down,1)
    fudgedb = [fudgebdown,fudgebup]
    return fudgedb

def disjointCheck(checkvec):
    if length(uniq(checkvec)) < length(checkvec):
        return 1
    else:
        return 0

def disjointIntervals(IntStart,IntEnd,CheckVal):
    if IntStart < CheckVal and CheckVal < IntEnd:
        return 1
    else:
        return 0

def IntervalVecCheck(checkVec):
    veclen = len(checkVec)
    returnval = 0
    for i in range(veclen):
        for j in range(veclen):
            if (disjointIntervals(checkVec[j][0],checkVec[j][1],checkVec[i][0]) + disjointIntervals(checkVec[j][0],checkVec[j][1],checkVec[i][1])) > 0:
                returnval = returnval + 1
    if returnval > 0:
        return 1
    else:
        return 0



\end{sagesilent}

\begin{sagesilent}
funcvec = [x^2, x, e^x, log(x)]

###### Problem p1
p1c1 = NonZeroInt(-10,10)
p1c2 = NonZeroInt(-10,10,[0,p1c1])
p1pick1 = RandInt(0,3)
p1pick2 = NonZeroInt(0,3,[p1pick1])

p1rad = p1c1*funcvec[p1pick1]+p1c2*funcvec[p1pick2]


###### Problem p2
p2c1 = NonZeroInt(-10,10)
p2c2 = NonZeroInt(-10,10,[0,p2c1])
p2pick1 = RandInt(0,3)
p2pick2 = NonZeroInt(0,3,[p2pick1])

p2rad = p2c1*funcvec[p2pick1]*p2c2*funcvec[p2pick2]


###### Problem p3
p3c1 = NonZeroInt(-10,10)
p3c2 = NonZeroInt(-10,10,[0,p3c1])
p3pick1 = RandInt(0,3)
p3pick2 = NonZeroInt(0,3,[p3pick1])

p3rad = p3c1*funcvec[p3pick1]+p3c2*funcvec[p3pick2]


###### Problem p4
p4c1 = NonZeroInt(-10,10)
p4c2 = NonZeroInt(-10,10,[0,p4c1])
p4pick1 = RandInt(0,3)
p4pick2 = NonZeroInt(0,3,[p4pick1])

p4rad = p4c1*funcvec[p4pick1]*p4c2*funcvec[p4pick2]


\end{sagesilent}

\begin{problem}
    Is the following radical a Type 1 or Type 2 Radical?
    \[
        \sqrt{\sage{p1rad}}
    \]
    
    \begin{multipleChoice}
        \choice{Type 1 (aka only one term within the radical)}
        \choice[correct]{Type 2 (aka more than one term within the radical)}
    \end{multipleChoice}
    \begin{feedback}[correct]
        Yes! Since the radicand has more than one term (separated with a plus or minus sign) it is a type 2!
    \end{feedback}
    
    \begin{problem}
        Can you simplify this radical as it is currently written?
        \begin{multipleChoice}
            \choice{Only if the numbers happen to be perfect squares}
            \choice{Potentially; depending on the values, since it is a Type 2 it is at least possible to simplify.}
            \choice[correct]{Not as it is written. Type 2 radicals cannot be simplified without manipulating them first into Type 1 radicals.}
            \choice{No, there is never anything we can do to simplify Type 2 radicals.}
        \end{multipleChoice}
        \begin{feedback}[correct]
            As mentioned in our lesson, the whole point of distinguishing the type 2 vs type 1, is that the radicand of a type 2 radical must be factored (or otherwise manipulated) before we have any hope of simplifying the radical!
        \end{feedback}
        
    \end{problem}
\end{problem}

\begin{problem}
    Is the following radical a Type 1 or Type 2 Radical?
    \[
        \sqrt{\sage{p2rad}}
    \]
    
    \begin{multipleChoice}
        \choice[correct]{Type 1 (aka only one term within the radical)}
        \choice{Type 2 (aka more than one term within the radical)}
    \end{multipleChoice}
    \begin{feedback}[correct]
        Yes! Since this has everything inside looking like it has already been factored (i.e. each term is multiplying every other term) we have that this is a type 1 radical!
    \end{feedback}
    
    \begin{problem}
        Can you simplify this radical as it is currently written?
        \begin{multipleChoice}
            \choice{Only if the numbers happen to be perfect squares}
            \choice[correct]{Potentially; depending on the values, since it is a Type 1 it is at least possible to simplify.}
            \choice{Not as it is written. Type 1 radicals cannot be simplified without manipulating them first into Type 2 radicals.}
            \choice{No, there is never anything we can do to simplify Type 1 radicals.}
        \end{multipleChoice}
        \begin{feedback}[correct]
            Since the original radical's radicand is just one term (a.k.a. it appears ``factored'' already) we can potentially simplify the radical... there is hope! However, this also depends on the powers of the individual factors in the radicand, as well as the power of the radical. Thus it is potentially something that can be simplified further, but it may already be as simplified as it can get. 
        \end{feedback}
    \end{problem}
\end{problem}

\begin{problem}
    Is the following radical a Type 1 or Type 2 Radical?
    \[
        \sqrt{\sage{p3rad}}
    \]
    
    \begin{multipleChoice}
        \choice{Type 1 (aka only one term within the radical)}
        \choice[correct]{Type 2 (aka more than one term within the radical)}
    \end{multipleChoice}
    \begin{feedback}[correct]
        Yes! Since the radicand has more than one term (separated with a plus or minus sign) it is a type 2!
    \end{feedback}
        
    \begin{problem}
        Can you simplify this radical as it is currently written?
        \begin{multipleChoice}
            \choice{Only if the numbers happen to be perfect squares}
            \choice{Potentially; depending on the values, since it is a Type 2 it is at least possible to simplify.}
            \choice[correct]{Not as it is written. Type 2 radicals cannot be simplified without manipulating them first into Type 1 radicals.}
            \choice{No, there is never anything we can do to simplify Type 2 radicals.}
        \end{multipleChoice}
        \begin{feedback}[correct]
            As mentioned in our lesson, the whole point of distinguishing the type 2 vs type 1, is that the radicand of a type 2 radical must be factored (or otherwise manipulated) before we have any hope of simplifying the radical!
        \end{feedback}
    \end{problem}
\end{problem}

\begin{problem}
    Is the following radical a Type 1 or Type 2 Radical?
    \[
        \sqrt{\sage{p4rad}}
    \]
    
    \begin{multipleChoice}
        \choice[correct]{Type 1 (aka only one term within the radical)}
        \choice{Type 2 (aka more than one term within the radical)}
    \end{multipleChoice}
    \begin{feedback}[correct]
        Yes! Since this has everything inside looking like it has already been factored (i.e. each term is multiplying every other term) we have that this is a type 1 radical!
    \end{feedback}
        
    \begin{problem}
        Can you simplify this radical as it is currently written?
        \begin{multipleChoice}
            \choice{Only if the numbers happen to be perfect squares}
            \choice[correct]{Potentially; depending on the values, since it is a Type 1 it is at least possible to simplify.}
            \choice{Not as it is written. Type 1 radicals cannot be simplified without manipulating them first into Type 2 radicals.}
            \choice{No, there is never anything we can do to simplify Type 1 radicals.}
        \end{multipleChoice}
        \begin{feedback}[correct]
            Since the original radical's radicand is just one term (a.k.a. it appears ``factored'' already) we can potentially simplify the radical... there is hope! However, this also depends on the powers of the individual factors in the radicand, as well as the power of the radical. Thus it is potentially something that can be simplified further, but it may already be as simplified as it can get. 
        \end{feedback}
    \end{problem}
\end{problem}


\end{document}