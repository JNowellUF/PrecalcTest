\documentclass{ximera}
\title{Factor Coefficients Method Practice 1}



\begin{document}


\begin{sagesilent}
var('x')

def RandInt(a,b):
    """ Returns a random integer in [`a`,`b`]. Note that `a` and `b` should be integers themselves to avoid unexpected behavior.
    """
    return QQ(randint(int(a),int(b)))
    # return choice(range(a,b+1))

def NonZeroInt(b,c, avoid = [0]):
    """ Returns a random integer in [`b`,`c`] which is not in `av`. 
        If `av` is not specified, defaults to a non-zero integer.
    """
    while True:
        a = RandInt(b,c)
        if a not in avoid:
            return a

###### Problem p1
p1f1(x) = x + 999999
while p1f1(x=1)>500 or p1f1(x=-1)>500 or p1f1(x=0)>500:
    p1c1 = RandInt(1,5)
    p1c2 = RandInt(1,5)
    p1c3 = RandInt(1,5)
    p1c4 = RandInt(1,5)
    p1c5 = RandInt(-7,7)
    p1c6 = RandInt(-7,7)
    p1c7 = RandInt(-7,7)
    p1c8 = RandInt(-7,7)
    p1tog1 = RandInt(0,1)
    p1tog2 = RandInt(1-p1tog1,1)
    
    p1f1t = p1c1*x-p1c5
    p1f2t = p1c2*x-p1c6
    p1f3t = (p1c3*x-p1c7)^p1tog1
    p1f4t = (p1c4*x-p1c8)^p1tog2
    
    
    p1f1 = (p1f1t).mul(p1f2t,p1f3t,p1f4t,hold=true)

p1rad = expand(p1f1)


###### Problem p2
p2f1(x) = x + 999999
while p2f1(x=1)>500 or p2f1(x=-1)>500 or p2f1(x=0)>500:
    p2c1 = RandInt(1,5)
    p2c2 = RandInt(1,5)
    p2c3 = RandInt(1,5)
    p2c4 = RandInt(1,5)
    p2c5 = RandInt(-7,7)
    p2c6 = RandInt(-7,7)
    p2c7 = RandInt(-7,7)
    p2c8 = RandInt(-7,7)
    p2tog1 = RandInt(0,1)
    p2tog2 = RandInt(0,1)
    
    p2f1t = p2c1*x-p2c5
    p2f2t = p2c2*x-p2c6
    p2f3t = (p2c3*x-p2c7)^p2tog1
    p2f4t = (p2c4*x-p2c8)^p2tog2
    
    
    p2f1 = (p2f1t).mul(p2f2t,p2f3t,p2f4t,hold=true)

p2rad = expand(p2f1)


###### Problem p3
p3f1(x) = x + 999999
while p3f1(x=1)>500 or p3f1(x=-1)>500 or p3f1(x=0)>500:
    p3c1 = RandInt(1,5)
    p3c2 = RandInt(1,5)
    p3c3 = RandInt(1,5)
    p3c4 = RandInt(1,5)
    p3c5 = RandInt(-7,7)
    p3c6 = RandInt(-7,7)
    p3c7 = RandInt(-7,7)
    p3c8 = RandInt(-7,7)
    p3tog1 = RandInt(0,1)
    p3tog2 = RandInt(0,1)
    
    p3f1t = p3c1*x-p3c5
    p3f2t = p3c2*x-p3c6
    p3f3t = (p3c3*x-p3c7)^p3tog1
    p3f4t = (p3c4*x-p3c8)^p3tog2
    
    
    p3f1 = (p3f1t).mul(p3f2t,p3f3t,p3f4t,hold=true)

p3rad = expand(p3f1)




\end{sagesilent}

\begin{javascript}
    var x;

    // sameDerivative checks to see if the derivative with respect to x and C are equal.
    sameDerivative = function(a,b) {
        return (a.derivative('x').equals( b.derivative('x') ) && a.derivative('C').equals( b.derivative('C') )) ;
    };


    function factorCheck(f,g) {
        // This validator is designed to check that a student is submitting a factored polynomial. It works by:
        //  Checking that there are the correct number of non-numeric and non-inverse factors as expected,
        //  Checking that the submitted answer and the expected answer are the same via real Xronos evaluation,
        //  Checking that the outer most (last to be computed when following order of operations) operation is multiplication.
        
        var operCheck = f.tree[0];// Check to see if the root operation is multiplication at end.
        var studentFactors = f.tree.length;// Temporary number of student-provided factors (+1 because of root operation)
        
        // Now we adjust the length to remove any numeric factors, or division factors, etc to avoid ``padding'' by students.
        for (var i = 0; i < f.tree.length; i++) {
            if ((typeof f.tree[i] === 'number')||(f.tree[i][0] == '-')||(f.tree[i][0] == '/')) {
                studentFactors = studentFactors - 1;
            }
        }
        
        // Now we do the same with the provided answer, in case sage or something provides a weird format.
        var answerFactors = g.tree.length;
        
        // Adjust length in the same way, so that it will match the students if it should.
        for (var i = 0; i < g.tree.length; i++) {
            if (typeof g.tree[i] === 'number') {
                answerFactors = answerFactors - 1;
            }
        }
        
        // Note: An especially dedicated student could pad with weird factors that all happen to cancel down to 1.
        // For example, a student could enter sin^2(x)+cos^2(x) as a multiplicative factor to pad the number of factors.
        // This would be somewhat difficult to think of, even on purpose.
        // Until I can reliably evaluate the factors themselves as functions though, there isn't a lot to be done here.
        
        return ((f.equals(g))&&(studentFactors==answerFactors)&&(operCheck=='*'))
    }
    
\end{javascript}


\begin{problem}
    Factor the following radicand to force the type two radical into being a type one radical:
    \[
        \sqrt{\sage{p1rad}} = \sqrt{\answer[validator=factorCheck]{\sage{p1f1}}}
    \]
    \begin{feedback}
        You should have all linear factors above. In particular you should not have any factors higher than degree 1 polynomials. Essentially you can treat the radicand as if it were a polynomial factoring problem and factor the polynomial just like you would have normally.
        
        \textbf{Note:} Repeated factors should be entered in distinctly for the validator. For example, instead of writing something like $x^2(x-1)^2$ you should enter it in as $(x-0)(x-0)(x-1)(x-1)$.
    \end{feedback}
\end{problem}


\begin{problem}
    Factor the following radicand to force the type two radical into being a type one radical:
    \[
        \sqrt{\sage{p2rad}} = \sqrt{\answer[validator=factorCheck]{\sage{p2f1}}}
    \]
    \begin{feedback}
        You should have all linear factors above. In particular you should not have any factors higher than degree 1 polynomials. Essentially you can treat the radicand as if it were a polynomial factoring problem and factor the polynomial just like you would have normally.
        
        \textbf{Note:} Repeated factors should be entered in distinctly for the validator. For example, instead of writing something like $x^2(x-1)^2$ you should enter it in as $(x-0)(x-0)(x-1)(x-1)$.
    \end{feedback}
\end{problem}


\begin{problem}
    Factor the following radicand to force the type two radical into being a type one radical:
    \[
        \sqrt{\sage{p3rad}} = \sqrt{\answer[validator=factorCheck]{\sage{p3f1}}}
    \]
    \begin{feedback}
        You should have all linear factors above. In particular you should not have any factors higher than degree 1 polynomials. Essentially you can treat the radicand as if it were a polynomial factoring problem and factor the polynomial just like you would have normally.
        
        \textbf{Note:} Repeated factors should be entered in distinctly for the validator. For example, instead of writing something like $x^2(x-1)^2$ you should enter it in as $(x-0)(x-0)(x-1)(x-1)$.
    \end{feedback}
\end{problem}



\end{document}