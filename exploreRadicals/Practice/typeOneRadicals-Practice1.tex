\documentclass{ximera}
\title{Simplifying Algebraic Radicals: Practice 1}


\begin{document}


\begin{sagesilent}
var('x')

def RandInt(a,b):
    """ Returns a random integer in [`a`,`b`]. Note that `a` and `b` should be integers themselves to avoid unexpected behavior.
    """
    return QQ(randint(int(a),int(b)))
    # return choice(range(a,b+1))

def NonZeroInt(b,c, avoid = [0]):
    """ Returns a random integer in [`b`,`c`] which is not in `av`. 
        If `av` is not specified, defaults to a non-zero integer.
    """
    while True:
        a = RandInt(b,c)
        if a not in avoid:
            return a


#### Note: Because Sage does everything in the complex plane, 
#           it cannot be trusted to simplify radicals in any sane way for a precalc student.
#           Thus we will need to manually build out the correct answer.

###### Problem p1
### Decide on the root value.
p1rootVal = 2*RandInt(1,4)+1# They haven't discussed even roots yielding absolute value signs yet, so let's only give them odds for now.

### Build coefficients. In order to avoid crazy leading coefficients it is useful to ensure that there are no common factors. Start with a baseline of 2 all around, then redefine them to have properties we want.
p1c1 = 2
p1c2 = 2
p1c3 = 2
p1c4 = 2
p1c5 = 2
p1c6 = 2
p1c7 = 2
p1c8 = 2

# We will ensure GCD = 1 to avoid a common factor from being pulled out of the factor (ax-b)^n and get crazy leading coefficient.
# We will also force the leading coefficients of each factor to be different, which will force each factor to stay independent and not merge into one factor when computed. This follows due to the relatively prime aspect of the a,b in (ax-b)^n forms.
while gcd(p1c1,p1c2)>1:
    p1c1 = RandInt(1,9)
    p1c2 = RandInt(-10,10)
while gcd(p1c3,p1c4)>1:
    p1c3 = NonZeroInt(1,9,[p1c1])
    p1c4 = RandInt(-10,10)
while gcd(p1c5,p1c6)>1:
    p1c5 = NonZeroInt(1,9,[p1c1,p1c3])
    p1c6 = RandInt(-10,10)
while gcd(p1c7,p1c8)>1:
    p1c7 = NonZeroInt(1,9,[p1c1,p1c3,p1c5])
    p1c8 = RandInt(-10,10)


### Now we should have four relatively prime factors for inside the radical. Now we need to determine how many will ``evenly'' get removed and now many will stay within the radical.

# Choose the remainder power for each factor; the part that remains in radicand after simplifying.
p1f1remainPwr = RandInt(1,p1rootVal-1)
p1f2remainPwr = RandInt(0,p1rootVal-1)
p1f3remainPwr = RandInt(0,p1rootVal-1)
p1f4remainPwr = RandInt(0,p1rootVal-1)


# Choose the factorable power; the part that will be removed from the radicand during simplifying.
p1f1factorPwr = RandInt(0,5)
p1f2factorPwr = RandInt(0,5)
p1f3factorPwr = RandInt(0,5)
p1f4factorPwr = RandInt(0,5)


### Now we have the number of remainders and the number of even quantities of each factor are going to exist, so we need to make the display values for the function and the solution.

# Build individual function elements that will remain inside the radical
p1f1remain = (p1c1*x-p1c2)^p1f1remainPwr
p1f2remain = (p1c3*x-p1c4)^p1f2remainPwr
p1f3remain = (p1c5*x-p1c6)^p1f3remainPwr
p1f4remain = (p1c7*x-p1c8)^p1f4remainPwr

# Build individual function elements that will be displayed as the original function (the total of all powers)
p1f1total = (p1c1*x-p1c2)^(p1f1remainPwr+p1f1factorPwr*p1rootVal)
p1f2total = (p1c3*x-p1c4)^(p1f2remainPwr+p1f2factorPwr*p1rootVal)
p1f3total = (p1c5*x-p1c6)^(p1f3remainPwr+p1f3factorPwr*p1rootVal)
p1f4total = (p1c7*x-p1c8)^(p1f4remainPwr+p1f4factorPwr*p1rootVal)


### Now we build the displayed total function, the total factored piece function, and the remainder (inside radical) function:
p1fDisp = p1f1total*p1f2total*p1f3total*p1f4total
p1fRemainder = p1f1remain*p1f2remain*p1f3remain*p1f4remain
p1fFactored = (p1c1*x-p1c2)^p1f1factorPwr*(p1c3*x-p1c4)^p1f2factorPwr*(p1c5*x-p1c6)^p1f3factorPwr*(p1c7*x-p1c8)^p1f4factorPwr



###### Problem p2
### Decide on the root value.
p2rootVal = 2*RandInt(1,4)+1# They haven't discussed even roots yielding absolute value signs yet, so let's only give them odds for now.

### Build coefficients. In order to avoid crazy leading coefficients it is useful to ensure that there are no common factors. Start with a baseline of 2 all around, then redefine them to have properties we want.
p2c1 = 2
p2c2 = 2
p2c3 = 2
p2c4 = 2
p2c5 = 2
p2c6 = 2
p2c7 = 2
p2c8 = 2

# We will ensure GCD = 1 to avoid a common factor from being pulled out of the factor (ax-b)^n and get crazy leading coefficient.
# We will also force the leading coefficients of each factor to be different, which will force each factor to stay independent and not merge into one factor when computed. This follows due to the relatively prime aspect of the a,b in (ax-b)^n forms.
while gcd(p2c1,p2c2)>1:
    p2c1 = RandInt(1,9)
    p2c2 = RandInt(-10,10)
while gcd(p2c3,p2c4)>1:
    p2c3 = NonZeroInt(1,9,[p2c1])
    p2c4 = RandInt(-10,10)
while gcd(p2c5,p2c6)>1:
    p2c5 = NonZeroInt(1,9,[p2c1,p2c3])
    p2c6 = RandInt(-10,10)
while gcd(p2c7,p2c8)>1:
    p2c7 = NonZeroInt(1,9,[p2c1,p2c3,p2c5])
    p2c8 = RandInt(-10,10)


### Now we should have four relatively prime factors for inside the radical. Now we need to determine how many will ``evenly'' get removed and now many will stay within the radical.

# Choose the remainder power for each factor; the part that remains in radicand after simplifying.
p2f1remainPwr = RandInt(1,p2rootVal-1)
p2f2remainPwr = RandInt(0,p2rootVal-1)
p2f3remainPwr = RandInt(0,p2rootVal-1)
p2f4remainPwr = RandInt(0,p2rootVal-1)


# Choose the factorable power; the part that will be removed from the radicand during simplifying.
p2f1factorPwr = RandInt(0,5)
p2f2factorPwr = RandInt(0,5)
p2f3factorPwr = RandInt(0,5)
p2f4factorPwr = RandInt(0,5)


### Now we have the number of remainders and the number of even quantities of each factor are going to exist, so we need to make the display values for the function and the solution.

# Build individual function elements that will remain inside the radical
p2f1remain = (p2c1*x-p2c2)^p2f1remainPwr
p2f2remain = (p2c3*x-p2c4)^p2f2remainPwr
p2f3remain = (p2c5*x-p2c6)^p2f3remainPwr
p2f4remain = (p2c7*x-p2c8)^p2f4remainPwr

# Build individual function elements that will be displayed as the original function (the total of all powers)
p2f1total = (p2c1*x-p2c2)^(p2f1remainPwr+p2f1factorPwr*p2rootVal)
p2f2total = (p2c3*x-p2c4)^(p2f2remainPwr+p2f2factorPwr*p2rootVal)
p2f3total = (p2c5*x-p2c6)^(p2f3remainPwr+p2f3factorPwr*p2rootVal)
p2f4total = (p2c7*x-p2c8)^(p2f4remainPwr+p2f4factorPwr*p2rootVal)


### Now we build the displayed total function, the total factored piece function, and the remainder (inside radical) function:
p2fDisp = p2f1total*p2f2total*p2f3total*p2f4total
p2fRemainder = p2f1remain*p2f2remain*p2f3remain*p2f4remain
p2fFactored = (p2c1*x-p2c2)^p2f1factorPwr*(p2c3*x-p2c4)^p2f2factorPwr*(p2c5*x-p2c6)^p2f3factorPwr*(p2c7*x-p2c8)^p2f4factorPwr




###### Problem p3
### Decide on the root value.
p3rootVal = 2*RandInt(1,4)+1# They haven't discussed even roots yielding absolute value signs yet, so let's only give them odds for now.

### Build coefficients. In order to avoid crazy leading coefficients it is useful to ensure that there are no common factors. Start with a baseline of 2 all around, then redefine them to have properties we want.
p3c1 = 2
p3c2 = 2
p3c3 = 2
p3c4 = 2
p3c5 = 2
p3c6 = 2
p3c7 = 2
p3c8 = 2

# We will ensure GCD = 1 to avoid a common factor from being pulled out of the factor (ax-b)^n and get crazy leading coefficient.
# We will also force the leading coefficients of each factor to be different, which will force each factor to stay independent and not merge into one factor when computed. This follows due to the relatively prime aspect of the a,b in (ax-b)^n forms.
while gcd(p3c1,p3c2)>1:
    p3c1 = RandInt(1,9)
    p3c2 = RandInt(-10,10)
while gcd(p3c3,p3c4)>1:
    p3c3 = NonZeroInt(1,9,[p3c1])
    p3c4 = RandInt(-10,10)
while gcd(p3c5,p3c6)>1:
    p3c5 = NonZeroInt(1,9,[p3c1,p3c3])
    p3c6 = RandInt(-10,10)
while gcd(p3c7,p3c8)>1:
    p3c7 = NonZeroInt(1,9,[p3c1,p3c3,p3c5])
    p3c8 = RandInt(-10,10)


### Now we should have four relatively prime factors for inside the radical. Now we need to determine how many will ``evenly'' get removed and now many will stay within the radical.

# Choose the remainder power for each factor; the part that remains in radicand after simplifying.
p3f1remainPwr = RandInt(1,p3rootVal-1)
p3f2remainPwr = RandInt(0,p3rootVal-1)
p3f3remainPwr = RandInt(0,p3rootVal-1)
p3f4remainPwr = RandInt(0,p3rootVal-1)


# Choose the factorable power; the part that will be removed from the radicand during simplifying.
p3f1factorPwr = RandInt(0,5)
p3f2factorPwr = RandInt(0,5)
p3f3factorPwr = RandInt(0,5)
p3f4factorPwr = RandInt(0,5)


### Now we have the number of remainders and the number of even quantities of each factor are going to exist, so we need to make the display values for the function and the solution.

# Build individual function elements that will remain inside the radical
p3f1remain = (p3c1*x-p3c2)^p3f1remainPwr
p3f2remain = (p3c3*x-p3c4)^p3f2remainPwr
p3f3remain = (p3c5*x-p3c6)^p3f3remainPwr
p3f4remain = (p3c7*x-p3c8)^p3f4remainPwr

# Build individual function elements that will be displayed as the original function (the total of all powers)
p3f1total = (p3c1*x-p3c2)^(p3f1remainPwr+p3f1factorPwr*p3rootVal)
p3f2total = (p3c3*x-p3c4)^(p3f2remainPwr+p3f2factorPwr*p3rootVal)
p3f3total = (p3c5*x-p3c6)^(p3f3remainPwr+p3f3factorPwr*p3rootVal)
p3f4total = (p3c7*x-p3c8)^(p3f4remainPwr+p3f4factorPwr*p3rootVal)


### Now we build the displayed total function, the total factored piece function, and the remainder (inside radical) function:
p3fDisp = p3f1total*p3f2total*p3f3total*p3f4total
p3fRemainder = p3f1remain*p3f2remain*p3f3remain*p3f4remain
p3fFactored = (p3c1*x-p3c2)^p3f1factorPwr*(p3c3*x-p3c4)^p3f2factorPwr*(p3c5*x-p3c6)^p3f3factorPwr*(p3c7*x-p3c8)^p3f4factorPwr




###### Problem p4
### Decide on the root value.
p4rootVal = 2*RandInt(1,4)+1# They haven't discussed even roots yielding absolute value signs yet, so let's only give them odds for now.

### Build coefficients. In order to avoid crazy leading coefficients it is useful to ensure that there are no common factors. Start with a baseline of 2 all around, then redefine them to have properties we want.
p4c1 = 2
p4c2 = 2
p4c3 = 2
p4c4 = 2
p4c5 = 2
p4c6 = 2
p4c7 = 2
p4c8 = 2

# We will ensure GCD = 1 to avoid a common factor from being pulled out of the factor (ax-b)^n and get crazy leading coefficient.
# We will also force the leading coefficients of each factor to be different, which will force each factor to stay independent and not merge into one factor when computed. This follows due to the relatively prime aspect of the a,b in (ax-b)^n forms.
while gcd(p4c1,p4c2)>1:
    p4c1 = RandInt(1,9)
    p4c2 = RandInt(-10,10)
while gcd(p4c3,p4c4)>1:
    p4c3 = NonZeroInt(1,9,[p4c1])
    p4c4 = RandInt(-10,10)
while gcd(p4c5,p4c6)>1:
    p4c5 = NonZeroInt(1,9,[p4c1,p4c3])
    p4c6 = RandInt(-10,10)
while gcd(p4c7,p4c8)>1:
    p4c7 = NonZeroInt(1,9,[p4c1,p4c3,p4c5])
    p4c8 = RandInt(-10,10)


### Now we should have four relatively prime factors for inside the radical. Now we need to determine how many will ``evenly'' get removed and now many will stay within the radical.

# Choose the remainder power for each factor; the part that remains in radicand after simplifying.
p4f1remainPwr = RandInt(1,p4rootVal-1)
p4f2remainPwr = RandInt(0,p4rootVal-1)
p4f3remainPwr = RandInt(0,p4rootVal-1)
p4f4remainPwr = RandInt(0,p4rootVal-1)


# Choose the factorable power; the part that will be removed from the radicand during simplifying.
p4f1factorPwr = RandInt(0,5)
p4f2factorPwr = RandInt(0,5)
p4f3factorPwr = RandInt(0,5)
p4f4factorPwr = RandInt(0,5)


### Now we have the number of remainders and the number of even quantities of each factor are going to exist, so we need to make the display values for the function and the solution.

# Build individual function elements that will remain inside the radical
p4f1remain = (p4c1*x-p4c2)^p4f1remainPwr
p4f2remain = (p4c3*x-p4c4)^p4f2remainPwr
p4f3remain = (p4c5*x-p4c6)^p4f3remainPwr
p4f4remain = (p4c7*x-p4c8)^p4f4remainPwr

# Build individual function elements that will be displayed as the original function (the total of all powers)
p4f1total = (p4c1*x-p4c2)^(p4f1remainPwr+p4f1factorPwr*p4rootVal)
p4f2total = (p4c3*x-p4c4)^(p4f2remainPwr+p4f2factorPwr*p4rootVal)
p4f3total = (p4c5*x-p4c6)^(p4f3remainPwr+p4f3factorPwr*p4rootVal)
p4f4total = (p4c7*x-p4c8)^(p4f4remainPwr+p4f4factorPwr*p4rootVal)


### Now we build the displayed total function, the total factored piece function, and the remainder (inside radical) function:
p4fDisp = p4f1total*p4f2total*p4f3total*p4f4total
p4fRemainder = p4f1remain*p4f2remain*p4f3remain*p4f4remain
p4fFactored = (p4c1*x-p4c2)^p4f1factorPwr*(p4c3*x-p4c4)^p4f2factorPwr*(p4c5*x-p4c6)^p4f3factorPwr*(p4c7*x-p4c8)^p4f4factorPwr



###### Problem p5
### Decide on the root value.
p5rootVal = 2*RandInt(1,4)+1# They haven't discussed even roots yielding absolute value signs yet, so let's only give them odds for now.

### Build coefficients. In order to avoid crazy leading coefficients it is useful to ensure that there are no common factors. Start with a baseline of 2 all around, then redefine them to have properties we want.
p5c1 = 2
p5c2 = 2
p5c3 = 2
p5c4 = 2
p5c5 = 2
p5c6 = 2
p5c7 = 2
p5c8 = 2

# We will ensure GCD = 1 to avoid a common factor from being pulled out of the factor (ax-b)^n and get crazy leading coefficient.
# We will also force the leading coefficients of each factor to be different, which will force each factor to stay independent and not merge into one factor when computed. This follows due to the relatively prime aspect of the a,b in (ax-b)^n forms.
while gcd(p5c1,p5c2)>1:
    p5c1 = RandInt(1,9)
    p5c2 = RandInt(-10,10)
while gcd(p5c3,p5c4)>1:
    p5c3 = NonZeroInt(1,9,[p5c1])
    p5c4 = RandInt(-10,10)
while gcd(p5c5,p5c6)>1:
    p5c5 = NonZeroInt(1,9,[p5c1,p5c3])
    p5c6 = RandInt(-10,10)
while gcd(p5c7,p5c8)>1:
    p5c7 = NonZeroInt(1,9,[p5c1,p5c3,p5c5])
    p5c8 = RandInt(-10,10)


### Now we should have four relatively prime factors for inside the radical. Now we need to determine how many will ``evenly'' get removed and now many will stay within the radical.

# Choose the remainder power for each factor; the part that remains in radicand after simplifying.
p5f1remainPwr = RandInt(1,p5rootVal-1)
p5f2remainPwr = RandInt(0,p5rootVal-1)
p5f3remainPwr = RandInt(0,p5rootVal-1)
p5f4remainPwr = RandInt(0,p5rootVal-1)


# Choose the factorable power; the part that will be removed from the radicand during simplifying.
p5f1factorPwr = RandInt(0,5)
p5f2factorPwr = RandInt(0,5)
p5f3factorPwr = RandInt(0,5)
p5f4factorPwr = RandInt(0,5)


### Now we have the number of remainders and the number of even quantities of each factor are going to exist, so we need to make the display values for the function and the solution.

# Build individual function elements that will remain inside the radical
p5f1remain = (p5c1*x-p5c2)^p5f1remainPwr
p5f2remain = (p5c3*x-p5c4)^p5f2remainPwr
p5f3remain = (p5c5*x-p5c6)^p5f3remainPwr
p5f4remain = (p5c7*x-p5c8)^p5f4remainPwr

# Build individual function elements that will be displayed as the original function (the total of all powers)
p5f1total = (p5c1*x-p5c2)^(p5f1remainPwr+p5f1factorPwr*p5rootVal)
p5f2total = (p5c3*x-p5c4)^(p5f2remainPwr+p5f2factorPwr*p5rootVal)
p5f3total = (p5c5*x-p5c6)^(p5f3remainPwr+p5f3factorPwr*p5rootVal)
p5f4total = (p5c7*x-p5c8)^(p5f4remainPwr+p5f4factorPwr*p5rootVal)


### Now we build the displayed total function, the total factored piece function, and the remainder (inside radical) function:
p5fDisp = p5f1total*p5f2total*p5f3total*p5f4total
p5fRemainder = p5f1remain*p5f2remain*p5f3remain*p5f4remain
p5fFactored = (p5c1*x-p5c2)^p5f1factorPwr*(p5c3*x-p5c4)^p5f2factorPwr*(p5c5*x-p5c6)^p5f3factorPwr*(p5c7*x-p5c8)^p5f4factorPwr




\end{sagesilent}


\begin{problem}
    Simplify the following type one radical. Notice that the root symbol is already supplied for you so you only need to supply the inside and outside functions (no need to expand them!)
    \[
        \sqrt[\sage{p1rootVal}]{\sage{p1fDisp}} = \left(\answer{\sage{p1fFactored}}\right)\sqrt[\sage{p1rootVal}]{\answer{\sage{p1fRemainder}}}
    \]
    \begin{feedback}
        Remember you want to break up the powers of each factor into the part that the root's power goes into ``evenly'' versus the remainder. So, for example, $\sage{p1f1total}$ can be written as $\sage{p1f1total} = \sage{(p1c1*x-p1c2)^(p1f1factorPwr*p1rootVal)}\cdot (\sage{p1c1*x-p1c2})^{\sage{p1f1remainPwr}}$. Then you can pull out the $\sage{(p1c1*x-p1c2)^(p1f1factorPwr*p1rootVal)}$ part from the radical, leaving the $(\sage{p1c1*x-p1c2})^{\sage{p1f1remainPwr}}$ part behind in the simplified version's radicand.
    \end{feedback}
\end{problem}

\begin{problem}
    Simplify the following type one radical. Notice that the root symbol is already supplied for you so you only need to supply the inside and outside functions (no need to expand them!)
    \[
        \sqrt[\sage{p2rootVal}]{\sage{p2fDisp}} = \left(\answer{\sage{p2fFactored}}\right)\sqrt[\sage{p2rootVal}]{\answer{\sage{p2fRemainder}}}
    \]
    \begin{feedback}
        Remember you want to break up the powers of each factor into the part that the root's power goes into ``evenly'' versus the remainder. So, for example, $\sage{p2f1total}$ can be written as $\sage{p2f1total} = \sage{(p2c1*x-p2c2)^(p2f1factorPwr*p2rootVal)}\cdot (\sage{p2c1*x-p2c2})^{\sage{p2f1remainPwr}}$. Then you can pull out the $\sage{(p2c1*x-p2c2)^(p2f1factorPwr*p2rootVal)}$ part from the radical, leaving the $(\sage{p2c1*x-p2c2})^{\sage{p2f1remainPwr}}$ part behind in the simplified version's radicand.
    \end{feedback}
\end{problem}


\begin{problem}
    Simplify the following type one radical. Notice that the root symbol is already supplied for you so you only need to supply the inside and outside functions (no need to expand them!)
    \[
        \sqrt[\sage{p3rootVal}]{\sage{p3fDisp}} = \left(\answer{\sage{p3fFactored}}\right)\sqrt[\sage{p3rootVal}]{\answer{\sage{p3fRemainder}}}
    \]
    \begin{feedback}
        Remember you want to break up the powers of each factor into the part that the root's power goes into ``evenly'' versus the remainder. So, for example, $\sage{p3f1total}$ can be written as $\sage{p3f1total} = \sage{(p3c1*x-p3c2)^(p3f1factorPwr*p3rootVal)}\cdot (\sage{p3c1*x-p3c2})^{\sage{p3f1remainPwr}}$. Then you can pull out the $\sage{(p3c1*x-p3c2)^(p3f1factorPwr*p3rootVal)}$ part from the radical, leaving the $(\sage{p3c1*x-p3c2})^{\sage{p3f1remainPwr}}$ part behind in the simplified version's radicand.
    \end{feedback}
\end{problem}


\begin{problem}
    Simplify the following type one radical. Notice that the root symbol is already supplied for you so you only need to supply the inside and outside functions (no need to expand them!)
    \[
        \sqrt[\sage{p4rootVal}]{\sage{p4fDisp}} = \left(\answer{\sage{p4fFactored}}\right)\sqrt[\sage{p4rootVal}]{\answer{\sage{p4fRemainder}}}
    \]
    \begin{feedback}
        Remember you want to break up the powers of each factor into the part that the root's power goes into ``evenly'' versus the remainder. So, for example, $\sage{p4f1total}$ can be written as $\sage{p4f1total} = \sage{(p4c1*x-p4c2)^(p4f1factorPwr*p4rootVal)}\cdot (\sage{p4c1*x-p4c2})^{\sage{p4f1remainPwr}}$. Then you can pull out the $\sage{(p4c1*x-p4c2)^(p4f1factorPwr*p4rootVal)}$ part from the radical, leaving the $(\sage{p4c1*x-p4c2})^{\sage{p4f1remainPwr}}$ part behind in the simplified version's radicand.
    \end{feedback}
\end{problem}


\begin{problem}
    Simplify the following type one radical. Notice that the root symbol is already supplied for you so you only need to supply the inside and outside functions (no need to expand them!)
    \[
        \sqrt[\sage{p5rootVal}]{\sage{p5fDisp}} = \left(\answer{\sage{p5fFactored}}\right)\sqrt[\sage{p5rootVal}]{\answer{\sage{p5fRemainder}}}
    \]
    \begin{feedback}
        Remember you want to break up the powers of each factor into the part that the root's power goes into ``evenly'' versus the remainder. So, for example, $\sage{p5f1total}$ can be written as $\sage{p5f1total} = \sage{(p5c1*x-p5c2)^(p5f1factorPwr*p5rootVal)}\cdot (\sage{p5c1*x-p5c2})^{\sage{p5f1remainPwr}}$. Then you can pull out the $\sage{(p5c1*x-p5c2)^(p5f1factorPwr*p5rootVal)}$ part from the radical, leaving the $(\sage{p5c1*x-p5c2})^{\sage{p5f1remainPwr}}$ part behind in the simplified version's radicand.
    \end{feedback}
\end{problem}


\end{document}