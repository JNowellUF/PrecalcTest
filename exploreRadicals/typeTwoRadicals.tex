\documentclass{ximera}
\input{../preamble}

\title{Type 2 Radicals}
\begin{document}
\begin{abstract}
    This section discusses how to handle type two radicals.
\end{abstract}
\maketitle

\subsection*{Type Two Radicals and how to `simplify' them.}
    You can watch a video on this topic here:
    
    \youtube{OUF6PRWy5M0}

    In some senses, type two radicals are the most straight forward type which is why we start with the type two instead of the type one. In short, a type two radical is `easy' purely because there isn't anything we can do when it is in this form in terms of `simplifying' the radical (without manipulating the radicand). The key thing here is that you \textit{\textbf{absolutely cannot simplify a type two radical!}} This may seem straight forward, but it is often easy to overlook this and simplify things because it seems like it should work. For example;

    \begin{example}[Find values of $x$ so that $\sqrt{x^2 + 9} = 10$]%
        Notice that the radical in the problem is type 2 because it has more than one term in the radicand. Nonetheless many reading this example will want to immediately `simplify' the radical to get the equation: $x + 3 = 10$ (by square rooting each of the terms). Unfortunately this is invalid... to see that it doesn't work, let's try it this way and see what we get.

        Doing this `simplification' gets us that $x + 3 = 10$ which we can solve to find $x = 7$. An astute student might want to include the $\pm$ option because of the $x^2$, so let's consider that possibility as well; ie $x + 3 = - 10$ which gives $x = -13$. Now let's plug these ``solutions" of $x$ back into the original equation to see what happens;
        
        \begin{tabular}{rccccccc}
            $x=7$:  & $\sqrt{x^2 + 9}$ & $=$ & $\sqrt{7^2 + 9}$     & $=$ & $\sqrt{49 + 9}$     & $=$ & $\sqrt{58} \neq 10$.\\
            $x=13$: & $\sqrt{x^2 + 9}$ & $=$ & $\sqrt{(-13)^2 + 9}$ & $=$ & $\sqrt{169 + 9}$    & $=$ & $\sqrt{178} \neq 10$.
        \end{tabular}

        As we can see; in both these cases our `solutions' didn't work, this is because \wordChoice{\choice[correct]{\textbf{\textit{you cannot simplify type two radicals}}}\choice{math loves wasting our time.}\choice{because there is a better way to simplify these radicals.}}. Thus we will need a different way of solving these kinds of equations, which we will discuss later in this topic.
    \end{example}% end of the Example.

    So, when you are confronted with the need to simplify a type two radical, your only practical approach (at least at this point) is to try and manipulate the radicand to make the type two radical into a type one radical. In short, you need to \textbf{factor} the radicand. If possible this is almost always the best way to go, as other techniques to `solve' equations with type two radicals come with a variety of possible drawbacks.
    
    In the next section we will be discussing how to simplify type one radicals. So once you have factored the type two radical into a type one radical, you can use the techniques in the next section to then simplify the radical.




%
%
%\begin{question}
%    This is a purely Place Holder type question that will be replaced.
%    \begin{multipleChoice}
%        \choice{This question shouldn't be possible to get correct.}
%    \end{multipleChoice}
%\end{question}
%
%
%





\end{document}