\documentclass{ximera}

\title{What is mathematical reasoning?}
\begin{document}
\begin{abstract}
    This section aims to introduce the idea of mathematical reasoning and give an example of how it is used.
\end{abstract}
\maketitle

Here is an example video!

\youtube{daHbOACbZpw}

Mathematical reasoning is a method to work through a problem. It is a very powerful tool, but it is important to remember that it is still only one possible approach. Throughout this class we will be focusing on this skill, but when you move on into your careers or post-graduate work, this will be one of several ways you will be able to approach problems. As such, it is useful to know when it is, or isn't, appropriate to use it.


\subsubsection*{So; what is mathematical reasoning?}
    Simply put, mathematical reasoning is the process of quantifying generic information into data, and then using deductive reasoning to extrapolate the results you are after. That may not seem like it is ``simply put", but let's see an example.
    
    \begin{exploration}
        {\large \bfseries How much would it cost to build a patio?}
        
        This is an example of a question that might occur to (or be asked of) you at some point in your life. If this is all you are given, then it seems impossible to answer the question. In reality it is most often the case that this is how a problem is presented; someone will ask a generic question or hand you a problem, with little information attached and tell you to ``just figure it out". This means we must use our own problem solving skills to come to a solution.\\
        
        {\bfseries Question Phase:} First we need to acquire information. In this case we might ask questions like:
        \begin{itemize}
            \item What do we want to make the patio out of?
            \item What size is the patio?
            \item What kind of expertise is required to build the patio?
        \end{itemize}
        Usually it's not clear what information is useful and what information is not (information that is not useful is often called \textit{extraneous} information). One way to determine this, and a way to figure out how to get to an answer to our original question, is to try and convert the information we have into quantified information or "data". Once we have done that, we can create a model which relates the data together, and then manipulate the model to arrive at our result (ie the answer to our problem).\\
        
        {\bfseries Modeling Phase:}
        Thus, we might end our process with something that looks like this:
        \begin{itemize}
            \item $x$ = The number of bricks needed to build our patio
            \item $C$ = The cost of the patio (in dollars).
            \item $A$ = The area of the patio (in square feet).
            \item $C(x) = 2.5x$
            \item $x(A) = \frac{A}{0.75}$
            \item $C(A) = \frac{15}{8}A$
        \end{itemize}
        At first glance the above may seem confusing, and in fact it should be. The problem is that we've "jumped to the end" and have just a bunch of equations and variables. But, the intervening steps that let us go from the "questioning" phase to the "modeling" phase are where the "mathematical reasoning" comes into play.\\
        
        What kind of problems are most likely to involve/require `mathematical reasoning'?
        \begin{multipleChoice}
            \choice{Interpreting information given on a survey.}
            \choice[correct]{Synthesizing or interpret (primarily) numeric information.}
            \choice{Assigning numbers to non-numeric information, such as quantifying emotions.}
            \choice{Trying to calculate values in any situation... math is basically a superpower. It is useful everywhere and with everything.}
        \end{multipleChoice}
    \end{exploration}

If this seems confusing, don't worry. This is sort of like skipping to the end of a book you've never seen before and reading the last sentence, then thinking you should know the plot. Mathematical reasoning is something you need to practice and learn in order to understand and get good at. But the upside is that, once you do, you will have a very powerful tool at your disposal for solving problems. And if there is one thing that is ubiquitous in life (and careers), it's problems that need solving. If you get good at problem solving, you will be a powerful candidate at job applications, and you boss's favorite employee.






\end{document}