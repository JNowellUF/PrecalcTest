\documentclass{ximera}

\title{Embrace Laziness!}
\begin{document}
\begin{abstract}
    This section aims to show the virtues, and techniques, in generalizing numeric models into `generalized' models.
\end{abstract}
\maketitle

Here is a video on embracing laziness!

\youtube{eX9ZfbFJIqc}

So far we have spent a large amount of time and effort learning how to solve very narrow (and occasionally non-specific) problems. It gets tiresome to go through this process for every single question you are posed however, and often you can take many questions and group them together under one single problem with minor variations. This is where generalized models will come in handy.

Let's consider our earlier patio example. We were asked to determine the cost of building a patio, presumably by an \textit{individual} wanting a patio. But what if you worked for a construction company? You'd have to go through a similar process many times a week, and doing so would get irritatingly repetitive. But, things that are repetitive are usually susceptible to being generalized, and thus you can streamline the process. So, rather than brute forcing the work for many patios, embrace laziness and generalize the process to come up with a ``shorthand" version that works for almost any desired patio!

\textbf{Author's note on the best way to read the following sections}: I will first describe the process in general and then present a short example. It would be ideal to read these things in parallel, but for the sake of continuity I will present one and then the other. I actually recommend reading this content in order first and not worrying too much about fully understanding the ``concept" portion on the first time through. Then read the example, and come back to reread each section of the ``concept" description with the corresponding section in the example. This may seem much more time consuming, but it has a much higher chance of helping you ``learn" the content (as oppose to ``memorize" the content).

\subsubsection*{First step to generalizing: Determine what can be generalized.}
    The first step in generalizing a numeric solution may seem ``obvious" (in the same way that the first phase of solving the numeric model; ``clarifying the problem" seemed like it should be obvious), but it is again often deceptively important. A good way to start determining what can be generalized, is to consider what information you need and already know what (type of) units it would be supplied in.
    \footnote{One may restate this to say that a good first step would be to look at what you have as \textit{data}, and not simply \textit{information}. Data is often easier to generalize as it is quantifiable, and that quantity is what you are generalizing. It is very important to remember that this is only a start however. Often, in industry, the most pivotal piece of information to generalize isn't found by doing this, but it at least gives you a place to start.}
    In our patio example we needed to know what the patio was made of (which we later determined would be cement pavers) as well as what size the patio would be. This first piece of information (what the patio was made of) isn't in units that we can expect (ie it isn't going to be \textit{data}) and as such may not be a good candidate for generalizing. The size however is going to be some form of area (and thus will be \textit{data}), thus we can expect some kind of square units (eg square feet or square yards) and so that piece of information may be a good candidate to try and generalize.
    
    \begin{exploration}
        {\large\bfseries A common difficulty}\\
        
        A common error is to overgeneralize. Just because you may be able to generalize a piece of information, doesn't mean you will want to. The first step is to \emph{identify} which elements we \emph{are able} to generalize, but that doesn't mean we will generalize every piece of information we can. This is the art-form of modeling; there is no definite rule to follow to know what is the `right' amount of generalizing, but with practice one can develop a talent for determining the `perfect' level of generalizing.
        
        In other words, modeling is...
        
        \begin{multipleChoice}
            \choice{Obnoxious?}
            \choice{Awesome!}
            \choice[correct]{More of an artform than a science}
            \choice{A difficult process that isn't really worth practicing.}
            \choice{The perfect way to be lazy at work without getting fired.}
        \end{multipleChoice}
    \end{exploration}

\subsubsection*{Next Step: Determine what \emph{should} be generalized... and how.}

    The art of modeling (in the mathematical sense) comes into play when you are trying to decide which variables to generalize. On the one hand, the more you generalize, the more versatile and applicable your model becomes. On the other hand, generalizing takes time and effort, meaning you may spend so much time generalizing that you take far longer to do the same job... something your boss will undoubtedly not appreciate.
    
    The broad rule of thumb is to ask yourself ``which of the things that I can generalize are likely to need to change from project to project." If a piece of information is likely to change between different variations of projects, then it's a good candidate for generalizing. For example, not everyone will want the same size or dimension patio... so that is likely to change from project to project, ie from patio to patio. Keep in mind that these things differ by situation; consider the possibility that you are in a business where all you make are twenty by twenty patios from a variety of materials, generalizing the size of the patio will be pointless (it's always the same size after all), but generalizing the materials becomes key; something that might prove very difficult given the previous comments about the difficulty of generalizing non-data information.
    
    Once you have identified the information that you wish to generalize, the \textit{how} is ``straightforward"
    \footnote{I put straightforward in quotes because, in practice, executing the generalizing step itself might be easy, but keeping track of it as you update/build your model can be very difficult. This is why the advice about keeping a written list of variables and what they mean is absolutely key, especially early in learning this process.}.
    You generalize a piece of information by replacing it in your model with the correct type of \textit{variable}.
    In order to understand what we mean when we say the ``correct type of variable", you should first understand the role of variables in the model, and what different types exist.
    
    
    \begin{question}
        The goal of generalizing numeric models is to...
        \begin{selectAll}
            \choice[correct]{expend more time and effort up front, to save considerable time and effort later.}
            \choice{annoy your boss with a million questions for every project.}
            \choice[correct]{set up a standard by which to calculate values. That is to say, to create a `form' where certain known quantities can be ``plugged in" and the answer is immediately calculated.}
            \choice{torture students with insanity inducing pointless exercises.}
        \end{selectAll}
    \end{question}


\end{document}