\documentclass{ximera}
\usepackage{longdivision}
\usepackage{polynom}
\usepackage{float}% Use `H' as the figure optional argument to force it's vertical placement to conform to source.
%\usepackage{caption}% Allows us to describe the figures without having "figure 1:" in it. :: Apparently Caption isn't supported.
%    \captionsetup{labelformat=empty}% Actually does the figure configuration stated above.
\usetikzlibrary{arrows.meta,arrows}% Allow nicer arrow heads for tikz.
\usepackage{gensymb, pgfplots}
\usepackage{tabularx}
\usepackage{arydshln}
\usepackage[margin=1.5cm]{geometry}
\usepackage{indentfirst}

\setlength\parindent{16pt}

\graphicspath{
  {./}
  {./explorePolynomials/}
  {./exploreRadicals/}
  {./graphing/}
}

%% Default style for tikZ
\pgfplotsset{my style/.append style={axis x line=middle, axis y line=
middle, xlabel={$x$}, ylabel={$y$}, axis equal }}


%% Because log being natural log is too hard for people.
\let\logOld\log% Keep the old \log definition, just in case we need it.
\renewcommand{\log}{\ln}


%%% Changes in polynom to show the zero coefficient terms
\makeatletter
\def\pld@CF@loop#1+{%
    \ifx\relax#1\else
        \begingroup
          \pld@AccuSetX11%
          \def\pld@frac{{}{}}\let\pld@symbols\@empty\let\pld@vars\@empty
          \pld@false
          #1%
          \let\pld@temp\@empty
          \pld@AccuIfOne{}{\pld@AccuGet\pld@temp
                            \edef\pld@temp{\noexpand\pld@R\pld@temp}}%
           \pld@if \pld@Extend\pld@temp{\expandafter\pld@F\pld@frac}\fi
           \expandafter\pld@CF@loop@\pld@symbols\relax\@empty
           \expandafter\pld@CF@loop@\pld@vars\relax\@empty
           \ifx\@empty\pld@temp
               \def\pld@temp{\pld@R11}%
           \fi
          \global\let\@gtempa\pld@temp
        \endgroup
        \ifx\@empty\@gtempa\else
            \pld@ExtendPoly\pld@tempoly\@gtempa
        \fi
        \expandafter\pld@CF@loop
    \fi}
\def\pld@CMAddToTempoly{%
    \pld@AccuGet\pld@temp\edef\pld@temp{\noexpand\pld@R\pld@temp}%
    \pld@CondenseMonomials\pld@false\pld@symbols
    \ifx\pld@symbols\@empty \else
        \pld@ExtendPoly\pld@temp\pld@symbols
    \fi
    \ifx\pld@temp\@empty \else
        \pld@if
            \expandafter\pld@IfSum\expandafter{\pld@temp}%
                {\expandafter\def\expandafter\pld@temp\expandafter
                    {\expandafter\pld@F\expandafter{\pld@temp}{}}}%
                {}%
        \fi
        \pld@ExtendPoly\pld@tempoly\pld@temp
        \pld@Extend\pld@tempoly{\pld@monom}%
    \fi}
\makeatother




%%%%% Code for making prime factor trees for numbers, taken from user Qrrbrbirlbel at: https://tex.stackexchange.com/questions/131689/how-to-automatically-draw-tree-diagram-of-prime-factorization-with-latex

\usepackage{forest,mathtools,siunitx}
\makeatletter
\def\ifNum#1{\ifnum#1\relax
  \expandafter\pgfutil@firstoftwo\else
  \expandafter\pgfutil@secondoftwo\fi}
\forestset{
  num content/.style={
    delay={
      content/.expanded={\noexpand\num{\forestoption{content}}}}},
  pt@prime/.style={draw, circle},
  pt@start/.style={},
  pt@normal/.style={},
  start primeTree/.style={%
    /utils/exec=%
      % \pt@start holds the current minimum factor, we'll start with 2
      \def\pt@start{2}%
      % \pt@result will hold the to-be-typeset factorization, we'll start with
      % \pgfutil@gobble since we don't want a initial \times
      \let\pt@result\pgfutil@gobble
      % \pt@start@cnt holds the number of ^factors for the current factor
      \def\pt@start@cnt{0}%
      % \pt@lStart will later hold "l"ast factor used
      \let\pt@lStart\pgfutil@empty,
    alias=pt-start,
    pt@start/.try,
    delay={content/.expanded={$\noexpand\num{\forestove{content}}
                            \noexpand\mathrlap{{}= \noexpand\pt@result}$}},
    primeTree},
  primeTree/.code=%
    % take the content of the node and save it in the count
    \c@pgf@counta\forestove{content}\relax
    % if it's 2 we're already finished with the factorization
    \ifNum{\c@pgf@counta=2}{%
      % add the factor
      \pt@addfactor{2}%
      % finalize the factorization of the result
      \pt@addfactor{}%
      % and set the style to the prime style
      \forestset{pt@prime/.try}%
    }{%
      % this simply calculates content/2 and saves it in \pt@end
      % this is later used for an early break of the recursion since no factor
      % can be greater then content/2 (for integers of course)
      \edef\pt@content{\the\c@pgf@counta}%
      \divide\c@pgf@counta2\relax
      \advance\c@pgf@counta1\relax % to be on the safe side
      \edef\pt@end{\the\c@pgf@counta}%
      \pt@do}}

%%% our main "function"
\def\pt@do{%
  % let's test if the current factor is already greather then the max factor
  \ifNum{\pt@end<\pt@start}{%
    % great, we're finished, the same as above
    \expandafter\pt@addfactor\expandafter{\pt@content}%
    \pt@addfactor{}%
    \def\pt@next{\forestset{pt@prime/.try}}%
  }{%
    % this calculates int(content/factor)*factor
    % if factor is a factor of content (without remainder), the result will
    % equal content. The int(content/factor) is saved in \pgf@temp.
    \c@pgf@counta\pt@content\relax
    \divide\c@pgf@counta\pt@start\relax
    \edef\pgf@temp{\the\c@pgf@counta}%
    \multiply\c@pgf@counta\pt@start\relax
    \ifNum{\the\c@pgf@counta=\pt@content}{%
      % yeah, we found a factor, add it to the result and ...
      \expandafter\pt@addfactor\expandafter{\pt@start}%
      % ... add the factor as the first child with style pt@prime
      % and the result of int(content/factor) as another child.
      \edef\pt@next{\noexpand\forestset{%
        append={[\pt@start, pt@prime/.try]},
        append={[\pgf@temp, pt@normal/.try]},
        % forest is complex, this makes sure that for the second child, the
        % primeTree style is not executed too early (there must be a better way).
        delay={
          for descendants={
            delay={if n'=1{primeTree, num content}{}}}}}}%
    }{%
      % Alright this is not a factor, let's get the next factor
      \ifNum{\pt@start=2}{%
        % if the previous factor was 2, the next one will be 3
        \def\pt@start{3}%
      }{%
        % hmm, the previos factor was not 2,
        % let's add 2, maybe we'll hit the next prime number
        % and maybe a factor
        \c@pgf@counta\pt@start
        \advance\c@pgf@counta2\relax
        \edef\pt@start{\the\c@pgf@counta}%
      }%
      % let's do that again
      \let\pt@next\pt@do
    }%
  }%
  \pt@next
}

%%% this builds the \pt@result macro with the factors
\def\pt@addfactor#1{%
  \def\pgf@tempa{#1}%
  % is it the same factor as the previous one
  \ifx\pgf@tempa\pt@lStart
    % add 1 to the counter
    \c@pgf@counta\pt@start@cnt\relax
    \advance\c@pgf@counta1\relax
    \edef\pt@start@cnt{\the\c@pgf@counta}%
  \else
    % a new factor! Add the previous one to the product of factors
    \ifx\pt@lStart\pgfutil@empty\else
      % as long as there actually is one, the \ifnum makes sure we do not add ^1
      \edef\pgf@tempa{\noexpand\num{\pt@lStart}\ifnum\pt@start@cnt>1 
                                           ^{\noexpand\num{\pt@start@cnt}}\fi}%
      \expandafter\pt@addfactor@\expandafter{\pgf@tempa}%
    \fi
    % setup the macros for the next round
    \def\pt@lStart{#1}% <- current (new) factor
    \def\pt@start@cnt{1}% <- first time
  \fi
}
%%% This simply appends "\times #1" to \pt@result, with etoolbox this would be
%%% \appto\pt@result{\times#1}
\def\pt@addfactor@#1{%
  \expandafter\def\expandafter\pt@result\expandafter{\pt@result \times #1}}

%%% Our main macro:
%%% #1 = possible optional argument for forest (can be tikz too)
%%% #2 = the number to factorize
\newcommand*{\PrimeTree}[2][]{%
  \begin{forest}%
    % as the result is set via \mathrlap it doesn't update the bounding box
    % let's fix this:
    tikz={execute at end scope={\pgfmathparse{width("${}=\pt@result$")}%
                         \path ([xshift=\pgfmathresult pt]pt-start.east);}},
    % other optional arguments
    #1
    % And go!
    [#2, start primeTree]
  \end{forest}}
\makeatother


\providecommand\tabitem{\makebox[1em][r]{\textbullet~}}
\providecommand{\letterPlus}{\makebox[0pt][l]{$+$}}
\providecommand{\letterMinus}{\makebox[0pt][l]{$-$}}

\renewcommand{\texttt}[1]{#1}% Renew the command to prevent it from showing up in the sage strings for some weird reason.
%\renewcommand{\text}[1]{#1}% Renew the command to prevent it from showing up in the sage strings for some weird reason.



\title{The Point of Grades}
\begin{document}
\begin{abstract}
This is the grading rubric for the course, including the assignments, how many points things are worth, and how many points are needed for each letter grade.
\end{abstract}
\maketitle

\subsubsection*{Grading Scheme}
The syllabus in Canvas will have a breakdown of specific point values for all assignments, and how many points are required for each letter grade.
%See the tables below to see what will contribute to your grade, and what is necessary to attain a specific grade.\\
%\hspace{1cm}
%\begin{tabular}{|lcc|}
%    \hline \textbf{Assignment}          & \textbf{Point Value}          & \textbf{Total Points}\\ \hline 
%    Xronos                              & 50                            & 50\\
%    Participation                       & 40                            & 40\\
%    Quizzes (10 of 13)                  & 6                             & 60\\
%    Model Project                       & 50                            & 50\\    
%    Exams (3 total)                     & 50                            & 150\\
%    Final                               & 100                           & 100\\ \hline
%    \textbf{Total Points}               &                               & 450\\ \hline
%\end{tabular}
%\begin{tabular}{|cccc|}
%    \hline \textbf{Grade}       & \textbf{Point Range}      &\textbf{Grade}         & \textbf{Point Range}\\ \hline
%    A                           & 405-450                   & C                     & 315-329\\
%    A\letterMinus               & 390-404                   & C\letterMinus         & 300-314\\
%    B\letterPlus                & 375-389                   & D\letterPlus          & 285-299\\
%    B                           & 360-374                   & D                     & 270-284 \\
%    B\letterMinus               & 345-359                   & D\letterMinus         & 255-269\\
%    C\letterPlus                & 330-344                   & E                     & 0-254\\ \hline
%\end{tabular}


    \subsubsection*{Online Coursework}
    In this course we will utilize an in-house interactive online homework system developed by the math department at UF. This platform, called Xronos, is free of charge and will be explained during class. There is a single Xronos `assignment' in Canvas which is an interactive set of course notes that follows our course. It has numerous interactive features as well as examples and problems scattered throughout. The entire assignment is due the day after the final exam, but I will be posting regular updates about where you should be, and what sections you should cover, in preparation for each exam. I recommend you do  not try to complete the entire assignment at the end. First, there is simply too much to do all at once, and second it is intended as a supplemental source of learning for the exams and content. Be diligent and do them while you learn the material.\\
    
    There are some notes to keep in mind about how Xronos works:
    \begin{itemize}
    	\item You \textbf{MUST} access Xronos via Canvas \textbf{every single time you do your homework.} Do not bookmark the page, do not save the page, do not access Xronos directly via a link -- you \textbf{MUST} go through Canvas \textbf{EVERY TIME}. If you do not -- you will not receive credit for the problems you solve. This cannot be stressed enough.
    	    
    	\item Throughout the text there are problems embedded in the text to monitor learning and give examples. These are counted as part of the grade, and you are required to complete these to get credit for the assignment. These are static problems, ie each student will have the same problems with no randomization. You are free to work together on these problems, but keep in mind they are intended as practice, and as such \textbf{you are responsible for knowing the material covered in the homework}. 
    	
    	\item Also throughout the text there will be `practice' tiles. These problems will \textit{not} count for any credit and is entirely optional, but will give access to unlimited practice problems for the previous content. In each review page there will be a "Try Another" button at the top; whenever you wish to have a new set of problems, simply click this button to regenerate fresh problems. \textbf{Something to keep in mind:} randomly generated problems, no matter how well written, are susceptible to the occasional (unfortunate) confluence of randomization that make problems unreasonable to solve. If this seems to be the case, rather than slamming your head against the wall trying to solve it, hit the ``Try Another" button to get a different problem to solve. \textit{If this seems to happen several times in a row} for the same problem, you should see your TA to see if you are misunderstanding the problem/solution method, or perhaps the problem is broken (this is unlikely but it is in beta after all). Either way, your TA will be able to help you, either by showing you how to correctly solve the problem, or by determining that the problem is broken and referring it to me to be fixed.
    \end{itemize}


%\subsubsection*{Attendance and Participation}
%    Attendance to lecture is optional, but keep in mind that it is the exceptionally rare student that manages to pass (let alone get a desirable grade) without regularly attending lecture. Nonetheless we are all adults and if you don't want to go to lecture or something comes up that requires your focus rather than attending lecture, that is your decision. \textbf{Despite there being no mandatory attendance at lecture, you are responsible for all the material covered in lecture!} Part of being an adult means suffering the consequences of your decisions; keep this in mind as the semester progresses.
%    
%    As to your discussion sections, just about every discussion has a quiz. Moreover you \textbf{do have attendance grades in discussion class!} Lecture is large and difficult to be as interactive as we all would wish, which is why you have discussion classes. These classes have much smaller sizes and are designed to be interactive and helpful to individual students. If you have any questions about how attendance will be tracked or graded, talk to your individual TA.
%    
%    Finally, please note: \textbf{YOU  MAY  NOT  TURN  IN  WORK  FOR  A  STUDENT  WHO  IS NOT  IN  CLASS} (see  honor code below).
%
%\subsubsection*{Quizzes}
%    Quizzes will be administered by your Teaching Assistant during your discussion class. These will be (approximately) fifteen minute assessments to keep you up to date on the content as we progress through the course. There are twelve quizzes offered, but we will count the top ten grades (meaning you get to drop two quizzes). Keep in mind, with the way the course is structured, assessments will get progressively harder as we go through the semester. This means if you skip a quiz early on and decide it will be a "drop" quiz, that you will be trading a much easier quiz for a much harder one later on.
%    
%    Since we are dropping two quizzes, we will \textbf{not be offering makeup quizzes}. Keep this in mind when you are considering your scheduling for the semester. 
%
%\subsubsection*{Exams}
%    There are 3 exams during the semester, with a final at the end (for a total of 4 tests). The exam dates and content are as follows:
%    \begin{center}
%    \begin{tabular}{ |l l l| }
%    \hline
%    \textbf{Exam}       & \textbf{Date}     & \textbf{Content}\\ \hline 
%    Exam 1              & TBD     & Topics TBD\\
%    Exam 2              & TBD        & Topics TBD\\
%    Exam 3              & TBD        & Topics TBD\\
%    Final               & TBD        & Cumulative (All Topics)\\ \hline
%    \end{tabular}
%    \end{center}
%    
%    Keep in mind that this class is a pilot program and as such the content may take more or less time to get through. This means the ``Content" section above is the \textit{intended} content for that exam, but it may require some adjustment depending on how the pace of the course turns out. There will be notification well ahead of time if there is any intended changes to the content of any given exam, both in lecture and via email/announcements on Canvas.
    
\subsubsection*{An Important Note About Exam Design}
    
    Another remark about the exams is necessary. Typically, for most math courses, the class mean average exam score is in the $63\%-68$\% range. This often comes as a (rather unpleasant) shock to students, especially those that are newer to UF and are use to getting consistently excellent grades. The instructor and TAs will provide all the help they can, and there is unlimited practice offered as well (see `On-line homework' above), but ultimately you are on your own for exams, and they are exceptionally challenging. The exams are \textit{not} written with the intention that the problems are ones that you have already seen with different numbers. The purpose of this course is to teach you how to use mathematical tools to solve mathematical problems, which requires knowledge, understanding, and creativity to figure out which tool to use, when to use it, and how to use it correctly. We aren't trying to teach you to (only) follow a preset list of instructions. We are trying to teach you to be a problem solver; one who can utilize their knowledge and skills to unravel a completely new problem when they are confronted with one.

\subsubsection*{Final}
    There will be a final exam (See the official syllabus on Canvas for specifics on date/time). Your final will be cumulative, thus any content covered this semester is ``fair game" for the final (including any content covered after the third exam). The exact format of the final will be announced as we get closer to the date. Since the final is cumulative, I will replace your lowest exam score with half the grade of the final (only if it helps. Notice that the final is worth twice the amount of a standard exam, thus half the final grade will be equivalent to a single exam). This will be done automatically, \textbf{You do not need to request this}. 

%\subsubsection*{Model Project}
%    The beginning of this semester will cover how to apply mathematical modeling and problem solving in the real world. This is intended as a way to introduce functions, variables, and the other basic tools of mathematics, as well as deductive logic. Since this is intended to reflect a `real world' scenario, instead of having a standard exam, we will be doing a small group project. These will be assigned and graded by your TA who will be playing the role of your boss, giving your group an assignment to complete for your job. This will be explained further in lecture and in your discussion section.
            
\subsubsection*{Makeup Policies} %% Comment out if there are none.

    \begin{itemize}
        \item{\textbf{Xronos}:} There are no make-ups for Xronos.
        \item{\textbf{Class Participation}:} There are no make-ups for class participation.
        \item{\textbf{Quizzes:}} There are no makeups for quizzes; see syllabus for details.
        \item{\textbf{Exams:}} In order to get a makeup exam you must have a documented (and valid) reason to miss the exam. Otherwise you must rely on the half-final-grade option mentioned above. Only one makeup will be offered, and \textit{it will be held at the end of the semester}. Thus if you missed Exam 1 and have a valid (and documented) reason that warrants a makeup, it will still not be held until the end of the semester. Since there is only one makeup time, \textbf{only one makeup exam will be offered}. Even if you miss more than one exam, you may only make up \textbf{at most} one exam.
    \end{itemize}

%\subsubsection*{Grading Examples}
%
%
%The end of the semester approaches and a student look at their current grades which are as follows:
%\begin{itemize}
%    \item Discussion Quizzes:
%    \begin{itemize}
%        \item Quiz 1:   6
%        \item Quiz 2:   6
%        \item Quiz 3:   4
%        \item Quiz 4:   3
%        \item Quiz 5:   4
%        \item Quiz 6:   5
%        \item Quiz 7:   5
%        \item Quiz 8:   2
%        \item Quiz 9:   6
%        \item Quiz 10:  5.25
%        \item Quiz 11:  3.75
%        \item Quiz 12:  0
%    \end{itemize}
%    
%    \item Exams:
%    \begin{itemize}
%        \item Exam 1:   43
%        \item Exam 2:   48
%        \item Exam 3:   38
%        \item Final Exam:   88
%    \end{itemize}
%
%    \item Model Project:    48
%    \item Participation:    40
%    \item Xronos:           50
%\end{itemize}

%\begin{question}
%    How many points have they earned from the Discussion Quizzes in total? (As in, their total points earned out of 60)
%    $\answer{48}$.
%\end{question}
%
%\begin{question}
%    How many points have they earned under the exam category, including the Final Exam substitution if it applies, but \textbf{not} including the Final Exam itself?
%    $\answer{135}$.
%\end{question}
%
%\begin{question}
%    What are their total points earned in the course? $\answer{409}$.
%    \begin{question}
%        So their final grade is what? (Use capital letters) $\answer{A}$.
%    \end{question}
%\end{question}
%
%\begin{question}
%    The primary purpose of a recitation \textit{quiz} is which of the following?
%    \begin{multipleChoice}
%        \choice{To do many examples and demonstrate mechanics and individual steps of new techniques.}
%        \choice{To introduce new techniques and give broader context for how and why the technique should be used.}
%        \choice{To assess your problem solving skills, originality, and creativity in overcoming challenges.}
%        \choice[correct]{To assess your mechanical skills and computational skills.}
%        \choice{To assess your ability to synthesize (new and old) techniques to solve a given problem.}
%    \end{multipleChoice}
%\end{question}

\begin{question}
    The primary purpose of each exam is which of the following?
    \begin{multipleChoice}
        \choice{To do many examples and demonstrate mechanics and individual steps of new techniques.}
        \choice{To introduce new techniques and give broader context for how and why the technique should be used.}
        \choice{To assess your problem solving skills, originality, and creativity in overcoming challenges.}
        \choice{To assess your mechanical skills and computational skills.}
        \choice[correct]{To assess your ability to synthesize (new and old) techniques to solve a given problem.}
    \end{multipleChoice}
\end{question}

%\begin{question}
%    The primary purpose of the Modeling Project is which of the following?
%    \begin{multipleChoice}
%        \choice{To do many examples and demonstrate mechanics and individual steps of new techniques.}
%        \choice{To introduce new techniques and give broader context for how and why the technique should be used.}
%        \choice[correct]{To assess your problem solving skills, originality, and creativity in overcoming challenges.}
%        \choice{To assess your mechanical skills and computational skills.}
%        \choice{To assess your ability to synthesize (new and old) techniques to solve a given problem.}
%    \end{multipleChoice}
%\end{question}



\end{document}