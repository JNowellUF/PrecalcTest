\documentclass{ximera}

\title{Methods to Prepare}
\begin{document}
\begin{abstract}
Suggestions on Studying and Learning
\end{abstract}
\maketitle


Mathematics is something that needs to be practiced to learn and understand it. In many cases this means that students are given hordes of practice problems and told to complete them all for a grade; the hope being that you will gain understanding by repetition. Unfortunately in my experience, students often memorize the process, but don't stop to consider \textit{why} they are doing whatever they are doing (something often exasperated by not getting a satisfactory answer to the "why am I doing this" question). Although there is definitely value in repetitious practice, the value is usually gained by hoping that the repetition does one of two things;
\begin{enumerate}
\item More practice means it is less likely to make simple computational mistakes. Your teachers/professors are usually not much better at computation than you are, but they have vastly more experience and practice, meaning they make far fewer mistakes.
\item The ``why does this work" will suddenly ``click" into place. The hope is that by repeatedly practicing, the human instinct of pattern recognition will suddenly pick up on the underlying structure that is making whatever technique you are practicing suddenly clear as to why it works. Although this happens (and is often necessary, especially in higher level math) it can only happen if the student is looking for it. So the ``turn my brain off and get this done" approach is usually rather antithetical to the intent of the homework.
\end{enumerate}

This means that it is important, not only to practice the skills we discuss, but to constantly ask yourself the following questions;
\begin{itemize}
    \item Why does this work?
    \item What steps are necessary (and what steps are not necessary), for a given problem?
    \item What else could I use this technique on? 
    \item What does this technique really require?
\end{itemize}

My teaching philosophy is to put the impetus (responsibility) of learning on the student. I will answer any questions, and provide limitless practice in the form of reviews (see the syllabus for more information). Your TAs (and other TAs, and I) have office hours you can come and ask questions and get more information/help to understand techniques and content. But in the end it comes down to this: You get out of this course what you put into it. If you don't do any optional work, you will be lucky to pass the class, and you will \emph{definitely not} pass calculus 1.


\begin{question}
    The primary purpose of lectures is which of the following?
    \begin{multipleChoice}
        \choice{To do many examples and demonstrate mechanics and individual steps of new techniques.}
        \choice[correct]{To introduce new techniques and give broader context for how and why the technique should be used.}
        \choice{To assess your problem solving skills, originality, and creativity in overcoming challenges.}
        \choice{To assess your mechanical skills and computational skills.}
        \choice{To assess your ability to synthesize (new and old) techniques to solve a given problem.}
    \end{multipleChoice}
\end{question}

\begin{question}
    The primary purpose of recitation is which of the following?
    \begin{multipleChoice}
        \choice[correct]{To do many examples and demonstrate mechanics and individual steps of new techniques.}
        \choice{To introduce new techniques and give broader context for how and why the technique should be used.}
        \choice{To assess your problem solving skills, originality, and creativity in overcoming challenges.}
        \choice{To assess your mechanical skills and computational skills.}
        \choice{To assess your ability to synthesize (new and old) techniques to solve a given problem.}
    \end{multipleChoice}
\end{question}


\end{document}