\documentclass{ximera}

\title{Expect Differences}
\begin{document}
\begin{abstract}
What makes this course different from previous courses?
\end{abstract}
\maketitle
This course is under development as it is being rewritten from the ground up with a more practical use in mind, using more modern education theory techniques and emphasizing more mathematical concepts rather than raw techniques in isolation. Being in development has up and downsides. There is a video covering this below:

\youtube{x4p89W85qFQ}

In most math courses in the past you have likely been taught mechanics of mathematics; the computational tools like adding, subtracting, multiplying and dividing. These are definitely important things, but they are not what math \textit{is}. To use an analogy, ``learning math" by learning these mechanical skills, would be like trying to learn a foreign language by reading a translation dictionary. The vocabulary is important, but reading a dictionary is dry and dull and still doesn't teach you any of the important aspects of the foreign language (eg grammar, culture, history, etc).

Thus, this class is going to embrace the other aspects of mathematics you may not have had to learn in the past (or at least, not learn as deeply). This means that this class will often seem fundamentally at odds with what you think of when you think of a ``math class". My hope is that you will find this class more engaging and helpful; but regardless of the outcome, the techniques that will be discussed will be very different than what you have experienced in the past. I would recommend that you try to come to this class with an open mind and not treat this as ``just another math class". With any luck, that will be a good thing.


\begin{question}
    In lecture, the instructor goes through a problem but seems to be skipping several steps. Which of the following are good responses (select all that apply)?
    \begin{selectAll}
        \choice[correct]{Ask your TA to explain or redo the problem during recitation.}
        \choice[correct]{Raise your hand to ask the instructor to explain a specific step.}
        \choice[correct]{Go to the Instructor or any of the course TAs (not only your own) during office hours to ask for clarification.}
        \choice[correct]{Contact your TA or the instructor (via email) to setup a time to go over your questions at another time.}
    \end{selectAll}
\end{question}

\begin{question}
    What is the primary purpose/goal of this course (select all that apply)?
    \begin{selectAll}
        \choice{To memorize a bunch of pointless stuff you will never use.}
        \choice{To pretend to list uses for pointless stuff that you will never actually encounter.}
        \choice[correct]{To get everyone to the same mathematical level to progress to future math courses.}
        \choice[correct]{To teach students how to problem solve and think logically and deductively.}
        \choice[correct]{To introduce the necessary fundamentals of notation and mechanics for calculus (except trigonometry).}
    \end{selectAll}
\end{question}

\begin{problem}
    Since this course is in development it is still in flux. This means that you... (select all that apply)
    \begin{selectAll}
        \choice[correct]{Can impact what is improved and potentially how.}
        \choice[correct]{Influence the priority of improvements via discussion board posts in Canvas.}
        \choice{Will just have to deal with what you are given, even if it is terrible.}
        \choice[correct]{Some of the content may be incomplete, but you can ask questions about this in office hours or via email.}
        \choice[correct]{Some content you may have already covered will occasionally be updated with new resources and content. 
        
        So you should occasionally review sections you have already covered, looking for updates 
        
        (like the update button in the top right) for new content to help learning and understanding.}
    \end{selectAll}
\end{problem}




\end{document}