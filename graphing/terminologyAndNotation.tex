\documentclass{ximera}

\title{Terminology To Know}
\begin{document}
\begin{abstract}
    These are important terms and notations for this section.
\end{abstract}
\maketitle

Below is a quick-reference for definitions in this chapter.

\begin{definition}[Graph (of a function)]
    A visual representation of the relationship between domain and range, ie the ``x-y coordinate picture" of a function.
\end{definition}

\begin{definition}[(Cartesian) Coordinates]
    A method of graping a function where the domain and range meet at a right angle (ie the so-called ``x-y plane".)
\end{definition}

\begin{definition}[Precision]
    How exact (aka how specific) a value is. For example, $2.1343435$ is more \emph{precisely} determined than $2.134$ since it has considerably more digits given.
\end{definition}

\begin{definition}[Accuracy]
    How close to correct a value is. For example, $3.14$ is a more \emph{accurate} value of $\pi$ than $3.151592$, even though $3.151592$ is a \emph{more precise number} than $3.14$.
\end{definition}

\begin{definition}[Parent Function]
    A parent functions is the `prototypical' form of the given function type. That is to say, the `parent function' of a function type is the base (ie most basic) version of that function without any manipulations, shifts, or changes to it's form. \\
    \textbf{For example:} The parent function of the quadratic function would be $f(x) = x^2$. This is the base type without anything added to it.\\
    This is most commonly referenced by asking a question. \textbf{For example:} `What is the parent function type of the function $f(x) = x^2 + 2x - 3$?' In this case the answer would be $f(x) = x^2$ since the given function was a quadratic, and $x^2$ is the parent function for a quadratic.
\end{definition}

\end{document}