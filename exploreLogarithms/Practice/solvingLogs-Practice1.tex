\documentclass{ximera}
\title{Practice with Solving Logarithmic Equations}



\begin{document}
\input{Useful-Sage-Macros}

\begin{sagesilent}

###### Problem p1
p1b1 = RandInt(2,7)

p1c1 = NonZeroInt(-5,5)
p1c2 = RandInt(-5,5)
p1c3 = RandInt(-5,5)

p1f1 = p1c1*x + p1c2

p1ans1 = (p1b1^p1c3-p1c2)/p1c1


###### Problem p2
p2b1 = RandInt(2,7)

p2c1 = NonZeroInt(-5,5)
p2c2 = RandInt(-5,5)
p2c3 = RandInt(-5,5)

p2f1 = p2c1*x + p2c2

p2ans1 = (p2b1^p2c3-p2c2)/p2c1


###### Problem p3
p3b1 = RandInt(2,7)

p3c1 = NonZeroInt(-5,5)
p3c2 = RandInt(-5,5)
p3c3 = RandInt(-5,5)

p3f1 = p3c1*x + p3c2

p3ans1 = (p3b1^p3c3-p3c2)/p3c1


###### Problem p4
p4b1 = RandInt(2,7)

p4c1 = NonZeroInt(-5,5)
p4c2 = RandInt(-5,5)
p4c3 = RandInt(-5,5)

p4f1 = p4c1*x + p4c2

p4ans1 = (p4b1^p4c3-p4c2)/p4c1



\end{sagesilent}

\begin{problem}
    Solve for $x$ in the following logarithmic equation:
    \[
        \log_{\sage{p1b1}}\left(\sage{p1f1}\right) = \sage{p1c3}
    \]

    $x = \answer{\sage{p1ans1}}$.
    \begin{feedback}
        Start by rewriting the log into an exponential form (i.e. put each side of the equal sign as an exponent to a base which is the same value as the log's base). For example, start by rewriting $\log_{\sage{p1b1}}\left(\sage{p1f1}\right) = \sage{p1c3}$ as $\sage{p1f1} = \sage{p1b1}^{\sage{p1c3}}$, then proceed to solve $x$.
    \end{feedback}
\end{problem}


\begin{problem}
    Solve for $x$ in the following logarithmic equation:
    \[
        \log_{\sage{p2b1}}\left(\sage{p2f1}\right) = \sage{p2c3}
    \]

    $x = \answer{\sage{p2ans1}}$.
    \begin{feedback}
        Start by rewriting the log into an exponential form (i.e. put each side of the equal sign as an exponent to a base which is the same value as the log's base). For example, start by rewriting $\log_{\sage{p2b1}}\left(\sage{p2f1}\right) = \sage{p2c3}$ as $\sage{p2f1} = \sage{p2b1}^{\sage{p2c3}}$, then proceed to solve $x$.
    \end{feedback}
\end{problem}


\begin{problem}
    Solve for $x$ in the following logarithmic equation:
    \[
        \log_{\sage{p3b1}}\left(\sage{p3f1}\right) = \sage{p3c3}
    \]

    $x = \answer{\sage{p3ans1}}$.
    \begin{feedback}
        Start by rewriting the log into an exponential form (i.e. put each side of the equal sign as an exponent to a base which is the same value as the log's base). For example, start by rewriting $\log_{\sage{p3b1}}\left(\sage{p3f1}\right) = \sage{p3c3}$ as $\sage{p3f1} = \sage{p3b1}^{\sage{p3c3}}$, then proceed to solve $x$.
    \end{feedback}
\end{problem}


\begin{problem}
    Solve for $x$ in the following logarithmic equation:
    \[
        \log_{\sage{p4b1}}\left(\sage{p4f1}\right) = \sage{p4c3}
    \]

    $x = \answer{\sage{p4ans1}}$.
    \begin{feedback}
        Start by rewriting the log into an exponential form (i.e. put each side of the equal sign as an exponent to a base which is the same value as the log's base). For example, start by rewriting $\log_{\sage{p4b1}}\left(\sage{p4f1}\right) = \sage{p4c3}$ as $\sage{p4f1} = \sage{p4b1}^{\sage{p4c3}}$, then proceed to solve $x$.
    \end{feedback}
\end{problem}


\end{document}