\documentclass{ximera}
\input{../preamble.tex}


\title{Vertical Asymptotes}
\begin{document}
\begin{abstract}
    We discuss the circumstances that generate vertical asymptotes in rational functions.
\end{abstract}
\maketitle

You can watch a lecture video on this here!

\youtube{YMO6DUAOKPc}

\begin{javascript}
// A validator to check and verify something has a factored form...
function factorCheck(f,g) {
    // This validator is designed to check that a student is submitting a factored polynomial. It works by:
    //  Checking that there are the correct number of non-numeric and non-inverse factors as expected,
    //  Checking that the submitted answer and the expected answer are the same via real Xronos evaluation,
    //  Checking that the outer most (last to be computed when following order of operations) operation is multiplication.
    
    var operCheck = f.tree[0];// Check to see if the root operation is multiplication at end.
    var studentFactors = f.tree.length;// Temporary number of student-provided factors (+1 because of root operation)
    
    // Now we adjust the length to remove any numeric factors, or division factors, etc to avoid ``padding'' by students.
    for (var i = 0; i < f.tree.length; i++) {
        if ((typeof f.tree[i] === 'number')||(f.tree[i][0] == '-')||(f.tree[i][0] == '/')) {
            studentFactors = studentFactors - 1;
        }
    }
    
    // Now we do the same with the provided answer, in case sage or something provides a weird format.
    var answerFactors = g.tree.length;
    
    // Adjust length in the same way, so that it will match the students if it should.
    for (var i = 0; i < g.tree.length; i++) {
        if (typeof g.tree[i] === 'number') {
            answerFactors = answerFactors - 1;
        }
    }
    
    // Note: An especially dedicated student could pad with weird factors that are happen to cancel down to 1.
    // For example, a student could enter sin^2(x)+cos^2(x) as a multiplicative factor to pad the number of factors.
    // This would be somewhat difficult to think of, even on purpose.
    // Until I can reliably evaluate the factors themselves as functions though, there isn't a lot to be done here.
    
    return ((f.equals(g))&&(studentFactors==answerFactors)&&(operCheck=='*'))
};

\end{javascript}

We have encountered vertical asymptotes in the context of certain function types, but rational functions are special in that they give a way to generate multiple vertical asymptotes. Moreover, their location and functional behavior around the vertical asymptotes is not always the same. For this reason it warrants special attention. Unfortunately to properly study the nature of domain restrictions one needs limits, which we won't introduce until calculus. Despite this, there are some general guidelines we can use that will work so long as the functions in the numerator and denominator are \textit{continuous} functions (independently) near the domain restrictions.

In general, assuming the continuity condition mentioned above, if an $x$ value resolves to zero in the denominator function, and non-zero in the numerator function, then there is a vertical asymptote at that $x$ value. This is true even in a simplified version of the rational function.

\begin{example}
    Let $f(x) = x^2 - 3x + 2$ and $g(x) = x^2 + 3x - 10$ and define $r(x) = \left(\frac{f}{g}\right)(x) = \frac{x^2 - 3x + 2}{x^2 + 3x - 10}$. For what $x$ values does $r(x)$ definitely have a vertical asymptote?
    
    The first thing we want to do to solve this problem is to factor the numerator and denominator if possible. Doing so gets us $\frac{\answer[validator=factorCheck]{(x-2)(x-1)}}{\answer[validator=factorCheck]{(x-2)(x+5)}}$. This gives us domain restrictions of $x = -5$ and $x = 2$. 
    
    At this point we could check both these values. If we do; we get the numerator and denominator are both $0$ for $x=2$ and we get the numerator is $42$ and the denominator is $0$ when $x = -5$. From this we know that there is a vertical asymptote at $x=-5$. However, something of the form ``$\frac{0}{0}$'' it is called an \textit{indeterminate form} and we don't know what is happening at this $x$ value without doing further work. 
    
    Ideally we hope to simplify the function to resolve our indeterminate issue. Since we were able to factor the top and bottom, we can simplify the fraction down to $\frac{\answer{x-1}}{\answer{x+5}}$. Now if we check $x=2$ we end up with $\frac{1}{7}$, which is not ``non-zero over zero'' and thus is not a vertical asymptote (indeed, it is a hole in the graph, which we will discuss in the next section).
    
    So, we can conclude that the only vertical asymptote that we know exists for sure is at $x=-5$.
\end{example}

Now lets consider a more complicated example.

\begin{example}
    Consider the function
    \[
        r(x) = \frac{e^{3x} - e^{2x} - 2e^x}{3x^4 - 5x^3 - 17x^2 + 13 x + 6}
    \]
    Here we need to do some work to try and factor the top and bottom, but it is still possible. In the case of the bottom we can use rational root theorem and division to eventually factor it down to: $(x-1)(x+2)(x-3)(3x+1)$. For the top we can rewrite using rules of exponents to get $(e^x)^3 - (e^x)^2 -2x^x$. Substituting $u=e^x$ gives:
    \[
        (e^x)^3 - (e^x)^2 -2x^x = u^3 - u^2 - 2u = u(u-2)(u+1) = e^x(e^x-2)(e^x+1).
    \]
    So, the factored form becomes:
    \[
        r(x) = \frac{e^x(e^x-2)(e^x+1)}{(x-1)(x+2)(x-3)(3x+1)}
    \]
    So, the domain restrictions for $r(x)$ are (list from most negative to most positive): $\answer{-2}$, $\answer{-\frac{1}{3}}$,$\answer{1}$, $\answer{3}$.
    Plugging these in to the numerator to see which of these have the form ``non-zero over zero'' reveal that there are how many vertical asymptotes for $r(x)$?
    \begin{multipleChoice}
        \choice{$0$}
        \choice{$1$}
        \choice{$2$}
        \choice{$3$}
        \choice[correct]{$4$}
    \end{multipleChoice}
\end{example}

The takeaway is that, under certain continuity conditions near the domain restriction, if you try to evaluate a value and get something of the form ``non-zero over zero'' that domain restriction represents a vertical asymptote. If you still get something of the form ``zero over zero'' then it is indeterminate and you need better tools (limits) to determine what is happening. That leaves one more situation; ``something over non-zero''. In this last case, we have a hole at the domain restriction, and we discuss that next.



\end{document}