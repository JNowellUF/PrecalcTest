\documentclass{ximera}
\title{Determining which are Rational functions Practice 1}



\begin{document}


\begin{sagesilent}
var('x')

def RandInt(a,b):
    """ Returns a random integer in [`a`,`b`]. Note that `a` and `b` should be integers themselves to avoid unexpected behavior.
    """
    return QQ(randint(int(a),int(b)))
    # return choice(range(a,b+1))

def NonZeroInt(b,c, avoid = [0]):
    """ Returns a random integer in [`b`,`c`] which is not in `av`. 
        If `av` is not specified, defaults to a non-zero integer.
    """
    while True:
        a = RandInt(b,c)
        if a not in avoid:
            return a

## Build a vector of functions that are all zero at 0, so we can manipulate them and know where their zeros are regardless of which function type we roll.

###### Problem p1

p1c1 = NonZeroInt(-5,5)
p1c2 = NonZeroInt(-5,5,[0,p1c1])
p1c3 = NonZeroInt(-5,5,[0,p1c1,p1c2])

p1pwr1 = RandInt(0,3)
p1pwr2 = RandInt(0,3)
p1fTop = expand((x-p1c1)*(x-p1c2)^p1pwr1)
p1fBot = expand((x-p1c3)*(x-p1c2)^p1pwr2)

if p1pwr2 > p1pwr1:
    p1ansA = p1c3 + p1c2
else:
    p1ansA = p1c3



###### Problem p2

p2c1 = NonZeroInt(-5,5)
p2c2 = NonZeroInt(-5,5,[0,p2c1])
p2c3 = NonZeroInt(-5,5,[0,p2c1,p2c2])

p2pwr1 = RandInt(0,3)
p2pwr2 = RandInt(0,3)
p2fTop = expand((x-p2c1)*(x-p2c2)^p2pwr1)
p2fBot = expand((x-p2c3)*(x-p2c2)^p2pwr2)

if p2pwr2 > p2pwr1:
    p2ansA = p2c3 + p2c2
else:
    p2ansA = p2c3



###### Problem p3

p3c1 = NonZeroInt(-5,5)
p3c2 = NonZeroInt(-5,5,[0,p3c1])
p3c3 = NonZeroInt(-5,5,[0,p3c1,p3c2])

p3pwr1 = RandInt(0,3)
p3pwr2 = RandInt(0,3)
p3fTop = expand((x-p3c1)*(x-p3c2)^p3pwr1)
p3fBot = expand((x-p3c3)*(x-p3c2)^p3pwr2)

if p3pwr2 > p3pwr1:
    p3ansA = p3c3 + p3c2
else:
    p3ansA = p3c3




###### Problem p4

p4c1 = NonZeroInt(-8,8)
p4c2 = NonZeroInt(-8,8,[0,p4c1])
p4c3 = NonZeroInt(-8,8,[0,p4c1,p4c2])
p4c4 = NonZeroInt(-8,8,[0,p4c1,p4c2,p4c3])
p4c5 = NonZeroInt(-8,8,[0,p4c1,p4c2,p4c3,p4c4])
p4c6 = NonZeroInt(-8,8,[0,p4c1,p4c2,p4c3,p4c4,p4c5])

p4pwr1 = RandInt(0,3)
p4pwr2 = RandInt(0,3)
p4pwr3 = RandInt(0,3)
p4pwr4 = RandInt(0,3)
p4pwr5 = RandInt(0,3)
p4pwr6 = RandInt(0,3)
p4fTop = (x-p4c1)*(x-p4c2)^p4pwr1*(x-p4c3)^p4pwr3*(x-p4c4)^p4pwr4
p4fBot = (x-p4c5)*(x-p4c2)^p4pwr2*(x-p4c3)^p4pwr5*(x-p4c6)^p4pwr6

if p4pwr2 > p4pwr1:
    p4ansA1 = p4c2
else:
    p4ansA1 = 0
if p4pwr5 > p4pwr3:
    p4ansA2 = p4c3
else:
    p4ansA2 = 0
if p4pwr6 > 0:
    p4ansA3 = p4c6
else:
    p4ansA3 = 0
p4ansA = p4c5+p4ansA1+p4ansA2+p4ansA3



###### Problem p5

p5c1 = NonZeroInt(-8,8)
p5c2 = NonZeroInt(-8,8,[0,p5c1])
p5c3 = NonZeroInt(-8,8,[0,p5c1,p5c2])
p5c4 = NonZeroInt(-8,8,[0,p5c1,p5c2,p5c3])
p5c5 = NonZeroInt(-8,8,[0,p5c1,p5c2,p5c3,p5c4])
p5c6 = NonZeroInt(-8,8,[0,p5c1,p5c2,p5c3,p5c4,p5c5])

p5pwr1 = RandInt(0,3)
p5pwr2 = RandInt(0,3)
p5pwr3 = RandInt(0,3)
p5pwr4 = RandInt(0,3)
p5pwr5 = RandInt(0,3)
p5pwr6 = RandInt(0,3)
p5fTop = (x-p5c1)*(x-p5c2)^p5pwr1*(x-p5c3)^p5pwr3*(x-p5c4)^p5pwr4
p5fBot = (x-p5c5)*(x-p5c2)^p5pwr2*(x-p5c3)^p5pwr5*(x-p5c6)^p5pwr6

if p5pwr2 > p5pwr1:
    p5ansA1 = p5c2
else:
    p5ansA1 = 0
if p5pwr5 > p5pwr3:
    p5ansA2 = p5c3
else:
    p5ansA2 = 0
if p5pwr6 > 0:
    p5ansA3 = p5c6
else:
    p5ansA3 = 0
p5ansA = p5c5+p5ansA1+p5ansA2+p5ansA3



###### Problem p6

p6c1 = NonZeroInt(-8,8)
p6c2 = NonZeroInt(-8,8,[0,p6c1])
p6c3 = NonZeroInt(-8,8,[0,p6c1,p6c2])
p6c4 = NonZeroInt(-8,8,[0,p6c1,p6c2,p6c3])
p6c5 = NonZeroInt(-8,8,[0,p6c1,p6c2,p6c3,p6c4])
p6c6 = NonZeroInt(-8,8,[0,p6c1,p6c2,p6c3,p6c4,p6c5])

p6pwr1 = RandInt(0,3)
p6pwr2 = RandInt(0,3)
p6pwr3 = RandInt(0,3)
p6pwr4 = RandInt(0,3)
p6pwr5 = RandInt(0,3)
p6pwr6 = RandInt(0,3)
p6fTop = (x-p6c1)*(x-p6c2)^p6pwr1*(x-p6c3)^p6pwr3*(x-p6c4)^p6pwr4
p6fBot = (x-p6c5)*(x-p6c2)^p6pwr2*(x-p6c3)^p6pwr5*(x-p6c6)^p6pwr6

if p6pwr2 > p6pwr1:
    p6ansA1 = p6c2
else:
    p6ansA1 = 0
if p6pwr5 > p6pwr3:
    p6ansA2 = p6c3
else:
    p6ansA2 = 0
if p6pwr6 > 0:
    p6ansA3 = p6c6
else:
    p6ansA3 = 0
p6ansA = p6c5+p6ansA1+p6ansA2+p6ansA3



\end{sagesilent}

\begin{problem}
    Consider the following rational function:
    \[
        f(x) = \frac{\sage{p1fTop}}{\sage{p1fBot}}
    \]
    
    What is the sum of the $x$ values which have vertical asymptotes? $\answer{\sage{p1ansA}}$.
    \begin{feedback}
        First you need to find all the domain restrictions, i.e. where the denominator is zero. To do this, you may need to factor the denominator to find all the zeros. Remember though, that the values that give \textit{vertical asymptotes} are the values that remain after you fully simplify the rational function (by canceling factors in the top with factors in the bottom).
    \end{feedback}
\end{problem}




\begin{problem}
    Consider the following rational function:
    \[
        f(x) = \frac{\sage{p2fTop}}{\sage{p2fBot}}
    \]
    
    What is the sum of the $x$ values which have vertical asymptotes? $\answer{\sage{p2ansA}}$.
    \begin{feedback}
        First you need to find all the domain restrictions, i.e. where the denominator is zero. To do this, you may need to factor the denominator to find all the zeros. Remember though, that the values that give \textit{vertical asymptotes} are the values that remain after you fully simplify the rational function (by canceling factors in the top with factors in the bottom).
    \end{feedback}
\end{problem}




\begin{problem}
    Consider the following rational function:
    \[
        f(x) = \frac{\sage{p3fTop}}{\sage{p3fBot}}
    \]
    
    What is the sum of the $x$ values which have vertical asymptotes? $\answer{\sage{p3ansA}}$.
    \begin{feedback}
        First you need to find all the domain restrictions, i.e. where the denominator is zero. To do this, you may need to factor the denominator to find all the zeros. Remember though, that the values that give \textit{vertical asymptotes} are the values that remain after you fully simplify the rational function (by canceling factors in the top with factors in the bottom).
    \end{feedback}
\end{problem}




\begin{problem}
    Consider the following rational function:
    \[
        f(x) = \frac{\sage{p4fTop}}{\sage{p4fBot}}
    \]
    
    What is the sum of the $x$ values which have vertical asymptotes? $\answer{\sage{p4ansA}}$.
    \begin{feedback}
        First you need to find all the domain restrictions, i.e. where the denominator is zero. To do this, you may need to factor the denominator to find all the zeros. Remember though, that the values that give \textit{vertical asymptotes} are the values that remain after you fully simplify the rational function (by canceling factors in the top with factors in the bottom).
    \end{feedback}
\end{problem}




\begin{problem}
    Consider the following rational function:
    \[
        f(x) = \frac{\sage{p5fTop}}{\sage{p5fBot}}
    \]
    
    What is the sum of the $x$ values which have vertical asymptotes? $\answer{\sage{p5ansA}}$.
    \begin{feedback}
        First you need to find all the domain restrictions, i.e. where the denominator is zero. To do this, you may need to factor the denominator to find all the zeros. Remember though, that the values that give \textit{vertical asymptotes} are the values that remain after you fully simplify the rational function (by canceling factors in the top with factors in the bottom).
    \end{feedback}
\end{problem}




\begin{problem}
    Consider the following rational function:
    \[
        f(x) = \frac{\sage{p6fTop}}{\sage{p6fBot}}
    \]
    
    What is the sum of the $x$ values which have vertical asymptotes? $\answer{\sage{p6ansA}}$.
    \begin{feedback}
        First you need to find all the domain restrictions, i.e. where the denominator is zero. To do this, you may need to factor the denominator to find all the zeros. Remember though, that the values that give \textit{vertical asymptotes} are the values that remain after you fully simplify the rational function (by canceling factors in the top with factors in the bottom).
    \end{feedback}
\end{problem}





\end{document}