\documentclass{ximera}
\title{Determining which are Rational functions Practice 1}



\begin{document}


\begin{sagesilent}
var('x')

def RandInt(a,b):
    """ Returns a random integer in [`a`,`b`]. Note that `a` and `b` should be integers themselves to avoid unexpected behavior.
    """
    return QQ(randint(int(a),int(b)))
    # return choice(range(a,b+1))

def NonZeroInt(b,c, avoid = [0]):
    """ Returns a random integer in [`b`,`c`] which is not in `av`. 
        If `av` is not specified, defaults to a non-zero integer.
    """
    while True:
        a = RandInt(b,c)
        if a not in avoid:
            return a

## Build a vector of functions that are all zero at 0, so we can manipulate them and know where their zeros are regardless of which function type we roll.

###### Problem p1

p1c1 = NonZeroInt(-5,5)
p1c2 = NonZeroInt(-5,5,[0,p1c1])
p1c3 = NonZeroInt(-5,5,[0,p1c1,p1c2])

p1pwr1 = RandInt(0,2)
p1pwr2 = RandInt(0,2)
p1pwr3 = RandInt(1,2)
p1fTop = expand((x-p1c1)*(x-p1c2)^p1pwr1*(x-p1c3)^p1pwr3)
p1fBot = expand((x-p1c3)*(x-p1c2)^p1pwr2)

if p1pwr2 > p1pwr1 or p1pwr2 == 0:
    p1ansA = p1c3
else:
    p1ansA = p1c3 + p1c2

p1fcalc = (x-p1c1)*(x-p1c2)^(p1pwr1-p1pwr2)*(x-p1c3)^(p1pwr3-1)
p1xhole = p1c3
p1yhole = p1fcalc(x=p1xhole)


###### Problem p2

p2c1 = NonZeroInt(-5,5)
p2c2 = NonZeroInt(-5,5,[0,p2c1])
p2c3 = NonZeroInt(-5,5,[0,p2c1,p2c2])

p2pwr1 = RandInt(0,2)
p2pwr2 = RandInt(0,2)
p2pwr3 = RandInt(1,2)
p2fTop = expand((x-p2c1)*(x-p2c2)^p2pwr1*(x-p2c3)^p2pwr3)
p2fBot = expand((x-p2c3)*(x-p2c2)^p2pwr2)

if p2pwr2 > p2pwr1 or p2pwr2 == 0:
    p2ansA = p2c3
else:
    p2ansA = p2c3 + p2c2

p2fcalc = (x-p2c1)*(x-p2c2)^(p2pwr1-p2pwr2)*(x-p2c3)^(p2pwr3-1)
p2xhole = p2c3
p2yhole = p2fcalc(x=p2xhole)


###### Problem p3

p3c1 = NonZeroInt(-5,5)
p3c2 = NonZeroInt(-5,5,[0,p3c1])
p3c3 = NonZeroInt(-5,5,[0,p3c1,p3c2])

p3pwr1 = RandInt(0,2)
p3pwr2 = RandInt(0,2)
p3pwr3 = RandInt(1,2)
p3fTop = expand((x-p3c1)*(x-p3c2)^p3pwr1*(x-p3c3)^p3pwr3)
p3fBot = expand((x-p3c3)*(x-p3c2)^p3pwr2)

if p3pwr2 > p3pwr1 or p3pwr2 == 0:
    p3ansA = p3c3
else:
    p3ansA = p3c3 + p3c2

p3fcalc = (x-p3c1)*(x-p3c2)^(p3pwr1-p3pwr2)*(x-p3c3)^(p3pwr3-1)
p3xhole = p3c3
p3yhole = p3fcalc(x=p3xhole)





###### Problem p4

p4c1 = NonZeroInt(-8,8)
p4c2 = NonZeroInt(-8,8,[0,p4c1])
p4c3 = NonZeroInt(-8,8,[0,p4c1,p4c2])
p4c4 = NonZeroInt(-8,8,[0,p4c1,p4c2,p4c3])
p4c5 = NonZeroInt(-8,8,[0,p4c1,p4c2,p4c3,p4c4])
p4c6 = NonZeroInt(-8,8,[0,p4c1,p4c2,p4c3,p4c4,p4c5])

p4pwr1 = RandInt(0,3)
p4pwr2 = RandInt(0,3)
p4pwr3 = RandInt(0,3)
p4pwr4 = RandInt(1,3)
p4pwr5 = RandInt(0,3)
p4pwr6 = RandInt(0,3)
p4fTop = (x-p4c4)^p4pwr4*   (x-p4c2)^p4pwr1*   (x-p4c3)^p4pwr3*    (x-p4c1)
p4fBot = (x-p4c4)*          (x-p4c2)^p4pwr2*   (x-p4c3)^p4pwr5*    (x-p4c6)^p4pwr6

if p4pwr2 > p4pwr1 or p4pwr2 == 0:
    p4ansA1 = 0
else:
    p4ansA1 = p4c2

if p4pwr5 > p4pwr3 or p4pwr5 == 0:
    p4ansA2 = 0
else:
    p4ansA2 = p4c3

p4ansA = p4c4 + p4ansA1 + p4ansA2

p4fcalc = (x-p4c4)^(p4pwr4-1)*(x-p4c2)^(p4pwr1-p4pwr2)*(x-p4c3)^(p4pwr3-p4pwr5)*(x-p4c1)/(x-p4c6)^p4pwr6
p4xhole = p4c4
p4yhole = p4fcalc(x=p4xhole)





###### Problem p5

p5c1 = NonZeroInt(-8,8)
p5c2 = NonZeroInt(-8,8,[0,p5c1])
p5c3 = NonZeroInt(-8,8,[0,p5c1,p5c2])
p5c4 = NonZeroInt(-8,8,[0,p5c1,p5c2,p5c3])
p5c5 = NonZeroInt(-8,8,[0,p5c1,p5c2,p5c3,p5c4])
p5c6 = NonZeroInt(-8,8,[0,p5c1,p5c2,p5c3,p5c4,p5c5])

p5pwr1 = RandInt(0,3)
p5pwr2 = RandInt(0,3)
p5pwr3 = RandInt(0,3)
p5pwr4 = RandInt(1,3)
p5pwr5 = RandInt(0,3)
p5pwr6 = RandInt(0,3)
p5fTop = (x-p5c4)^p5pwr4*   (x-p5c2)^p5pwr1*   (x-p5c3)^p5pwr3*    (x-p5c1)
p5fBot = (x-p5c4)*          (x-p5c2)^p5pwr2*   (x-p5c3)^p5pwr5*    (x-p5c6)^p5pwr6

if p5pwr2 > p5pwr1 or p5pwr2 == 0:
    p5ansA1 = 0
else:
    p5ansA1 = p5c2

if p5pwr5 > p5pwr3 or p5pwr5 == 0:
    p5ansA2 = 0
else:
    p5ansA2 = p5c3

p5ansA = p5c4 + p5ansA1 + p5ansA2

p5fcalc = (x-p5c4)^(p5pwr4-1)*(x-p5c2)^(p5pwr1-p5pwr2)*(x-p5c3)^(p5pwr3-p5pwr5)*(x-p5c1)/(x-p5c6)^p5pwr6
p5xhole = p5c4
p5yhole = p5fcalc(x=p5xhole)




###### Problem p6

p6c1 = NonZeroInt(-8,8)
p6c2 = NonZeroInt(-8,8,[0,p6c1])
p6c3 = NonZeroInt(-8,8,[0,p6c1,p6c2])
p6c4 = NonZeroInt(-8,8,[0,p6c1,p6c2,p6c3])
p6c5 = NonZeroInt(-8,8,[0,p6c1,p6c2,p6c3,p6c4])
p6c6 = NonZeroInt(-8,8,[0,p6c1,p6c2,p6c3,p6c4,p6c5])

p6pwr1 = RandInt(0,3)
p6pwr2 = RandInt(0,3)
p6pwr3 = RandInt(0,3)
p6pwr4 = RandInt(1,3)
p6pwr5 = RandInt(0,3)
p6pwr6 = RandInt(0,3)
p6fTop = (x-p6c4)^p6pwr4*   (x-p6c2)^p6pwr1*   (x-p6c3)^p6pwr3*    (x-p6c1)
p6fBot = (x-p6c4)*          (x-p6c2)^p6pwr2*   (x-p6c3)^p6pwr5*    (x-p6c6)^p6pwr6

if p6pwr2 > p6pwr1 or p6pwr2 == 0:
    p6ansA1 = 0
else:
    p6ansA1 = p6c2

if p6pwr5 > p6pwr3 or p6pwr5 == 0:
    p6ansA2 = 0
else:
    p6ansA2 = p6c3

p6ansA = p6c4 + p6ansA1 + p6ansA2

p6fcalc = (x-p6c4)^(p6pwr4-1)*(x-p6c2)^(p6pwr1-p6pwr2)*(x-p6c3)^(p6pwr3-p6pwr5)*(x-p6c1)/(x-p6c6)^p6pwr6
p6xhole = p6c4
p6yhole = p6fcalc(x=p6xhole)




\end{sagesilent}

\begin{problem}
    Consider the following rational function:
    \[
        f(x) = \frac{\sage{p1fTop}}{\sage{p1fBot}}
    \]
    
    What is the sum of the $x$ values which have holes? $\answer{\sage{p1ansA}}$.
    \begin{feedback}
        First you need to find all the domain restrictions, i.e. where the denominator is zero. To do this, you may need to factor the denominator to find all the zeros. Remember though, that the values that give \textit{holes} are the values that are ``canceled out'' of the denominator when you fully simplify the rational function (by canceling factors in the top with factors in the bottom).
    \end{feedback}
    \begin{problem}
        What are the coordinates of the hole with $x$-value of $\sage{p1xhole}$? $(\answer{\sage{p1xhole}},\answer{\sage{p1yhole}})$
        \begin{feedback}
            Since you found the hole's $x$-value to be $\sage{p1xhole}$ in the previous step, you want to plug this number \textit{into the simplified function} (after you have canceled factors in the top and bottom) to get the $y$-value of the hole.
        \end{feedback}
    \end{problem}

\end{problem}


\begin{problem}
    Consider the following rational function:
    \[
        f(x) = \frac{\sage{p2fTop}}{\sage{p2fBot}}
    \]
    
    What is the sum of the $x$ values which have holes? $\answer{\sage{p2ansA}}$.
    \begin{feedback}
        First you need to find all the domain restrictions, i.e. where the denominator is zero. To do this, you may need to factor the denominator to find all the zeros. Remember though, that the values that give \textit{holes} are the values that are ``canceled out'' of the denominator when you fully simplify the rational function (by canceling factors in the top with factors in the bottom).
    \end{feedback}
    \begin{problem}
        What are the coordinates of the hole with $x$-value of $\sage{p2xhole}$? $(\answer{\sage{p2xhole}},\answer{\sage{p2yhole}})$
        \begin{feedback}
            Since you found the hole's $x$-value to be $\sage{p2xhole}$ in the previous step, you want to plug this number \textit{into the simplified function} (after you have canceled factors in the top and bottom) to get the $y$-value of the hole.
        \end{feedback}
    \end{problem}

\end{problem}


\begin{problem}
    Consider the following rational function:
    \[
        f(x) = \frac{\sage{p3fTop}}{\sage{p3fBot}}
    \]
    
    What is the sum of the $x$ values which have holes? $\answer{\sage{p3ansA}}$.
    \begin{feedback}
        First you need to find all the domain restrictions, i.e. where the denominator is zero. To do this, you may need to factor the denominator to find all the zeros. Remember though, that the values that give \textit{holes} are the values that are ``canceled out'' of the denominator when you fully simplify the rational function (by canceling factors in the top with factors in the bottom).
    \end{feedback}
    \begin{problem}
        What are the coordinates of the hole with $x$-value of $\sage{p3xhole}$? $(\answer{\sage{p3xhole}},\answer{\sage{p3yhole}})$
        \begin{feedback}
            Since you found the hole's $x$-value to be $\sage{p3xhole}$ in the previous step, you want to plug this number \textit{into the simplified function} (after you have canceled factors in the top and bottom) to get the $y$-value of the hole.
        \end{feedback}
    \end{problem}

\end{problem}




\begin{problem}
    Consider the following rational function:
    \[
        f(x) = \frac{\sage{p4fTop}}{\sage{p4fBot}}
    \]
    
    What is the sum of the $x$ values which have holes? $\answer{\sage{p4ansA}}$.
    \begin{feedback}
        First you need to find all the domain restrictions, i.e. where the denominator is zero. To do this, you may need to factor the denominator to find all the zeros. Remember though, that the values that give \textit{holes} are the values that are ``canceled out'' of the denominator when you fully simplify the rational function (by canceling factors in the top with factors in the bottom).
    \end{feedback}
    \begin{problem}
        What are the coordinates of the hole with $x$-value of $\sage{p4xhole}$? $(\answer{\sage{p4xhole}},\answer{\sage{p4yhole}})$
        \begin{feedback}
            Since you found the hole's $x$-value to be $\sage{p4xhole}$ in the previous step, you want to plug this number \textit{into the simplified function} (after you have canceled factors in the top and bottom) to get the $y$-value of the hole.
        \end{feedback}
    \end{problem}

\end{problem}


\begin{problem}
    Consider the following rational function:
    \[
        f(x) = \frac{\sage{p5fTop}}{\sage{p5fBot}}
    \]
    
    What is the sum of the $x$ values which have holes? $\answer{\sage{p5ansA}}$.
    \begin{feedback}
        First you need to find all the domain restrictions, i.e. where the denominator is zero. To do this, you may need to factor the denominator to find all the zeros. Remember though, that the values that give \textit{holes} are the values that are ``canceled out'' of the denominator when you fully simplify the rational function (by canceling factors in the top with factors in the bottom).
    \end{feedback}
    \begin{problem}
        What are the coordinates of the hole with $x$-value of $\sage{p5xhole}$? $(\answer{\sage{p5xhole}},\answer{\sage{p5yhole}})$
        \begin{feedback}
            Since you found the hole's $x$-value to be $\sage{p5xhole}$ in the previous step, you want to plug this number \textit{into the simplified function} (after you have canceled factors in the top and bottom) to get the $y$-value of the hole.
        \end{feedback}
    \end{problem}

\end{problem}


\begin{problem}
    Consider the following rational function:
    \[
        f(x) = \frac{\sage{p6fTop}}{\sage{p6fBot}}
    \]
    
    What is the sum of the $x$ values which have holes? $\answer{\sage{p6ansA}}$.
    \begin{feedback}
        First you need to find all the domain restrictions, i.e. where the denominator is zero. To do this, you may need to factor the denominator to find all the zeros. Remember though, that the values that give \textit{holes} are the values that are ``canceled out'' of the denominator when you fully simplify the rational function (by canceling factors in the top with factors in the bottom).
    \end{feedback}
    \begin{problem}
        What are the coordinates of the hole with $x$-value of $\sage{p6xhole}$? $(\answer{\sage{p6xhole}},\answer{\sage{p6yhole}})$
        \begin{feedback}
            Since you found the hole's $x$-value to be $\sage{p6xhole}$ in the previous step, you want to plug this number \textit{into the simplified function} (after you have canceled factors in the top and bottom) to get the $y$-value of the hole.
        \end{feedback}
    \end{problem}

\end{problem}





\end{document}