\documentclass{ximera}
\title{Determining which are Rational functions Practice 1}



\begin{document}


\begin{sagesilent}
var('x')

def RandInt(a,b):
    """ Returns a random integer in [`a`,`b`]. Note that `a` and `b` should be integers themselves to avoid unexpected behavior.
    """
    return QQ(randint(int(a),int(b)))
    # return choice(range(a,b+1))

def NonZeroInt(b,c, avoid = [0]):
    """ Returns a random integer in [`b`,`c`] which is not in `av`. 
        If `av` is not specified, defaults to a non-zero integer.
    """
    while True:
        a = RandInt(b,c)
        if a not in avoid:
            return a

## Build a vector of functions that are all zero at 0, so we can manipulate them and know where their zeros are regardless of which function type we roll.
funcvec = [x,x^2,x^3,sqrt(abs(x)),ln(abs(x)+1),e^x-1,abs(x)]


###### Problem p1

## Pick the relevant functions
p1f1pick = RandInt(0,6)
p1f2pick = RandInt(0,6)
p1f3pick = RandInt(0,6)
p1f4pick = RandInt(0,6)
p1f5pick = RandInt(0,6)
p1f6pick = RandInt(0,6)

## Generate the important constants that determine where things equal zero.
p1ans1 = 1
p1ans2 = 1
p1ans3 = 1
p1ans4 = 1
p1ans5 = 1
p1ans6 = 1

## Make sure we get 6 distinct magnitudes for answers, so the generated answers are all distinct in dropdown box.
while len(uniq([abs(p1ans1),abs(p1ans2),abs(p1ans3),abs(p1ans4),abs(p1ans5),abs(p1ans6)])) < 6:
    p1c1 = NonZeroInt(-10,10)
    p1c2 = NonZeroInt(-10,10)
    p1c3 = NonZeroInt(-10,10)
    p1c4 = NonZeroInt(-10,10)
    p1c5 = NonZeroInt(-10,10)
    p1c6 = NonZeroInt(-10,10)
    p1c7 = NonZeroInt(-10,10)
    p1c8 = NonZeroInt(-10,10)
    p1c9 = NonZeroInt(-10,10)
    p1c10 = NonZeroInt(-10,10)
    p1c11 = NonZeroInt(-10,10)
    p1c12 = NonZeroInt(-10,10)
    p1ans1 = p1c2/p1c1
    p1ans2 = p1c4/p1c3
    p1ans3 = p1c6/p1c5
    p1ans4 = p1c8/p1c7
    p1ans5 = p1c10/p1c9
    p1ans6 = p1c12/p1c11

## Build functions for top (f1,f2,f3) and bottom (f4,f5,f6)
p1f1 = funcvec[p1f1pick](x=(p1c1*x-p1c2))
p1f2 = funcvec[p1f2pick](x=(p1c3*x-p1c4))
p1f3 = funcvec[p1f3pick](x=(p1c5*x-p1c6))
p1f4 = funcvec[p1f4pick](x=(p1c7*x-p1c8))
p1f5 = funcvec[p1f5pick](x=(p1c9*x-p1c10))
p1f6 = funcvec[p1f6pick](x=(p1c11*x-p1c12))

## Make display functions for Top and Bottom of rational function.
p1fTop = p1f1*p1f2*p1f3
p1fBot = p1f4*p1f5*p1f6

## Build false answers and order correct answers to ensure the domain is built correctly.
p1ans1f = -p1ans1
p1ans2f = -p1ans2
p1ans3f = -p1ans3
p1ans4f = -p1ans4
p1ans5f = -p1ans5
p1ans6f = -p1ans6
p1ansvec = [p1ans4,p1ans5,p1ans6]
p1ansvec.sort()

## The answer to the ``sum of domain restriction x-values'' question.
p1ansA = p1ans4+p1ans5+p1ans6



###### Problem p2

## Pick the relevant functions
p2f1pick = RandInt(0,6)
p2f2pick = RandInt(0,6)
p2f3pick = RandInt(0,6)
p2f4pick = RandInt(0,6)
p2f5pick = RandInt(0,6)
p2f6pick = RandInt(0,6)

## Generate the important constants that determine where things equal zero.
p2ans1 = 1
p2ans2 = 1
p2ans3 = 1
p2ans4 = 1
p2ans5 = 1
p2ans6 = 1

## Make sure we get 6 distinct magnitudes for answers, so the generated answers are all distinct in dropdown box.
while len(uniq([abs(p2ans1),abs(p2ans2),abs(p2ans3),abs(p2ans4),abs(p2ans5),abs(p2ans6)])) < 6:
    p2c1 = NonZeroInt(-10,10)
    p2c2 = NonZeroInt(-10,10)
    p2c3 = NonZeroInt(-10,10)
    p2c4 = NonZeroInt(-10,10)
    p2c5 = NonZeroInt(-10,10)
    p2c6 = NonZeroInt(-10,10)
    p2c7 = NonZeroInt(-10,10)
    p2c8 = NonZeroInt(-10,10)
    p2c9 = NonZeroInt(-10,10)
    p2c10 = NonZeroInt(-10,10)
    p2c11 = NonZeroInt(-10,10)
    p2c12 = NonZeroInt(-10,10)
    p2ans1 = p2c2/p2c1
    p2ans2 = p2c4/p2c3
    p2ans3 = p2c6/p2c5
    p2ans4 = p2c8/p2c7
    p2ans5 = p2c10/p2c9
    p2ans6 = p2c12/p2c11

## Build functions for top (f1,f2,f3) and bottom (f4,f5,f6)
p2f1 = funcvec[p2f1pick](x=(p2c1*x-p2c2))
p2f2 = funcvec[p2f2pick](x=(p2c3*x-p2c4))
p2f3 = funcvec[p2f3pick](x=(p2c5*x-p2c6))
p2f4 = funcvec[p2f4pick](x=(p2c7*x-p2c8))
p2f5 = funcvec[p2f5pick](x=(p2c9*x-p2c10))
p2f6 = funcvec[p2f6pick](x=(p2c11*x-p2c12))

## Make display functions for Top and Bottom of rational function.
p2fTop = p2f1*p2f2*p2f3
p2fBot = p2f4*p2f5*p2f6

## Build false answers and order correct answers to ensure the domain is built correctly.
p2ans1f = -p2ans1
p2ans2f = -p2ans2
p2ans3f = -p2ans3
p2ans4f = -p2ans4
p2ans5f = -p2ans5
p2ans6f = -p2ans6
p2ansvec = [p2ans4,p2ans5,p2ans6]
p2ansvec.sort()

## The answer to the ``sum of domain restriction x-values'' question.
p2ansA = p2ans4+p2ans5+p2ans6



###### Problem p3

## Pick the relevant functions
p3f1pick = RandInt(0,6)
p3f2pick = RandInt(0,6)
p3f3pick = RandInt(0,6)
p3f4pick = RandInt(0,6)
p3f5pick = RandInt(0,6)
p3f6pick = RandInt(0,6)

## Generate the important constants that determine where things equal zero.
p3ans1 = 1
p3ans2 = 1
p3ans3 = 1
p3ans4 = 1
p3ans5 = 1
p3ans6 = 1

## Make sure we get 6 distinct magnitudes for answers, so the generated answers are all distinct in dropdown box.
while len(uniq([abs(p3ans1),abs(p3ans2),abs(p3ans3),abs(p3ans4),abs(p3ans5),abs(p3ans6)])) < 6:
    p3c1 = NonZeroInt(-10,10)
    p3c2 = NonZeroInt(-10,10)
    p3c3 = NonZeroInt(-10,10)
    p3c4 = NonZeroInt(-10,10)
    p3c5 = NonZeroInt(-10,10)
    p3c6 = NonZeroInt(-10,10)
    p3c7 = NonZeroInt(-10,10)
    p3c8 = NonZeroInt(-10,10)
    p3c9 = NonZeroInt(-10,10)
    p3c10 = NonZeroInt(-10,10)
    p3c11 = NonZeroInt(-10,10)
    p3c12 = NonZeroInt(-10,10)
    p3ans1 = p3c2/p3c1
    p3ans2 = p3c4/p3c3
    p3ans3 = p3c6/p3c5
    p3ans4 = p3c8/p3c7
    p3ans5 = p3c10/p3c9
    p3ans6 = p3c12/p3c11

## Build functions for top (f1,f2,f3) and bottom (f4,f5,f6)
p3f1 = funcvec[p3f1pick](x=(p3c1*x-p3c2))
p3f2 = funcvec[p3f2pick](x=(p3c3*x-p3c4))
p3f3 = funcvec[p3f3pick](x=(p3c5*x-p3c6))
p3f4 = funcvec[p3f4pick](x=(p3c7*x-p3c8))
p3f5 = funcvec[p3f5pick](x=(p3c9*x-p3c10))
p3f6 = funcvec[p3f6pick](x=(p3c11*x-p3c12))

## Make display functions for Top and Bottom of rational function.
p3fTop = p3f1*p3f2*p3f3
p3fBot = p3f4*p3f5*p3f6

## Build false answers and order correct answers to ensure the domain is built correctly.
p3ans1f = -p3ans1
p3ans2f = -p3ans2
p3ans3f = -p3ans3
p3ans4f = -p3ans4
p3ans5f = -p3ans5
p3ans6f = -p3ans6
p3ansvec = [p3ans4,p3ans5,p3ans6]
p3ansvec.sort()

## The answer to the ``sum of domain restriction x-values'' question.
p3ansA = p3ans4+p3ans5+p3ans6




\end{sagesilent}

\begin{problem}
    Consider the following rational function:
    \[
        f(x) = \frac{\sage{p1fTop}}{\sage{p1fBot}}
    \]
    
    What is the sum of the domain restriction x-values? $\answer{\sage{p1ansA}}$
    \begin{feedback}
        Start by finding all the zeros of the denominator. Notice that the denominator is already factored, so you can find the zeros by finding when each factor equals zero. Once you have those, add them up to find the sum!
    \end{feedback}
    
    \begin{problem}
        What is the domain of this function?
        
        $(-\infty,$
            \wordChoice{
                \choice[correct]{$\sage{p1ansvec[0]}$}
                \choice{$\sage{p1ans1f}$}
                \choice{$\sage{p1ans2}$}
                \choice{$\sage{p1ans2f}$}
                \choice{$\sage{p1ansvec[1]}$}
                \choice{$\sage{p1ans1}$}
                \choice{$\sage{p1ans3}$}
                \choice{$\sage{p1ans3f}$}
                \choice{$\sage{p1ansvec[2]}$}
                \choice{$\sage{p1ans4f}$}
                \choice{$\sage{p1ans5f}$}
                \choice{$\sage{p1ans6f}$}
                }
            \wordChoice{
                \choice{$($}
                \choice{$[$}
                \choice{$]$}
                \choice[correct]{$)$}
                }
                $\cup$
            \wordChoice{
                \choice[correct]{$($}
                \choice{$[$}
                \choice{$]$}
                \choice{$)$}
                }
            \wordChoice{
                \choice[correct]{$\sage{p1ansvec[0]}$}
                \choice{$\sage{p1ans1f}$}
                \choice{$\sage{p1ans2}$}
                \choice{$\sage{p1ans2f}$}
                \choice{$\sage{p1ansvec[1]}$}
                \choice{$\sage{p1ans1}$}
                \choice{$\sage{p1ans3}$}
                \choice{$\sage{p1ans3f}$}
                \choice{$\sage{p1ansvec[2]}$}
                \choice{$\sage{p1ans4f}$}
                \choice{$\sage{p1ans5f}$}
                \choice{$\sage{p1ans6f}$}
                },
            \wordChoice{
                \choice{$\sage{p1ansvec[0]}$}
                \choice{$\sage{p1ans1f}$}
                \choice{$\sage{p1ans2}$}
                \choice{$\sage{p1ans2f}$}
                \choice[correct]{$\sage{p1ansvec[1]}$}
                \choice{$\sage{p1ans1}$}
                \choice{$\sage{p1ans3}$}
                \choice{$\sage{p1ans3f}$}
                \choice{$\sage{p1ansvec[2]}$}
                \choice{$\sage{p1ans4f}$}
                \choice{$\sage{p1ans5f}$}
                \choice{$\sage{p1ans6f}$}
                }
            \wordChoice{
                \choice{$($}
                \choice{$[$}
                \choice{$]$}
                \choice[correct]{$)$}
                }
                $\cup$
            \wordChoice{
                \choice[correct]{$($}
                \choice{$[$}
                \choice{$]$}
                \choice{$)$}
                }
            \wordChoice{
                \choice{$\sage{p1ansvec[0]}$}
                \choice{$\sage{p1ans1f}$}
                \choice{$\sage{p1ans2}$}
                \choice{$\sage{p1ans2f}$}
                \choice[correct]{$\sage{p1ansvec[1]}$}
                \choice{$\sage{p1ans1}$}
                \choice{$\sage{p1ans3}$}
                \choice{$\sage{p1ans3f}$}
                \choice{$\sage{p1ansvec[2]}$}
                \choice{$\sage{p1ans4f}$}
                \choice{$\sage{p1ans5f}$}
                \choice{$\sage{p1ans6f}$}
                },
            \wordChoice{
                \choice{$\sage{p1ansvec[0]}$}
                \choice{$\sage{p1ans1f}$}
                \choice{$\sage{p1ans2}$}
                \choice{$\sage{p1ans2f}$}
                \choice{$\sage{p1ansvec[1]}$}
                \choice{$\sage{p1ans1}$}
                \choice{$\sage{p1ans3}$}
                \choice{$\sage{p1ans3f}$}
                \choice[correct]{$\sage{p1ansvec[2]}$}
                \choice{$\sage{p1ans4f}$}
                \choice{$\sage{p1ans5f}$}
                \choice{$\sage{p1ans6f}$}
                }
            \wordChoice{
                \choice{$($}
                \choice{$[$}
                \choice{$]$}
                \choice[correct]{$)$}
                }
                $\cup$
            \wordChoice{
                \choice[correct]{$($}
                \choice{$[$}
                \choice{$]$}
                \choice{$)$}
                }
            \wordChoice{
                \choice{$\sage{p1ansvec[0]}$}
                \choice{$\sage{p1ans1f}$}
                \choice{$\sage{p1ans2}$}
                \choice{$\sage{p1ans2f}$}
                \choice{$\sage{p1ansvec[1]}$}
                \choice{$\sage{p1ans1}$}
                \choice{$\sage{p1ans3}$}
                \choice{$\sage{p1ans3f}$}
                \choice[correct]{$\sage{p1ansvec[2]}$}
                \choice{$\sage{p1ans4f}$}
                \choice{$\sage{p1ans5f}$}
                \choice{$\sage{p1ans6f}$}
                },
            $\infty)$
            \begin{feedback}
                You need to make sure to avoid the values that make the denominator zero. Thus make sure you use the excluding symbol (the parentheses) and the end points of the intervals should be the the values that you found as zeros in the denominator (in increasing order; order matters!)
            \end{feedback}
    \end{problem}
\end{problem}




\begin{problem}
    Consider the following rational function:
    \[
        f(x) = \frac{\sage{p2fTop}}{\sage{p2fBot}}
    \]
    
    What is the sum of the domain restriction x-values? $\answer{\sage{p2ansA}}$
        \begin{feedback}
            Start by finding all the zeros of the denominator. Notice that the denominator is already factored, so you can find the zeros by finding when each factor equals zero. Once you have those, add them up to find the sum!
        \end{feedback}
    
    \begin{problem}
        What is the domain of this function?
        
        $(-\infty,$
            \wordChoice{
                \choice{$\sage{p2ans1f}$}
                \choice{$\sage{p2ans2}$}
                \choice{$\sage{p2ans2f}$}
                \choice{$\sage{p2ansvec[1]}$}
                \choice{$\sage{p2ans1}$}
                \choice{$\sage{p2ans3}$}
                \choice[correct]{$\sage{p2ansvec[0]}$}
                \choice{$\sage{p2ans3f}$}
                \choice{$\sage{p2ans4f}$}
                \choice{$\sage{p2ans5f}$}
                \choice{$\sage{p2ansvec[2]}$}
                \choice{$\sage{p2ans6f}$}
                }
            \wordChoice{
                \choice{$($}
                \choice{$[$}
                \choice{$]$}
                \choice[correct]{$)$}
                }
                $\cup$
            \wordChoice{
                \choice[correct]{$($}
                \choice{$[$}
                \choice{$]$}
                \choice{$)$}
                }
            \wordChoice{
                \choice{$\sage{p2ans1f}$}
                \choice{$\sage{p2ans2}$}
                \choice{$\sage{p2ans2f}$}
                \choice{$\sage{p2ansvec[1]}$}
                \choice{$\sage{p2ans1}$}
                \choice{$\sage{p2ans3}$}
                \choice[correct]{$\sage{p2ansvec[0]}$}
                \choice{$\sage{p2ans3f}$}
                \choice{$\sage{p2ans4f}$}
                \choice{$\sage{p2ans5f}$}
                \choice{$\sage{p2ansvec[2]}$}
                \choice{$\sage{p2ans6f}$}
                },
            \wordChoice{
                \choice{$\sage{p2ans1f}$}
                \choice{$\sage{p2ans2}$}
                \choice{$\sage{p2ans2f}$}
                \choice[correct]{$\sage{p2ansvec[1]}$}
                \choice{$\sage{p2ans1}$}
                \choice{$\sage{p2ans3}$}
                \choice{$\sage{p2ansvec[0]}$}
                \choice{$\sage{p2ans3f}$}
                \choice{$\sage{p2ans4f}$}
                \choice{$\sage{p2ans5f}$}
                \choice{$\sage{p2ansvec[2]}$}
                \choice{$\sage{p2ans6f}$}
                }
            \wordChoice{
                \choice{$($}
                \choice{$[$}
                \choice{$]$}
                \choice[correct]{$)$}
                }
                $\cup$
            \wordChoice{
                \choice[correct]{$($}
                \choice{$[$}
                \choice{$]$}
                \choice{$)$}
                }
            \wordChoice{
                \choice{$\sage{p2ans1f}$}
                \choice{$\sage{p2ans2}$}
                \choice{$\sage{p2ans2f}$}
                \choice[correct]{$\sage{p2ansvec[1]}$}
                \choice{$\sage{p2ans1}$}
                \choice{$\sage{p2ans3}$}
                \choice{$\sage{p2ansvec[0]}$}
                \choice{$\sage{p2ans3f}$}
                \choice{$\sage{p2ans4f}$}
                \choice{$\sage{p2ans5f}$}
                \choice{$\sage{p2ansvec[2]}$}
                \choice{$\sage{p2ans6f}$}
                },
            \wordChoice{
                \choice{$\sage{p2ans1f}$}
                \choice{$\sage{p2ans2}$}
                \choice{$\sage{p2ans2f}$}
                \choice{$\sage{p2ansvec[1]}$}
                \choice{$\sage{p2ans1}$}
                \choice{$\sage{p2ans3}$}
                \choice{$\sage{p2ansvec[0]}$}
                \choice{$\sage{p2ans3f}$}
                \choice{$\sage{p2ans4f}$}
                \choice{$\sage{p2ans5f}$}
                \choice[correct]{$\sage{p2ansvec[2]}$}
                \choice{$\sage{p2ans6f}$}
                }
            \wordChoice{
                \choice{$($}
                \choice{$[$}
                \choice{$]$}
                \choice[correct]{$)$}
                }
                $\cup$
            \wordChoice{
                \choice[correct]{$($}
                \choice{$[$}
                \choice{$]$}
                \choice{$)$}
                }
            \wordChoice{
                \choice{$\sage{p2ans1f}$}
                \choice{$\sage{p2ans2}$}
                \choice{$\sage{p2ans2f}$}
                \choice{$\sage{p2ansvec[1]}$}
                \choice{$\sage{p2ans1}$}
                \choice{$\sage{p2ans3}$}
                \choice{$\sage{p2ansvec[0]}$}
                \choice{$\sage{p2ans3f}$}
                \choice{$\sage{p2ans4f}$}
                \choice{$\sage{p2ans5f}$}
                \choice[correct]{$\sage{p2ansvec[2]}$}
                \choice{$\sage{p2ans6f}$}
                },
            $\infty)$
            \begin{feedback}
                You need to make sure to avoid the values that make the denominator zero. Thus make sure you use the excluding symbol (the parentheses) and the end points of the intervals should be the the values that you found as zeros in the denominator (in increasing order; order matters!)
            \end{feedback}
    \end{problem}
\end{problem}




\begin{problem}
    Consider the following rational function:
    \[
        f(x) = \frac{\sage{p3fTop}}{\sage{p3fBot}}
    \]
    
    What is the sum of the domain restriction x-values? $\answer{\sage{p3ansA}}$
        \begin{feedback}
            Start by finding all the zeros of the denominator. Notice that the denominator is already factored, so you can find the zeros by finding when each factor equals zero. Once you have those, add them up to find the sum!
        \end{feedback}
    
    \begin{problem}
        What is the domain of this function?
        
        $(-\infty,$
            \wordChoice{
                \choice{$\sage{p3ans1f}$}
                \choice{$\sage{p3ans3}$}
                \choice{$\sage{p3ans2}$}
                \choice[correct]{$\sage{p3ansvec[0]}$}
                \choice{$\sage{p3ans1}$}
                \choice{$\sage{p3ans6f}$}
                \choice{$\sage{p3ans3f}$}
                \choice{$\sage{p3ansvec[2]}$}
                \choice{$\sage{p3ans4f}$}
                \choice{$\sage{p3ansvec[1]}$}
                \choice{$\sage{p3ans2f}$}
                \choice{$\sage{p3ans5f}$}
                }
            \wordChoice{
                \choice{$($}
                \choice{$[$}
                \choice{$]$}
                \choice[correct]{$)$}
                }
                $\cup$
            \wordChoice{
                \choice[correct]{$($}
                \choice{$[$}
                \choice{$]$}
                \choice{$)$}
                }
            \wordChoice{
                \choice{$\sage{p3ans1f}$}
                \choice{$\sage{p3ans3}$}
                \choice{$\sage{p3ans2}$}
                \choice[correct]{$\sage{p3ansvec[0]}$}
                \choice{$\sage{p3ans1}$}
                \choice{$\sage{p3ans6f}$}
                \choice{$\sage{p3ans3f}$}
                \choice{$\sage{p3ansvec[2]}$}
                \choice{$\sage{p3ans4f}$}
                \choice{$\sage{p3ansvec[1]}$}
                \choice{$\sage{p3ans2f}$}
                \choice{$\sage{p3ans5f}$}
                },
            \wordChoice{
                \choice{$\sage{p3ans1f}$}
                \choice{$\sage{p3ans3}$}
                \choice{$\sage{p3ans2}$}
                \choice{$\sage{p3ansvec[0]}$}
                \choice{$\sage{p3ans1}$}
                \choice{$\sage{p3ans6f}$}
                \choice{$\sage{p3ans3f}$}
                \choice{$\sage{p3ansvec[2]}$}
                \choice{$\sage{p3ans4f}$}
                \choice[correct]{$\sage{p3ansvec[1]}$}
                \choice{$\sage{p3ans2f}$}
                \choice{$\sage{p3ans5f}$}
                }
            \wordChoice{
                \choice{$($}
                \choice{$[$}
                \choice{$]$}
                \choice[correct]{$)$}
                }
                $\cup$
            \wordChoice{
                \choice[correct]{$($}
                \choice{$[$}
                \choice{$]$}
                \choice{$)$}
                }
            \wordChoice{
                \choice{$\sage{p3ans1f}$}
                \choice{$\sage{p3ans3}$}
                \choice{$\sage{p3ans2}$}
                \choice{$\sage{p3ansvec[0]}$}
                \choice{$\sage{p3ans1}$}
                \choice{$\sage{p3ans6f}$}
                \choice{$\sage{p3ans3f}$}
                \choice{$\sage{p3ansvec[2]}$}
                \choice{$\sage{p3ans4f}$}
                \choice[correct]{$\sage{p3ansvec[1]}$}
                \choice{$\sage{p3ans2f}$}
                \choice{$\sage{p3ans5f}$}
                },
            \wordChoice{
                \choice{$\sage{p3ans1f}$}
                \choice{$\sage{p3ans3}$}
                \choice{$\sage{p3ans2}$}
                \choice{$\sage{p3ansvec[0]}$}
                \choice{$\sage{p3ans1}$}
                \choice{$\sage{p3ans6f}$}
                \choice{$\sage{p3ans3f}$}
                \choice[correct]{$\sage{p3ansvec[2]}$}
                \choice{$\sage{p3ans4f}$}
                \choice{$\sage{p3ansvec[1]}$}
                \choice{$\sage{p3ans2f}$}
                \choice{$\sage{p3ans5f}$}
                }
            \wordChoice{
                \choice{$($}
                \choice{$[$}
                \choice{$]$}
                \choice[correct]{$)$}
                }
                $\cup$
            \wordChoice{
                \choice[correct]{$($}
                \choice{$[$}
                \choice{$]$}
                \choice{$)$}
                }
            \wordChoice{
                \choice{$\sage{p3ans1f}$}
                \choice{$\sage{p3ans3}$}
                \choice{$\sage{p3ans2}$}
                \choice{$\sage{p3ansvec[0]}$}
                \choice{$\sage{p3ans1}$}
                \choice{$\sage{p3ans6f}$}
                \choice{$\sage{p3ans3f}$}
                \choice[correct]{$\sage{p3ansvec[2]}$}
                \choice{$\sage{p3ans4f}$}
                \choice{$\sage{p3ansvec[1]}$}
                \choice{$\sage{p3ans2f}$}
                \choice{$\sage{p3ans5f}$}
                },
            $\infty)$
            \begin{feedback}
                You need to make sure to avoid the values that make the denominator zero. Thus make sure you use the excluding symbol (the parentheses) and the end points of the intervals should be the the values that you found as zeros in the denominator (in increasing order; order matters!)
            \end{feedback}
    \end{problem}
\end{problem}





\end{document}