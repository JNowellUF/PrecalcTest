\documentclass{ximera}
\input{../preamble.tex}

\title{What is a rational function?}
\begin{document}
\begin{abstract}
    We discuss what makes a rational function, and why they are useful.
\end{abstract}
\maketitle

You can watch a lecture video on this here!

\youtube{EAJOEcWZhcQ}

The last method of combining functions we will discuss is known as the \textit{rational function}. These functions get their name from the same place as rational numbers; in particular they are a ratio of functions represented by a fraction.

\begin{definition}[Rational Functions]
    A rational function is a ratio of functions. Specifically it is a fraction with a non-constant function in the denominator.
\end{definition}

Technically a rational function can be a ratio of any function types, but most often we study the ratio of polynomials specifically. There are a number of reasons for this, but it is important to note that in calculus one studies ratios of all kinds of functions, not just polynomials. For this reason, although we will largely restrict our examples to ratios of polynomials, one should keep in mind the application of our work to ratios of other kinds of functions as well.

\begin{example}
    Which of the following are rational functions? Select all that apply.
    \begin{selectAll}
        \choice[correct]{$\frac{1}{x}$}
        \choice{$x^2 - 3x + 1$}
        \choice[correct]{$\frac{3x^2+x+1}{e^x + 7}$}
        \choice{$\frac{x + 1}{7}$}
        \choice{$\frac{e^x}{3}$}
        \choice[correct]{$\frac{2x^2-3x+1}{x^3-1}$}
    \end{selectAll}
\end{example}



\end{document}