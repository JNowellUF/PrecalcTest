\documentclass{ximera}

\title{Functions Require Context}
\begin{document}
\begin{abstract}
    In this section we demonstrate that a relation requires context to be considered a function.
\end{abstract}
\maketitle

    In the previous section we established that a relationship is a function if each input has exactly one output. This condition can be even trickier than it may initially seem though. The same equation can be a function in one setting, and not a function in another. This is because every relationship \textbf{requires} context before we can decide if it is a function or not. Consider the following example.
    
    \begin{explanation}[A Drink Vending Machine]%
        You are thirsty and decide to get a drink from the vending machine nearby. After looking over your choices you see that the vending machine is setup like the following:
    
        \begin{tabular}{c|ccccc}
                & \#1           & \#2           & \#3           & \#4           & \#5           \\ \hline
            A   & Pepsi         & Pepsi         & Pepsi         & Pepsi         & Pepsi         \\
            B   & Fanta         & Fanta         & Sierra Mist   & Sierra Mist   & Sierra Mist   \\
            C   & Gatorade Blue & Gatorade Blue & Gatorade Green& Gatorade Green& Gatorade Green\\
            D   & Coke          & Coke          & Coke          & Coke          & Coke          \\
            E   & Sprite        & Sprite        & Sprite        & Sunkist       & Sunkist       \\
            F   & Crush         & Root beer     & Cream Soda    & Water         & Water         \\
        \end{tabular}
    
        If you punch in a letter and number combination you know exactly what you will get; for example if you enter C5 you know you will get a green Gatorade. However, if you approach the vending machine wanting a Pepsi, then there are several options you could enter to get one; A1, A2, A3, A4, or A5.
        
        In this example, the relationship that inputs the location of the drink you request (such as C5) and outputs the drink you get as a result (green Gatorade) would be a function. In contrast, the relation whose input is what drink you want (such as Pepsi) and outputs the location you must enter to get that drink (A1 through A5) would \textit{not} be a function, because there are multiple outputs for the same input.
        
        Note that it is perfectly natural to ask what drink is in a given location, as well as asking what location you should type in for a particular drink you want. Both of these situations are perfectly natural, and yet one is a function and one is not.
    
    \end{explanation}%
    
    \begin{problem}
        What is meant by context, with regards to mathematical relations?
        \begin{multipleChoice}
            \choice[correct]{The actual objects/ideas/etc that a symbolic input and output represent.}
            \choice{The values (eg numbers) that you can put in, or get out, of a relation.}
            \choice{How well the mathematical relation represents the real world problem.}
            \choice{The formulas/equations/etc that are used to symbolically represent the real world situation.}
        \end{multipleChoice}
    \end{problem}
    
    The context required to define a function has a special set of terminology in mathematics; the domain, codomain, and range, which we discuss in the next section.

\end{document}