\documentclass{ximera}
%\usepackage{sagetex}
%\usepackage{amsfonts}
\title{Functions Practice 1}
\renewcommand{\text}[1]{#1}
\newcommand{\sagecheck}[2]{
    \ifthenelse{}{}{}\textit{}
    }
\begin{document}
\input{Useful-Sage-Macros}

\begin{sagesilent}
def RandInt(a,b):
    """ Returns a random integer in [`a`,`b`]. Note that `a` and `b` should be integers themselves to avoid unexpected behavior.
    """
    return QQ(randint(int(a),int(b)))
    # return choice(range(a,b+1))

def NonZeroInt(b,c, avoid = [0]):
    """ Returns a random integer in [`b`,`c`] which is not in `av`. 
        If `av` is not specified, defaults to a non-zero integer.
    """
    while True:
        a = RandInt(b,c)
        if a not in avoid:
            return a


p1c1 = RandInt(-10,10)
p1c2 = NonZeroInt(-10,10)
p1c3 = RandInt(-10,10)
p1f1 = p1c1*x^2 + p1c2*x + p1c3

p1c4 = RandInt(-5,5)

p1ans1 = p1f1(x=p1c4)



p2c1 = RandInt(-10,10)
p2c2 = NonZeroInt(-10,10)
p2c3 = RandInt(-10,10)
p2f1 = p2c1*x^2 + p2c2*x + p2c3

p2c4 = RandInt(-5,5)

p2ans1 = p2f1(x=p2c4)



p3c1 = RandInt(-10,10)
p3c2 = NonZeroInt(-10,10)
p3c3 = RandInt(-10,10)
p3f1 = p3c1*x^2 + p3c2*x + p3c3

p3c4 = RandInt(-5,5)

p3ans1 = p3f1(x=p3c4)



\end{sagesilent}

\begin{problem}
    Consider the function $f:\mathbb{R}\rightarrow\mathbb{R}$ defined by $f(x) = \sage{p1f1}$. The name of the function is $\answer{f}$.
    \begin{feedback}
        Remember that the function is named before the colon when the function's domain and codomain are given.
    \end{feedback}
    \begin{problem}
        What is the rule that defines $f(x)$? $x$ is mapped to $\answer{\sage{p1f1}}$.
        \begin{feedback}
            Remember that the rule to define $f(x)$ is the function equation that is given after the ``$=$'' sign and containing the $x$.
        \end{feedback}
        \begin{problem}
            $f(\sage{p1c4}) = \answer{\sage{p1ans1}}$.
            \begin{feedback}
                Remember that the notation $f(x)$ is denoting the $x$ as the placeholder for the input; so $f(\sage{p1c4})$ means you should substitute $\sage{p1c4}$ wherever there is an $x$ in the rule for $f(x)$.
            \end{feedback}
        \end{problem}
    \end{problem}
    
\end{problem}


\begin{problem}
    Consider the function $f:\mathbb{R}\rightarrow\mathbb{R}$ defined by $f(x) = \sage{p2f1}$. The name of the function is $\answer{f}$.
    \begin{feedback}
        Remember that the function is named before the colon when the function's domain and codomain are given.
    \end{feedback}
    \begin{problem}
        What is the rule that defines $f(x)$? $x$ is mapped to $\answer{\sage{p2f1}}$.
        \begin{feedback}
            Remember that the rule to define $f(x)$ is the function equation that is given after the ``$=$'' sign and containing the $x$.
        \end{feedback}
        \begin{problem}
            $f(\sage{p2c4}) = \answer{\sage{p2ans1}}$.
            \begin{feedback}
                Remember that the notation $f(x)$ is denoting the $x$ as the placeholder for the input; so $f(\sage{p2c4})$ means you should substitute $\sage{p2c4}$ wherever there is an $x$ in the rule for $f(x)$.
            \end{feedback}
        \end{problem}
    \end{problem}
    
\end{problem}


\begin{problem}
    Consider the function $f:\mathbb{R}\rightarrow\mathbb{R}$ defined by $f(x) = \sage{p3f1}$. The name of the function is $\answer{f}$.
    \begin{feedback}
        Remember that the function is named before the colon when the function's domain and codomain are given.
    \end{feedback}
    \begin{problem}
        What is the rule that defines $f(x)$? $x$ is mapped to $\answer{\sage{p3f1}}$.
        \begin{feedback}
            Remember that the rule to define $f(x)$ is the function equation that is given after the ``$=$'' sign and containing the $x$.
        \end{feedback}
        \begin{problem}
            $f(\sage{p3c4}) = \answer{\sage{p3ans1}}$.
            \begin{feedback}
                Remember that the notation $f(x)$ is denoting the $x$ as the placeholder for the input; so $f(\sage{p3c4})$ means you should substitute $\sage{p3c4}$ wherever there is an $x$ in the rule for $f(x)$.
            \end{feedback}
        \end{problem}
    \end{problem}
    
\end{problem}






\end{document}