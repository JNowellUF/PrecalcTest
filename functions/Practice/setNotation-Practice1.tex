\documentclass{ximera}
%\usepackage{sagetex}
\title{Functions Practice 1}
\renewcommand{\text}[1]{#1}
\begin{document}
\begin{sagesilent}

######  Define a function to convert a sage number into a saved counter number.

#####Define default Sage variables.
#Default function variables
var('x,y,z,X,Y,Z')
#Default function names
var('f,g,h,dx,dy,dz,dh,df')
#Default Wild cards
w0 = SR.wild(0)

def DispSign(b):
    """ Returns the string of the 'signed' version of `b`, e.g. 3 -> "+3", -3 -> "-3", 0 -> "".
    """
    if b == 0:
        return ""
    elif b > 0:
        return "+" + str(b)
    elif b < 0:
        return str(b)
    else:
        # If we're here, then something has gone wrong.
        raise ValueError

def ISP(b):
    return DispSign(b)

def NoEval(f, c):
    # TODO
    """ Returns a non-evaluted version of the result f(c).
    """
    cStr = str(c)
    # fLatex = latex(f)
    fString = latex(f)
    fStrList = list(fString)
    length = len(fStrList)
    fStrList2 = range(length)
    for i in range(0, length):
        if fStrList[i] == "x":
            fStrList2[i] = "("+cstr+")"
        else:
            fStrList2[i] = fStrList[i]
    f2 = join(fStrList2,"")
    return LatexExpr(f2)

def HyperSimp(f):
    """ Returns the expression `f` without hyperbolic expressions.
    """
    subsDict = {
        sinh(w0) : (exp(w0) - exp(-w0))/2,
        cosh(w0) : (exp(w0) + exp(-w0))/2,
        tanh(w0) : (exp(w0) - exp(-w0))/(exp(w0) + exp(-w0)),
        sech(w0) : 2/(exp(w0) + exp(-w0)),                      # This seems to work, but Nowell said it didn't at one point.
        csch(w0) : 2/(exp(w0) - exp(-w0)),                      # This seems to work, but Nowell said it didn't at one point.
        coth(w0) : (exp(w0) + exp(-w0))/(exp(w0) - exp(-w0)),   # This seems to work, but Nowell said it didn't at one point.
        arcsinh(w0) :       ln( w0 + sqrt((w0)^2 + 1) ),
        arccosh(w0) :       ln( w0 + sqrt((w0)^2 - 1) ),
        arctanh(w0) : 1/2 * ln( (1 + w0) / (1 - w0) ),
        arccsch(w0) :       ln( (1 + sqrt((w0)^2 + 1))/w0 ),
        arcsech(w0) :       ln( (1 + sqrt(1 - (w0)^2))/w0 ),
        arccoth(w0) : 1/2 * ln( (1 + w0) / (w0 - 1) )
    }
    g = f.substitute(subsDict)
    return simplify(g)

def RandInt(a,b):
    """ Returns a random integer in [`a`,`b`]. Note that `a` and `b` should be integers themselves to avoid unexpected behavior.
    """
    return QQ(randint(int(a),int(b)))
    # return choice(range(a,b+1))

def NonZeroInt(b,c, avoid = [0]):
    """ Returns a random integer in [`b`,`c`] which is not in `av`. 
        If `av` is not specified, defaults to a non-zero integer.
    """
    while True:
        a = RandInt(b,c)
        if a not in avoid:
            return a

def RandVector(b, c, avoid=[], rep=1):
    """ Returns essentially a multiset permutation of ([b,c]-av) * rep.
        That is, a vector which contains each integer in [`b`,`c`] which is not in `av` a total of `rep` number of times.
        Example:
        sage: RandVector(1,3, [2], 2)
        [3, 1, 1, 3]
    """
    oneVec = [val for val in range(b,c+1) if val not in avoid]
    vec = oneVec * rep
    shuffle(vec)
    return vec

def fudge(b):
    up = b+RandInt(2,5)/10
    down = b-RandInt(2,5)/10
    fudgebup = round(up,1)
    fudgebdown = round(down,1)
    fudgedb = [fudgebdown,fudgebup]
    return fudgedb

def disjointCheck(checkvec):
    if length(uniq(checkvec)) < length(checkvec):
        return 1
    else:
        return 0

def disjointIntervals(IntStart,IntEnd,CheckVal):
    if IntStart < CheckVal and CheckVal < IntEnd:
        return 1
    else:
        return 0

def IntervalVecCheck(checkVec):
    veclen = len(checkVec)
    returnval = 0
    for i in range(veclen):
        for j in range(veclen):
            if (disjointIntervals(checkVec[j][0],checkVec[j][1],checkVec[i][0]) + disjointIntervals(checkVec[j][0],checkVec[j][1],checkVec[i][1])) > 0:
                returnval = returnval + 1
    if returnval > 0:
        return 1
    else:
        return 0



\end{sagesilent}

\begin{sagesilent}
p1c1 = NonZeroInt(-10,10)
p1c2 = -p1c1

p2c1 = NonZeroInt(-10,10)
p2c2 = -p2c1
p2rvec = [ 'x' + LatexExpr(r"\leq") + str(p1c1), str(x < p1c1), 'x' + LatexExpr(r"\geq") + str(p1c1), str(x > p1c1) ]
p2vecfalse = [RealSet.open_closed(-infinity,p2c1), RealSet(-infinity, p2c1), RealSet.closed_open(p2c1,infinity), RealSet(p2c1,infinity),RealSet.open_closed(-infinity,-p2c1), RealSet(-infinity, -p2c1), RealSet.closed_open(-p2c1,infinity), RealSet(-p2c1,infinity)]
p2vecpick = RandInt(0,3)
p2relate = p2rvec[p2vecpick]
p2false = [p2vecfalse[i] for i in range(8) if abs(i - p2vecpick)>0]

p2ans1 = p2false[0]
p2ans2 = p2false[1]
p2ans3 = p2vecfalse[p2vecpick]
p2ans4 = p2false[2]
p2ans5 = p2false[3]
p2ans6 = p2false[4]
p2ans7 = p2false[5]
p2ans8 = p2false[6]

p3c1 = NonZeroInt(-10,10)
p3c2 = -p3c1

p4c1 = NonZeroInt(-10,10)
p4c2 = -p4c1
p4rvec = [ 'x' + LatexExpr(r"\leq") + str(p4c1), str(x < p4c1), 'x' + LatexExpr(r"\geq") + str(p4c1), str(x > p4c1) ]
p4vecfalse = [RealSet.open_closed(-infinity,p4c1), RealSet(-infinity, p4c1), RealSet.closed_open(p4c1,infinity), RealSet(p4c1,infinity),RealSet.open_closed(-infinity,-p4c1), RealSet(-infinity, -p4c1), RealSet.closed_open(-p4c1,infinity), RealSet(-p4c1,infinity)]
p4vecpick = RandInt(0,3)
p4relate = p4rvec[p4vecpick]
p4false = [p4vecfalse[i] for i in range(8) if abs(i - p4vecpick)>0]

p4ans1 = p4false[0]
p4ans2 = p4false[1]
p4ans3 = p4vecfalse[p4vecpick]
p4ans4 = p4false[2]
p4ans5 = p4false[3]
p4ans6 = p4false[4]
p4ans7 = p4false[5]
p4ans8 = p4false[6]

p5c1 = NonZeroInt(-10,10)
p5c2 = -p5c1

p6c1 = NonZeroInt(-10,10)
p6c2 = -p6c1
p6rvec = [ 'x' + LatexExpr(r"\leq") + str(p6c1), str(x < p6c1), 'x' + LatexExpr(r"\geq") + str(p6c1), str(x > p6c1) ]
p6vecfalse = [RealSet.open_closed(-infinity,p6c1), RealSet(-infinity, p6c1), RealSet.closed_open(p6c1,infinity), RealSet(p6c1,infinity),RealSet.open_closed(-infinity,-p6c1), RealSet(-infinity, -p6c1), RealSet.closed_open(-p6c1,infinity), RealSet(-p6c1,infinity)]
p6vecpick =RandInt(0,3)
p6relate = p6rvec[p6vecpick]
p6false = [p6vecfalse[i] for i in range(8) if abs(i - p6vecpick)>0]

p6ans1 = p6false[0]
p6ans2 = p6false[1]
p6ans3 = p6vecfalse[p6vecpick]
p6ans4 = p6false[2]
p6ans5 = p6false[3]
p6ans6 = p6false[4]
p6ans7 = p6false[5]
p6ans8 = p6false[6]



\end{sagesilent}

\begin{problem}
    Which of the following is equivalent to: $(\sage{p1c1}, \infty)$?
    \begin{multipleChoice}
        \choice{ $\{ x \in \mathbb{R} : x < \sage{p1c1} \}$ }
        \choice[correct]{ $\{ x \in \mathbb{R} : x > \sage{p1c1} \}$ }
        \choice{ $\{ x \in \mathbb{R} : x \leq \sage{p1c1} \}$ }
        \choice{ $\{ x \in \mathbb{R} : x \geq \sage{p1c1} \}$ }
        \choice{ $\{ x \in \mathbb{R} : x < \sage{p1c2} \}$ }
        \choice{ $\{ x \in \mathbb{R} : x > \sage{p1c2} \}$ }
        \choice{ $\{ x \in \mathbb{R} : x \leq \sage{p1c2} \}$ }
        \choice{ $\{ x \in \mathbb{R} : x \geq \sage{p1c2} \}$ }
    \end{multipleChoice}
    
    \begin{feedback}
        Set notation should have the open brace and then the dummy variable (usually with what kind of number it is) hence the ``$\{x \in \mathbb{R}$'' above is saying ``the set of all $x$, a real number''. The colon should be translated as a ``such that'' and what follows is the condition that the variable needs to adhere to, followed by the closing brace. Thus for example: the set ``$\{x \in \mathbb{R}: x > 3\}$'' is saying ``the set of all $x$, a real number, such that $x$ is strictly larger than $3$.'' \\
        
        Since we want ``$(\sage{p1c1}, \infty)$'' we want $x$ to be strictly (since we have a parenthesis) larger than $\sage{p1c1}$ and less than ``$\infty$''. But clearly any number is less than infinity, so we can simplify this to just ``$x$ strictly larger $\sage{p1c1}$.'' We then just need to put it in the correct format with the braces and using the correct inequality sign!. 
    \end{feedback}
\end{problem}


\begin{problem}
    Which of the following is equivalent to: $\{ x \in \mathbb{R} : \sage{p2relate} \}$?
    \begin{multipleChoice}
        \choice{ $ \sage{p2ans1} $ }
        \choice{ $ \sage{p2ans2} $ }
        \choice[correct]{ $ \sage{p2ans3} $ }
        \choice{ $ \sage{p2ans4} $ }
        \choice{ $ \sage{p2ans5} $ }
        \choice{ $ \sage{p2ans6} $ }
        \choice{ $ \sage{p2ans7} $ }
        \choice{ $ \sage{p2ans8} $ }
    \end{multipleChoice}
    
    \begin{feedback}
        Remember that strict inequalities (i.e. ``$>$'' or ``$<$'') need parenthesis and non-strict (i.e. ``$\leq$'' or ``$\geq$'') inequalities use brackets. You also need to account for both endpoints. So if you are trying to interpret $x > 5$ then you need ``$x$ strictly larger than $5$'', you would want the interval $(5,\infty)$; the initial ``$($'' is because it is a strict inequality, and the ``$\infty$'' is because you need the other ``endpoint'' (which, since we want ``anything bigger than $5$'', must be infinity since there is no upper bound given; note that infinity always gets a parenthesis since we don't include it as a ``number'').
    \end{feedback}
\end{problem}


\begin{problem}
    Which of the following is equivalent to: $(\sage{p3c1}, \infty)$?
    \begin{multipleChoice}
        \choice{ $\{ x \in \mathbb{R} : x < \sage{p3c1} \}$ }
        \choice[correct]{ $\{ x \in \mathbb{R} : x > \sage{p3c1} \}$ }
        \choice{ $\{ x \in \mathbb{R} : x \leq \sage{p3c1} \}$ }
        \choice{ $\{ x \in \mathbb{R} : x \geq \sage{p3c1} \}$ }
        \choice{ $\{ x \in \mathbb{R} : x < \sage{p3c2} \}$ }
        \choice{ $\{ x \in \mathbb{R} : x > \sage{p3c2} \}$ }
        \choice{ $\{ x \in \mathbb{R} : x \leq \sage{p3c2} \}$ }
        \choice{ $\{ x \in \mathbb{R} : x \geq \sage{p3c2} \}$ }
    \end{multipleChoice}
    
    \begin{feedback}
        Since we want ``$(\sage{p3c1}, \infty)$'' we want $x$ to be strictly (since we have a parenthesis) larger than $\sage{p3c1}$ and less than ``$\infty$''. But clearly any number is less than infinity, so we can simplify this to just ``$x$ strictly larger $\sage{p3c1}$.'' We then just need to put it in the correct format with the braces and using the correct inequality sign!. 
    \end{feedback}
\end{problem}



\begin{problem}
    Which of the following is equivalent to: $\{ x \in \mathbb{R} : \sage{p4relate} \}$?
    \begin{multipleChoice}
        \choice{ $ \sage{p4ans1} $ }
        \choice{ $ \sage{p4ans2} $ }
        \choice{ $ \sage{p4ans4} $ }
        \choice{ $ \sage{p4ans5} $ }
        \choice{ $ \sage{p4ans6} $ }
        \choice[correct]{ $ \sage{p4ans3} $ }
        \choice{ $ \sage{p4ans7} $ }
        \choice{ $ \sage{p4ans8} $ }
    \end{multipleChoice}
    
    \begin{feedback}
        Remember that strict inequalities (i.e. ``$>$'' or ``$<$'') need parenthesis and non-strict (i.e. ``$\leq$'' or ``$\geq$'') inequalities use brackets. You also need to account for both endpoints. So if you are trying to interpret $x > 5$ then you need ``$x$ strictly larger than $5$'', you would want the interval $(5,\infty)$; the initial ``$($'' is because it is a strict inequality, and the ``$\infty$'' is because you need the other ``endpoint'' (which, since we want ``anything bigger than $5$'', must be infinity since there is no upper bound given; note that infinity always gets a parenthesis since we don't include it as a ``number'').
    \end{feedback}
\end{problem}


\begin{problem}
    Which of the following is equivalent to: $(\sage{p5c1}, \infty)$?
    \begin{multipleChoice}
        \choice{ $\{ x \in \mathbb{R} : x < \sage{p5c1} \}$ }
        \choice[correct]{ $\{ x \in \mathbb{R} : x > \sage{p5c1} \}$ }
        \choice{ $\{ x \in \mathbb{R} : x \leq \sage{p5c1} \}$ }
        \choice{ $\{ x \in \mathbb{R} : x \geq \sage{p5c1} \}$ }
        \choice{ $\{ x \in \mathbb{R} : x < \sage{p5c2} \}$ }
        \choice{ $\{ x \in \mathbb{R} : x > \sage{p5c2} \}$ }
        \choice{ $\{ x \in \mathbb{R} : x \leq \sage{p5c2} \}$ }
        \choice{ $\{ x \in \mathbb{R} : x \geq \sage{p5c2} \}$ }
    \end{multipleChoice}
    
    
    \begin{feedback}
        Since we want ``$(\sage{p5c1}, \infty)$'' we want $x$ to be strictly (since we have a parenthesis) larger than $\sage{p5c1}$ and less than ``$\infty$''. But clearly any number is less than infinity, so we can simplify this to just ``$x$ strictly larger $\sage{p5c1}$.'' We then just need to put it in the correct format with the braces and using the correct inequality sign!. 
    \end{feedback}
\end{problem}



\begin{problem}
    Which of the following is equivalent to: $\{ x \in \mathbb{R} : \sage{p6relate} \}$?
    \begin{multipleChoice}
        \choice{ $ \sage{p6ans1} $ }
        \choice{ $ \sage{p6ans2} $ }
        \choice{ $ \sage{p6ans4} $ }
        \choice{ $ \sage{p6ans5} $ }
        \choice{ $ \sage{p6ans6} $ }
        \choice{ $ \sage{p6ans7} $ }
        \choice{ $ \sage{p6ans8} $ }
        \choice[correct]{ $ \sage{p6ans3} $ }
    \end{multipleChoice}
    
    \begin{feedback}
        Remember that strict inequalities (i.e. ``$>$'' or ``$<$'') need parenthesis and non-strict (i.e. ``$\leq$'' or ``$\geq$'') inequalities use brackets. You also need to account for both endpoints. So if you are trying to interpret $x > 5$ then you need ``$x$ strictly larger than $5$'', you would want the interval $(5,\infty)$; the initial ``$($'' is because it is a strict inequality, and the ``$\infty$'' is because you need the other ``endpoint'' (which, since we want ``anything bigger than $5$'', must be infinity since there is no upper bound given; note that infinity always gets a parenthesis since we don't include it as a ``number'').
    \end{feedback}
\end{problem}




\end{document}