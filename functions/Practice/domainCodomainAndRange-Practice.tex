\documentclass{ximera}

\title{Functions Practice 1}
\begin{document}

%
\begin{sagesilent}
def RandInt(a,b):
    """ Returns a random integer in [`a`,`b`]. Note that `a` and `b` should be integers themselves to avoid unexpected behavior.
    """
    return QQ(randint(int(a),int(b)))
    # return choice(range(a,b+1))

def NonZeroInt(b,c, avoid = [0]):
    """ Returns a random integer in [`b`,`c`] which is not in `av`. 
        If `av` is not specified, defaults to a non-zero integer.
    """
    while True:
        a = RandInt(b,c)
        if a not in avoid:
            return a

def RandVector(b, c, avoid=[], rep=1):
    """ Returns essentially a multiset permutation of ([b,c]-av) * rep.
        That is, a vector which contains each integer in [`b`,`c`] which is not in `av` a total of `rep` number of times.
        Example:
        sage: RandVector(1,3, [2], 2)
        [3, 1, 1, 3]
    """
    oneVec = [val for val in range(b,c+1) if val not in avoid]
    vec = oneVec * rep
    shuffle(vec)
    return vec

p1vec1 = RandVector(-100,100)
p1x1 = p1vec1[0]
p1x2 = p1vec1[1]
p1x3 = p1vec1[2]
p1x4 = p1vec1[3]
p1x5 = p1vec1[4]
p1y1 = p1vec1[5]
p1y2 = p1vec1[6]
p1y3 = p1vec1[7]
p1y4 = p1vec1[8]
p1y5 = p1vec1[9]

p2vec1 = RandVector(-100,100)
p2x1 = p2vec1[0]
p2x2 = p2vec1[1]
p2x3 = p2vec1[2]
p2x4 = p2vec1[3]
p2x5 = p2vec1[4]
p2y1 = p2vec1[5]
p2y2 = p2vec1[6]
p2y3 = p2vec1[7]
p2y4 = p2vec1[8]
p2y5 = p2vec1[9]

p3vec1 = RandVector(-100,100)
p3x1 = p3vec1[0]
p3x2 = p3vec1[1]
p3x3 = p3vec1[2]
p3x4 = p3vec1[3]
p3x5 = p3vec1[4]
p3y1 = p3vec1[5]
p3y2 = p3vec1[6]
p3y3 = p3vec1[7]
p3y4 = p3vec1[8]
p3y5 = p3vec1[9]

\end{sagesilent}

\begin{problem}
    Consider the relation defined by the following coordinate pairs (recall that a coordinate pair is a pair of values, the first of which is the input and the second of which is the output, of the relation).
    \[
        (\sage{p1x1},\sage{p1y1}),(\sage{p1x2},\sage{p1y2}),(\sage{p1x3},\sage{p1y3}),(\sage{p1x4},\sage{p1y4}),(\sage{p1x5},\sage{p1y5})
    \]
    Which of the following are in the domain of the relation? (Select all that apply)
    \begin{selectAll}
        \choice[correct]{$\sage{p1x1}$}
        \choice{$\sage{p1y1}$}
        \choice{$\sage{p1y2}$}
        \choice[correct]{$\sage{p1x2}$}
        \choice[correct]{$\sage{p1x3}$}
        \choice{$\sage{p1y5}$}
        \choice{$\sage{p1y4}$}
        \choice[correct]{$\sage{p1x5}$}
        \choice[correct]{$\sage{p1x4}$}
        \choice{$\sage{p1y3}$}
    \end{selectAll}
    \begin{feedback}
        Remember that the first number in each pair is the input, and the domain is comprised of all valid inputs.
    \end{feedback}
    \begin{problem}
        Which of the following are in the range of the relation? (Select all that apply)
        \begin{selectAll}
            \choice{$\sage{p1x1}$}
            \choice[correct]{$\sage{p1y1}$}
            \choice[correct]{$\sage{p1y2}$}
            \choice{$\sage{p1x2}$}
            \choice{$\sage{p1x3}$}
            \choice[correct]{$\sage{p1y5}$}
            \choice[correct]{$\sage{p1y4}$}
            \choice{$\sage{p1x5}$}
            \choice{$\sage{p1x4}$}
            \choice[correct]{$\sage{p1y3}$}
        \end{selectAll}
        \begin{feedback}
            Remember that the second number in each pair is the output, and the range is comprised of all outputs that you can actually get.
        \end{feedback}
    \end{problem}
\end{problem}

\begin{problem}
    Consider the relation defined by the following coordinate pairs (recall that a coordinate pair is a pair of values, the first of which is the input and the second of which is the output, of the relation).
    \[
        (\sage{p2x1},\sage{p2y1}),(\sage{p2x2},\sage{p2y2}),(\sage{p2x3},\sage{p2y3}),(\sage{p2x4},\sage{p2y4}),(\sage{p2x5},\sage{p2y5})
    \]
    Which of the following are in the domain of the relation? (Select all that apply)
    \begin{selectAll}
        \choice[correct]{$\sage{p2x1}$}
        \choice{$\sage{p2y1}$}
        \choice{$\sage{p2y2}$}
        \choice[correct]{$\sage{p2x2}$}
        \choice[correct]{$\sage{p2x3}$}
        \choice{$\sage{p2y5}$}
        \choice{$\sage{p2y4}$}
        \choice[correct]{$\sage{p2x5}$}
        \choice[correct]{$\sage{p2x4}$}
        \choice{$\sage{p2y3}$}
    \end{selectAll}
    \begin{feedback}
        Remember that the first number in each pair is the input, and the domain is comprised of all valid inputs.
    \end{feedback}
    \begin{problem}
        Which of the following are in the range of the relation? (Select all that apply)
        \begin{selectAll}
            \choice{$\sage{p2x1}$}
            \choice[correct]{$\sage{p2y1}$}
            \choice[correct]{$\sage{p2y2}$}
            \choice{$\sage{p2x2}$}
            \choice{$\sage{p2x3}$}
            \choice[correct]{$\sage{p2y5}$}
            \choice[correct]{$\sage{p2y4}$}
            \choice{$\sage{p2x5}$}
            \choice{$\sage{p2x4}$}
            \choice[correct]{$\sage{p2y3}$}
        \end{selectAll}
        \begin{feedback}
            Remember that the second number in each pair is the output, and the range is comprised of all outputs that you can actually get.
        \end{feedback}
    \end{problem}
\end{problem}


\begin{problem}
    Consider the relation defined by the following coordinate pairs (recall that a coordinate pair is a pair of values, the first of which is the input and the second of which is the output, of the relation).
    \[
        (\sage{p3x1},\sage{p3y1}),(\sage{p3x2},\sage{p3y2}),(\sage{p3x3},\sage{p3y3}),(\sage{p3x4},\sage{p3y4}),(\sage{p3x5},\sage{p3y5})
    \]
    Which of the following are in the domain of the relation? (Select all that apply)
    \begin{selectAll}
        \choice[correct]{$\sage{p3x1}$}
        \choice{$\sage{p3y1}$}
        \choice{$\sage{p3y2}$}
        \choice[correct]{$\sage{p3x2}$}
        \choice[correct]{$\sage{p3x3}$}
        \choice{$\sage{p3y5}$}
        \choice{$\sage{p3y4}$}
        \choice[correct]{$\sage{p3x5}$}
        \choice[correct]{$\sage{p3x4}$}
        \choice{$\sage{p3y3}$}
    \end{selectAll}
    \begin{feedback}
        Remember that the first number in each pair is the input, and the domain is comprised of all valid inputs.
    \end{feedback}
    
    \begin{problem}
        Which of the following are in the range of the relation? (Select all that apply)
        \begin{selectAll}
            \choice{$\sage{p3x1}$}
            \choice[correct]{$\sage{p3y1}$}
            \choice[correct]{$\sage{p3y2}$}
            \choice{$\sage{p3x2}$}
            \choice{$\sage{p3x3}$}
            \choice[correct]{$\sage{p3y5}$}
            \choice[correct]{$\sage{p3y4}$}
            \choice{$\sage{p3x5}$}
            \choice{$\sage{p3x4}$}
            \choice[correct]{$\sage{p3y3}$}
        \end{selectAll}
        \begin{feedback}
            Remember that the second number in each pair is the output, and the range is comprised of all outputs that you can actually get.
        \end{feedback}
    \end{problem}
\end{problem}




\end{document}