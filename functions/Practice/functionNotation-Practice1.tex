\documentclass{ximera}
%\usepackage{sagetex}
%\usepackage{amsfonts}
\title{Functions Practice 1}
\renewcommand{\text}[1]{#1}
\newcommand{\sagecheck}[2]{
    \ifthenelse{}{}{}\textit{}
    }
\begin{document}
\begin{sagesilent}

######  Define a function to convert a sage number into a saved counter number.

#####Define default Sage variables.
#Default function variables
var('x,y,z,X,Y,Z')
#Default function names
var('f,g,h,dx,dy,dz,dh,df')
#Default Wild cards
w0 = SR.wild(0)

def DispSign(b):
    """ Returns the string of the 'signed' version of `b`, e.g. 3 -> "+3", -3 -> "-3", 0 -> "".
    """
    if b == 0:
        return ""
    elif b > 0:
        return "+" + str(b)
    elif b < 0:
        return str(b)
    else:
        # If we're here, then something has gone wrong.
        raise ValueError

def ISP(b):
    return DispSign(b)

def NoEval(f, c):
    # TODO
    """ Returns a non-evaluted version of the result f(c).
    """
    cStr = str(c)
    # fLatex = latex(f)
    fString = latex(f)
    fStrList = list(fString)
    length = len(fStrList)
    fStrList2 = range(length)
    for i in range(0, length):
        if fStrList[i] == "x":
            fStrList2[i] = "("+cstr+")"
        else:
            fStrList2[i] = fStrList[i]
    f2 = join(fStrList2,"")
    return LatexExpr(f2)

def HyperSimp(f):
    """ Returns the expression `f` without hyperbolic expressions.
    """
    subsDict = {
        sinh(w0) : (exp(w0) - exp(-w0))/2,
        cosh(w0) : (exp(w0) + exp(-w0))/2,
        tanh(w0) : (exp(w0) - exp(-w0))/(exp(w0) + exp(-w0)),
        sech(w0) : 2/(exp(w0) + exp(-w0)),                      # This seems to work, but Nowell said it didn't at one point.
        csch(w0) : 2/(exp(w0) - exp(-w0)),                      # This seems to work, but Nowell said it didn't at one point.
        coth(w0) : (exp(w0) + exp(-w0))/(exp(w0) - exp(-w0)),   # This seems to work, but Nowell said it didn't at one point.
        arcsinh(w0) :       ln( w0 + sqrt((w0)^2 + 1) ),
        arccosh(w0) :       ln( w0 + sqrt((w0)^2 - 1) ),
        arctanh(w0) : 1/2 * ln( (1 + w0) / (1 - w0) ),
        arccsch(w0) :       ln( (1 + sqrt((w0)^2 + 1))/w0 ),
        arcsech(w0) :       ln( (1 + sqrt(1 - (w0)^2))/w0 ),
        arccoth(w0) : 1/2 * ln( (1 + w0) / (w0 - 1) )
    }
    g = f.substitute(subsDict)
    return simplify(g)

def RandInt(a,b):
    """ Returns a random integer in [`a`,`b`]. Note that `a` and `b` should be integers themselves to avoid unexpected behavior.
    """
    return QQ(randint(int(a),int(b)))
    # return choice(range(a,b+1))

def NonZeroInt(b,c, avoid = [0]):
    """ Returns a random integer in [`b`,`c`] which is not in `av`. 
        If `av` is not specified, defaults to a non-zero integer.
    """
    while True:
        a = RandInt(b,c)
        if a not in avoid:
            return a

def RandVector(b, c, avoid=[], rep=1):
    """ Returns essentially a multiset permutation of ([b,c]-av) * rep.
        That is, a vector which contains each integer in [`b`,`c`] which is not in `av` a total of `rep` number of times.
        Example:
        sage: RandVector(1,3, [2], 2)
        [3, 1, 1, 3]
    """
    oneVec = [val for val in range(b,c+1) if val not in avoid]
    vec = oneVec * rep
    shuffle(vec)
    return vec

def fudge(b):
    up = b+RandInt(2,5)/10
    down = b-RandInt(2,5)/10
    fudgebup = round(up,1)
    fudgebdown = round(down,1)
    fudgedb = [fudgebdown,fudgebup]
    return fudgedb

def disjointCheck(checkvec):
    if length(uniq(checkvec)) < length(checkvec):
        return 1
    else:
        return 0

def disjointIntervals(IntStart,IntEnd,CheckVal):
    if IntStart < CheckVal and CheckVal < IntEnd:
        return 1
    else:
        return 0

def IntervalVecCheck(checkVec):
    veclen = len(checkVec)
    returnval = 0
    for i in range(veclen):
        for j in range(veclen):
            if (disjointIntervals(checkVec[j][0],checkVec[j][1],checkVec[i][0]) + disjointIntervals(checkVec[j][0],checkVec[j][1],checkVec[i][1])) > 0:
                returnval = returnval + 1
    if returnval > 0:
        return 1
    else:
        return 0



\end{sagesilent}

\begin{sagesilent}
#### Start of problem p1
p1c1 = NonZeroInt(-10,10)
p1c2 = NonZeroInt(-10,10,[p1c1])
p1domvec = [ 'x' + LatexExpr(r"\leq") + str(p1c1), str(x < p1c1), 'x' + LatexExpr(r"\geq") + str(p1c1), str(x > p1c1) ]
p1codomvec = [ 'x' + LatexExpr(r"\leq") + str(p1c2), str(x < p1c2), 'x' + LatexExpr(r"\geq") + str(p1c2), str(x > p1c2) ]

p1dompick =RandInt(0,3)
p1codompick = NonZeroInt(0,3,[p1dompick])

p1var1 = p1domvec[p1dompick]
p1var2 = p1codomvec[p1codompick]

p1ans1 = p1domvec[p1dompick]
p1ans2 = p1codomvec[p1codompick]
p1ans3 = p1domvec[3 - p1dompick]
p1ans4 = p1codomvec[3 - p1codompick]


#### Start of problem p2
p2c1 = NonZeroInt(-10,10)
p2c2 = NonZeroInt(-10,10,[p2c1])

p2domvec = [ 'x' + LatexExpr(r"\leq") + str(p2c1), str(x < p2c1), 'x' + LatexExpr(r"\geq") + str(p2c1), str(x > p2c1) ]
p2codomvec = [ 'x' + LatexExpr(r"\leq") + str(p2c2), str(x < p2c2), 'x' + LatexExpr(r"\geq") + str(p2c2), str(x > p2c2) ]

p2dompick =RandInt(0,3)
p2codompick = NonZeroInt(0,3,[p2dompick])

p2var1 = p2domvec[p2dompick]
p2var2 = p2codomvec[p2codompick]

p2ans1 = p2domvec[p2dompick]
p2ans2 = p2codomvec[p2codompick]
p2ans3 = p2domvec[3 - p2dompick]
p2ans4 = p2codomvec[3 - p2codompick]


#### Start of problem p3
p3c1 = NonZeroInt(-10,10)
p3c2 = NonZeroInt(-10,10,[p3c1])
p3domvec = [ 'x' + LatexExpr(r"\leq") + str(p3c1), str(x < p3c1), 'x' + LatexExpr(r"\geq") + str(p3c1), str(x > p3c1) ]
p3codomvec = [ 'x' + LatexExpr(r"\leq") + str(p3c2), str(x < p3c2), 'x' + LatexExpr(r"\geq") + str(p3c2), str(x > p3c2) ]

p3dompick =RandInt(0,3)
p3codompick = NonZeroInt(0,3,[p3dompick])

p3var1 = p3domvec[p3dompick]
p3var2 = p3codomvec[p3codompick]

p3ans1 = p3domvec[p3dompick]
p3ans2 = p3codomvec[p3codompick]
p3ans3 = p3domvec[3 - p3dompick]
p3ans4 = p3codomvec[3 - p3codompick]


#### Start of problem p4
p4c1 = NonZeroInt(-10,10)
p4c2 = NonZeroInt(-10,10,[p4c1])
p4domvec = [ 'x' + LatexExpr(r"\leq") + str(p4c1), str(x < p4c1), 'x' + LatexExpr(r"\geq") + str(p4c1), str(x > p4c1) ]
p4codomvec = [ 'x' + LatexExpr(r"\leq") + str(p4c2), str(x < p4c2), 'x' + LatexExpr(r"\geq") + str(p4c2), str(x > p4c2) ]


p4dompick =RandInt(0,3)
p4codompick = NonZeroInt(0,3,[p4dompick])

p4var1 = p4domvec[p4dompick]
p4var2 = p4codomvec[p4codompick]

p4ans1 = p4domvec[p4dompick]
p4ans2 = p4codomvec[p4codompick]
p4ans3 = p4domvec[3 - p4dompick]
p4ans4 = p4codomvec[3 - p4codompick]




\end{sagesilent}

\begin{problem}
    Consider the function 
    \[
        f:\{ x \in \mathbb{R} : \sage{p1var1} \} \longrightarrow \{ x \in \mathbb{R} : \sage{p1var2} \}
    \]
    
    What is the domain of the function $f$?
    \begin{multipleChoice}
        \choice[correct]{ $ \{ x \in \mathbb{R} : \sage{p1ans1} \}  $ }
        \choice{$\{ x \in \mathbb{R} : \sage{p1ans2} \} $}
        \choice{$\{ x \in \mathbb{R} : \sage{p1ans3} \} $}
        \choice{$\{ x \in \mathbb{R} : \sage{p1ans4} \} $}        
    \end{multipleChoice}
    
    \begin{feedback}
        Remember that the domain is the ``input'' of the function, and is the part listed first. So a function $f:A\rightarrow B$ would have a domain of ``$A$''.
    \end{feedback}    
    What is the codomain of the function $f$?
    
    \begin{multipleChoice}
        \choice{ $ \{ x \in \mathbb{R} : \sage{p1ans1} \}  $ }
        \choice[correct]{$\{ x \in \mathbb{R} : \sage{p1ans2} \} $}
        \choice{$\{ x \in \mathbb{R} : \sage{p1ans3} \} $}
        \choice{$\{ x \in \mathbb{R} : \sage{p1ans4} \} $}        
    \end{multipleChoice}
    
    \begin{feedback}
        Remember that the codomain is the ``type of output'' of the function, and is the part listed second. So a function $f:A\rightarrow B$ would have a codomain of ``$B$''.
    \end{feedback}    
    
\end{problem}


\begin{problem}
    Consider the function 
    \[
        f:\{ x \in \mathbb{R} : \sage{p2var1} \} \longrightarrow \{ x \in \mathbb{R} : \sage{p2var2} \}
    \]
    
    What is the domain of the function $f$?
    \begin{multipleChoice}
        \choice[correct]{ $ \{ x \in \mathbb{R} : \sage{p2ans1} \}  $ }
        \choice{$\{ x \in \mathbb{R} : \sage{p2ans2} \} $}
        \choice{$\{ x \in \mathbb{R} : \sage{p2ans3} \} $}
        \choice{$\{ x \in \mathbb{R} : \sage{p2ans4} \} $}        
    \end{multipleChoice}
    
    \begin{feedback}
        Remember that the domain is the ``input'' of the function, and is the part listed first. So a function $f:A\rightarrow B$ would have a domain of ``$A$''.
    \end{feedback}    
    
    What is the codomain of the function $f$?
    
    \begin{multipleChoice}
        \choice{ $ \{ x \in \mathbb{R} : \sage{p2ans1} \}  $ }
        \choice[correct]{$\{ x \in \mathbb{R} : \sage{p2ans2} \} $}
        \choice{$\{ x \in \mathbb{R} : \sage{p2ans3} \} $}
        \choice{$\{ x \in \mathbb{R} : \sage{p2ans4} \} $}        
    \end{multipleChoice}
        
    \begin{feedback}
        Remember that the codomain is the ``type of output'' of the function, and is the part listed second. So a function $f:A\rightarrow B$ would have a codomain of ``$B$''.
    \end{feedback}    
\end{problem}


\begin{problem}
    Consider the function 
    \[
        f:\{ x \in \mathbb{R} : \sage{p3var1} \} \longrightarrow \{ x \in \mathbb{R} : \sage{p3var2} \}
    \]
    
    What is the domain of the function $f$?
    \begin{multipleChoice}
        \choice[correct]{ $ \{ x \in \mathbb{R} : \sage{p3ans1} \}  $ }
        \choice{$\{ x \in \mathbb{R} : \sage{p3ans2} \} $}
        \choice{$\{ x \in \mathbb{R} : \sage{p3ans3} \} $}
        \choice{$\{ x \in \mathbb{R} : \sage{p3ans4} \} $}        
    \end{multipleChoice}
    
    \begin{feedback}
        Remember that the domain is the ``input'' of the function, and is the part listed first. So a function $f:A\rightarrow B$ would have a domain of ``$A$''.
    \end{feedback}    
    
    What is the codomain of the function $f$?
    
    \begin{multipleChoice}
        \choice{ $ \{ x \in \mathbb{R} : \sage{p3ans1} \}  $ }
        \choice[correct]{$\{ x \in \mathbb{R} : \sage{p3ans2} \} $}
        \choice{$\{ x \in \mathbb{R} : \sage{p3ans3} \} $}
        \choice{$\{ x \in \mathbb{R} : \sage{p3ans4} \} $}        
    \end{multipleChoice}
    
    \begin{feedback}
        Remember that the codomain is the ``type of output'' of the function, and is the part listed second. So a function $f:A\rightarrow B$ would have a codomain of ``$B$''.
    \end{feedback}    
\end{problem}


\begin{problem}
    Consider the function 
    \[
        f:\{ x \in \mathbb{R} : \sage{p4var1} \} \longrightarrow \{ x \in \mathbb{R} : \sage{p4var2} \}
    \]
    
    What is the domain of the function $f$?
    \begin{multipleChoice}
        \choice[correct]{ $ \{ x \in \mathbb{R} : \sage{p4ans1} \}  $ }
        \choice{$\{ x \in \mathbb{R} : \sage{p4ans2} \} $}
        \choice{$\{ x \in \mathbb{R} : \sage{p4ans3} \} $}
        \choice{$\{ x \in \mathbb{R} : \sage{p4ans4} \} $}        
    \end{multipleChoice}
    
    \begin{feedback}
        Remember that the domain is the ``input'' of the function, and is the part listed first. So a function $f:A\rightarrow B$ would have a domain of ``$A$''.
    \end{feedback}    
    
    What is the codomain of the function $f$?
    
    \begin{multipleChoice}
        \choice{ $ \{ x \in \mathbb{R} : \sage{p4ans1} \}  $ }
        \choice[correct]{$\{ x \in \mathbb{R} : \sage{p4ans2} \} $}
        \choice{$\{ x \in \mathbb{R} : \sage{p4ans3} \} $}
        \choice{$\{ x \in \mathbb{R} : \sage{p4ans4} \} $}        
    \end{multipleChoice}
    
    \begin{feedback}
        Remember that the codomain is the ``type of output'' of the function, and is the part listed second. So a function $f:A\rightarrow B$ would have a codomain of ``$B$''.
    \end{feedback}    
\end{problem}



\end{document}