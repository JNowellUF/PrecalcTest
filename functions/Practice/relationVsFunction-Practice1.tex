\documentclass{ximera}

\title{Functions Practice 1}
\begin{document}

%
%\begin{sagesilent}
%
%\end{sagesilent}

\begin{problem}
    Is it possible to have a relation that is not a function?
    \begin{multipleChoice}
        \choice[correct]{Yes.}
        \choice{No.}
    \end{multipleChoice}
    \begin{feedback}
        Although we restrict our attention almost exclusively to functions in this course, the reality is that there are a lot of relations that are not functions. Any time that there is uncertainty for example, the relation is most probably not a function.
    \end{feedback}
\end{problem}

\begin{problem}
    Is it possible to have an equation that is not a function?
    \begin{multipleChoice}
        \choice[correct]{Yes.}
        \choice{No.}
    \end{multipleChoice}
    \begin{feedback}
        Although it is less frequent, it is indeed possible to have equations that do not represent functions. Most commonly these involve equality between variables that represent far more complex objects than numbers. Since we will be primarily focusing on functions we won't be encountering these things in practice in this course, but it is important to remember that \textit{just because you see an equality does not automatically mean it is a function}.
    \end{feedback}
\end{problem}
\begin{problem}
    Which of the following would be considered functions? (Select all that apply)
    \begin{selectAll}
        \choice{$A = $ the area of a provided square.}
        \choice[correct]{$f(x) = x^2 + 3x - 4$.}
        \choice{$y^2 = x^2$.}
        \choice{The formula for the area of an elipse is $AB\pi$.}
        \choice[correct]{$I = e^{rt}$}
        \choice[correct]{$e = mc^2$}
    \end{selectAll}
    \begin{feedback}
        In order: The first choice has an equals sign but it is actually an improperly defined variable rather than an actual equation, let alone a function.
        
        The second is a classic quadratic function.
        
        The third is probably the most tricky as it appears to be a function at first glance, but even if we assume the dependent variable is $y$ and the independent variable is $x$, the equality fails the property required for a function. For example, if $x=2$ then both $y=2$ and $y=-2$ are valid $y$ values that satisfy the equality (check for yourself!) and so the property that ``each input has exactly one output'' fails in this case.
        
        The fourth is a statement of a function, but it is not written as a function. Indeed if this were translated into symbols, it would \textit{then} be a function.
        
        The fifth is a classic function, typically associated with continually accumulated interest.
        
        The last is another classic function that relates mass to the quantity of energy that it generates when it is entirely converted to energy.
    \end{feedback}
\end{problem}




\end{document}