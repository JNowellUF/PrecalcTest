\documentclass{ximera}

\title{A Review of Exponential Functions}
\begin{document}
\begin{abstract}
    This section reviews the basics of exponential functions and how to compute numeric exponentials.
\end{abstract}
\maketitle


\subsection*{What are Exponentials?}

    One typically learns exponentials as a purely mechanical action. In essence, one can think of exponentials as ``repeated multiplication", much like how multiplication is ``repeated addition". But this is hiding some of the more intuitive and important understanding of exponential functions. Nonetheless it helps to learn the mechanics (at least for numbers) before considering the modeling aspects of the function.

\subsection*{Numeric Computation of Exponentials; a brief review.}

    Exponentials are given in the form $(\text{base})^{(\text{exponent})}$. It is perhaps easiest to understand the notation by explicitly considering the parallel we mentioned earlier; that of multiplication and addition.
    
        {\bfseries Comparing exponents to multiplication and multiplication to addition}\\%
        
        When you were (very) young and learning addition, you probably had moments of adding the same number to itself repeatedly. In fact, counting is doing exactly this with the number $1$. So, counting $1,2,3,4,5$ is really just adding $1$ to itself 5 times and saying aloud each substep. As you got (slightly) older it probably got bothersome to add the same number to itself over and over, and this is often when multiplication is introduced.\\
        Thus we can think of multiplication (and it is in fact defined to be) just adding something over and over. Instead of computing $4 + 4 + 4$ we can say we have $3$ `fours' and compute it as $3 \cdot 4 = 12$. This is probably just about boring you into a stupor here, so let's move to exponents.\\
        Exponentials are the same idea. Instead of multiplying $4 \cdot 4 \cdot 4$ we can say we have $3$ `fours' that are being multiplied. Obviously we can't use the same symbol to represent this since if we wrote $3 \cdot 4$ we would interpret that as $4 + 4 + 4$ not $4 \cdot 4 \cdot 4$. Thus to tell the difference we put the $3$ in a different spot, namely we write $4^3$.\\
        This may seem obvious and a waste of time, but recognizing that all an exponent is, is a way of writing repeated multiplication, we can actually extract \textit{almost all} the important features of exponents from just remembering that one little detail. This is what we will do next.

%\begin{question}
%    This is a purely Place Holder type question that will be replaced.
%    \begin{multipleChoice}
%        \choice{This question shouldn't be possible to get correct.}
%    \end{multipleChoice}
%\end{question}

\end{document}