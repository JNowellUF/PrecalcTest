\documentclass{ximera}
\title{Unlimited Practice for Properties of Exponentials}
\usepackage{longdivision}
\usepackage{polynom}
\usepackage{float}% Use `H' as the figure optional argument to force it's vertical placement to conform to source.
%\usepackage{caption}% Allows us to describe the figures without having "figure 1:" in it. :: Apparently Caption isn't supported.
%    \captionsetup{labelformat=empty}% Actually does the figure configuration stated above.
\usetikzlibrary{arrows.meta,arrows}% Allow nicer arrow heads for tikz.
\usepackage{gensymb, pgfplots}
\usepackage{tabularx}
\usepackage{arydshln}
\usepackage[margin=1.5cm]{geometry}
\usepackage{indentfirst}

\setlength\parindent{16pt}

\graphicspath{
  {./}
  {./explorePolynomials/}
  {./exploreRadicals/}
  {./graphing/}
}

%% Default style for tikZ
\pgfplotsset{my style/.append style={axis x line=middle, axis y line=
middle, xlabel={$x$}, ylabel={$y$}, axis equal }}


%% Because log being natural log is too hard for people.
\let\logOld\log% Keep the old \log definition, just in case we need it.
\renewcommand{\log}{\ln}


%%% Changes in polynom to show the zero coefficient terms
\makeatletter
\def\pld@CF@loop#1+{%
    \ifx\relax#1\else
        \begingroup
          \pld@AccuSetX11%
          \def\pld@frac{{}{}}\let\pld@symbols\@empty\let\pld@vars\@empty
          \pld@false
          #1%
          \let\pld@temp\@empty
          \pld@AccuIfOne{}{\pld@AccuGet\pld@temp
                            \edef\pld@temp{\noexpand\pld@R\pld@temp}}%
           \pld@if \pld@Extend\pld@temp{\expandafter\pld@F\pld@frac}\fi
           \expandafter\pld@CF@loop@\pld@symbols\relax\@empty
           \expandafter\pld@CF@loop@\pld@vars\relax\@empty
           \ifx\@empty\pld@temp
               \def\pld@temp{\pld@R11}%
           \fi
          \global\let\@gtempa\pld@temp
        \endgroup
        \ifx\@empty\@gtempa\else
            \pld@ExtendPoly\pld@tempoly\@gtempa
        \fi
        \expandafter\pld@CF@loop
    \fi}
\def\pld@CMAddToTempoly{%
    \pld@AccuGet\pld@temp\edef\pld@temp{\noexpand\pld@R\pld@temp}%
    \pld@CondenseMonomials\pld@false\pld@symbols
    \ifx\pld@symbols\@empty \else
        \pld@ExtendPoly\pld@temp\pld@symbols
    \fi
    \ifx\pld@temp\@empty \else
        \pld@if
            \expandafter\pld@IfSum\expandafter{\pld@temp}%
                {\expandafter\def\expandafter\pld@temp\expandafter
                    {\expandafter\pld@F\expandafter{\pld@temp}{}}}%
                {}%
        \fi
        \pld@ExtendPoly\pld@tempoly\pld@temp
        \pld@Extend\pld@tempoly{\pld@monom}%
    \fi}
\makeatother




%%%%% Code for making prime factor trees for numbers, taken from user Qrrbrbirlbel at: https://tex.stackexchange.com/questions/131689/how-to-automatically-draw-tree-diagram-of-prime-factorization-with-latex

\usepackage{forest,mathtools,siunitx}
\makeatletter
\def\ifNum#1{\ifnum#1\relax
  \expandafter\pgfutil@firstoftwo\else
  \expandafter\pgfutil@secondoftwo\fi}
\forestset{
  num content/.style={
    delay={
      content/.expanded={\noexpand\num{\forestoption{content}}}}},
  pt@prime/.style={draw, circle},
  pt@start/.style={},
  pt@normal/.style={},
  start primeTree/.style={%
    /utils/exec=%
      % \pt@start holds the current minimum factor, we'll start with 2
      \def\pt@start{2}%
      % \pt@result will hold the to-be-typeset factorization, we'll start with
      % \pgfutil@gobble since we don't want a initial \times
      \let\pt@result\pgfutil@gobble
      % \pt@start@cnt holds the number of ^factors for the current factor
      \def\pt@start@cnt{0}%
      % \pt@lStart will later hold "l"ast factor used
      \let\pt@lStart\pgfutil@empty,
    alias=pt-start,
    pt@start/.try,
    delay={content/.expanded={$\noexpand\num{\forestove{content}}
                            \noexpand\mathrlap{{}= \noexpand\pt@result}$}},
    primeTree},
  primeTree/.code=%
    % take the content of the node and save it in the count
    \c@pgf@counta\forestove{content}\relax
    % if it's 2 we're already finished with the factorization
    \ifNum{\c@pgf@counta=2}{%
      % add the factor
      \pt@addfactor{2}%
      % finalize the factorization of the result
      \pt@addfactor{}%
      % and set the style to the prime style
      \forestset{pt@prime/.try}%
    }{%
      % this simply calculates content/2 and saves it in \pt@end
      % this is later used for an early break of the recursion since no factor
      % can be greater then content/2 (for integers of course)
      \edef\pt@content{\the\c@pgf@counta}%
      \divide\c@pgf@counta2\relax
      \advance\c@pgf@counta1\relax % to be on the safe side
      \edef\pt@end{\the\c@pgf@counta}%
      \pt@do}}

%%% our main "function"
\def\pt@do{%
  % let's test if the current factor is already greather then the max factor
  \ifNum{\pt@end<\pt@start}{%
    % great, we're finished, the same as above
    \expandafter\pt@addfactor\expandafter{\pt@content}%
    \pt@addfactor{}%
    \def\pt@next{\forestset{pt@prime/.try}}%
  }{%
    % this calculates int(content/factor)*factor
    % if factor is a factor of content (without remainder), the result will
    % equal content. The int(content/factor) is saved in \pgf@temp.
    \c@pgf@counta\pt@content\relax
    \divide\c@pgf@counta\pt@start\relax
    \edef\pgf@temp{\the\c@pgf@counta}%
    \multiply\c@pgf@counta\pt@start\relax
    \ifNum{\the\c@pgf@counta=\pt@content}{%
      % yeah, we found a factor, add it to the result and ...
      \expandafter\pt@addfactor\expandafter{\pt@start}%
      % ... add the factor as the first child with style pt@prime
      % and the result of int(content/factor) as another child.
      \edef\pt@next{\noexpand\forestset{%
        append={[\pt@start, pt@prime/.try]},
        append={[\pgf@temp, pt@normal/.try]},
        % forest is complex, this makes sure that for the second child, the
        % primeTree style is not executed too early (there must be a better way).
        delay={
          for descendants={
            delay={if n'=1{primeTree, num content}{}}}}}}%
    }{%
      % Alright this is not a factor, let's get the next factor
      \ifNum{\pt@start=2}{%
        % if the previous factor was 2, the next one will be 3
        \def\pt@start{3}%
      }{%
        % hmm, the previos factor was not 2,
        % let's add 2, maybe we'll hit the next prime number
        % and maybe a factor
        \c@pgf@counta\pt@start
        \advance\c@pgf@counta2\relax
        \edef\pt@start{\the\c@pgf@counta}%
      }%
      % let's do that again
      \let\pt@next\pt@do
    }%
  }%
  \pt@next
}

%%% this builds the \pt@result macro with the factors
\def\pt@addfactor#1{%
  \def\pgf@tempa{#1}%
  % is it the same factor as the previous one
  \ifx\pgf@tempa\pt@lStart
    % add 1 to the counter
    \c@pgf@counta\pt@start@cnt\relax
    \advance\c@pgf@counta1\relax
    \edef\pt@start@cnt{\the\c@pgf@counta}%
  \else
    % a new factor! Add the previous one to the product of factors
    \ifx\pt@lStart\pgfutil@empty\else
      % as long as there actually is one, the \ifnum makes sure we do not add ^1
      \edef\pgf@tempa{\noexpand\num{\pt@lStart}\ifnum\pt@start@cnt>1 
                                           ^{\noexpand\num{\pt@start@cnt}}\fi}%
      \expandafter\pt@addfactor@\expandafter{\pgf@tempa}%
    \fi
    % setup the macros for the next round
    \def\pt@lStart{#1}% <- current (new) factor
    \def\pt@start@cnt{1}% <- first time
  \fi
}
%%% This simply appends "\times #1" to \pt@result, with etoolbox this would be
%%% \appto\pt@result{\times#1}
\def\pt@addfactor@#1{%
  \expandafter\def\expandafter\pt@result\expandafter{\pt@result \times #1}}

%%% Our main macro:
%%% #1 = possible optional argument for forest (can be tikz too)
%%% #2 = the number to factorize
\newcommand*{\PrimeTree}[2][]{%
  \begin{forest}%
    % as the result is set via \mathrlap it doesn't update the bounding box
    % let's fix this:
    tikz={execute at end scope={\pgfmathparse{width("${}=\pt@result$")}%
                         \path ([xshift=\pgfmathresult pt]pt-start.east);}},
    % other optional arguments
    #1
    % And go!
    [#2, start primeTree]
  \end{forest}}
\makeatother


\providecommand\tabitem{\makebox[1em][r]{\textbullet~}}
\providecommand{\letterPlus}{\makebox[0pt][l]{$+$}}
\providecommand{\letterMinus}{\makebox[0pt][l]{$-$}}

\renewcommand{\texttt}[1]{#1}% Renew the command to prevent it from showing up in the sage strings for some weird reason.
%\renewcommand{\text}[1]{#1}% Renew the command to prevent it from showing up in the sage strings for some weird reason.





\begin{document}
\begin{sagesilent}

######  Define a function to convert a sage number into a saved counter number.

#####Define default Sage variables.
#Default function variables
var('x,y,z,X,Y,Z')
#Default function names
var('f,g,h,dx,dy,dz,dh,df')
#Default Wild cards
w0 = SR.wild(0)

def DispSign(b):
    """ Returns the string of the 'signed' version of `b`, e.g. 3 -> "+3", -3 -> "-3", 0 -> "".
    """
    if b == 0:
        return ""
    elif b > 0:
        return "+" + str(b)
    elif b < 0:
        return str(b)
    else:
        # If we're here, then something has gone wrong.
        raise ValueError

def ISP(b):
    return DispSign(b)

def NoEval(f, c):
    # TODO
    """ Returns a non-evaluted version of the result f(c).
    """
    cStr = str(c)
    # fLatex = latex(f)
    fString = latex(f)
    fStrList = list(fString)
    length = len(fStrList)
    fStrList2 = range(length)
    for i in range(0, length):
        if fStrList[i] == "x":
            fStrList2[i] = "("+cstr+")"
        else:
            fStrList2[i] = fStrList[i]
    f2 = join(fStrList2,"")
    return LatexExpr(f2)

def HyperSimp(f):
    """ Returns the expression `f` without hyperbolic expressions.
    """
    subsDict = {
        sinh(w0) : (exp(w0) - exp(-w0))/2,
        cosh(w0) : (exp(w0) + exp(-w0))/2,
        tanh(w0) : (exp(w0) - exp(-w0))/(exp(w0) + exp(-w0)),
        sech(w0) : 2/(exp(w0) + exp(-w0)),                      # This seems to work, but Nowell said it didn't at one point.
        csch(w0) : 2/(exp(w0) - exp(-w0)),                      # This seems to work, but Nowell said it didn't at one point.
        coth(w0) : (exp(w0) + exp(-w0))/(exp(w0) - exp(-w0)),   # This seems to work, but Nowell said it didn't at one point.
        arcsinh(w0) :       ln( w0 + sqrt((w0)^2 + 1) ),
        arccosh(w0) :       ln( w0 + sqrt((w0)^2 - 1) ),
        arctanh(w0) : 1/2 * ln( (1 + w0) / (1 - w0) ),
        arccsch(w0) :       ln( (1 + sqrt((w0)^2 + 1))/w0 ),
        arcsech(w0) :       ln( (1 + sqrt(1 - (w0)^2))/w0 ),
        arccoth(w0) : 1/2 * ln( (1 + w0) / (w0 - 1) )
    }
    g = f.substitute(subsDict)
    return simplify(g)

def RandInt(a,b):
    """ Returns a random integer in [`a`,`b`]. Note that `a` and `b` should be integers themselves to avoid unexpected behavior.
    """
    return QQ(randint(int(a),int(b)))
    # return choice(range(a,b+1))

def NonZeroInt(b,c, avoid = [0]):
    """ Returns a random integer in [`b`,`c`] which is not in `av`. 
        If `av` is not specified, defaults to a non-zero integer.
    """
    while True:
        a = RandInt(b,c)
        if a not in avoid:
            return a

def RandVector(b, c, avoid=[], rep=1):
    """ Returns essentially a multiset permutation of ([b,c]-av) * rep.
        That is, a vector which contains each integer in [`b`,`c`] which is not in `av` a total of `rep` number of times.
        Example:
        sage: RandVector(1,3, [2], 2)
        [3, 1, 1, 3]
    """
    oneVec = [val for val in range(b,c+1) if val not in avoid]
    vec = oneVec * rep
    shuffle(vec)
    return vec

def fudge(b):
    up = b+RandInt(2,5)/10
    down = b-RandInt(2,5)/10
    fudgebup = round(up,1)
    fudgebdown = round(down,1)
    fudgedb = [fudgebdown,fudgebup]
    return fudgedb

def disjointCheck(checkvec):
    if length(uniq(checkvec)) < length(checkvec):
        return 1
    else:
        return 0

def disjointIntervals(IntStart,IntEnd,CheckVal):
    if IntStart < CheckVal and CheckVal < IntEnd:
        return 1
    else:
        return 0

def IntervalVecCheck(checkVec):
    veclen = len(checkVec)
    returnval = 0
    for i in range(veclen):
        for j in range(veclen):
            if (disjointIntervals(checkVec[j][0],checkVec[j][1],checkVec[i][0]) + disjointIntervals(checkVec[j][0],checkVec[j][1],checkVec[i][1])) > 0:
                returnval = returnval + 1
    if returnval > 0:
        return 1
    else:
        return 0



\end{sagesilent}

\begin{sagesilent}
var('x','y','z','r')
###### Problem p1

p1c1 = RandInt(-6,6)
p1c2 = RandInt(-6,6)
p1c3 = RandInt(-6,6)
p1c4 = RandInt(-6,6)
p1c5 = RandInt(-6,6)
p1c6 = RandInt(-6,6)
p1c7 = RandInt(-6,6)
p1c8 = RandInt(-6,6)
p1c9 = RandInt(-6,6)
p1c10 = RandInt(-6,6)
p1c11 = RandInt(-6,6)
p1c12 = RandInt(-6,6)
p1c13 = RandInt(-6,6)
p1c14 = RandInt(-6,6)
p1c15 = RandInt(-6,6)
p1c16 = RandInt(-6,6)
p1pwr1 = RandInt(-6,6)
p1pwr2 = RandInt(-6,6)

p1f1 = x^p1c1*y^p1c2*z^p1c3*r^p1c4
p1f2 = x^p1c5*y^p1c6*z^p1c7*r^p1c8

p1f3 = x^p1c9*y^p1c10*z^p1c11*r^p1c12
p1f4 = x^p1c13*y^p1c14*z^p1c15*r^p1c16

p1ans1 = (p1c1-p1c5)*p1pwr1 + (p1c9-p1c13)*p1pwr2
p1ans2 = (p1c2-p1c6)*p1pwr1 + (p1c10-p1c14)*p1pwr2
p1ans3 = (p1c3-p1c7)*p1pwr1 + (p1c11-p1c15)*p1pwr2
p1ans4 = (p1c4-p1c8)*p1pwr1 + (p1c12-p1c16)*p1pwr2


###### Problem p2

p2c1 = RandInt(-6,6)
p2c2 = RandInt(-6,6)
p2c3 = RandInt(-6,6)
p2c4 = RandInt(-6,6)
p2c5 = RandInt(-6,6)
p2c6 = RandInt(-6,6)
p2c7 = RandInt(-6,6)
p2c8 = RandInt(-6,6)
p2c9 = RandInt(-6,6)
p2c10 = RandInt(-6,6)
p2c11 = RandInt(-6,6)
p2c12 = RandInt(-6,6)
p2c13 = RandInt(-6,6)
p2c14 = RandInt(-6,6)
p2c15 = RandInt(-6,6)
p2c16 = RandInt(-6,6)
p2pwr1 = RandInt(-6,6)
p2pwr2 = RandInt(-6,6)

p2f1 = x^p2c1*y^p2c2*z^p2c3*r^p2c4
p2f2 = x^p2c5*y^p2c6*z^p2c7*r^p2c8

p2f3 = x^p2c9*y^p2c10*z^p2c11*r^p2c12
p2f4 = x^p2c13*y^p2c14*z^p2c15*r^p2c16

p2ans1 = (p2c1-p2c5)*p2pwr1 + (p2c9-p2c13)*p2pwr2
p2ans2 = (p2c2-p2c6)*p2pwr1 + (p2c10-p2c14)*p2pwr2
p2ans3 = (p2c3-p2c7)*p2pwr1 + (p2c11-p2c15)*p2pwr2
p2ans4 = (p2c4-p2c8)*p2pwr1 + (p2c12-p2c16)*p2pwr2


###### Problem p3

p3c1 = RandInt(-6,6)
p3c2 = RandInt(-6,6)
p3c3 = RandInt(-6,6)
p3c4 = RandInt(-6,6)
p3c5 = RandInt(-6,6)
p3c6 = RandInt(-6,6)
p3c7 = RandInt(-6,6)
p3c8 = RandInt(-6,6)
p3c9 = RandInt(-6,6)
p3c10 = RandInt(-6,6)
p3c11 = RandInt(-6,6)
p3c12 = RandInt(-6,6)
p3c13 = RandInt(-6,6)
p3c14 = RandInt(-6,6)
p3c15 = RandInt(-6,6)
p3c16 = RandInt(-6,6)
p3pwr1 = RandInt(-6,6)
p3pwr2 = RandInt(-6,6)

p3f1 = x^p3c1*y^p3c2*z^p3c3*r^p3c4
p3f2 = x^p3c5*y^p3c6*z^p3c7*r^p3c8

p3f3 = x^p3c9*y^p3c10*z^p3c11*r^p3c12
p3f4 = x^p3c13*y^p3c14*z^p3c15*r^p3c16

p3ans1 = (p3c1-p3c5)*p3pwr1 + (p3c9-p3c13)*p3pwr2
p3ans2 = (p3c2-p3c6)*p3pwr1 + (p3c10-p3c14)*p3pwr2
p3ans3 = (p3c3-p3c7)*p3pwr1 + (p3c11-p3c15)*p3pwr2
p3ans4 = (p3c4-p3c8)*p3pwr1 + (p3c12-p3c16)*p3pwr2


###### Problem p4

p4c1 = RandInt(-6,6)
p4c2 = RandInt(-6,6)
p4c3 = RandInt(-6,6)
p4c4 = RandInt(-6,6)
p4c5 = RandInt(-6,6)
p4c6 = RandInt(-6,6)
p4c7 = RandInt(-6,6)
p4c8 = RandInt(-6,6)
p4c9 = RandInt(-6,6)
p4c10 = RandInt(-6,6)
p4c11 = RandInt(-6,6)
p4c12 = RandInt(-6,6)
p4c13 = RandInt(-6,6)
p4c14 = RandInt(-6,6)
p4c15 = RandInt(-6,6)
p4c16 = RandInt(-6,6)
p4pwr1 = RandInt(-6,6)
p4pwr2 = RandInt(-6,6)

p4f1 = x^p4c1*y^p4c2*z^p4c3*r^p4c4
p4f2 = x^p4c5*y^p4c6*z^p4c7*r^p4c8

p4f3 = x^p4c9*y^p4c10*z^p4c11*r^p4c12
p4f4 = x^p4c13*y^p4c14*z^p4c15*r^p4c16

p4ans1 = (p4c1-p4c5)*p4pwr1 + (p4c9-p4c13)*p4pwr2
p4ans2 = (p4c2-p4c6)*p4pwr1 + (p4c10-p4c14)*p4pwr2
p4ans3 = (p4c3-p4c7)*p4pwr1 + (p4c11-p4c15)*p4pwr2
p4ans4 = (p4c4-p4c8)*p4pwr1 + (p4c12-p4c16)*p4pwr2



\end{sagesilent}

Here is a walk-through example of how to do a problem like this:

\begin{example}
Use rules of exponents to rewrite the following expression without any fractions (using negative exponents if needed):

\[
    \left(\frac{\frac{a^3b^2c^{-4}d^6}{a^2d^{-2}}}{\frac{a^4c^{-2}}{c^3d}}\right)^5
        \left(\frac{\frac{b^2cd^3}{a^2}}{\frac{b^3c}{a^{-2}d^3}}\right)^{-3}
\]
Solution: First, we should try to simplify the parts inside the large parentheses. Keep in mind this means we will need to do a bunch of algebra and our last line is likely to involve a ton of cancellation, so remember to use all the rules to combine exponentials with the same base to see how the below happens (especially the last step!)\\

\begin{tabular}{rll}
    $\left(\frac{\frac{a^3b^2c^{-4}d^6}{a^2d^{-2}}}{\frac{a^4c^{-2}}{c^3d}}\right)^5
        \left(\frac{\frac{b^2cd^3}{a^2}}{\frac{b^3c}{a^{-2}d^3}}\right)^{-3}$ 
        & $=\left(\frac{a^3b^2c^{-4}d^6}{a^2d^{-2}}\cdot\frac{c^3d}{a^4c^{-2}}\right)^5
            \left(\frac{b^2cd^3}{a^2}\cdot\frac{a^{-2}d^3}{b^3c}\right)^{-3}$ 
            & Step 1: Invert the bottom fractions and multiply \\
        & $= \left(\frac{a^3b^2c^{-4}d^6c^3d}{a^2d^{-2}a^4c^{-2}}\right)^5
            \left(\frac{b^2cd^3a^{-2}d^3}{a^2b^3c}\right)^{-3}$
            & Step 2: Multiply straight across\\
        & $= \left(\frac{b^2cd^9}{a^3}\right)^5
            \left(\frac{d^6}{a^4b}\right)^{-3}$
            & Step 3: Combine like terms to simplify.
\end{tabular}

Next we want to distribute the large power to each of the terms inside the parentheses. We can do this \textbf{only because} everything inside is being multiplied! In essence, the inside is factored, so we can distribute the outer power.
\renewcommand{\arraystretch}{2.5}
\begin{tabular}{rll}
    $\left(\frac{\frac{a^3b^2c^{-4}d^6}{a^2d^{-2}}}{\frac{a^4c^{-2}}{c^3d}}\right)^5
        \left(\frac{\frac{b^2cd^3}{a^2}}{\frac{b^3c}{a^{-2}d^3}}\right)^{-3}$ 
    & $= \left(\frac{b^2cd^9}{a^3}\right)^5
        \left(\frac{d^6}{a^4b}\right)^{-3}$
        & From previous work.\\
    & $= \left(\frac{b^{10}c^5d^{45}}{a^{15}}\right)
        \left(\frac{d^{-18}}{a^{-12}b^{-3}}\right)$
        & Distribute Power.
        \footnote{You can also distribute the $-1$ separately from the $3$ in the $-3$ power in the second factor. Essentially you can just ``flip'' the fraction and make the $-3$ into a $3$ and then distribute the $3$ to each term. Either way is perfectly correct!}\\
    & $= \frac{b^{10}c^5d^{45}d^{-18}}{a^{15}a^{-12}b^{-3}}$
        & Multiple Straight Across.\\
    & $= \frac{b^{13}c^5d^{27}}{a^3}$
        & Simplify powers using Exponential Properties.\\
    & $= a^{-3}b^{13}c^5d^{27}$
        & Rewrite using negative exponents to avoid having a fraction.
        
\end{tabular}
\renewcommand{\arraystretch}{1}
\end{example}

\begin{problem}
    Use rules of exponents to rewrite the following expression without any fractions (using negative exponents if needed).
    \[
        \left(\frac{\sage{p1f1}}{\sage{p1f2}}\right)^{\sage{p1pwr1}} \cdot \left(\frac{\sage{p1f3}}{\sage{p1f4}}\right)^{\sage{p1pwr2}} = x^{\answer{\sage{p1ans1}}}\cdot y^{\answer{\sage{p1ans2}}}\cdot z^{\answer{\sage{p1ans3}}} \cdot r^{\answer{\sage{p1ans4}}}
    \]
    \begin{feedback}
        Follow the walkthrough above closely; start with simplifying the inside of each set of parentheses, then distribute the power from the parentheses, then combine the two results into a single fraction. 
    \end{feedback}
\end{problem}


\begin{problem}
    Use rules of exponents to rewrite the following expression without any fractions (using negative exponents if needed).
    \[
        \left(\frac{\sage{p2f1}}{\sage{p2f2}}\right)^{\sage{p2pwr1}} \cdot \left(\frac{\sage{p2f3}}{\sage{p2f4}}\right)^{\sage{p2pwr2}} = x^{\answer{\sage{p2ans1}}}\cdot y^{\answer{\sage{p2ans2}}}\cdot z^{\answer{\sage{p2ans3}}} \cdot r^{\answer{\sage{p2ans4}}}
    \]
    \begin{feedback}
        Follow the walkthrough above closely; start with simplifying the inside of each set of parentheses, then distribute the power from the parentheses, then combine the two results into a single fraction. 
    \end{feedback}
\end{problem}


\begin{problem}
    Use rules of exponents to rewrite the following expression without any fractions (using negative exponents if needed).
    \[
        \left(\frac{\sage{p3f1}}{\sage{p3f2}}\right)^{\sage{p3pwr1}} \cdot \left(\frac{\sage{p3f3}}{\sage{p3f4}}\right)^{\sage{p3pwr2}} = x^{\answer{\sage{p3ans1}}}\cdot y^{\answer{\sage{p3ans2}}}\cdot z^{\answer{\sage{p3ans3}}} \cdot r^{\answer{\sage{p3ans4}}}
    \]
    \begin{feedback}
        Follow the walkthrough above closely; start with simplifying the inside of each set of parentheses, then distribute the power from the parentheses, then combine the two results into a single fraction. 
    \end{feedback}
\end{problem}


\begin{problem}
    Use rules of exponents to rewrite the following expression without any fractions (using negative exponents if needed).
    \[
        \left(\frac{\sage{p4f1}}{\sage{p4f2}}\right)^{\sage{p4pwr1}} \cdot \left(\frac{\sage{p4f3}}{\sage{p4f4}}\right)^{\sage{p4pwr2}} = x^{\answer{\sage{p4ans1}}}\cdot y^{\answer{\sage{p4ans2}}}\cdot z^{\answer{\sage{p4ans3}}} \cdot r^{\answer{\sage{p4ans4}}}
    \]
    \begin{feedback}
        Follow the walkthrough above closely; start with simplifying the inside of each set of parentheses, then distribute the power from the parentheses, then combine the two results into a single fraction. 
    \end{feedback}
\end{problem}


\end{document}