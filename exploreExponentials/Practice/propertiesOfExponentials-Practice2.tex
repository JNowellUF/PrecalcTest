\documentclass{ximera}
\usepackage{longdivision}
\usepackage{polynom}
\usepackage{float}% Use `H' as the figure optional argument to force it's vertical placement to conform to source.
%\usepackage{caption}% Allows us to describe the figures without having "figure 1:" in it. :: Apparently Caption isn't supported.
%    \captionsetup{labelformat=empty}% Actually does the figure configuration stated above.
\usetikzlibrary{arrows.meta,arrows}% Allow nicer arrow heads for tikz.
\usepackage{gensymb, pgfplots}
\usepackage{tabularx}
\usepackage{arydshln}
\usepackage[margin=1.5cm]{geometry}
\usepackage{indentfirst}

\setlength\parindent{16pt}

\graphicspath{
  {./}
  {./explorePolynomials/}
  {./exploreRadicals/}
  {./graphing/}
}

%% Default style for tikZ
\pgfplotsset{my style/.append style={axis x line=middle, axis y line=
middle, xlabel={$x$}, ylabel={$y$}, axis equal }}


%% Because log being natural log is too hard for people.
\let\logOld\log% Keep the old \log definition, just in case we need it.
\renewcommand{\log}{\ln}


%%% Changes in polynom to show the zero coefficient terms
\makeatletter
\def\pld@CF@loop#1+{%
    \ifx\relax#1\else
        \begingroup
          \pld@AccuSetX11%
          \def\pld@frac{{}{}}\let\pld@symbols\@empty\let\pld@vars\@empty
          \pld@false
          #1%
          \let\pld@temp\@empty
          \pld@AccuIfOne{}{\pld@AccuGet\pld@temp
                            \edef\pld@temp{\noexpand\pld@R\pld@temp}}%
           \pld@if \pld@Extend\pld@temp{\expandafter\pld@F\pld@frac}\fi
           \expandafter\pld@CF@loop@\pld@symbols\relax\@empty
           \expandafter\pld@CF@loop@\pld@vars\relax\@empty
           \ifx\@empty\pld@temp
               \def\pld@temp{\pld@R11}%
           \fi
          \global\let\@gtempa\pld@temp
        \endgroup
        \ifx\@empty\@gtempa\else
            \pld@ExtendPoly\pld@tempoly\@gtempa
        \fi
        \expandafter\pld@CF@loop
    \fi}
\def\pld@CMAddToTempoly{%
    \pld@AccuGet\pld@temp\edef\pld@temp{\noexpand\pld@R\pld@temp}%
    \pld@CondenseMonomials\pld@false\pld@symbols
    \ifx\pld@symbols\@empty \else
        \pld@ExtendPoly\pld@temp\pld@symbols
    \fi
    \ifx\pld@temp\@empty \else
        \pld@if
            \expandafter\pld@IfSum\expandafter{\pld@temp}%
                {\expandafter\def\expandafter\pld@temp\expandafter
                    {\expandafter\pld@F\expandafter{\pld@temp}{}}}%
                {}%
        \fi
        \pld@ExtendPoly\pld@tempoly\pld@temp
        \pld@Extend\pld@tempoly{\pld@monom}%
    \fi}
\makeatother




%%%%% Code for making prime factor trees for numbers, taken from user Qrrbrbirlbel at: https://tex.stackexchange.com/questions/131689/how-to-automatically-draw-tree-diagram-of-prime-factorization-with-latex

\usepackage{forest,mathtools,siunitx}
\makeatletter
\def\ifNum#1{\ifnum#1\relax
  \expandafter\pgfutil@firstoftwo\else
  \expandafter\pgfutil@secondoftwo\fi}
\forestset{
  num content/.style={
    delay={
      content/.expanded={\noexpand\num{\forestoption{content}}}}},
  pt@prime/.style={draw, circle},
  pt@start/.style={},
  pt@normal/.style={},
  start primeTree/.style={%
    /utils/exec=%
      % \pt@start holds the current minimum factor, we'll start with 2
      \def\pt@start{2}%
      % \pt@result will hold the to-be-typeset factorization, we'll start with
      % \pgfutil@gobble since we don't want a initial \times
      \let\pt@result\pgfutil@gobble
      % \pt@start@cnt holds the number of ^factors for the current factor
      \def\pt@start@cnt{0}%
      % \pt@lStart will later hold "l"ast factor used
      \let\pt@lStart\pgfutil@empty,
    alias=pt-start,
    pt@start/.try,
    delay={content/.expanded={$\noexpand\num{\forestove{content}}
                            \noexpand\mathrlap{{}= \noexpand\pt@result}$}},
    primeTree},
  primeTree/.code=%
    % take the content of the node and save it in the count
    \c@pgf@counta\forestove{content}\relax
    % if it's 2 we're already finished with the factorization
    \ifNum{\c@pgf@counta=2}{%
      % add the factor
      \pt@addfactor{2}%
      % finalize the factorization of the result
      \pt@addfactor{}%
      % and set the style to the prime style
      \forestset{pt@prime/.try}%
    }{%
      % this simply calculates content/2 and saves it in \pt@end
      % this is later used for an early break of the recursion since no factor
      % can be greater then content/2 (for integers of course)
      \edef\pt@content{\the\c@pgf@counta}%
      \divide\c@pgf@counta2\relax
      \advance\c@pgf@counta1\relax % to be on the safe side
      \edef\pt@end{\the\c@pgf@counta}%
      \pt@do}}

%%% our main "function"
\def\pt@do{%
  % let's test if the current factor is already greather then the max factor
  \ifNum{\pt@end<\pt@start}{%
    % great, we're finished, the same as above
    \expandafter\pt@addfactor\expandafter{\pt@content}%
    \pt@addfactor{}%
    \def\pt@next{\forestset{pt@prime/.try}}%
  }{%
    % this calculates int(content/factor)*factor
    % if factor is a factor of content (without remainder), the result will
    % equal content. The int(content/factor) is saved in \pgf@temp.
    \c@pgf@counta\pt@content\relax
    \divide\c@pgf@counta\pt@start\relax
    \edef\pgf@temp{\the\c@pgf@counta}%
    \multiply\c@pgf@counta\pt@start\relax
    \ifNum{\the\c@pgf@counta=\pt@content}{%
      % yeah, we found a factor, add it to the result and ...
      \expandafter\pt@addfactor\expandafter{\pt@start}%
      % ... add the factor as the first child with style pt@prime
      % and the result of int(content/factor) as another child.
      \edef\pt@next{\noexpand\forestset{%
        append={[\pt@start, pt@prime/.try]},
        append={[\pgf@temp, pt@normal/.try]},
        % forest is complex, this makes sure that for the second child, the
        % primeTree style is not executed too early (there must be a better way).
        delay={
          for descendants={
            delay={if n'=1{primeTree, num content}{}}}}}}%
    }{%
      % Alright this is not a factor, let's get the next factor
      \ifNum{\pt@start=2}{%
        % if the previous factor was 2, the next one will be 3
        \def\pt@start{3}%
      }{%
        % hmm, the previos factor was not 2,
        % let's add 2, maybe we'll hit the next prime number
        % and maybe a factor
        \c@pgf@counta\pt@start
        \advance\c@pgf@counta2\relax
        \edef\pt@start{\the\c@pgf@counta}%
      }%
      % let's do that again
      \let\pt@next\pt@do
    }%
  }%
  \pt@next
}

%%% this builds the \pt@result macro with the factors
\def\pt@addfactor#1{%
  \def\pgf@tempa{#1}%
  % is it the same factor as the previous one
  \ifx\pgf@tempa\pt@lStart
    % add 1 to the counter
    \c@pgf@counta\pt@start@cnt\relax
    \advance\c@pgf@counta1\relax
    \edef\pt@start@cnt{\the\c@pgf@counta}%
  \else
    % a new factor! Add the previous one to the product of factors
    \ifx\pt@lStart\pgfutil@empty\else
      % as long as there actually is one, the \ifnum makes sure we do not add ^1
      \edef\pgf@tempa{\noexpand\num{\pt@lStart}\ifnum\pt@start@cnt>1 
                                           ^{\noexpand\num{\pt@start@cnt}}\fi}%
      \expandafter\pt@addfactor@\expandafter{\pgf@tempa}%
    \fi
    % setup the macros for the next round
    \def\pt@lStart{#1}% <- current (new) factor
    \def\pt@start@cnt{1}% <- first time
  \fi
}
%%% This simply appends "\times #1" to \pt@result, with etoolbox this would be
%%% \appto\pt@result{\times#1}
\def\pt@addfactor@#1{%
  \expandafter\def\expandafter\pt@result\expandafter{\pt@result \times #1}}

%%% Our main macro:
%%% #1 = possible optional argument for forest (can be tikz too)
%%% #2 = the number to factorize
\newcommand*{\PrimeTree}[2][]{%
  \begin{forest}%
    % as the result is set via \mathrlap it doesn't update the bounding box
    % let's fix this:
    tikz={execute at end scope={\pgfmathparse{width("${}=\pt@result$")}%
                         \path ([xshift=\pgfmathresult pt]pt-start.east);}},
    % other optional arguments
    #1
    % And go!
    [#2, start primeTree]
  \end{forest}}
\makeatother


\providecommand\tabitem{\makebox[1em][r]{\textbullet~}}
\providecommand{\letterPlus}{\makebox[0pt][l]{$+$}}
\providecommand{\letterMinus}{\makebox[0pt][l]{$-$}}

\renewcommand{\texttt}[1]{#1}% Renew the command to prevent it from showing up in the sage strings for some weird reason.
%\renewcommand{\text}[1]{#1}% Renew the command to prevent it from showing up in the sage strings for some weird reason.



\title{Practice Solving Exponential Equations 2}

\begin{document}
\begin{abstract}
    Unlimited Practice for Exponential Equations.
\end{abstract}
\maketitle

\textbf{NOTE:} These are all randomized problems. As a result, it is entirely possible to get pretty awful numbers if you are suitably unlucky. Some of these may look bad until you start doing them, but if you see problems that look excessively awful, remember that you can always hit the `Another' button in the top (green refresh arrow) to get new numbers. If you find yourself doing this frequently, you may want to discuss it with your TA to see if you have a gap in your understanding, or to see if the problems are just really that bad (in which case the TA will forward the info to the content authors).

%\begin{sagesilent}

######  Define a function to convert a sage number into a saved counter number.

#####Define default Sage variables.
#Default function variables
var('x,y,z,X,Y,Z')
#Default function names
var('f,g,h,dx,dy,dz,dh,df')
#Default Wild cards
w0 = SR.wild(0)

def DispSign(b):
    """ Returns the string of the 'signed' version of `b`, e.g. 3 -> "+3", -3 -> "-3", 0 -> "".
    """
    if b == 0:
        return ""
    elif b > 0:
        return "+" + str(b)
    elif b < 0:
        return str(b)
    else:
        # If we're here, then something has gone wrong.
        raise ValueError

def ISP(b):
    return DispSign(b)

def NoEval(f, c):
    # TODO
    """ Returns a non-evaluted version of the result f(c).
    """
    cStr = str(c)
    # fLatex = latex(f)
    fString = latex(f)
    fStrList = list(fString)
    length = len(fStrList)
    fStrList2 = range(length)
    for i in range(0, length):
        if fStrList[i] == "x":
            fStrList2[i] = "("+cstr+")"
        else:
            fStrList2[i] = fStrList[i]
    f2 = join(fStrList2,"")
    return LatexExpr(f2)

def HyperSimp(f):
    """ Returns the expression `f` without hyperbolic expressions.
    """
    subsDict = {
        sinh(w0) : (exp(w0) - exp(-w0))/2,
        cosh(w0) : (exp(w0) + exp(-w0))/2,
        tanh(w0) : (exp(w0) - exp(-w0))/(exp(w0) + exp(-w0)),
        sech(w0) : 2/(exp(w0) + exp(-w0)),                      # This seems to work, but Nowell said it didn't at one point.
        csch(w0) : 2/(exp(w0) - exp(-w0)),                      # This seems to work, but Nowell said it didn't at one point.
        coth(w0) : (exp(w0) + exp(-w0))/(exp(w0) - exp(-w0)),   # This seems to work, but Nowell said it didn't at one point.
        arcsinh(w0) :       ln( w0 + sqrt((w0)^2 + 1) ),
        arccosh(w0) :       ln( w0 + sqrt((w0)^2 - 1) ),
        arctanh(w0) : 1/2 * ln( (1 + w0) / (1 - w0) ),
        arccsch(w0) :       ln( (1 + sqrt((w0)^2 + 1))/w0 ),
        arcsech(w0) :       ln( (1 + sqrt(1 - (w0)^2))/w0 ),
        arccoth(w0) : 1/2 * ln( (1 + w0) / (w0 - 1) )
    }
    g = f.substitute(subsDict)
    return simplify(g)

def RandInt(a,b):
    """ Returns a random integer in [`a`,`b`]. Note that `a` and `b` should be integers themselves to avoid unexpected behavior.
    """
    return QQ(randint(int(a),int(b)))
    # return choice(range(a,b+1))

def NonZeroInt(b,c, avoid = [0]):
    """ Returns a random integer in [`b`,`c`] which is not in `av`. 
        If `av` is not specified, defaults to a non-zero integer.
    """
    while True:
        a = RandInt(b,c)
        if a not in avoid:
            return a

def RandVector(b, c, avoid=[], rep=1):
    """ Returns essentially a multiset permutation of ([b,c]-av) * rep.
        That is, a vector which contains each integer in [`b`,`c`] which is not in `av` a total of `rep` number of times.
        Example:
        sage: RandVector(1,3, [2], 2)
        [3, 1, 1, 3]
    """
    oneVec = [val for val in range(b,c+1) if val not in avoid]
    vec = oneVec * rep
    shuffle(vec)
    return vec

def fudge(b):
    up = b+RandInt(2,5)/10
    down = b-RandInt(2,5)/10
    fudgebup = round(up,1)
    fudgebdown = round(down,1)
    fudgedb = [fudgebdown,fudgebup]
    return fudgedb

def disjointCheck(checkvec):
    if length(uniq(checkvec)) < length(checkvec):
        return 1
    else:
        return 0

def disjointIntervals(IntStart,IntEnd,CheckVal):
    if IntStart < CheckVal and CheckVal < IntEnd:
        return 1
    else:
        return 0

def IntervalVecCheck(checkVec):
    veclen = len(checkVec)
    returnval = 0
    for i in range(veclen):
        for j in range(veclen):
            if (disjointIntervals(checkVec[j][0],checkVec[j][1],checkVec[i][0]) + disjointIntervals(checkVec[j][0],checkVec[j][1],checkVec[i][1])) > 0:
                returnval = returnval + 1
    if returnval > 0:
        return 1
    else:
        return 0



\end{sagesilent}

\begin{sagesilent}
#####Define default Sage variables.
#Default function variables
var('x,y,z,X,Y,Z')
#Default function names
var('f,g,h,dx,dy,dz,dh,df')
#Default Wild cards
w0 = SR.wild(0)

def RandInt(a,b):
    """ Returns a random integer in [`a`,`b`]. Note that `a` and `b` should be integers themselves to avoid unexpected behavior.
    """
    return QQ(randint(int(a),int(b)))
    # return choice(range(a,b+1))

def NonZeroInt(b,c, avoid = [0]):
    """ Returns a random integer in [`b`,`c`] which is not in `av`. 
        If `av` is not specified, defaults to a non-zero integer.
    """
    while True:
        a = RandInt(b,c)
        if a not in avoid:
            return a

\end{sagesilent}
\begin{sagesilent}
### Problem p1

p1b1 = NonZeroInt(2,15,[4,8,9])
p1denom = RandInt(2,10)

p1r1 = RandInt(-5,5)/RandInt(1,10)
p1r2 = RandInt(-5,5)/p1denom
p1r3 = RandInt(-5,5)/RandInt(1,10)
p1r4 = RandInt(-5,5)/p1denom
p1r5 = RandInt(-5,5)/RandInt(1,10)
p1r6 = RandInt(-5,5)/p1denom
p1r7 = RandInt(-5,5)/RandInt(1,10)
p1r8 = RandInt(-5,5)/p1denom
p1r9 = RandInt(-5,5)/RandInt(1,10)
p1r10 = RandInt(-5,5)/p1denom
p1r11 = RandInt(-5,5)/RandInt(1,10)
p1r12 = RandInt(-5,5)/p1denom

p1a1 = p1r1*x+p1r2+p1r3*y+p1r4+p1r5*z+p1r6-(p1r7*x+p1r8+p1r9*y+p1r10+p1r11*z+p1r12)

p1q1 = (p1b1^(p1r1*x + p1r2)*p1b1^(p1r3*y + p1r4)*p1b1^(p1r5*z + p1r6))/(p1b1^(p1r7*x + p1r8)*p1b1^(p1r9*y + p1r10)*p1b1^(p1r11*z + p1r12))


### Problem p2

p2b1 = NonZeroInt(2,11,[4,8,9])
p2b2 = RandInt(1,3)
p2b3 = RandInt(1,3)
p2b4 = RandInt(1,3)
p2denom = RandInt(2,10)

p2r1 = NonZeroInt(-5,5)/RandInt(1,10)
p2r2 = RandInt(-5,5)/p2denom
p2r3 = NonZeroInt(-5,5)/RandInt(1,10)
p2r4 = RandInt(-5,5)/p2denom
p2r5 = NonZeroInt(-5,5)/RandInt(1,10)
p2r6 = RandInt(-5,5)/p2denom
p2r7 = NonZeroInt(2,10)/RandInt(1,10)
p2r8 = RandInt(-5,5)/p2denom
p2r9 = NonZeroInt(-5,5)/RandInt(1,10)
p2r10 = RandInt(-5,5)/p2denom
p2r11 = NonZeroInt(-5,5)/RandInt(1,10)
p2r12 = RandInt(-5,5)/p2denom

p2a1 = p2b2*(p2r1*x+p2r2)+p2b3*(p2r3*y+p2r4)+p2b4*(p2r5*z+p2r6)-(p2b4*(p2r7*x+p2r8)+p2b2*(p2r9*y+p2r10)+p2b3*(p2r11*z+p2r12))

p2q1 = ((p2b1^(p2b2))^(p2r1*x + p2r2)*(p2b1^(p2b3))^(p2r3*y + p2r4)*(p2b1^(p2b4))^(p2r5*z + p2r6))/((p2b1^(p2b4))^(p2r7*x + p2r8)*(p2b1^(p2b2))^(p2r9*y + p2r10)*(p2b1^(p2b3))^(p2r11*z + p2r12))



### Problem p3

p3b1 = NonZeroInt(2,15,[4,8,9])
p3denom = RandInt(2,10)

p3r1 = RandInt(-5,5)/RandInt(1,10)
p3r2 = RandInt(-5,5)/p3denom
p3r3 = RandInt(-5,5)/RandInt(1,10)
p3r4 = RandInt(-5,5)/p3denom
p3r5 = RandInt(-5,5)/RandInt(1,10)
p3r6 = RandInt(-5,5)/p3denom
p3r7 = RandInt(-5,5)/RandInt(1,10)
p3r8 = RandInt(-5,5)/p3denom
p3r9 = RandInt(-5,5)/RandInt(1,10)
p3r10 = RandInt(-5,5)/p3denom
p3r11 = RandInt(-5,5)/RandInt(1,10)
p3r12 = RandInt(-5,5)/p3denom

p3a1 = p3r1*x+p3r2+p3r3*y+p3r4+p3r5*z+p3r6-(p3r7*x+p3r8+p3r9*y+p3r10+p3r11*z+p3r12)

p3q1 = (p3b1^(p3r1*x + p3r2)*p3b1^(p3r3*y + p3r4)*p3b1^(p3r5*z + p3r6))/(p3b1^(p3r7*x + p3r8)*p3b1^(p3r9*y + p3r10)*p3b1^(p3r11*z + p3r12))


### Problem p4

p4b1 = NonZeroInt(2,11,[4,8,9])
p4b2 = RandInt(1,3)
p4b3 = RandInt(1,3)
p4b4 = RandInt(1,3)
p4denom = RandInt(2,10)

p4r1 = NonZeroInt(-5,5)/RandInt(1,10)
p4r2 = RandInt(-5,5)/p4denom
p4r3 = NonZeroInt(-5,5)/RandInt(1,10)
p4r4 = RandInt(-5,5)/p4denom
p4r5 = NonZeroInt(-5,5)/RandInt(1,10)
p4r6 = RandInt(-5,5)/p4denom
p4r7 = NonZeroInt(2,10)/RandInt(1,10)
p4r8 = RandInt(-5,5)/p4denom
p4r9 = NonZeroInt(-5,5)/RandInt(1,10)
p4r10 = RandInt(-5,5)/p4denom
p4r11 = NonZeroInt(-5,5)/RandInt(1,10)
p4r12 = RandInt(-5,5)/p4denom

p4a1 = p4b2*(p4r1*x+p4r2)+p4b3*(p4r3*y+p4r4)+p4b4*(p4r5*z+p4r6)-(p4b4*(p4r7*x+p4r8)+p4b2*(p4r9*y+p4r10)+p4b3*(p4r11*z+p4r12))

p4q1 = ((p4b1^(p4b2))^(p4r1*x + p4r2)*(p4b1^(p4b3))^(p4r3*y + p4r4)*(p4b1^(p4b4))^(p4r5*z + p4r6))/((p4b1^(p4b4))^(p4r7*x + p4r8)*(p4b1^(p4b2))^(p4r9*y + p4r10)*(p4b1^(p4b3))^(p4r11*z + p4r12))



### Problem p5

p5b1 = NonZeroInt(2,15,[4,8,9])
p5denom = RandInt(2,10)

p5r1 = RandInt(-5,5)/RandInt(1,10)
p5r2 = RandInt(-5,5)/p5denom
p5r3 = RandInt(-5,5)/RandInt(1,10)
p5r4 = RandInt(-5,5)/p5denom
p5r5 = RandInt(-5,5)/RandInt(1,10)
p5r6 = RandInt(-5,5)/p5denom
p5r7 = RandInt(-5,5)/RandInt(1,10)
p5r8 = RandInt(-5,5)/p5denom
p5r9 = RandInt(-5,5)/RandInt(1,10)
p5r10 = RandInt(-5,5)/p5denom
p5r11 = RandInt(-5,5)/RandInt(1,10)
p5r12 = RandInt(-5,5)/p5denom

p5a1 = p5r1*x+p5r2+p5r3*y+p5r4+p5r5*z+p5r6-(p5r7*x+p5r8+p5r9*y+p5r10+p5r11*z+p5r12)

p5q1 = (p5b1^(p5r1*x + p5r2)*p5b1^(p5r3*y + p5r4)*p5b1^(p5r5*z + p5r6))/(p5b1^(p5r7*x + p5r8)*p5b1^(p5r9*y + p5r10)*p5b1^(p5r11*z + p5r12))


### Problem p6

p6b1 = NonZeroInt(2,11,[4,8,9])
p6b2 = RandInt(1,3)
p6b3 = RandInt(1,3)
p6b4 = RandInt(1,3)
p6denom = RandInt(2,10)

p6r1 = NonZeroInt(-5,5)/RandInt(1,10)
p6r2 = RandInt(-5,5)/p6denom
p6r3 = NonZeroInt(-5,5)/RandInt(1,10)
p6r4 = RandInt(-5,5)/p6denom
p6r5 = NonZeroInt(-5,5)/RandInt(1,10)
p6r6 = RandInt(-5,5)/p6denom
p6r7 = NonZeroInt(2,10)/RandInt(1,10)
p6r8 = RandInt(-5,5)/p6denom
p6r9 = NonZeroInt(-5,5)/RandInt(1,10)
p6r10 = RandInt(-5,5)/p6denom
p6r11 = NonZeroInt(-5,5)/RandInt(1,10)
p6r12 = RandInt(-5,5)/p6denom

p6a1 = p6b2*(p6r1*x+p6r2)+p6b3*(p6r3*y+p6r4)+p6b4*(p6r5*z+p6r6)-(p6b4*(p6r7*x+p6r8)+p6b2*(p6r9*y+p6r10)+p6b3*(p6r11*z+p6r12))

p6q1 = ((p6b1^(p6b2))^(p6r1*x + p6r2)*(p6b1^(p6b3))^(p6r3*y + p6r4)*(p6b1^(p6b4))^(p6r5*z + p6r6))/((p6b1^(p6b4))^(p6r7*x + p6r8)*(p6b1^(p6b2))^(p6r9*y + p6r10)*(p6b1^(p6b3))^(p6r11*z + p6r12))


### Problem p7

p7b1 = RandInt(2,6)
p7b2 = RandInt(2,5)
p7b3 = NonZeroInt(2,5,[p7b2])
p7b4 = NonZeroInt(2,5,[p7b2,p7b3])

p7r1 = NonZeroInt(-5,5)/RandInt(1,10)
p7r2 = NonZeroInt(-5,5)/RandInt(1,10)
p7r3 = NonZeroInt(-5,5)/RandInt(1,10)
p7r4 = NonZeroInt(-5,5)/RandInt(1,10)

p7q1 = p7b1^(p7b2*x^p7r1 + p7b3*y^p7r2 + p7b4*z^p7r3 + p7r4)

p7a1 = abs(p7b1^p7r4)


### Problem p8

p8b1 = RandInt(2,6)
p8b2 = RandInt(2,5)
p8b3 = NonZeroInt(2,5,[p8b2])
p8b4 = NonZeroInt(2,5,[p8b2,p8b3])

p8r1 = NonZeroInt(-5,5)/RandInt(1,10)
p8r2 = NonZeroInt(-5,5)/RandInt(1,10)
p8r3 = NonZeroInt(-5,5)/RandInt(1,10)
p8r4 = NonZeroInt(-5,5)/RandInt(1,10)

p8q1 = p8b1^(p8b2*x^p8r1 + p8b3*y^p8r2 + p8b4*z^p8r3 + p8r4)

p8a1 = abs(p8b1^p8r4)


### Problem p9

p9b1 = RandInt(2,6)
p9b2 = RandInt(2,5)
p9b3 = NonZeroInt(2,5,[p9b2])
p9b4 = NonZeroInt(2,5,[p9b2,p9b3])

p9r1 = NonZeroInt(-5,5)/RandInt(1,10)
p9r2 = NonZeroInt(-5,5)/RandInt(1,10)
p9r3 = NonZeroInt(-5,5)/RandInt(1,10)
p9r4 = NonZeroInt(-5,5)/RandInt(1,10)

p9q1 = p9b1^(p9b2*x^p9r1 + p9b3*y^p9r2 + p9b4*z^p9r3 + p9r4)

p9a1 = abs(p9b1^p9r4)


### Problem p10

p10b1 = RandInt(2,6)
p10b2 = RandInt(2,5)
p10b3 = NonZeroInt(2,5,[p10b2])
p10b4 = NonZeroInt(2,5,[p10b2,p10b3])

p10r1 = NonZeroInt(-5,5)/RandInt(1,10)
p10r2 = NonZeroInt(-5,5)/RandInt(1,10)
p10r3 = NonZeroInt(-5,5)/RandInt(1,10)
p10r4 = NonZeroInt(-5,5)/RandInt(1,10)

p10q1 = p10b1^(p10b2*x^p10r1 + p10b3*y^p10r2 + p10b4*z^p10r3 + p10r4)

p10a1 = abs(p10b1^p10r4)

\end{sagesilent}


\textbf{Note to the below problems:} I included some hints that will almost certainly be redundant or stupid (like $14^1 = 14$), but this is a byproduct of generating these things using random numbers. I wanted to make sure to provide you with any crazy computation without you needing to figure it out on it's own (since that wasn't the point of these problems), so I supplied everything, even if it's redundant or silly.

\textbf{However:} Keep in mind that on any assessments (quizzes, exams, etc) the ability to recognize different `bases' as powers of the same base (eg $4 = 2^2$ or $\frac{1}{81} = 3^{-4}$ \textit{will be expected and probably required}. On assessments however, we will make sure to keep the numbers very reasonable; nothing bigger than 3 digits, and they should be fairly recognizable. Remember, if worst-comes-to-worst, you can always make a factor tree to find the prime factors and figure out the lowest base.



Here is a walk-through example of how to do a problem like this:

\begin{example}
Condense the following expression into a single exponential.

\[
    \frac{% Top of fraction
        \left(125^{\frac{1}{3}z}\right)\left(125^{\frac{4}{5}x-\frac{5}{8}}\right)\left(25^{-5y-\frac{5}{8}}\right)
        }% Bottom of fraction.
        {
        \left(125^{\frac{3}{4}x+\frac{1}{2}}\right)\left(25^{-\frac{1}{3}y}\right)\left(25^{-\frac{1}{8}z-\frac{3}{8}}\right)
        }
\]
(Hint: $5^2= 25$, $5^3 = 125$)

 
Solution: Before we can do a lot, it helps to get everything to the same base. This is why a hint is provided in the situation where they don't all start the same base (Note: You shouldn't always expect such a hint to be given though, so keep an eye out!) The easiest way to do this is to simply replace the larger number with the universal base to the appropriate power in parentheses. So in our case we will replace $125$ by $(5^3)$ and $25$ by $(5^2)$ and then simplify.

\noindent\begin{tabular}{rll}
    $\frac{% Top of fraction
        \left(125^{\frac{1}{3}z}\right)\left(125^{\frac{4}{5}x-\frac{5}{8}}\right)\left(25^{-5y-\frac{5}{8}}\right)
        }% Bottom of fraction.
        {
        \left(125^{\frac{3}{4}x+\frac{1}{2}}\right)\left(25^{-\frac{1}{3}y}\right)\left(25^{-\frac{1}{8}z-\frac{3}{8}}\right)
        }$
    & $=\frac{% Top of fraction
        \left((5^3)^{\frac{1}{3}z}\right)\left((5^3)^{\frac{4}{5}x-\frac{5}{8}}\right)\left((5^2)^{-5y-\frac{5}{8}}\right)
        }% Bottom of fraction.
        {
        \left((5^3)^{\frac{3}{4}x+\frac{1}{2}}\right)\left((5^2)^{-\frac{1}{3}y}\right)\left((5^2)^{-\frac{1}{8}z-\frac{3}{8}}\right)
        }$
    & Step 1: Replace each base with universal base and power.\\
    & $=\frac{% Top of fraction
        \left(5^{3\cdot\frac{1}{3}z}\right)\left(5^{3\cdot\left(\frac{4}{5}x-\frac{5}{8}\right)}\right)\left(5^{2\cdot\left(-5y-\frac{5}{8}\right)}\right)
        }% Bottom of fraction.
        {
        \left(5^{3\cdot\left(\frac{3}{4}x+\frac{1}{2}\right)}\right)\left(5^{2\cdot\left(-\frac{1}{3}y\right)}\right)\left(5^{2\cdot\left(-\frac{1}{8}z-\frac{3}{8}\right)}\right)
        }$
    & Step 2: Simplify power of power in each term.\\
    & $=\frac{% Top of fraction
        \left(5^{z}\right)\left(5^{\frac{12}{5}x-\frac{15}{8}}\right)\left(5^{-10y-\frac{5}{4}}\right)
        }% Bottom of fraction.
        {
        \left(5^{\frac{9}{4}x+\frac{3}{2}}\right)\left(5^{-\frac{2}{3}y}\right)\left(5^{-\frac{1}{4}z-\frac{3}{4}}\right)
        }$
    & Step 3: Distribute and Simplify.\\     
\end{tabular}

Now that we everything in terms of the same base, we can begin merging. First we merge all the top bases together and all the bottom bases together. Then when we are down to only one base with a (large and complicated) exponent, we will merge the top and bottom bases together.\\

\renewcommand{\arraystretch}{2.5}
\noindent\begin{tabular}{rll}
    $\frac{% Top of fraction
        \left(125^{\frac{1}{3}z}\right)\left(125^{\frac{4}{5}x-\frac{5}{8}}\right)\left(25^{-5y-\frac{5}{8}}\right)
        }% Bottom of fraction.
        {
        \left(125^{\frac{3}{4}x+\frac{1}{2}}\right)\left(25^{-\frac{1}{3}y}\right)\left(25^{-\frac{1}{8}z-\frac{3}{8}}\right)
        }$
    & $=\frac{% Top of fraction
        \left(5^{z}\right)\left(5^{\frac{12}{5}x-\frac{15}{8}}\right)\left(5^{-10y-\frac{5}{4}}\right)
        }% Bottom of fraction.
        {
        \left(5^{\frac{9}{4}x+\frac{3}{2}}\right)\left(5^{-\frac{2}{3}y}\right)\left(5^{-\frac{1}{4}z-\frac{3}{4}}\right)
        }$
    & From above.\\     
    & $=\frac{% Top of fraction
        5^{z+\frac{12}{5}x-\frac{15}{8}+-10y-\frac{5}{4}}
        }% Bottom of fraction.
        {
        5^{\frac{9}{4}x+\frac{3}{2} + -\frac{2}{3}y + -\frac{1}{4}z-\frac{3}{4}}
        }$
    & Product of bases equals sum of powers.\\     
    & $=\frac{% Top of fraction
        5^{\frac{12}{5}x - 10y + z - \frac{25}{8}}
        }% Bottom of fraction.
        {
        5^{\frac{9}{4}x - \frac{2}{3}y - \frac{1}{4}z + \frac{3}{4}}
        }$
    & Simplify exponents.\\
    & $= 5^{\frac{12}{5}x - 10y + z - \frac{25}{8} - \left(\frac{9}{4}x - \frac{2}{3}y - \frac{1}{4}z + \frac{3}{4}\right)}$
    & Division of bases is subtraction of exponents.\\
    & $= 5^{\frac{3}{20}x - \frac{28}{3}y + \frac{5}{4}z - \frac{17}{4}}$
    & Simplify Exponent.\\
\end{tabular}
\renewcommand{\arraystretch}{1}
\end{example}





\begin{problem}
    Condense the following expression into a single exponential.\\
    
    \[
        \sage{p1q1} = {\sage{p1b1}}^{\answer{\sage{p1a1}}}
    \]
    \begin{feedback}
        Follow the walkthrough above closely; start with making sure the base is the same for each term. Then combine the bases in the top together using the property $a^ba^c = a^{b+c}$, and do the same to the bottom. Simplify the top and bottom exponents to make your life easier in the next couple steps. Next you want to subtract the bottom power from the top power, and finally simplify.
    \end{feedback}
\end{problem}

\begin{problem}
    Condense the following expression into a single exponential.\\
    
    \[
        \sage{p3q1} = {\sage{p3b1}}^{\answer{\sage{p3a1}}}
    \]
    \begin{feedback}
        Follow the walkthrough above closely; start with making sure the base is the same for each term. Then combine the bases in the top together using the property $a^ba^c = a^{b+c}$, and do the same to the bottom. Simplify the top and bottom exponents to make your life easier in the next couple steps. Next you want to subtract the bottom power from the top power, and finally simplify.
    \end{feedback}
\end{problem}

\begin{problem}
    Condense the following expression into a single exponential.\\
    
    \[
        \sage{p5q1} = {\sage{p5b1}}^{\answer{\sage{p5a1}}}
    \]
    \begin{feedback}
        Follow the walkthrough above closely; start with making sure the base is the same for each term. Then combine the bases in the top together using the property $a^ba^c = a^{b+c}$, and do the same to the bottom. Simplify the top and bottom exponents to make your life easier in the next couple steps. Next you want to subtract the bottom power from the top power, and finally simplify.
    \end{feedback}
\end{problem}

\begin{problem}
    Condense the following expression into a single exponential.\\
    
    \[
        \sage{p2q1} = {\sage{p2b1}}^{\answer{\sage{p2a1}}}
    \]
    (Hint: $\sage{p2b1}^{\sage{p2b2}} = \sage{p2b1^p2b2}$, 
        $\sage{p2b1}^{\sage{p2b3}} = \sage{p2b1^p2b3}$, 
        $\sage{p2b1}^{\sage{p2b4}} = \sage{p2b1^p2b4}$)
    \begin{feedback}
        Follow the walkthrough above closely; start with making sure the base is the same for each term. Then combine the bases in the top together using the property $a^ba^c = a^{b+c}$, and do the same to the bottom. Simplify the top and bottom exponents to make your life easier in the next couple steps. Next you want to subtract the bottom power from the top power, and finally simplify.
    \end{feedback}
\end{problem}

\begin{problem}
    Condense the following expression into a single exponential.\\
    
    \[
        \sage{p4q1} = {\sage{p4b1}}^{\answer{\sage{p4a1}}}
    \]
    (Hint: $\sage{p4b1}^{\sage{p4b2}} = \sage{p4b1^p4b2}$, 
        ${\sage{p4b1}}^{\sage{p4b3}} = \sage{p4b1^p4b3}$, 
        $\sage{p4b1}^{\sage{p4b4}} = \sage{p4b1^p4b4}$)
    \begin{feedback}
        Follow the walkthrough above closely; start with making sure the base is the same for each term. Then combine the bases in the top together using the property $a^ba^c = a^{b+c}$, and do the same to the bottom. Simplify the top and bottom exponents to make your life easier in the next couple steps. Next you want to subtract the bottom power from the top power, and finally simplify.
    \end{feedback}
\end{problem}

\begin{problem}
    Condense the following expression into a single exponential.\\
    
    \[
        \sage{p6q1} = {\sage{p6b1}}^{\answer{\sage{p6a1}}}
    \]
    (Hint: ${\sage{p6b1}}^{\sage{p6b2}} = \sage{p6b1^p6b2}$, 
        $\sage{p6b1}^{\sage{p6b3}} = \sage{p6b1^p6b3}$, 
        $\sage{p6b1}^{\sage{p6b4}} = \sage{p6b1^p6b4}$)
    \begin{feedback}
        Follow the walkthrough above closely; start with making sure the base is the same for each term. Then combine the bases in the top together using the property $a^ba^c = a^{b+c}$, and do the same to the bottom. Simplify the top and bottom exponents to make your life easier in the next couple steps. Next you want to subtract the bottom power from the top power, and finally simplify.
    \end{feedback}
\end{problem}



\begin{example}
Expand the following exponential so that each exponent has at most one term.

\[
    5^{
        \left(\frac{3}{x^{\frac{2}{9}}}\right) + \left( \frac{5}{y^{\frac{4}{9}}} \right) + \left( \frac{2}{z} \right) + \left(\frac{3}{4}\right)
    } = ? \cdot 125^{?} \cdot 3125^{?} \cdot 25^{?}
\]

Solution: Here we are essentially doing the reverse process of the last examples. Our goal is to expand out the expression by writing the given single base as a product of bases with various powers. Moreover, the problems give you the expected bases. Thus we will begin by separating the base on each addition symbol and then pull out the constant factor from each term to form the different numeric bases.

\noindent\begin{tabular}{rll}
    $5^{
        \left(\frac{3}{x^{\frac{2}{9}}}\right) + \left( \frac{5}{y^{\frac{4}{9}}} \right) + \left( \frac{2}{z} \right) + \left(\frac{3}{4}\right)
    }$
    & $= 5^{\frac{3}{x^{\frac{2}{9}}}}\cdot 5^{\frac{5}{y^{\frac{4}{9}}}} \cdot 5^{\frac{2}{z}} \cdot 5^{\frac{3}{4}}$
    & Step 1: Separate terms as product of bases.\\
    & $= 5^{3\left(\frac{1}{x^{\frac{2}{9}}}\right)}\cdot 
        5^{5\left(\frac{1}{y^{\frac{4}{9}}}\right)} \cdot 
        5^{2\left(\frac{1}{z}\right)} \cdot 
        5^{3\left(\frac{1}{4}\right)}$
    & Step 2: Factor out largest constant from each exponent.\\
    & $= (5^3)^{\left(\frac{1}{x^{\frac{2}{9}}}\right)}\cdot 
        (5^5)^{\left(\frac{1}{y^{\frac{4}{9}}}\right)} \cdot 
        (5^2)^{\left(\frac{1}{z}\right)} \cdot 
        (5^3)^{\left(\frac{1}{4}\right)}$
    & Step 3: Product of exponent is repeated power.\\  
    & $= (125)^{\left(\frac{1}{x^{\frac{2}{9}}}\right)}\cdot 
        (3125)^{\left(\frac{1}{y^{\frac{4}{9}}}\right)} \cdot 
        (25)^{\left(\frac{1}{z}\right)} \cdot 
        (125)^{\left(\frac{1}{4}\right)}$
    & Step 4: Calculate bases.\\  
\end{tabular}

We've done the majority of the work here to get the different bases that were expected (notice in the original problem we wanted bases of $125$, $3125$, and $25$, which is exactly what we ended up with!) Now we need to simplify the exponents for each term by making them negative if needed. Remember that the \textbf{power} doesn't change magnitude, \textit{only the sign changes} when you move a term from the bottom to the top of a fraction (or from the top to the bottom).


\renewcommand{\arraystretch}{2.5}
\noindent\begin{tabular}{rll}
    $5^{
        \left(\frac{3}{x^{\frac{2}{9}}}\right) + \left( \frac{5}{y^{\frac{4}{9}}} \right) + \left( \frac{2}{z} \right) + \left(\frac{3}{4}\right)
    }$
    & $= 125^{\left(\frac{1}{x^{\frac{2}{9}}}\right)}\cdot 
        3125^{\left(\frac{1}{y^{\frac{4}{9}}}\right)} \cdot 
        25^{\left(\frac{1}{z}\right)} \cdot 
        125^{\left(\frac{1}{4}\right)}$
    & From above.\\
    & $= 125^{\left(x^{-\frac{2}{9}}\right)}\cdot 
        3125^{\left(y^{-\frac{4}{9}}\right)}\cdot 
        25^{\left(z^{-1}\right)}\cdot 
        125^{\left(\frac{1}{4}\right)}$
    & Rewrite fractional exponents with negatives.\\
    & $= \left(125^{\left(\frac{1}{4}\right)}\right)\cdot
        \left(125^{\left(x^{-\frac{2}{9}}\right)}\right)\cdot
        \left(3125^{\left(y^{-\frac{4}{9}}\right)}\right)\cdot
        \left(25^{\left(z^{-1}\right)}\right)$
    & Rewrite to match original base order.
\end{tabular}
\renewcommand{\arraystretch}{1}
\end{example}


\begin{problem}
    Expand the following exponential so that each exponent has at most one term.
    
    \[
        \sage{p7q1} 
            = \answer{\sage{p7a1}} \cdot {\sage{p7b1^p7b2}}^{\answer{x^{\sage{p7r1}}}} \cdot  {\sage{p7b1^p7b3}}^{\answer{y^{\sage{p7r2}}}} \cdot {\sage{p7b1^p7b4}}^{\answer{z^{\sage{p7r3}}}}
    \]
    \begin{feedback}
        Follow the walkthrough above closely; essentially doing the previous problem steps in reverse. Start by writing the exponent without any variables in the bottom of fractions (move them to the top and make them negative). Then separate the exponent by using the property that $a^{b+c} = a^ba^c$. Once the exponents are separated and you have a bunch of parts with the same base, rewrite each base with any numeric coefficient; for example instead of $4^{2x^2}$ you want to rewrite this as $\left(4^2\right)^{x^2}$ which you can compute to $16^{x^2}$. Match up the base to the answer box for that base (or the exponent with the answer box with that exponent) and fill in the missing info.
    \end{feedback}
\end{problem}

\begin{problem}
    Expand the following exponential so that each exponent has at most one term.
    
    \[
        \sage{p8q1} 
            = \answer{\sage{p8a1}} \cdot {\answer{\sage{p8b1^p8b2}}}^{{x^{\sage{p8r1}}}} \cdot  {\sage{p8b1^p8b3}}^{\answer{y^{\sage{p8r2}}}} \cdot {\answer{\sage{p8b1^p8b4}}}^{{z^{\sage{p8r3}}}}
    \]
    \begin{feedback}
        Follow the walkthrough above closely; essentially doing the previous problem steps in reverse. Start by writing the exponent without any variables in the bottom of fractions (move them to the top and make them negative). Then separate the exponent by using the property that $a^{b+c} = a^ba^c$. Once the exponents are separated and you have a bunch of parts with the same base, rewrite each base with any numeric coefficient; for example instead of $4^{2x^2}$ you want to rewrite this as $\left(4^2\right)^{x^2}$ which you can compute to $16^{x^2}$. Match up the base to the answer box for that base (or the exponent with the answer box with that exponent) and fill in the missing info.
    \end{feedback}
\end{problem}

\begin{problem}
    Expand the following exponential so that each exponent has at most one term.
    
    \[
        \sage{p9q1} 
            = \answer{\sage{p9a1}} \cdot {\sage{p9b1^p9b2}}^{\answer{x^{\sage{p9r1}}}} \cdot  \answer{\sage{p9b1^p9b3}}^{y^{\answer{\sage{p9r2}}}} \cdot {\sage{p9b1^p9b4}}^{z^{\answer{\sage{p9r3}}}}
    \]
    \begin{feedback}
        Follow the walkthrough above closely; essentially doing the previous problem steps in reverse. Start by writing the exponent without any variables in the bottom of fractions (move them to the top and make them negative). Then separate the exponent by using the property that $a^{b+c} = a^ba^c$. Once the exponents are separated and you have a bunch of parts with the same base, rewrite each base with any numeric coefficient; for example instead of $4^{2x^2}$ you want to rewrite this as $\left(4^2\right)^{x^2}$ which you can compute to $16^{x^2}$. Match up the base to the answer box for that base (or the exponent with the answer box with that exponent) and fill in the missing info.
    \end{feedback}
\end{problem}

\begin{problem}
    Expand the following exponential so that each exponent has at most one term.
    
    \[
        \sage{p10q1} 
            = \answer{\sage{p10a1}} \cdot {\sage{p10b1^p10b2}}^{\answer{x^{\sage{p10r1}}}} \cdot  {\sage{p10b1^p10b3}}^{\answer{y^{\sage{p10r2}}}} \cdot {\sage{p10b1^p10b4}}^{\answer{z^{\sage{p10r3}}}}
    \]
    \begin{feedback}
        Follow the walkthrough above closely; essentially doing the previous problem steps in reverse. Start by writing the exponent without any variables in the bottom of fractions (move them to the top and make them negative). Then separate the exponent by using the property that $a^{b+c} = a^ba^c$. Once the exponents are separated and you have a bunch of parts with the same base, rewrite each base with any numeric coefficient; for example instead of $4^{2x^2}$ you want to rewrite this as $\left(4^2\right)^{x^2}$ which you can compute to $16^{x^2}$. Match up the base to the answer box for that base (or the exponent with the answer box with that exponent) and fill in the missing info.
    \end{feedback}
\end{problem}


\end{document}